\documentclass[12pt, preprint]{aastex}
\usepackage{hyperref}
\usepackage{environ}
\hypersetup{
    colorlinks,
    citecolor=black,
    filecolor=black,
    linkcolor=black,
    urlcolor=black
}

\newif\ifanswers

\NewEnviron{answer}[1][]{\ifanswers\color{blue}\expandafter\BODY
{\par \it #1}\fi}

\answersfalse
\answerstrue

\begin{document}

\newcommand{\degree}{\ensuremath{{}^\circ}}

\section{Time and Date of Transit}

\begin{itemize}
\item If a star transits at 10pm tonight, when will it transits tomorrow?

  \begin{answer}
    It rises 4 minutes earlier each day, so 9:56pm.
  \end{answer}
\item If a star transits at 10pm tonight, when will it transits a month from
  now?

  \begin{answer}
    It rises $\sim$ 4 minutes $\times$ 30 days $=$ 120 minutes $=$ 2
    hours earlier each month, so 8:00pm.
  \end{answer}
\item What RA is transiting on April 21 at midnight?

  \begin{answer}
    We know RA$=$12h transits on March 21 at midnight. As the year
    goes on, the RA transiting at a given time goes up. So one month
    later, the RA transiting at midnight will be 2h higher, or
    14h. The RA transiting at any given time is called the Local
    Sidereal Time.
    \begin{equation}
    \begin{array}{ccc}
      \mbox{\bf RA} & \mbox{\bf Day} & \mbox{\bf Time} \cr
      \mbox{12h} & \mbox{March 21} & \mbox{12 am}\cr
      \mbox{2h} & \mbox{April 21}  & \mbox{12 am}\cr
    \end{array}
    \end{equation}
  \end{answer}
\item What RA is transiting on October 25 at 3:30pm?

  \begin{answer}
    We know RA$=$0h transits on Sept 23 at midnight. As the year
    goes on, the RA transiting at a given time goes up. So one month
    later, the RA transiting at midnight will be 2h higher, or
    RA$=$2h. Eight and a half hours earlier, at 3:30pm, the transiting
    RA will be 17h 30m.
    \begin{equation}
    \begin{array}{ccc}
      \mbox{\bf RA} & \mbox{\bf Day} & \mbox{\bf Time} \cr
      \mbox{0h} & \mbox{Sept 23} & \mbox{12 am}\cr
      \mbox{2h} & \mbox{Oct 25}  & \mbox{12 am}\cr
      \mbox{17h 30m} & \mbox{Oct 25}  & \mbox{3:30 pm}\cr
    \end{array}
    \end{equation}
  \end{answer}
\item If a star is at RA$=$17h, on what day does it transit at midnight?

  \begin{answer}
    We know that RA$=$18h transits at midnight around June 21, and the
    RA transiting at a given time increases throughout the year.
    Therefore RA$=$17h transits earlier in the year, by about half a
    month, so around June 6.
    \begin{equation}
    \begin{array}{ccc}
      \mbox{\bf RA} & \mbox{\bf Day} & \mbox{\bf Time} \cr
      \mbox{18h} & \mbox{Jun 21} & \mbox{12 am}\cr
      \mbox{17h} & \mbox{Jun 6}  & \mbox{12 am}\cr
    \end{array}
    \end{equation}
  \end{answer}
\item If a star is at RA$=$8h, on what day does it transit at midnight?

  \begin{answer}
    We know that RA$=$6h transits at midnight around December 21, and
    the RA transiting at a given time increases throughout the year.
    Therefore RA$=$8h transits later in the year, by about a month,
    so around January 21.
    \begin{equation}
    \begin{array}{ccc}
      \mbox{\bf RA} & \mbox{\bf Day} & \mbox{\bf Time} \cr
      \mbox{6h} & \mbox{Dec 21} & \mbox{12 am}\cr
      \mbox{8h} & \mbox{Jan 21}  & \mbox{12 am}\cr
    \end{array}
    \end{equation}
  \end{answer}
\item What time does RA$=$18h transit on October 21?

  \begin{answer}
    We know RA$=$0h transits on Sept 23 at midnight. So RA$=$18h
    transits six hours earlier that night around 6pm.
    The transit occurs
    earlier as the year goes on. So a month later it will transit two
    hours earlier, at 4pm. Note how on the chart below, if I keep the
    RA fixed and change the {\it day}, I must change the {\it time} in
    the opposite way.
    \begin{equation}
    \begin{array}{ccc}
      \mbox{\bf RA} & \mbox{\bf Day} & \mbox{\bf Time} \cr
      \mbox{0h} & \mbox{Sept 23} & \mbox{12 am}\cr
      \mbox{18h} & \mbox{Sept 23} & \mbox{6 pm}\cr
      \mbox{18h} & \mbox{Oct 21}  & \mbox{4 pm}\cr
    \end{array}
    \end{equation}
  \end{answer}
\item If a star is at RA$=$3h, on what day does it transit at midnight?

  \begin{answer}
    We know that RA$=$0h transits at midnight around September 23, and
    the RA transiting at a given time increases throughout the year.
    Therefore RA$=$3h transits later in the year, by about 1.5 months,
    so around November 5.
    \begin{equation}
    \begin{array}{ccc}
      \mbox{\bf RA} & \mbox{\bf Day} & \mbox{\bf Time} \cr
      \mbox{0h} & \mbox{Sept 23} & \mbox{12 am}\cr
      \mbox{3h} & \mbox{Nov 5}  & \mbox{12 am}\cr
    \end{array}
    \end{equation}
  \end{answer}
\item If a star is at RA$=$3h, on what day does it transit at 3am?

  \begin{answer}
    On the day that RA$=$3h transits at 3am, RA$=$0h must transit
    three hours earlier at midnight (12am). RA$=$0h transits at
    midnight on September 23, so RA$=$3h transits at 3am that same
    night.
    \begin{equation}
    \begin{array}{ccc}
      \mbox{\bf RA} & \mbox{\bf Day} & \mbox{\bf Time} \cr
      \mbox{0h} & \mbox{Sept 23} & \mbox{12 am}\cr
      \mbox{3h} & \mbox{Sept 23}  & \mbox{3 am}\cr
    \end{array}
    \end{equation}
  \end{answer}
\item If a star is at RA$=$15h, on what day does it transit at 8pm?

  \begin{answer}
    On the day that RA$=15$h transits at 8pm, RA$=$19h must transit
    four hours later at midnight (12am). RA$=$18h transits at
    midnight on June 21. RA$=$19h must therefore transit at
    midnight around July 5. Then RA$=$15h must transit 4 hours earlier
    at 8pm around July 5.
    \begin{equation}
    \begin{array}{ccc}
      \mbox{\bf RA} & \mbox{\bf Day} & \mbox{\bf Time} \cr
      \mbox{18h} & \mbox{June 21} & \mbox{12 am}\cr
      \mbox{19h} & \mbox{July 5}  & \mbox{12 am}\cr
      \mbox{15h} & \mbox{July 5}  & \mbox{8 pm}\cr
    \end{array}
    \end{equation}
  \end{answer}
\item If a star is at RA$=$1h, on what day does it transit at 2am?

  \begin{answer}
    On the day that RA$=$1h transits at 2am, RA$=$23h must transit two
    hours earlier at midnight (12am). RA$=$0h transits at midnight on
    September 23. RA$=$23h must therefore transit at midnight half a
    month earlier, around September 8. Then RA$=$1h must transit 2
    hours later at 2am around September 8.
    \begin{equation}
    \begin{array}{ccc}
      \mbox{\bf RA} & \mbox{\bf Day} & \mbox{\bf Time} \cr
      \mbox{0h} & \mbox{Sept 23} & \mbox{12 am}\cr
      \mbox{2h} & \mbox{Sept 23}  & \mbox{2 am}\cr
      \mbox{1h} & \mbox{Sept 8}  & \mbox{2 am}\cr
    \end{array}
    \end{equation}
  \end{answer}
\item What time does RA$=$7h transit on April 5?

  \begin{answer}
    We know RA$=$12h transits on March 21 at midnight. So RA$=$7h
    transits at 7pm on March 21. It transits
    earlier as the year goes on. So half a month later it will transit
    one hour earlier, at 6pm. 
    \begin{equation}
    \begin{array}{ccc}
      \mbox{\bf RA} & \mbox{\bf Day} & \mbox{\bf Time} \cr
      \mbox{12h} & \mbox{March 21} & \mbox{12 am}\cr
      \mbox{7h} & \mbox{March 21}  & \mbox{7 pm}\cr
      \mbox{7h} & \mbox{April 5}  & \mbox{6 pm}\cr
    \end{array}
    \end{equation}
  \end{answer}
\item What is the hour angle of a star with an RA$=$9h observed at
  7pm on March 7?

  \begin{answer}
    We know RA$=$12h transits on March 21 at midnight. So RA$=$11h
    transits at midnight on March 7. Five hours earlier at 7pm on
    March 7, RA$=$6h is transiting. That is, the current LST$=$6h at
    that time. Since the hour angle HA $=$ LST $-$ RA, the
    HA$=$-3h. That is, this star is rising.
  \end{answer}
\end{itemize}

\section{Magnitudes}

\begin{itemize}
\item Which is brighter, a magnitude 3 star or a magnitude 5 star?

  \begin{answer}
    Higher magnitudes indicate fainter objects, so the magnitude 3
    star is brighter.
  \end{answer}

\item How much brighter is a magnitude 0 star than a magnitude 7.5
  star?

  \begin{answer}
    Each difference of 2.5 in the magnitude is a factor of 10 in the
    brightness. The difference of $7.5 = 3\times 2.5$, and three
    factors of $10 = 10^3 = 1000$. So the magnitude 0 star is 1000
    times brighter than a magnitude 7.5 star.
  \end{answer}

\item If there is a factor of a million times difference in
  brightness between two stars, how many magnitudes does that
  correspond to?

  \begin{answer}
    Each factor of 10 in brightness is a difference of 2.5 in
    magnitude. One million is $10^6$, or six factors of ten, so the
    difference in magnitudes is $6\times 2.5 = 15$.
  \end{answer}

\end{itemize}

\section{Altitude at Transit}

What are the altitudes at transit for stars observed under the
following situations?

\begin{itemize}
  \item Observer at latitude 45\degree, star at a declination of
    10\degree

    \begin{answer}
      \begin{eqnarray}
      {\rm max. alt} &=& 90\degree - \left|{\rm Dec} - {\rm
        Lat}\right| \cr
      &=& 90\degree - 35\degree = 55\degree
      \end{eqnarray}
    \end{answer}
  \item Observer at latitude 45\degree, star at a declination of
    75\degree

    \begin{answer}
      \begin{eqnarray}
      {\rm max. alt} &=& 90\degree - \left|{\rm Dec} - {\rm
        Lat}\right| \cr
      &=& 90\degree - 30\degree = 60\degree
      \end{eqnarray}
    \end{answer}
  \item Observer at latitude 90\degree, star at a declination of
    $-$5\degree

    \begin{answer}
      \begin{eqnarray}
      {\rm max. alt} &=& 90\degree - \left|{\rm Dec} - {\rm
        Lat}\right| \cr
      &=& 90\degree - 95\degree = -5\degree
      \end{eqnarray}
      Thus the star never rises.
    \end{answer}
  \item Observer at latitude -50\degree, star at a declination of
    $-$5\degree

    \begin{answer}
      \begin{eqnarray}
      {\rm max. alt} &=& 90\degree - \left|{\rm Dec} - {\rm
        Lat}\right| \cr
      &=& 90\degree - 45\degree = 55\degree
      \end{eqnarray}
    \end{answer}
\end{itemize}

\section{Small Angle Approximation}

Please note that the small angle approximation is only good for small
angles! Less than a degree is best, although more approximately it can
be used up to about 10 degrees. 

\begin{itemize}
  \item I am standing 15 feet away from a window 3 feet wide. I see a
    person outside (probably 6 feet tall) lying on the ground, and
    from my point view they ``just fit'' inside the window. How far
    away are they from me?

    \begin{answer}
      This is just a problem of {\it similar triangles}, which are
      triangles with the same angles but just scaled up or down from
      each other in size. The triangle formed by me and the two edges
      of the window frame is {\it similar} to the triangle formed by
      me and the edges (head and feet) of the person. Since the person
      is twice the length of the window, the second triangle is twice
      as big on every side. So the distance to the person is twice the
      distance to the window, or 30 feet.
    \end{answer}
  \item My finger is a centimeter in width, and my arm is about 60 cm
    long. If I view my finger with my arm stretched out, what angle on
    the sky does my finger cover? If a person (5 feet tall, say) is
    standing some distance away from me, such that my finger just
    about completely covers my view of them, how far away are they?

    \begin{answer}
      Now we have to use the small angle approximation formula:
      \begin{equation}
        \frac{\theta}{60\degree} \sim \frac{s}{D}
      \end{equation}
      where here $s$ is the width of my finger, and $D$ is the length
      of my arm. So:
      \begin{equation}
        \frac{\theta}{60\degree} \sim \frac{1 {\rm ~cm}}{60 {\rm ~cm}}
        = \frac{1}{60}
      \end{equation}
      and therefore $\theta\sim 1\degree$.

      To answer the second part of the question, we can use the method
      of similar triangles as in the last question, which would tell
      us that the person must be 60 times further away than they are
      tall, so about 300 feet.

      Or we can use the small angle approximation again, this time
      using the fact that we know what angle our finger subtends. So
      $\theta \sim 1\degree$ and $s=$5 feet in this formula:
      \begin{equation}
        \frac{\theta}{60\degree} \sim \frac{s}{D}
      \end{equation}
      which means:
      \begin{equation}
        D \sim \frac{60\degree}{\theta} \times s = 60\times (5{\rm
          ~ft}) = 300 {\rm ~ft}.
      \end{equation}
    \end{answer}
  \item The Moon's diameter is 0.5\degree\ on the sky, and is 400,000 km
    away. How big is the Moon's diameter in km?

    \begin{answer}
      In this case $s$ is the Moon's diameter, and $D$ is 400,000
      km. So:
      \begin{eqnarray}
        \frac{\theta}{60\degree} &\sim& \frac{s}{D} \cr
        \frac{0.5 \degree}{60\degree} &\sim& \frac{s}{400,000 {\rm
            ~km}} \cr
        \frac{400,000 {\rm ~km}}{120} &\sim& s \cr
        s \sim 3300 {\rm ~km},
      \end{eqnarray}
      which is pretty close to the right answer of 3,500 km.
    \end{answer}
  \item Assume that Andromeda is the same size as the Milky Way (about
    100,000 light years in diameter). It spans about 3\degree\ on the
    sky. How far away is it?

    \begin{answer}
      In this case, $\theta\sim 3\degree$ and $s=100,000$ light
      years. So
      \begin{eqnarray}
        \frac{\theta}{60\degree} &\sim& \frac{s}{D} \cr
        D \sim \frac{60\degree}{\theta} \times s \cr
        D \sim 20 \times s \sim 2,000,000 {\rm ~lyr}\cr
      \end{eqnarray}
      This is a bit of an underestimate, because Andromeda is a bit
      bigger in light years than the Milky Way.
    \end{answer}
\end{itemize}

\section{Telescopes}

\begin{itemize}
  \item Through the finder scope of my telescope, which has an
    aperture of 50 mm, I can see down to 8th magnitude. If all that
    mattered was the telescope aperture, down to what magnitude should
    I be able to see with the telescope itself, if it has an aperture
    of 50 cm (20 inches)?

    \begin{answer}
      First we calculate the difference in collecting area, which is
      $\pi r^2$. The ratio of the collecting areas (telescope over
      finder scope) is:
      \begin{equation}
        R = \frac{\pi r_{t}^2}{\pi r_f^2} =
        \frac{r_{t}^2}{r_f^2} = \left(\frac{r_t}{r_f}\right)^2 =
        \left(\frac{50 {\rm ~cm}}{50 {\rm ~mm}}\right)^2
        = \left(\frac{500 {\rm ~mm}}{50 {\rm ~mm}}\right)^2
        = \left(10\right)^2 = 100
      \end{equation}
      So if all that matters is the telescope aperture, it collects
      100 times more light so should see things 100 times fainter. In
      magnitudes that is 5 magnitudes (see problems above!), so the
      telescope would see down to 13th mag.

      In reality other factors may come into play. The higher
      magnification of the telescope will (usually) help because it
      will make stars stand out better against the background sky. So
      the improvement could be more than 5 magnitudes.
    \end{answer}

  \item If I am using a telescope with an objective focal length of
    2500 mm (like the ones in class) and an eyepiece with a focal
    length of 25 mm, what is the magnification and the field of view?

    \begin{answer}
      The magnification can be calculated from the objective focal
      length divided by the eyepiece focal length:
      \begin{equation}
        M = \frac{2500 {\rm ~mm}}{25 {\rm ~mm}} = 100
      \end{equation}
      This means that two stars an arcminute apart on the sky will
      appear to be 100 arcminutes apart when you view them through the
      eyepiece. 

      The field of view through the eyepiece typically covered about
      50\degree\ of your vision. But of course what you see is highly
      magnified by a factor $M$. So what you are seeing is not
      50\degree\ on the sky, but is 50\degree$/M \sim 0.5\degree$.
    \end{answer}
\end{itemize}



\end{document}
