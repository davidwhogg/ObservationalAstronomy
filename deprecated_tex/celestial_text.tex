

\bigskip

\noindent
{Objective:} The celestial globe is a simple device but one of the
best ways to develop clear ideas on how the sky works. Go slowly
through sections 1--3 to make sure you understand each point
clearly. Answer the questions in the remaining sections as you work
through them. If in doubt, ask the instructor.

\medskip
\noindent
{\bf 1. Finding your way around the globe.}

\medskip
\noindent
The celestial globe represents a view of the universe as seen from planet
Earth. Because one does not perceive the depth of space, it is
convenient to represent the \emph{directions} of things by their
positions on a sphere, centered on Earth. It is important to realize
that in this scheme, the Earth should really be represented by a
very small sphere rather that the 6 inch diameter one in our
model, so you need to mentally shrink it down in size.

\medskip\noindent
\emph{Stars}. Stars are painted on the globe as yellow dots. The
larger the dot the brighter the star, i.e., the dot size is an
indication of the star's \emph{magnitude}. For many of the stars the
Greek letters or proper names are given.  The constellation names are
given in blue boxes, and the official constellation boundaries are
given by the blue dotted lines. You will see that the whole sphere is
completely covered by the constellations (88 in number), like a
quilt. Note that the sphere is to be read by looking through one side
to the other, so that we see the same patterns as the small Earthlings
living on the plastic Earth.

\medskip\noindent
\emph{Sky coordinates}. Special points on the celestial sphere are the
North celestial pole (NCP) and the South celestial pole (SCP) at the
convergence points of the black lines, and the celestial equator (CE)
which is around the join of the two hemispheres of the celestial
globe. You can see that the poles and the equator are directly above
Earth's poles and equator, respectively.  The black grid lines
represent the celestial graph paper of Right Ascension (RA) and
Declination (Dec). The lines joining the poles are RA lines: they are
labeled in hours from 1--24 hr (smaller subdivisions not shown on the
globe are minutes and seconds). The lines parallel to the equator are
Dec lines; they run from 0\deg\ (on the equator) to +90\deg\ at the
NCP and $-$90\deg\ at the SCP.

\medskip\noindent
\emph{Sun and sky motion}.  The black knob controls a yellow blob
which represents the sun -- not to scale relative to the Earth! As you
move the sun around it moves along a path on the globe called the
\emph{ecliptic}. Along the ecliptic are dates, which tell you where
the sun is on the globe during the course of a year. The silver Knob
turns the earth relative to the sky, or allows you to hold the Earth
steady and twist the whole sky around if you prefer (this is more like
what seems to happen when you actually go outside and see the
heavens). The sky turns round Earth (or the Earth spins round relative
to the sky) once in 23 hr and 56 min (nearly 24 hr, but not quite).

\medskip\noindent
\emph{Horizon coordinates}.
Finally, the flat, plastic ring around the Earth represents the horizon for
an Earthling centered on the hemisphere that is sliced off by the
ring. For this to be convincing you have to imagine the Earth model
shrunk down to be tiny. An observer therefore sees half the celestial
sphere at one time. The ring is marked with N, NE, E etc, and is also
marked in degrees from 0\deg\ to 360\deg\ which we call the
azimuth. This angle starts at North and increases to the
East. Relative to an observer's horizon, the position of a star can be
given by the angular distance above the horizon (Alt) and this azimuth
(Az). The convenient hole in the celestial sphere centered near RA 2
hr 30 min, Dec $-$25\deg\ (check this!) allows you to adjust the
horizon so it refers to any place on Earth.

\bigskip
\noindent
{\bf 2. Setting up the globe for tonight}

\medskip
\noindent
Now we shall set up the globe to represent the sky in New York city, for this
evening at 8:00 pm.
\begin{itemize}
\item Set the sun at its position on the globe corresponding to
today's date.
\item Turn and tilt the celestial globe so that the point at the top
is RA = 4 hr, Dec = +40\deg. Why these numbers ? We will get to that
later, but note for now that the Dec setting is related to the
latitude of NY which is +40\deg\, and the RA setting is related to the
date and time of day.
\item Turn the earth by the silver handle so that New York is on the
top of the Earth. It is easiest to do this by looking down from above
and lining up the RA and Dec position with NYC.
\item Move and flatten the horizon ring so that it is horizontal.
\item recheck the above steps to see that the settings have not moved.
\end{itemize}
If you now imagine the Earth and yourself shrunk much smaller, and
you go and stand on NY on the model Earth, the hemisphere of the globe above
the horizon ring represents the sky as seem from NY. Simple !?

\bigskip\noindent
{\bf 3. A first look at the stars}

\medskip
\noindent 
Examine the the stars and constellations visible this evening in NY
(remember, look through the globe). At
the same time of day they will be roughly the same throughout the
semester so you will get to know them well. Note also the location of
the celestial poles (only one visible) and the CE in the sky. At this
stage it is very useful to compare this view and identify some of the
stars with the maps in the Field Guide (FG) and atlas (M5).

In the FG sky map 12 (following page 53) most closely corresponds to
this time and date. There are 2 versions, one with star names and one
without; there are also two views, one looking North and one looking
South.

In the M5, map 1 corresponds to the region near the NCP and map 3
corresponds to the section of the sphere near the zenith (straight up)
and extending to the South.


\newpage

\bigskip\noindent
{\bf 4. The brightest stars}

\bigskip\noindent
Locate the following bright stars on the globe,
and fill in the table from the information on the globe. Estimate the
Dec in degrees to the nearest degree, and the RA in hr and minutes
(there are 60 min in each hr division).
\begin{center}
\begin{tabular}{lccc} \hline
Star  & \ \ \ \ Constellation \ \ &\hspace{1.5cm} RA \hspace{1.5cm} & \ \ Dec \ \ \ \ \\
          &     & (hr:min) & (degrees)   \\ \hline
Sirius   &  &   &        \\ \hline
Regulus   &  &   &       \\ \hline
Betelgeuse  &  &   &       \\ \hline
Pollux  &  &   &       \\ \hline
Aldebaran  &  &   &       \\  \hline
Capella &  &   &     \\  \hline 
\end{tabular}
\end{center}

\bigskip
Which of the above stars is:
\medskip
Nearest the zenith (Alt=90\deg) ? \ \makebox[4cm]{\hrulefill}

Nearest the NCP  ?\ \makebox[4cm]{\hrulefill} \ \ 
Nearest the CE   ?\ \makebox[4cm]{\hrulefill}

\bigskip\noindent
{\bf 5. Angles}

\bigskip\noindent
Make an angle ruler along the edge of a piece of paper by making several
successive divisions of 3 cm which corresponds to 10\deg\ on the globe
-- check this on the globe against the Dec scale. (If you have a non
standard sized globe, ask the instructor or use the Dec scale to figure
it out.) Use this ruler to estimate the angular separation (the
shortest distance on the globe) between the
following pairs of stars.

\bigskip
Betelgeuse--Sirius: \makebox[4cm]{\hrulefill}

Pollux-Castor (the Twins): \makebox[4cm]{\hrulefill}

Capella--Aldebaran: \makebox[4cm]{\hrulefill}
  
\bigskip
\noindent
{\bf 6. Altitude and Azimuth}

\bigskip
\noindent
Using the angle ruler and the Az gradations on the horizon ring, 
estimate the Alt and Az of the following:

\medskip   
Sirius: Alt:= \makebox[4cm]{\hrulefill} Az= \makebox[4cm]{\hrulefill}

The NCP:  Alt:= \makebox[4cm]{\hrulefill} Az= \makebox[4cm]{\hrulefill}

Regulus: Alt:= \makebox[4cm]{\hrulefill} Az= \makebox[4cm]{\hrulefill}

\bigskip

\noindent
{\bf 7. Where are the planets this evening?}

\bigskip
\noindent
Below is list of the RA and Dec coordinates of the planets for
(approximately) this evening. They do not change by very much during
the course of the semester. Enter the coordinates in the table below
and locate the planets on the globe with small pieces of paper. Find
the constellation each planet is in, and estimate its Alt and Az. If
it is below the horizon, enter ``Not visible'' in the Alt-Az columns.

\medskip
\begin{center}
\begin{tabular}{lccccc} \hline
Planet  & \ \ \ \ RA  \ \ \ & \hspace{1.5cm} Dec \hspace{1.5cm} &
         Constellation &\ \ \ \  Alt \ \ \ \ & \ \ \ \  Az\ \ \  \ \\ 
         \hline
Mercury   & 19h~03m & $-19^\circ 40'$  & & &       \\ \hline
Venus   & 18h~10m & $-22^\circ 26'$  &     & &    \\ \hline
Mars & 14h~39m & $-13^\circ 56'$  &     & &    \\ \hline
Jupiter  & 11h~35m & $04^\circ 13'$  &   & &      \\ \hline
Saturn  & 16h~49m & $-20^\circ 46'$  &   & &      \\  \hline
Uranus & 01h~03m & $-06^\circ 00'$  &    & &   \\  \hline 
Neptune  & 22h~40m & $-09^\circ 17'$  &   & &    \\  \hline 
Pluto  & 19h~07m &  $-20^\circ 58'$ &   & &    \\  \hline 
\end{tabular}
\end{center}

\noindent The planets will be seen to lie near a particular feature of the
celestial sphere. 

\noindent What is this? \makebox[4cm]{\hrulefill}

\noindent Where is Mercury relative to the Sun, and why?

\bigskip
\bigskip
\bigskip
\noindent 
{\bf 8. Postscript}

\bigskip\noindent
Try turning the Earth (with the silver knob clockwise) to see how the
stars change over the course of a day (which corresponds to one full
turn) to an observer in NY.  You will see that the RA and Dec of the
stars remains the same, but their Alt and Az change. Stars rise on the
East horizon, and eventually the sun rises in the East and day
begins. Some stars are always above the horizon, and some stars never
are.


