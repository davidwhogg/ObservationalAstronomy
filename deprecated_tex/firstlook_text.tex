
\marginpar{
\vspace{-1.6cm}
\hfill 00$\deg$\makebox[2cm]{\hrulefill} \\
\vspace{-0.028cm}
\hfill 01$\deg$\makebox[2cm]{\hrulefill} \\
\vspace{-0.028cm}
\hfill 02$\deg$\makebox[2cm]{\hrulefill} \\
\vspace{-0.028cm}
\hfill 03$\deg$\makebox[2cm]{\hrulefill} \\
\vspace{-0.028cm}
\hfill 04$\deg$\makebox[2cm]{\hrulefill} \\
\vspace{-0.028cm}
\hfill 05$\deg$\makebox[2cm]{\hrulefill} \\
\vspace{-0.028cm}
\hfill 06$\deg$\makebox[2cm]{\hrulefill} \\
\vspace{-0.028cm}
\hfill 07$\deg$\makebox[2cm]{\hrulefill} \\
\vspace{-0.028cm}
\hfill 08$\deg$\makebox[2cm]{\hrulefill} \\
\vspace{-0.028cm}
\hfill 09$\deg$\makebox[2cm]{\hrulefill} \\
\vspace{-0.028cm}
\hfill 10$\deg$\makebox[2cm]{\hrulefill} \\
\vspace{-0.028cm}
\hfill 11$\deg$\makebox[2cm]{\hrulefill} \\
\vspace{-0.028cm}
\hfill 12$\deg$\makebox[2cm]{\hrulefill} \\
\vspace{-0.028cm}
\hfill 13$\deg$\makebox[2cm]{\hrulefill} \\
\vspace{-0.028cm}
\hfill 14$\deg$\makebox[2cm]{\hrulefill} \\
\vspace{-0.028cm}
\hfill 15$\deg$\makebox[2cm]{\hrulefill} \\
\vspace{-0.028cm}
\hfill 16$\deg$\makebox[2cm]{\hrulefill} \\
\vspace{-0.028cm}
\hfill 17$\deg$\makebox[2cm]{\hrulefill} \\
\vspace{-0.028cm}
\hfill 18$\deg$\makebox[2cm]{\hrulefill} \\
\vspace{-0.028cm}
\hfill 19$\deg$\makebox[2cm]{\hrulefill} \\
\vspace{-0.028cm}
\hfill 20$\deg$\makebox[2cm]{\hrulefill} \\
\vspace{-0.028cm}
\hfill 21$\deg$\makebox[2cm]{\hrulefill} \\
\vspace{-0.028cm}
\hfill 22$\deg$\makebox[2cm]{\hrulefill} \\
\vspace{-0.028cm}
\hfill 23$\deg$\makebox[2cm]{\hrulefill} \\
\vspace{-0.028cm}
\hfill 24$\deg$\makebox[2cm]{\hrulefill} \\
\vspace{-0.028cm}
\hfill 25$\deg$\makebox[2cm]{\hrulefill} \\
}


\bigskip

\noindent
{Objective:} To find our way around the sky by identifying the
brightest stars and constellations, and by measuring some angles
in the sky.

\bigskip\noindent
{\bf INSIDE PREPARATIONS}
\bigskip

\noindent
{\bf 1. Review of tonight's sky}

\medskip
\noindent
The stars visible outside tonight are shown in the Field Guide (FG) on
sky map 12, in the section following p 53. There are two versions of
the map, with and without star names. The maps are to be held above
your head and tilted down to the horizon in the direction you are
looking.

Outside we are going to first use a simplified version of this map
shown in Fig.~1. Check that it is essentially the same as in the FG,
and identify a few of the main constellations in the M5 atlas (e.g.,
Orion in map 3). If there are any bright planets visible, the instructor
will give you their coordinates: using the M5 as a guide to the RA and
Dec lines, draw them in on Fig.~1.

Examine map 1 in the M5 atlas, centered on Polaris which lies within
1\deg\ of the North celestial pole (NCP). Note the position of Polaris
with respect to the W of Cassiopeia and the pointers Merak and Duhbe
of the Big Dipper in Ursa Major. Note also the hr markers around the
outside of the map.

\bigskip
\noindent
{\bf 2. Measuring angles}

\medskip
\noindent
An important aspect of locating and
identifying celestial objects is their positions. There is no
perception of depth in space, but we can describe the apparent
position of a star or planet using angles. For example, we can specify
the angle between two stars, or the altitude (Alt) and azimuth (Az) of
a single star. Recall that Alt is the angle from the horizon up to the
star, and Az is the angle from the North horizon (measured Eastwards)
to the point on the horizon immediately beneath the star. In all these
cases, the ``angle'' means the angle between lines in the two
directions that meet at us, or at our eyes.

One rough way to measure large angles is to point with straight arms
in the two directions at the same time, and to estimate the angle
between your arms.

A second way is to remember some typical angle sizes: e.g.,  at
arms length, the width of a fist including your thumb is about 10\deg\
and the width of your little finger is about 1\deg.

A more accurate way to measure angles is with some kind of measuring
device. We shall use a simple angle ruler that, when held at arm's
length at right angles to your arm, will measure angles.

Simple trigonometry tells us that the angle $\theta$ (in degrees)
subtended at our eye by an object of size $s$ (across our line of
sight) at a distance D from us, is given by the formula $\theta =
(360/2\pi)\,s/D$ or $57.3\, s/D$, when the angle is not too large.
For $\theta$ equal to 1\deg, then $D = 57.3\, s$.  We use this to
construct an angle ruler.  It turns out that the distance from my eye
to my outstretched hand is about 57~cm, and this is a reasonable approximation
for most ordinary sized people. Thus from the formula, a length ($s$) of
1~cm held across the line of sight is not a bad approximation to an
angle of $1\deg$.  Down the side margin of the front page I have
marked intervals of 1~cm and labeled them in degrees.

Now you can use this ruler to measure the angle between two points or
stars. Hold it at arm's length, at right angles to your line of
sight. Line the zero position up with one star, and read off the angle
where it matches the other star.

Use your angle ruler to check the finger and fist estimates given
above. Unless you have unusual hands or arms, the angles should be
fairly close to those given.

\medskip
\bigskip\noindent
{\bf AT THE OBSERVATORY}
\bigskip


\noindent 
Take a look around the sky and identify some of the brightest stars
and constellations. Circle the constellations you recognize in
Fig.~1. 
Locate your zenith (Alt = 90\deg) and convince yourself that you
could see half the celestial sphere down to the horizon, if it wasn't
for the city buildings.

\bigskip\noindent
{\bf 3. Polaris}

\bigskip\noindent
Using the star maps locate Polaris. It is not so easy to find as it is
fairly faint and the city lights of mid town are bright. It lies
30\deg\ from the W of Cassiopeia and the pointers of the big dipper;
the pointers are 5\deg\ apart.  Once you can see Polaris you know
(roughly) where the NCP is, and the direction of due North on Earth
(the horizon point directly below Polaris). Keep walking in this
direction and you will eventually reach the North pole. You will
notice that contrary to common belief, the avenues of Manhattan (and
Broadway in our part of town) are not oriented N--S.
 
\noindent Measure as accurately as possible the altitude of Polaris. You cannot
see the exact horizon so start from one arm held accurately
horizontal. Your angle ruler will not be long enough so you will need to
place more than one, end to end to build up the height.

\bigskip
{\hfill Alt of Polaris = \makebox[4cm]{\hrulefill} \hfill}

\clearpage

\noindent Use your binoculars, and look at Polaris through them. Can you verify
you are actually looking at Polaris?  It takes some practice to know
how bright a star should look through the binoculars. Draw the field
of stars centered on Polaris as you see it in the binoculars:

\centerline{\psfig{figure={o4s_f2.eps},width=4.5cm}}


\bigskip\noindent
{\bf 4. Altitude and Azimuth}

\bigskip\noindent
Locate and measure the Alt and Az of the objects in the following
table. For Alt use the angle ruler, for Az use the arm
method. Remember Az is measured from North (below Polaris) through East.

\begin{center}
\begin{tabular}{lcc} \hline \\ [-6pt]
Star   &\hspace{1.5cm} Altitude \hspace{1.5cm} & \ \ Azimuth \ \ \ \
 \\ [6pt]  \hline
Sirius    &   &        \\ \hline
Betelgeuse    &   &       \\ \hline
Aldebaran    &   &       \\  \hline
\end{tabular}
\end{center}

\bigskip\noindent
{\bf 5. Angular separation}

\bigskip\noindent
Locate and measure with your ruler the angular separation of the following.
Note too, which is the brighter of each pair.

\begin{center}
\begin{tabular}{lccc} \hline \\ [-6pt]
Pair   & \hspace{1cm} Separation \hspace{1cm}& Brighter Component \\ [6pt]
\hline
The ends of Orion's belt      & &        \\ \hline
Betelgeuse -- Rigel     & &       \\ \hline
Castor -- Pollux       & &       \\ \hline
\end{tabular}
\end{center}

\noindent Using these estimates, estimate the field-of-view of the binoculars
you are using.

\clearpage 
\medskip\noindent
{\bf 6. Colors}

\bigskip\noindent 
The colors of stars are subtly different. 
They are typically described as: blue-white, white,
yellow-white, yellow, orange, red.  Describe the colors of the following.
  
\begin{center}
\begin{tabular}{lcc} \hline \\  [-6pt]
Star   &\hspace{1.5cm}  Color \hspace{1.5cm} &  \\  [6pt]
\hline
Rigel    &   &        \\ \hline
Betelgeuse     &   &       \\ \hline
Pollux     &   &       \\ \hline
Aldebaran    &   &       \\  \hline
Capella    &   &     \\  \hline 
\end{tabular}
\end{center}


\medskip\noindent
{\bf 7. Star time}

\bigskip\noindent
Study the sky near Polaris using the M5 map~1.  
The circles are lines of Declination and 
and the radial lines are lines of Right Ascension, marked around the 
outside in hours. Compare the map with the sky and
orient it so that they match up. Read the RA grid marker that is
highest in the sky. 

\medskip
{\hfill RA = \makebox[4cm]{\hrulefill} \hfill }

\medskip
This line of RA runs from Polaris through the zenith to the South
horizon. Its value is called the sidereal time. As the sky turns its
value increases like a clock. 

\bigskip\noindent
{\bf 8. Magnitudes}

\bigskip\noindent Compare Orion
or Gemini with the detailed star maps
in the M5 atlas.

Find the faintest star you can just see in the sky in this region. Read its
magnitude from the atlas:  \makebox[4cm]{\hrulefill}

\smallskip\noindent
Find a faint star in the atlas in this region that you cannot see with
your naked eye.
 
What magnitude is it: \makebox[4cm]{\hrulefill}
 
Locate it with binoculars. Can you see it now: \makebox[4cm]{\hrulefill}


\newpage

\begin{figure*}[h]
        \centerline{\psfig{figure={brightstar.ps},width=6.5in,angle=90.}}
        \caption{}
         \end{figure*}









