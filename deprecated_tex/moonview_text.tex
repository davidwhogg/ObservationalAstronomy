
\bigskip
\noindent
{\bf 1. Indoors  }

\medskip\noindent 
a) \emph{The Position of the Sun.} \ \ 
One of the things we shall measure tonight is the angle between the Sun
and the Moon, which is directly related to the lunar phases. Before going
out, estimate the R.A. of the Sun today using the approximate
method we used to estimate sidereal time: i.e., count the months and
days from the last equinox/solstice position. Enter the result in the
space for the R.A. of the Sun in section 3.

\medskip\noindent 
b) \emph{Scale of the Moon Maps.} \ \
To get some idea of the size of things we shall see on the Moon, we want to
calibrate the Moon maps in miles or km. In the M5 map (last page), the
Moon diameter is given as 2160 miles. Measure the diameter in cm (ruler
on back page of field guide), and figure out how many miles in 1~cm
(\makebox[1.5cm]{\hrulefill} miles) and 1~mm (\makebox[1.5
cm]{\hrulefill} miles) on the map. Mark these scales above the map for
later reference. Check also that you can relate the detailed Moon maps
in the FG (starting at page 365) with the M5 map.



\bigskip
\noindent
{\bf OUTSIDE}

\bigskip
\noindent
{\bf 2. The Moon's Phase and Age}

\medskip\noindent 
Examine the Moon with the naked eye and verify that the side facing
the Sun is the side that is lit up! 
See if you can detect the dark side of the Moon
which is dimly lit by sunlight reflected off the Earth, back to the
Moon, and back to the Earth again (Earth-shine).

The figure below is a circle representing the lunar disk. The
horizontal line is the equator, and the tick marks along it show the
position of the terminator (at the equator) for each day since the new
Moon phase. Draw in the terminator as you see it, shade in the unlit
side, and estimate the age of the Moon this evening.
\medskip

{ \hfill The age of the Moon is: \makebox[2cm]{\hrulefill} }

\begin{figure*}[h]
        \centerline{\psfig{figure={moon_phase.ps},width=10.0cm}}
        \caption{}
         \end{figure*}

\newpage

\bigskip
\noindent
{\bf 3. The Angle Between the Sun and Moon}

\medskip\noindent 
Using the M5 atlas, identify the stars nearest the Moon, and thereby locate as
accurately as possible the position of the Moon on the map.
Enter the R.A. of the Moon below. Determine the angle in R.A. between
the Sun and
Moon in hr:min, and then convert this into \deg\ (remember 1 hr = 15
\deg\, and 4 min = 1\deg). Using the fact that the Moon moves 12.2\deg\
east of the Sun each day, find the age of the Moon since it was last
new. Fill out the little table as you go. Your final answer should be
within one day of the estimate made from the phase.

\medskip
The the R.A. of the  Moon today is: \hfill \makebox[2cm]{\hrulefill} 

\medskip
The the R.A. of the  Sun today is: \hfill \makebox[2cm]{\hrulefill} 

\medskip
The angle between the Sun and Moon in hr:min is: \hfill \makebox[2cm]{\hrulefill} 

\medskip
The angle between the Sun and Moon in degrees is: \hfill
\makebox[2cm]{\hrulefill} 

\medskip
The age of the Moon from the Sun-Moon angle is: \hfill
\makebox[2cm]{\hrulefill} 
 
\bigskip
\medskip
\noindent
{\bf 4. Observing}

\medskip\noindent
Set up the telescope and study the Moon through the main eyepiece. The
telescope need only be very roughly aligned with the Polaris; no need
to set the RA and Dec circles. Check carefully the orientation of the
image so you can find your way about; it will be upside down in the
finder, but is probably upright in the main eyepiece, but possibly
flipped left-right.

Note how the shadows on the Moon become more prominent the nearer you
get to the terminator, and the craters look more elongated the closer
you get to the limb (edge) of the Moon, because of the effect of
projection. Once you have found your way around, work through the
following list of things to find, using the M5 map and the detailed FG
maps where needed.  If you cannot find an item, write
``none visible''. After the first question try using
a high power eyepiece so you can see more details.

\bigskip 

a) Identify the most prominent mares visible tonight: \hfill
\makebox[4cm]{\hrulefill}

\medskip
b) Identify two prominent craters close to the terminator: \hfill
\makebox[4cm]{\hrulefill}

\medskip
c) Estimate the diameter of one of these craters in miles:\hfill
\makebox[4cm]{\hrulefill}

(From the scale on your map)

\medskip
d) Estimate the size of the smallest thing you can see: \hfill
\makebox[4cm]{\hrulefill}

(Look in the neighborhood of the above crater and scale down from its size.)

\medskip
e) Find and identify by name an old crater: \hfill
\makebox[4cm]{\hrulefill}

(one that has smaller craters inside or on the rim)

\medskip
f) Find and identify a crater with a dome in the center: \hfill \makebox[4cm]{\hrulefill}

\medskip
g) Find and identify a crater whose floor has been flooded:
\hfill \makebox[4cm]{\hrulefill}


\medskip
h) Examine several craters close to the terminator: are they holes in
the ground 

or do they have elevated rims as well: how can you tell? 
\hfill \makebox[4cm]{\hrulefill}

\medskip
i) Find and identify by name a mountain range: \hfill
\makebox[4cm]{\hrulefill}


\medskip
j) Estimate the length of the mountain range in miles: \hfill
\makebox[4cm]{\hrulefill}


\medskip
k) Depending on the phase, choose one or two of these special lunar
features 

(or any others that you can find and identify).

Describe what you see in the space below: 

\smallskip
Valley or rille of Ariadaeus and Hygius (few days before 1st quarter, see M5
map) 

\smallskip
Straight wall (1-2 days after 1st quarter, M5 map)

\smallskip
Straight range (1-2 days after 1st quarter, see M5 map)

\smallskip
Undulations/ridges in the flat mares (any phase)

\smallskip
Bright craters and rays (any phase)

\medskip
l) Before quitting, take a look at the place where man first landed. 

The Apollo 11 touchdown is marked in the M5 map. 

Can you see any footprints or pod marks? They are still there!

\bigskip 
       


