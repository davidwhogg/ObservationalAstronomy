
\noindent
In this project we shall explore characteristics of the Milky Way and
some of the deep sky objects found in the Messier catalog.


\medskip \bigskip
\noindent
{\bf 1. The Local neighborhood in 3-D} 

\medskip\noindent
The stars in our neighborhood are dotted about at random, typically
separated by about 1~pc or a few light years (recall 1~pc =
3.26~lt~yr).  To see our stellar neighborhood, launch SN and choose
Favorites/Local Neighborhood: it takes you about 5 parsecs out and
gives you a little tour.  Double click Sun in Find, so you look back
to the Solar System.  Many of the very faintest, nearby stars are not
included -- but it gives the correct general impression of where we
live. You can move nearer to or farther away from the Sun using the
up/down arrows under Viewing Location.


\bigskip
\noindent
{\bf 2. The Milky Way} 

\medskip\noindent
On much larger scales of thousands of parsecs or light years our star
system is flat, so it appears brightest in a ring on the celestial
sphere that we call the Milky Way.

Return to NY. Get rid of daylight and the horizon, and click
View/Stars and Milky Way. Set the clock turning and see how the MW moves
around the sky.  Click on equatorial (celestial) coordinates and see
that the MW is fixed in RA and Dec. Estimate the coordinates of the of
the MW at about 10 or more positions along its whole length, and plot the
results on the graph below. To do this, it is easier to stop the clock..

\vspace{0.1cm}

\begin{figure*}[h]
        \centerline{\psfig{figure={i6s_f1.eps},width=10.0cm}}
        \caption{}
         \end{figure*}

\newpage
\noindent
Join the dots you have plotted with a smooth line: this
shows the plane of the MW -- the Galactic plane. The
center of the system lies towards  RA = 17 hr 40
min, Dec = --29\deg\ about 8,000~pc away. Mark it with a cross.

\bigskip
\noindent
Is the Galactic plane aligned with the celestial equator? Or with the
the ecliptic?: \\ \makebox[4cm]{\hrulefill}

\bigskip
\noindent
How many times per day (if at all) does the MW pass through the zenith in NYC:
\makebox[4cm]{\hrulefill}

\bigskip
\noindent
When (if at all) does the Galactic Center cross the meridian today:
\makebox[4cm]{\hrulefill}



\bigskip
\bigskip
\bigskip
\bigskip
\noindent
{\bf 3. The Messier Objects} 

\medskip
\noindent
In addition to single and binary stars, our star system has many other
components. Examples are to be found in the Messier Catalog (109
in number), and in the later NGC and IC catalogs which are similar but with
many more entries (1000s).

These deep sky objects are classified as several main types: regions of ionized
gas (called H II regions); planetary nebulae (PN); open star clusters
(OC); globular clusters (GC); and galaxies. Supernova remnants are
very rare: M1 is the only example in the Messier catalog.  Note that
H II regions which are defined by their nebulosity often have a newly
formed star cluster in the center, and at least one of the stars is
blue and luminous so that it can excite the nebula; in these objects
there is also often evidence of dark cloud (DC) obscuration where the
unexcited material blocks the light from the background.

\bigskip
\noindent
Look up the representative M objects in the following table (over the
page) in SN, using Find. (Be Careful, SN may try to find M81 rather
than M8 unless you correct it, and make sure View/Deep Sky/Messier
is checked).

\bigskip
\noindent
Sketch what each looks like in the box provided. Measure a
representative angular size in arc minutes, and record the result in
the space provided.

\bigskip
\noindent
From the distances and angular sizes of these objects: \\
Which is the physically largest: \makebox[3cm]{\hrulefill}\\
Which is the physically smallest.\makebox[3cm]{\hrulefill}

\newpage

\noindent
{\framebox[3.0cm]{\rule[-1.5cm]{0cm}{3cm}\,} \hspace{0.5cm}
M3: GC \hfill $\theta_{\rm diam}$: \makebox[3cm]{\hrulefill} \ \ Distance: 
30,000 lt yr}

\bigskip
\noindent
{\framebox[3.0cm]{\rule[-1.5cm]{0cm}{3cm}\,} \hspace{0.5cm}
M6: OC  \hfill $\theta_{\rm diam}$: \makebox[3cm]{\hrulefill} \ \ 
Distance:\ 2,000 lt yr}

\bigskip
\noindent
{\framebox[3.0cm]{\rule[-1.5cm]{0cm}{3cm}\,} \hspace{0.5cm}
M8: H II \hfill $\theta_{\rm diam}$: \makebox[3cm]{\hrulefill} \ \ 
Distance: 6,500 lt yr}

\bigskip
\noindent
{\framebox[3.0cm]{\rule[-1.5cm]{0cm}{3cm}\,} \hspace{0.5cm}
M27: PN \hfill $\theta_{\rm diam}$: \makebox[3cm]{\hrulefill} \ \ 
Distance: 1,200 lt yr}

\bigskip
\bigskip
\noindent
Using your experience, find the following with SN and classify them:

\medskip
M44: \makebox[3cm]{\hrulefill} 

\medskip
M57: \makebox[3cm]{\hrulefill}

\medskip
M13: \makebox[3cm]{\hrulefill}

\medskip
M16: \makebox[3cm]{\hrulefill}

\vspace{.3cm}
\bigskip


\clearpage
\noindent
{\bf 4. Where Are The Deep Sky Objects?}

\medskip
\noindent
If you click  View/Deep Sky/NGC--IC, SN draws little circles around
the numerous NGC and IC catalog objects. If you examine the location
of these in the sky you will find the distribution quite striking. 

\bigskip
\noindent
The best place to find a star cluster or a nebula of some kind is : 
\makebox[3cm]{\hrulefill}


\bigskip
\medskip
\noindent

\noindent
{\bf 5. A Close-up of the Milky Way}

\medskip
\noindent
To get a realistic picture of the Galactic plane, examine the two
photographic images of the MW from the Palomar Sky Survey. The two
images are of the same patch of sky with a field of view about $6\deg
\times 6\deg$ in size.  The correct orientation is with the small
rectangle in the top left corner. The bright star at the bottom edge
is Deneb, $\alpha$ Cygni (near 21 hr +45\deg).  The images are
negatives so the light of the stars etc. is dark. The image labeled E
(in the small rectangle) is taken with a red filter -- so the image
records the red light, and the one labeled O is taken with a blue
filter, and so records the blue light.

\noindent We only have four copies of each image, so teams may need to
take turns using these images.

\noindent There are some magnifiers in the lab room, which will make
it easier to examine some of the fine-scale features in the images. 

\medskip
\noindent
{\bf Please treat them with great care: keep them flat; no pencils or
  pens near them, and do not write on paper on top of the
  prints. Thanks!}

\medskip
\noindent
You will see that the MW is a pretty busy place. Each tiny speck is a
star, seen here down to about 20th magnitude. Locate the field in
the the Mag 5 and the FG, atlas chart 19. Study the images and answer
the following questions. Note: the large horizontal nebula structures which
cross the images are parts of a large, old supernova remnant. The star
exploded to the south of the picture.


\bigskip
\noindent
What color is the nebulosity, red or blue: \makebox[4cm]{\hrulefill}

\bigskip
\noindent
Identify and examine the star pairs labeled 1, 2, and 3, in the key
below.
Which of each pair (left or right) is the hottest (bluest) -- note it need not be the
brightest: \\
pair 1: \makebox[4cm]{\hrulefill} \  pair 2: \makebox[4cm]{\hrulefill}
\ \
pair 3:  \makebox[4cm]{\hrulefill}

\bigskip
\noindent
Classify the gas cloud marked with an H in the Key:  \makebox[4cm]{\hrulefill}

\bigskip
\noindent
What is at the center of gas cloud H, making it excited (check both images): 
\makebox[4cm]{\hrulefill}

\bigskip
\noindent
Classify the tiny feature seen in red a few cm up and to the right of
the cloud H:   \makebox[4cm]{\hrulefill}

\bigskip
\noindent
Explain the white patches that cross the feature I in the key: \makebox[4cm]{\hrulefill}


\bigskip
\noindent
Use SN with a field of 10\deg\ to identify the detailed star field of the
images.  Identify by name the star marked A in the Key:
\makebox[4cm]{\hrulefill} \\
It is 932 lt yr away. If you right click and check go there,
you can take a trip out there in 3-D.


\begin{figure*}[h]
        \centerline{\psfig{figure={poss-marked.ps},width=6.5in}}
        \caption{}
         \end{figure*}




