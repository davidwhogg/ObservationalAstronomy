
\begin{enumerate}
\item Describe in words why the Earth (and other planets) orbit the
  Sun. 
\vspace{80pt}
\item  What is the name of the circle on the sky in Equatorial
  Coordinates that the Sun traverses during the year?
\vspace{80pt}
\item  What is the declination of the Sun on March 21? June 21?
  September 23? December 21?
\vspace{80pt}
\item Why are there seasons?
\vspace{80pt}
\item At what latitude(s) does one have perpetual daylight for at least
  one day each year? What are such latitudes called?
\vspace{80pt}
\item In what latitudes does the Sun ever pass directly overhead? What
  are such latitudes called?
\vspace{80pt}
\item What latitudes would the Arctic and the Tropics be in if the
  ecliptic angle were 10$^\circ$?
\vspace{80pt}
\item In Tallahassee, what declinations do stars have that never set?
  that never rise?
\vspace{80pt}
\item How long is a siderial day?
\vspace{80pt}
\item Say you observe a star at Dec = 0 passing through a telescope's
  field of view in 4 minutes. What is the angular size of that field
  of view?
\vspace{80pt}
\item If a star is transiting at midnight tonight, when will it
  transit tomorrow night? A month from now?
\vspace{80pt}
\item If Betelgeuse is transiting, what is the current Local Siderial
  Time?
\vspace{80pt}
\item What is the mathematical relationship between HA, RA and LST?
\vspace{80pt}
\item On what day are the LST and standard local time nearly
  identical?
\vspace{80pt}
\item How is Daylight Savings Time defined?
\vspace{80pt}
\item What RA transits at midnight on March 21? June 21? September 23?
  December 21?
\vspace{80pt}
\end{enumerate}
