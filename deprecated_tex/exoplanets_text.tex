
\noindent
{Objectives:} To explore some indirect astronomical observing techniques that allow to discover and characterize objects not easily, sometime not at all visible.  To explore astronomical time series and connect the shape of a lightcurve to physical parameters (orbital period, size). To familiarize the student with  photometry, and the synergic use of multiple datasets to obtain more accurate measurements.


\medskip
\bigskip
\noindent
{\bf Discovering Exoplanets}

\medskip
\noindent
A recent astronomical discovery is that the Universe is populated by
an incredibly high number of planets in a variety of sizes and
conditions. Statistically speaking, every star has a planet!
Exoplanets that ``transit'', i.e.: we see them pass in front of their
star, will cause light from the star to be dimmed by a few percent,
allowing us to discover the plantes, and providing critical insight on
the properties of the planet. Transits tell us the size of the planet
(from the dimming) and the semi-major axis of the orbit (from the
duration of the transit).

In this Lab, we will use CCD images from the Faulkes Telescope North
to try and detect the transit of a known extrasolar planet and
practice photometric techniques. Then, an online interactive exoplanet
hunter technique will be used to get a better transit for this
exoplanet.

\begin{figure*}[h]
        \centerline{\psfig{figure={29WASP.eps},width=10.0cm}}
        \caption{When the planet WASP-10b crosses the disk of its star, WASP-10, the brightness of the star decreases, allowing scientists to measure the precise size of the planet. \emph{Credit:} Prof. John Johnson}
         \end{figure*}

\noindent
{\bf 1. Understanding the Transiting Technique}:


If you know the mass of the star, the transit technique can give you the ratio of the radius of the star and the planet (it also provides us with the transit duration time ($dt$) which in turn gives us an estimate of the period of the orbit, which you can do for extra credit). If we know the stellar mass $M_∗$ then we can use this and the semi-major orbital radius $a$ to estimate the ratio of the planet and star radii.
\begin{itemize}
\item
the dip in the light curve due to the transiting planet is directly related to the ratio of the area of the projected disks of the star and planet, since it is proportional to the fracion of star light that the planet blocks. the area of a circle is proportional to the square of the radius, hence the depth of the dip in the transit lightcurve will be  $(R_p/R_∗)^2$.
\item
the orbital radius a of the planet is estimated using Kepler’s 3rd Law given the mass and radius of the star and the transit duration. 
Recall Kepler’s 3rd Law:
$$\frac{P^2}{a^3}=\mathrm{const}$$
or in words: the period of an orbiting body and its orbit (semi major axis) are bound. If you know one, you know the other, at least for circular orbits where the proportionality constant is: $\mathrm{const} = 4\pi^2/(GM_*)$.
We can wait for another transit, and get the period, then we have the semi major axis. . 
Note however that there are ways to do this even with a single transit, if you know the radius of the star $R_*$  and the inclination of the planet's orbit $i$, by using Kepler's third law and the following equation:
$$ dt~=~\frac{P}{\pi}*\sqrt{\left(\frac{R_*}{a}\right)^2 ~-~\mathrm{cos}^2 i} $$
\noindent
or if the planet is seen edge on simply: $dt~=~\frac{P~R_*}{\pi~a}$, where $t_\mathrm{tr}$ is the transit duration. You should try combining the equation above with Kepler's third law to derive the semi major axis of the planet when you have your Agent Exoplanet lightcurve!
\end{itemize}

You will use data from the Faulkes Telescope North for the planet
Qatar-1b. Qatar-1b is the first extrasolar planet to be detected by a
Qatar-led survey program, and orbits the star 3UC311-087990 at a
distance of 500 ly from Earth. Its discovery, announced on 20 December
2010, was through the transit technique and subsequently confirmed
using the radial velocity method.
\begin{table}
\begin{center}
\begin{tabular}{lc}
Parameter & Value\\
\hline
RA (J2000) & 20h13m31s.61\\
Dec (J2000) & +65$^o$09′43′′.4\\
$M_∗ (M_\odot)$ & 0.85 ± 0.03 \\
$R_∗ (R_\odot)$ & 0.823 ± 0.025 
\end{tabular}\caption{Physical Parameters for Qatar-1b system}
\end{center}
\end{table}

\bigskip
\noindent
{\bf 2. Using Agent Exoplanet}

You will use an online program called Agent Exoplanet to produce light
curves of Qatar 1b. You are doing \emph{photometry} on the data. In a
nutshell, you will select your target star, the star hosting Qatar-1b,
and measure its brightness image after image. The planet passing in
front of it will cause the brightness to decrease. You will need to
specify the \emph{centroid}: the position of the star in the
image. You will need to specify an \emph{aperture}: basically the
extent of the star in your image. The brightness of the pixels within
the aperture will be summed to get the total brightness of the
star. However there is a bright sky behind the star, so if you do not
take care of that your total brightness will be overestimated (and
more importantly changes that you may see rom one image to the next
may not be only due to changes in the star
brightness. To \emph{remove} the sky from the star brightness an area
around your aperture is selected, and the average sky luminosity in
this area will be assumed to be the average sky luminosity underneath
the star. Good news: Agent Exoplanet does this for you!!! But you
should still know what is going on. Lastly: it is not sufficient to
measure the brightness of your target star: changes in the registered
star brightness in a CCD image may be due to many thing, not just to
actual changes in the star brightness. The easiest example is if a
cloud passes in fromt of your field of view: that will cause a dimming
of the star. But all the stars in the field will also get dimmer
accordingly! To correct for these changes in the luminosity of the
stars in your image, you will select other stars,
called \emph{reference} stars, and measure their brightness as
well. Changes in brightness that are common to all stars can then be
removed (Agent Exoplanet does this for you) and what you are left with
is changes that are specific to your target star.

Go to: {\tt http://lcogt.net/agentexoplanet/}, and make sure you have
read the instructions at {\tt
http://lcogt.net/agentexoplanet/briefing/}.

Find the Qatar-1b data set in this program. Now you need to to click
on the button ‘Analyze images for this exoplanet’ (see
Figure \ref{AE}).  You will now be asked to log into your account. If
you do not already have one, simply register for one on this site and
you can move straight onto using the program.  The Faulkes Telescope
Dataset for Qatar-1b is vast! You will not need to analyze all the
images in it.  The online program uses combined photometric
measurements from many users to produce an ‘average’ light curve.  You
can measure as many of these images as you like, because Agent
Exoplanet will combine your measurements with those of other
volunteers, and produce a combined result.  However, make sure you do
enough to obtain a decent light curve.  Once you have done enough
photometry, click on the lightcurve tabs to the right of the page


\begin{figure*}[h]
        \centerline{\psfig{figure={AEqatar1b.eps},width=20.0cm}}
        \caption{A screenshot of the Agent Exoplanet website with Qatar 1b dataset.}\label{AE}
\end{figure*}

You will see all of your data and the resulting light curve you have
produced. This allows you to investigate your analysis and classify
your light curves, including checking whether you want to remove what
you believe to be ``bad'' sets. Carefully go through and check your
measurements. If you see anything that is not a transit dip (e.g. a
sharp vertical spike) then use the buttons to mark it. You can change 
your mind at any time, just click the relevant button. To check on the
classifications you have made, click the ``Toggle Table'' link. You
must remember that other people will see this lightcurve so make sure
you do a good job with your analysis.  Describe what you do with Agent
Exoplanet below. You must keep a note of everything you have done in
using Agent Exoplanet including the number of images you analysed and
the number of stars you used in each image. Include the details for
the final lightcurve produced by Agent Exoplanet and print this final
lightcurve out.

\clearpage

\noindent
{\bf 3. Getting the parameters for the planet:} When you are happy
with your lightcurve, measure the relevant parameters: the duration
and depth of the transit. Use the equations and star parameters
provided earlier derive the radius of the planet $R_p$ and the period
$P$ of the orbit and report them below (showing your calculation):

\bigskip

\medskip\noindent 
{\bf$R_p$:}
\vspace{30pt}

\medskip\noindent 
{\bf$P$:}
\vspace{60pt}

Ultimately Agent Exoplanet will produce a \emph{Master light curve}
from everyone’s data with a fitted a transit curve. Once the transit
‘curve’ has been drawn, Agent Exoplanet produces a scaled cartoon of
this extra-solar planetary system based on your and all other
volunteer’s combined measurements. It will also give you the
parameters which fit the light curve including the \% dip in
brightness, the diameter of the planet, the duration of the transit,
and the ratio of planet-to-star radius. Compared them from what you
found from your own lightcurve.

\medskip\noindent
{\bf Questions.}

\medskip\noindent How close are the parameters you derive from your 
own lightcurve to those derived by Agent Exoplanet from
the \emph{Master} lightcurve ?

\medskip\noindent What may have caused the discrepancy?

