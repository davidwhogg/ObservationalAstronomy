\documentclass[12pt]{article}
\input{physics1}
\begin{document}

\section*{Observational Astronomy---Midterm Exam}

\textsl{This midterm is composed of short-answer questions about the
  reading component of this course. They should be straightforward to
  answer if you review the reading assignments. Do your work on your
  own. If you discuss these questions with another class member, say
  who that is on your answer sheet. Your answers are due on
  \textbf{2017 Monday March 27 at 15:30 EDT}.}

\begin{enumerate}
\item Translate RA $=$ 18h37m32.1s and Dec $=$ 51$^\circ$ 13$'$ 27$"$
  to decimal units.
\item Measure the width of your fist, and the length of your arm. You
  see a person approaching. If you hold your fist extended, it just
  covers your view of them.  Approximately how far away are they?
\item What is the difference between a great circle and a small
  circle? (Draw examples).
\item What declination is the zenith in Los Angeles?
\item What right ascension is at zenith in Los Angeles on March 21,
  12:00 Universal Time (UT)?
\item What is the maximum altitude that the star Sirius attains in New
  York City?
\item If you point your arm horizontally, pointing to the south-east,
to what altitude and azimuth are you pointing?
\item  What is the declination of the Sun on March 21? June 21?
  September 23? December 21?
\item In what latitudes does the Sun ever pass directly overhead? What
  are such latitudes called?
\item In Tallahassee, at what declinations are the stars that never set?
  that never rise?
\item Say you observe a star at Dec = 0 passing through a telescope's
  field of view in 4 minutes. What is the angular size of that field
  of view?
\item If a star is transiting at midnight tonight, when will it
  transit tomorrow night? A month from now?
\item If Betelgeuse is transiting, what is the current Local Siderial
  Time?
\item On what day are the LST and standard local time nearly
  identical?
\item How is Daylight Savings Time defined?
\item You go outside one night at midnight standard time in New York
  State, and see Merak and Dubhe pointing down vertically at
  Polaris. What month is it? 
\item What is field of view of a 2m focal length telescope with a 40mm
  eyepiece? 10mm?
\item What are the siderial and synodic periods of the Moon's orbit, in days?
  Which one governs the period of the Moon's phases as seen from
  the Earth?
\item If you see the moon high in the sky near sunrise, what phase is
  it going to be at? If it is high in the sky near sunset?
\item What is the average distance to the Moon? Its perigee and apogee?
\item What are craters on the Moon formed by? How old are heavily
  cratered areas relative to uncratered areas?
\item Which planets are visible to the naked eye?
\item What are Kepler's laws?
\item When and where are Venus and Mercury likely to be found on the
  sky?
\item Why does Venus have ``phases'' like the Moon but Jupiter does
  not?
\end{enumerate}

\end{document}
