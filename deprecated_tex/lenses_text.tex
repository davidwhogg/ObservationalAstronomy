
\noindent Today we will make a simple telescope using some lenses.  We
will begin by examining the properties of a lens, and learn how to
measure focal lengths. We will then use the lenses to make a
telescope, and examine its magnification.  

\noindent {\bf Equipment:} optical bench, magnetic stands, light
source, 1 mm aperture mask, plane mirror, concave mirror, glass,
imaging screen, and several lenses.

\noindent
{\bf 1. Properties of a single lens} 

\noindent Set up the optical bench as follows, with the 1 mm aperture
attached to the light source, a 70 mm lens and the screen, as shown
below. Position the source so the front face corresponds to the zero
position on the bench --- this makes measuring distances easier.

\begin{figure*}[h]
\centerline{\psfig{figure={uv.ps},width=13.0cm}}
\caption{}
\end{figure*}

\noindent In what follows, we shall refer to the distance between the
source and the lens as $v$ (the object distance), and the distance
between the lens and the screen (when the image is in focus on the
screen) as $u$ (the image distance). 

\noindent Move the lens so that $v= 60$ cm.  Then move the screen
until the image of the light source is in focus.  Determine the
corresponding value for $u$ and enter the result in the table below.
Repeat the procedure for the other values of $u$ given in the table.
Note as you proceed the relative brightness of the focused light at
each configuration --- mark whether the image is brightest for $v=20$
or $v=60$ cm.  Plot the results on the attached sheet of graph paper,
and join the points with a neat, smooth curve.
 
\begin{center}
\begin{tabular}{lc} \hline \\ [-6pt]
$v$ (cm)  & \hspace{1cm} $u$ (cm) \hspace{1cm} \\ [6pt]
\hline
60 &        \\ \hline
50 &        \\ \hline
40 &        \\ \hline
30 &        \\ \hline
25 &        \\ \hline
20 &        \\ \hline
\end{tabular}
\end{center}

\noindent On the schematic of the optics above, draw the path the
light takes from the center point of the aperture, assuming that the
light is in focus. Start at the arrows shown and continue the lines
until the light hits the screen. Remember the light goes straight when
going through the air and bends when it goes through the lens! Note of
course that not {\it all} of the light ends up going through the lens.

\noindent Mathematically, the relation between $u$, $v$ and the focal
length $f$ is given by the equation:

\begin{equation}
\frac{1}{u} + \frac{1}{v} = \frac{1}{f}
\end{equation}

\noindent A special case occurs when $u=v$.  Then:

\begin{equation}
\frac{1}{u}+\frac{1}{v} = \frac{1}{u}+\frac{1}{u} = \frac{2}{u}
=\frac{1}{f}
\end{equation}

\noindent In other words: $f=u/2$ when $u=v$.  From your plotted curve
find the point when $u=v$ and deduce $f$:

\vspace{30pt}

\noindent Examine your graph and comment:
\begin{enumerate}
\item What happens to $u$ as $v$ gets very large? \vspace{20pt}
\item What happens to $u$ as $v$ gets very close to $f$? \vspace{20pt}
\end{enumerate}
\noindent These special cases can be investigated with the above
equation.  When $v$ is very large (at infinity) then $1/v=0$.  What
then is the value of $u$?

\vspace{30pt}

\noindent When $v=f$ (i.e. the source is at the focus of the lens, what then is
the value of $u$?

\vspace{30pt}

\noindent {\bf 2. Parallel light and focal lengths}

\noindent We can use the results from part (1) in two ways: first, to
simulate on the optical bench a very distant light source, and then to
use this to directly measure focal lengths.

\noindent First, set up the optical bench as follows, with the light
source at one end as before, but now with the 48 mm focal length lens
and the plane mirror:

\begin{figure*}[h]
\centerline{\psfig{figure={parallel.ps},width=13.0cm}}
\caption{}
\end{figure*}

\noindent Adjust the position of the lens (keeping the mirror close to
it) so that the light from the source is reflected back off the mirror
and through the lens again, and is in sharp focus on the front face of
the source.  This may require some fiddling so you can actually see
the image on the source screen (if things are aligned too well, the
image will form on the aperture itself, where you won't be able to see
it).

\noindent Some thought should convince you that (a) the source-lens
distance is equal to the focal length of the lens, and (b) on removing
the mirror, the beam emerging from the lens will be equivalent to that
coming from a very distant object (that is, the rays will be
essentially parallel). 
Draw the light paths in the above diagram.

\clearpage

\noindent With a parallel (``collimated'') beam, we can directly
measure focal lengths of lenses.  Simply mount a lens on the bench in
the parallel beam close to the 48 mm lens, as follows:

\begin{figure*}[h]
\centerline{\psfig{figure={measure.ps},width=13.0cm}}
\caption{}
\end{figure*}


\noindent Adjust the position of the screen from the lens until a
sharp image is obtained.  From part (1) it should be clear hat this
distance equals the focal length of the lens. Compare your measurement
with the label on the lens itself:

\vspace{30pt}

\noindent Again, draw the light paths for an in-focus configuration in
the above diagram.

\noindent While we have a parallel beam set up, let us look at how a
mirror focuses a beam. Install the concave mirror and a piece of
straight, clear glass, as shown below:

\begin{figure*}[h]
\centerline{\psfig{figure={mirror.ps},width=13.0cm}}
\caption{}
\end{figure*}

\noindent Configure the mirror so that you can see the image form on
the {\it glass}. The distance between the concave mirror and the image
it forms from a parallel beam is its focal length.  Compare your
measured focal length to that stated on the mirror:

\vspace{30pt} 

\noindent Draw the paths of the rays in the figure above.

\noindent {\bf 3. A simple telescope}

\noindent Clear the optical bench of everything including the light
source, and mount two lenses: an ``objective'' lens O with focal
length $f_O$ and an ``eyepiece'' lens with focal length $f_e$. To
begin, put the eyepiece at one end, and position the objective length
a length $f_O$ from it. Try to look at a source on the far end of the
room; you will see that the telescope is not in focus.  Move the
objective back until the distant object is clearly in focus.

\begin{figure*}[h]
\centerline{\psfig{figure={telescope.ps},width=13.0cm}}
\caption{}
\end{figure*}

\noindent Draw the light paths in the diagram above.

\noindent If properly configured, this set up should produce a
magnified image of distant objects --- if you find that the images are
{\it demagnified} how do you have to change the configuration? 

\vspace{30pt}

\noindent Measure the distance from lens O to lens E.  How does that
compare with the number you expect? 

\vspace{30pt}

\noindent We now have a telescope that magnifies distant objects.
According to the theory, the angular magnification of the telescope is
given by $M=f_O/f_e$.  Using the labeled values of $f_O$ and $f_e$ on
your lenses, calculate the theoretical magnification of the telescope.

\vspace{30pt}

\noindent Now pick a different lens for the objective, and report if
the magnification changes in the expected manner:

\vspace{50pt}


\noindent {\bf 4. Extra credit}

\noindent Notice that the image in the telescope we have built is
reversed.  Suggest and implement a change in the design to make the
images upright.

\clearpage

