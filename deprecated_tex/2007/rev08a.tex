 
\documentclass[11pt]{article} 
\topmargin -.6in 
\textheight 8.7in 
\oddsidemargin -.0in 
\textwidth 6.5in 
\title{The Analysis of Starlight: Lab Projects} 
\date{Fall 1997} 
%\renewcommand{\baselinestretch}{1.2} 
\begin{document} 
\setcounter{page}{1} 
\setcounter{equation}{0} 
\pagestyle{empty} 
\parindent 0pt 
\parskip 8pt 
%\pagestyle{myheadings} 
\markright{{\bf LAB B: Measuring the Wavelength of Light} \
\hrulefill \ } 
\def\arcsec{\ifmmode {^{\scriptscriptstyle\prime\prime}}
          \else $^{\scriptscriptstyle\prime\prime}$\fi}
\def\arcmin{\ifmmode {^{\scriptscriptstyle\prime}}
          \else $^{\scriptscriptstyle\prime}$\fi}
\def\deg{\ifmmode^\circ\else$^\circ$\fi}


   
 
\noindent 
%\vspace{0.15in} 
{\bf Observational Astronomy      \hfill  V85.0013}\\ 
 
\bigskip 
 
\bigskip 
 
\noindent 
{\hfill \Large {\bf Review Sheet 8a} \hfill} 
 
 
\bigskip 

\begin{enumerate} 

\item
The brightest part is a flat system, we are in it (off to one side) so
it appears as a band around the sky.


\item 
2 times


\item 
planetary nebulae, H II regions

\item 
planetary nebula: Ring nebula (M57),  star cluster: Pleiades (M45) , \\
H II region: Orion nebula (M 42),  external galaxy: Andromeda galaxy (M31)\\
Which can be seen with the naked eye: M45, M31 

\item
Messier


\item 
The light is spread out, so more difficult to see against the general
sky background 


\item 
It is a dark cloud. It absorbs the light from the stars behind it


\item 
H II region, it is gas excited by the ultraviolet light of the stars

\item
bluer, red giants or red supergiants

\item 
The turn off on the main sequence

\item
They look round like planets; they are more symmetric than H II regions
because they are ejected from a central star 

\item 
It was a red giant, it will become a white dwarf


\item 
10 mags is a factor of 10,000 

\item 
0.3\arcsec\ per year


\item 
M 31, 50 Mpc, or 150,000 lt yr.


\item 
2, external galaxies, in southern sky dec about --70\deg.



\item 
spirals, ellipticals, irregulars; MW is a spiral


\item 
--20.6 (use formula); difference from Sun -25.4 say 25 = 10 billion


\item 
It is disk-like, but appears elliptical because it is tilted to the
line of sight; can see edges of spiral arms marked in dark clouds


\item 
Very long exposure image taken by HST; galaxies all the way out.

\end{enumerate} 

\end{document}










