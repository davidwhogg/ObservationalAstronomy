
\documentclass[12pt]{article}
\usepackage{psfig}
\topmargin -.6in
\textheight 8.7in
\oddsidemargin -.0in
\textwidth 6.5in
\title{The Analysis of Starlight: Lab Projects}
%\renewcommand{\baselinestretch}{1.2}

\begin{document}
\setcounter{page}{1}
\setcounter{equation}{0}
\pagestyle{plain}
\thispagestyle{empty}  % suppress number on first page
%\pagestyle{myheadings}
\newcommand{\kms}{\hbox{km\,s$^{\rm -1}$}}
%\def\kms{\ifmmode {\,{\rm km\,s^{-1}}}                          % km s-1
%determine the RA of the Sun        \else {\hbox{$\,$ {\rm km$\,$s$^{\rm -1}$}}}\fi}
%\def\solar {\ifmmode_{\mathord\odot} \else $_{\mathord\odot}$\fi} % _solar
%\def\mo {\ifmmode {\,{\it M}\solar} \else $\,M$\solar\fi}       % M solar
\def\lo {\ifmmode {\,{\it L}\solar} \else $\,L$\solar\fi}       % L solar
\def\my {\ifmmode {\,{\it M}\solar\,{\rm yr^{-1}}}              % Msol/year
        \else {$\,M$\solar$\,$yr$^{\rm -1}$}\fi}
\def\BD {BD$\,$+30{\degr}3639}
\def\HUNO{\rm H$\,$I}                   % molecular hydrogen
\def\HDOS{\rm H$_2$}                    % molecular hydrogen
\def\arcsec{\ifmmode {^{\scriptscriptstyle\prime\prime}}
          \else $^{\scriptscriptstyle\prime\prime}$\fi}
\def\arcmin{\ifmmode {^{\scriptscriptstyle\prime}}
          \else $^{\scriptscriptstyle\prime}$\fi}
\def\deg{\ifmmode^\circ\else$^\circ$\fi}

\sloppy



%\markright{{\bf LAB E: Hubble's Law} \ \hrulefill \ }


\noindent
%\vspace{0.15in}
{\bf Observational Astronomy    \hfill}\\
{\bf Lab: O-6S} 


\bigskip

\medskip

\noindent
{\hfill \Large {\bf Deep Sky Objects} \hfill}


\bigskip

\noindent
Objective: You will be given the names of some faint, deep sky objects
which are (hopefully) visible in the sky this evening. Finding them
requires considerable observing skill.  You will first make finding
charts to help locate the objects, and then 
record what you find at the telescope.

\bigskip\noindent
{\bf INSIDE PREPARATIONS} 

\medskip\noindent
The field of view of the finding charts is $20\deg \times 20 \deg$
which is usually big enough to include some reasonably
bright stars.  The inner circle on the chart is 5\deg\ in diameter,
and roughly corresponds to the field of view of the finder scope. The
cross at the center will correspond to the location of the object. To
make the finding chart:

\medskip\noindent
Locate the object in the Mag 5 Atlas and the Field Guide (pp 228 +).

\medskip\noindent
Mark on the chart the brightest stars in the Mag 5 Atlas that are within the
20\deg\ field centered on the
object.  Use neat pencil dots about 2 mm in diameter for the stars.
In case you later get lost, circle one of the stars as your reference,
estimate its RA and Dec as accurately as possible, and write the
coordinates next to the star.

\medskip\noindent Now locate fainter stars in the FG and draw them in
on the chart in the central circle, and (if necessary) out as far as
your reference star. Use appropriate sized dots for plotting. You need
only include enough of the brighter stars so that you will be able to
recognize that pattern and hop, using the finder, from the reference
star to match the view in the finder with the stars in the circle on
your chart.  Remember, the object may not be visible in the finder.

\bigskip
\bigskip\noindent
{\bf OUTDOORS} 

\medskip\noindent
Approximately align the telescope.

\medskip\noindent
Locate your reference star in the finder. If you have difficulty
identifying it.  Choose a very bright star, set up the setting circles
using the very bright star's coordinates from the atlas (page 33),
and offset to your reference star.

\medskip\noindent
Once you can see your reference star in the finder, star hop from star to star until the deep sky
object according to your map is located on the cross hair in the
finder. If you have been careful, the object should be visible in the
main eyepiece.

\medskip\noindent
Record what you see on the observing form.

\bigskip\noindent
{\bf NOTE 1:} The finder inverts your view so you have to use your
chart upside down.


\medskip\noindent
{\bf NOTE 2:} Many deep-sky objects are very faint, so you need to be well
dark adapted, i.e., in the dark with only a red flashlight for 20
minutes or more.

\medskip\noindent
{\bf NOTE 3:} Except for very small objects, a long focal length
eyepiece (e.g., 25 mm) is easier
to use because the field of view is larger.

\newpage
\pagestyle{empty}
\noindent
{\bf DEEP SKY OBJECT FINDING CHART    \hfill} {\bf First
  Name:\makebox[4cm]{\hrulefill}}

{\                 \hfill {\bf Last Name:\makebox[4cm]{\hrulefill}}

\vspace{1.5cm}


\noindent
{\bf OBJECT NAME:}

\vspace{1cm}
\noindent
{\bf MAG/COORDINATES:} \ \ \ Mag = \hspace{2cm} RA = \hspace{2cm} DEC =     

\vspace{1cm}
\noindent
{\bf CONSTELLATION:}


\vspace{1cm}
\noindent
{\bf OBJECT TYPE:}

\vspace{1cm}
\noindent
{\bf DISTANCE:}


\vspace{3cm}

\noindent
{\bf OBSERVATIONS:}

\bigskip
\noindent
Record what you see below in the form of:


\medskip\noindent
{\bf A sketch;} indicating the approximate shape (e.g., round,
elliptical, ring shape, amorphous), and components (e.g., main stars, gas
distribution and so on).


\medskip\noindent
{\bf Notes:} e.g., if a star cluster, an estimate of the number of
stars resolved (e.g, 10,
100, 1000, more than 100 etc.), colors (e.g., colors of main stars, or
colors of gas) and any other features.




\end{document}










