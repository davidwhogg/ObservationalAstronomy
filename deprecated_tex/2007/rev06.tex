 
\documentstyle[11pt]{article} 
\topmargin -.6in 
\textheight 8.7in 
\oddsidemargin -.0in 
\textwidth 6.5in 
\title{The Analysis of Starlight: Lab Projects} 
\date{Fall 1997} 
%\renewcommand{\baselinestretch}{1.2} 
\begin{document} 
\setcounter{page}{1} 
\setcounter{equation}{0} 
\pagestyle{empty} 
\parindent 0pt 
\parskip 8pt 
%\pagestyle{myheadings} 
\markright{{\bf LAB B: Measuring the Wavelength of Light} \
\hrulefill \ } 
\def\arcsec{\ifmmode {^{\scriptscriptstyle\prime\prime}}
          \else $^{\scriptscriptstyle\prime\prime}$\fi}
\def\arcmin{\ifmmode {^{\scriptscriptstyle\prime}}
          \else $^{\scriptscriptstyle\prime}$\fi}
\def\deg{\ifmmode^\circ\else$^\circ$\fi}


   
 
\noindent 
%\vspace{0.15in} 
{\bf Observational Astronomy      \hfill  V85.0013}\\ 
 
\bigskip 
 
\bigskip 
 
\noindent 
{\hfill \Large {\bf Review Sheet 6} \hfill} 
 
\bigskip 
 
\begin{enumerate} 

\item
A pair of 7$\times$50 binoculars has an apparent field of view of
50\deg. What is the true field on the sky.

\item
A Meade 8-inch telescope is fitted with an eyepiece that gives a magnification
of 80 and a field of view of 40\deg. What is the true field on the sky.

\item
Using the diameter and orbital radius for Mars (consult the texts),
calculate its maximum angular diameter as seen from Earth in arc
seconds.

\item
If the atmosphere in NY smears out the image of Mars (at maximum size)
to give
``pixels'' of 2\arcsec, how many pixels is it across the diameter of
the Mars image. 

\item
Using the maximum Mars diameter and its orbital radius,  determine the
minimum angular diameter of Mars.

\item
Using the diameter and orbital radius for Saturn (consult the texts),
calculate its maximum angular diameter as seen from Earth in arc seconds.

\item
From the above, and images of Saturn in the texts, estimate the maximum
angular diameter of Saturn's rings.

\item
The maximum angular diameter of Venus is about 60\arcsec. Calculate
the minimum.

\item
The relative change of the angular size of Neptune from largest to
smallest is less than that of Venus. Why.

\item
What phase is Venus in when its angular size is a maximum.

\item
Is the angular size of Venus in a crescent phase larger or smaller
than when it is in the gibbous phase. 

\item
What phase is Saturn in when its angular size is a maximum.

\item
Saturn, when fully illuminated is slightly elliptical rather than
round. Which dimension is larger and why.

\item
Jupiter and (to a lesser extent) Saturn are striped. What are the
stripes, and how are they oriented.

\item
Of the planets Mercury--Saturn:\\
Which has least atmosphere \\
Which is covered by a thick, uniform cloud layer \\
Which shows frozen polar caps like Earth \\

\item
List the 4 bright moons of Jupiter, in order from the center.

\item
What is very roughly the orbital time for the innermost moon of Jupiter: an
hour, a day, a month, a year, 12 years.

\item
To which planets do these moons belong: \\
Titan \\
Phobos \\
Rhea.

\item
The moons of Jupiter are seen close to a line through the center, but
those of Saturn are sometimes seen above and/or  below the planet as
well. Explain.  


\item 
The rings of Saturn change their orientation as seen from Earth in a
regular way with a period of 29 years. What does this time scale
correspond to, and why do the rings change with it.




\end{enumerate} 


\end{document}










