 
\documentstyle[11pt]{article} 
\topmargin -.6in 
\textheight 8.7in 
\oddsidemargin -.0in 
\textwidth 6.5in 
\title{The Analysis of Starlight: Lab Projects} 
\date{Fall 1997} 
%\renewcommand{\baselinestretch}{1.2} 
\begin{document} 
\setcounter{page}{1} 
\setcounter{equation}{0} 
\pagestyle{empty} 
\parindent 0pt 
\parskip 8pt 
%\pagestyle{myheadings} 
\markright{{\bf LAB B: Measuring the Wavelength of Light} \
\hrulefill \ } 
\def\arcsec{\ifmmode {^{\scriptscriptstyle\prime\prime}}
          \else $^{\scriptscriptstyle\prime\prime}$\fi}
\def\arcmin{\ifmmode {^{\scriptscriptstyle\prime}}
          \else $^{\scriptscriptstyle\prime}$\fi}
\def\deg{\ifmmode^\circ\else$^\circ$\fi}


   
 
\noindent 
%\vspace{0.15in} 
{\bf Observational Astronomy      \hfill  V85.0013}\\ 

 
\bigskip 
 
\bigskip 
 
\noindent 
{\hfill \Large {\bf Review Sheet 5} \hfill} 
 
 
\bigskip 
%{\hfill {\bf Short Answers} \hfill} 
 
\begin{enumerate} 

\item 
The planets are found on the celestial sphere close to what line. 

\item 
What is 1 AU in km. \\
How many Earths, placed edge-to-edge, are needed to stretch from the
Sun to the Earth. (Earth Radius = 6,400~km)

\item 
To an observer in NY during a typical 24 hr period, all the
planets rise and set: true/false?

\item 
To an observer on the North Pole during a typical 24 hr period, all the
planets rise and set: true/false?

\item 
For an observer in NY, roughly when does an outer planet like Saturn in
conjunction rise.

\item 
For an observer in NY, roughly when does an outer planet like Neptune in
opposition cross our meridian.

\item
Jupiter has a sidereal period of 11.9 years. How long on average does
it spend in each of the zodiacal constellations.

\item
An outer planet like Saturn moves (east/west) against the stars,
except near (conjunction/opposition/east quadrature/west quadrature),
when it moves (east/west) in retrograde motion.

\item
For an observer in NY, roughly when does Venus in inferior conjunction
set.

\item
When Venus is in western elongation, is it a morning or evening star.


\item
When Venus is an evening star, is it approaching or receding from
Earth.

\item
Place the following in order of brightness
(at their maximum), brightest first: Sirius, Jupiter, Venus.

\item
In what phase are the following: \\
Jupiter at opposition \\
Jupiter in conjunction \\
Venus at greatest eastern elongation \\
Venus at superior conjunction \\
Mercury at inferior conjunction

\item
When Mars is closest to Earth \\
What phase is it in: \\
When is it highest in the sky:

\item
Neptune has a sidereal period of 165 years. When it is in opposition,
roughly how long is it before it is next in opposition.


\item
Which planet has: \\
The longest orbital period\\
The shortest synodic period \\
The synodic period closest to one year.

\item
Venus has an orbital radius of 0.72 AU. How far away is it:\\
At its closest approach to Earth \\
At its farthest from Earth.

\item
Saturn has an orbital radius of 9.5 AU. How far away is it:\\
At its closest approach to Earth \\
At its farthest from Earth.

\item
What is a ``transit of Venus''.

\item 
For experts! At what latitude and sidereal time on
Earth will all the planets be close to the horizon.


\end{enumerate} 


\end{document}










