 
\documentstyle[11pt]{article} 
\topmargin -.6in 
\textheight 8.7in 
\oddsidemargin -.0in 
\textwidth 6.5in 
\title{The Analysis of Starlight: Lab Projects} 
\date{Fall 1997} 
%\renewcommand{\baselinestretch}{1.2} 
\begin{document} 
\setcounter{page}{1} 
\setcounter{equation}{0} 
\pagestyle{empty} 
\parindent 0pt 
\parskip 8pt 
%\pagestyle{myheadings} 
\markright{{\bf LAB B: Measuring the Wavelength of Light} \
\hrulefill \ } 
\def\arcsec{\ifmmode {^{\scriptscriptstyle\prime\prime}}
          \else $^{\scriptscriptstyle\prime\prime}$\fi}
\def\arcmin{\ifmmode {^{\scriptscriptstyle\prime}}
          \else $^{\scriptscriptstyle\prime}$\fi}
\def\deg{\ifmmode^\circ\else$^\circ$\fi}


   
 
\noindent 
%\vspace{0.15in} 
{\bf Observational Astronomy      \hfill  V85.0013}\\ 
 
\bigskip 
 
\bigskip 
 
\noindent 
{\hfill \Large {\bf Review Sheet 6} \hfill} 
 
\bigskip 
 
\begin{enumerate} 

\item
7.1\deg

\item
0.5\deg

\item
18\arcsec (actually gets bigger because orbit is ellipse)


\item
9

\item
6\arcsec

\item
19\arcsec


\item
about 42\arcsec 

\item
10\arcsec


\item
It is more distant so adding/subtracting Earth orbit makes less difference

\item
new


\item
larger when crescent

\item
full

\item
equator, spun out 

\item
cloud patterns, parallel to equator

\item
Mercury, Venus, Mars


\item
Io, Europa, Ganymede, Callisto

\item
day

\item
Saturn, Mars, Saturn

\item
Jupiter is edge-on; Saturn is tilted to ecliptic


\item 
The time is Saturn's orbital period. The ring plane is fixed in space,
so we see the whole range of orientations in one period


\end{enumerate} 


\end{document}










