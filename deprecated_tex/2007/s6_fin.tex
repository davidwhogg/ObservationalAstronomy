\documentclass[11pt]{article}
\topmargin -0.6in 
\textheight 8.7in 
\oddsidemargin -.0in 
\parindent 0pt
\textwidth 6.5in 
\pagestyle{empty}
\usepackage{graphicx}
\begin{document}
\def\arcsec{\ifmmode {^{\scriptscriptstyle\prime\prime}}
          \else $^{\scriptscriptstyle\prime\prime}$\fi}
\def\arcmin{\ifmmode {^{\scriptscriptstyle\prime}}
          \else $^{\scriptscriptstyle\prime}$\fi}
\def\deg{\ifmmode^\circ\else$^\circ$\fi}


%%%%%%%%%%%%%%%%%%%%%%%%% page 1

{  \hfill Last Name:\,\makebox[6cm]{\hrulefill}}

{ \hfill First Name:\,\makebox[6cm]{\hrulefill}}

\vspace{0.5cm} 
\begin{center}
{\large \bf Observational Astronomy: Final} \\
%\vskip 0.8cm
\end{center}

\noindent Answer all questions. The backs of the sheets may be used
for rough work if needed.

\bigskip
{\bf Part I}

\bigskip
1. Figure 1 below is an unlabeled map of the sky with Polaris at the center, and
going down to Dec = --45\deg\ around the edges.  For each of the
following: \\
a) carefully circle the single star, star group, or constellation
{\bf unambiguously} \\ 
b) write its name outside the map, and draw a line from
the name to the circled region.

\bigskip
Ursa Minor \hspace{0.5cm} Big Dipper \hspace{0.5cm} Rigel
\hspace{0.5cm} Castor and Pollux (together) \hspace{0.5cm} Sirius
\hspace{0.5cm} Cancer \hspace{0.5cm}
  

\newpage
2. Figure 2 (below) is a part of the celestial sphere from the atlas.

\bigskip
How many fists at arms length stretch from top to bottom of the map:
\makebox[4cm]{\hrulefill}

\medskip
How many full moons side-to-side are needed to stretch from lambda to
gamma Ceti: \makebox[4cm]{\hrulefill}

\medskip
What does the constellation name Aries mean in English:
\makebox[4cm]{\hrulefill} 

\medskip
When the center of the map crosses the meridian, what is the
sidereal time:  \makebox[4cm]{\hrulefill} 

\medskip
When the center of the map crosses the meridian in NY, what is its
altitude above the horizon:  \makebox[4cm]{\hrulefill} 

\medskip
How long does it take the Sun to pass across the map: 
\makebox[4cm]{\hrulefill} 

\medskip
How long does it take the Moon to pass across the map:
\makebox[4cm]{\hrulefill}  
 
\medskip 
What does the M in M45 in the map stand for: \makebox[4cm]{\hrulefill}

\medskip
If you observed the center of the map, centered in
the field of view of a  pair of 7 $\times$ 50
binoculars, how many stars of the map would you see: \makebox[4cm]{\hrulefill} 

\medskip
As accurately as possible draw in with a filled circle where the Sun
is on April 23.


\newpage
3. The figure below shows a schematic horizon for NYC and 
the celestial equator across the sky.
It is just before dawn. If any of the following are below the horizon,
write BELOW after the item, otherwise draw them in the figure and
label them with the question number (no phases needed).

\begin{tabbing}
1. Full Moon \ \ \ \ \ \ \ \ \ \ \ \ \ \ \ \ \ \ \ \ \ \ \  \ \ \ \ \
\ \ \ \ \ \ \ \ \ \ \ \ \ \ \    \=
2. Third quarter Moon \\
3. Waning crescent Moon \>
4. Venus at maximum western elongation \\
5. Mercury at superior conjunction \>
6. Saturn at opposition  \\
7. Neptune, 3 months after conjunction
\end{tabbing}
\vspace{7.0cm}

\bigskip 
4. 

%% commented out because I don't have the images...
%% {\hfill \resizebox{!}{5.0cm}{\includegraphics{pic_m57.eps}}
%%   \resizebox{!}{5.0cm}{\includegraphics{pic_horse.eps}}
%%     \resizebox{!}{5.0cm}{\includegraphics{obs_eclipse.eps}}  \hfill} 

Left image:\\
Specify the generic class of deep sky object: \makebox[3cm]{\hrulefill} \\
Specify exactly what it was immediately before:  \makebox[3cm]{\hrulefill}

\bigskip
Middle image: \\
Specify why the bright stuff shines:  \makebox[3cm]{\hrulefill}\\
Specify why the dark area is dark: \makebox[3cm]{\hrulefill}\\

Right image (this includes the Sun and Moon): \\
Is it a solar or lunar eclipse: \makebox[3cm]{\hrulefill} \\
Is the crescent getting bigger or smaller: \makebox[3cm]{\hrulefill} \\

 
\clearpage
\newpage
{\bf PART II} Answer with one or a few words only.

\begin{enumerate}

\item
What causes astronomical ``seeing'': \makebox[3cm]{\hrulefill}\\
Give a typical value for it (including units):  \makebox[3cm]{\hrulefill}

\vspace{0.5cm}
\item
During the semester Saturn was brighter/dimmer than Polaris:
\makebox[3cm]{\hrulefill} \\
Saturn was mostly on the border between two constellations: name one:
\makebox[3cm]{\hrulefill}  

\vspace{0.5cm}
\item 
Where in the sky are the planets to be found: \makebox[3cm]{\hrulefill}

\vspace{0.5cm}
\item 
What is the approximate angular diameter of the Sun in arc minutes:
  \makebox[3cm]{\hrulefill} \\ What is the time period between 
  maximum and minimum angular diameter: \makebox[3cm]{\hrulefill}

\vspace{0.5cm}
\item 
As you watch the stars  in NYC at alt = 10\deg, az = 200\deg, the az
(increases/decreases)  \makebox[3cm]{\hrulefill} and the alt
(increase/decreases)  \makebox[3cm]{\hrulefill}. 

\vspace{0.5cm}
\item
The distance to the Moon is roughly (give units):
\makebox[3cm]{\hrulefill} \\
The distance to the nearest stars is roughly (give units):
\makebox[3cm]{\hrulefill} 

\vspace{0.5cm}
\item
The spectral type is directly related to what physical stellar
quantity: \makebox[3cm]{\hrulefill} 

\vspace{0.5cm}
\item 
What is the Trapezium:  \makebox[3cm]{\hrulefill} \\
Where is the Trapezium:  \makebox[3cm]{\hrulefill}

\vspace{0.5cm}
\item 
The resolution of the human eye is about (include units): 
\makebox[3cm]{\hrulefill}  \\

\vspace{0.5cm}
\item
The most distant object visible to the naked human eye is :
\makebox[3cm]{\hrulefill} \\
The most distant objects visible to the Hubble Space Telescope are: 
\makebox[3cm]{\hrulefill}
 

\vspace{0.5cm}
\item
What two quantities determine the angular size of a planet:
\makebox[3cm]{\hrulefill} 


\vspace{0.5cm}
\item
Where can you see:  Plato: \makebox[3cm]{\hrulefill}  \ \ Io: \makebox[3cm]{\hrulefill}
  

\vspace{0.5cm}
\item
Why do atmospheric effects like twinkling increase at lower altitude: 
 \makebox[3cm]{\hrulefill} 



\vspace{0.5cm}
\item
Algol varies because: \makebox[3cm]{\hrulefill}

\vspace{0.5cm}
\item
Saturn's disk (not the rings) looks slightly elliptical when fully
illuminated. Why: \makebox[3cm]{\hrulefill}   

\vspace{0.5cm}
\item
At the summer solstice, where on Earth is the Sun in the zenith at
noon: \makebox[3cm]{\hrulefill}


\vspace{0.5cm}
\item
When observing a star with one of our Meade or Celestron telescopes, 
how many times does the light pass along the length of the telescope:
\makebox[3cm]{\hrulefill}

\vspace{0.5cm} 
\item
What property of a 7th magnitude
nebula makes it more difficult to see with a telescope than a 
7th magnitude star: 
\makebox[3cm]{\hrulefill}

\vspace{0.5cm} 
\item
A nearby comet can have a much larger angular size than a nearby
asteroid. Why:  \makebox[3cm]{\hrulefill}


\vspace{0.5cm} 
\item
Why don't close binary stars simply fall together: \makebox[3cm]{\hrulefill} 

\end{enumerate}

\vspace{0.7cm} 
{\bf PART III}

\begin{enumerate}


\vspace{0.5cm}
\item
In this diagram the + marks the position of the NCP \\
as seen from NYC, and the * marks a star 5\deg {\hspace{3cm}} *
{\hspace{1.5cm}}+ \\ 
from it. \\
a) What is the dec of the star: \makebox[3cm]{\hrulefill} \\
b) Indicate with another * where the star will be in 6 hrs time.


\vspace{0.5cm}
\item
In which constellations (Latin or English) are the following:

\medskip
a) Mizar:  \makebox[3cm]{\hrulefill} \\
b) Sirius:  \makebox[3cm]{\hrulefill}  \\
c) Big Dipper:  \makebox[3cm]{\hrulefill}

\vspace{0.5cm}
\item
In NY, what is the Dec of stars:\\
a) In the zenith:  \makebox[3cm]{\hrulefill} \\
b) On the north horizon:  \makebox[3cm]{\hrulefill} \\
c) On the east horizon:  \makebox[3cm]{\hrulefill}


\vspace{0.5cm}
\item
At a sidereal time of 21 hr in NYC, what is the RA: 
\\
a) In the zenith: \makebox[3cm]{\hrulefill} \\
b) On the west horizon: \makebox[3cm]{\hrulefill} \\
c) On the north horizon: \makebox[3cm]{\hrulefill} 

\vspace{0.5cm}
\item
As seen from the equator, the following are in the zenith approximately: \\
always, 1/hr, 2/hr, 1/day, 2/day, 1/week, 2/week, 1/month, 2/month,
1/yr, 2/yr, never) \\
a) Celestial equator:  \makebox[3cm]{\hrulefill} \\
b) Ecliptic: \makebox[3cm]{\hrulefill}\\
c) Milky Way: \makebox[3cm]{\hrulefill}


\vspace{0.5cm}
\item
What is the sidereal time at: \\
a) noon on March 21:  \makebox[3cm]{\hrulefill} \\
b) 6 pm on June 21:  \makebox[3cm]{\hrulefill} \\
c) midnight on Sept 23rd (the equinox):  \makebox[3cm]{\hrulefill}

\vspace{0.5cm}
\item
A star on the celestial equator appears on the meridian in NYC tonight
at 9:00 pm exactly.
At what watch time: \\
a) Will it cross the meridian tomorrow: \makebox[3cm]{\hrulefill} \\
b) Will it cross the meridian in 3 months time:
\makebox[3cm]{\hrulefill} \\

\vspace{0.5cm}
\item
Venus is currently prominent in the eastern sky before dawn. Describe
briefly the sequence of where and when it can/cannot be seen in the sky
over the next synodic cycle:



\vspace{0.8cm}
\item
For the planets Mercury through Saturn:
\\
a) Which shows polar caps:  \makebox[3cm]{\hrulefill}
\\
b) Which has the shortest synodic period: \makebox[3cm]{\hrulefill}
\\
c) Which has 2 moons:  \makebox[3cm]{\hrulefill}



\vspace{0.5cm}
\item
a)Draw a first quarter Moon \\ 
as it crosses the meridian \\
as seen from NYC. \\
Shade in the unlit part \\
b) How many days is it since the moon was new: \makebox[3cm]{\hrulefill}
\\
c) For a Moon resident, is it sunset or sunrise at the center of the disk: \makebox[3cm]{\hrulefill} 

\vspace{0.5cm}
\item 
a) The best time to see details of a given part of the lunar
 surface is when: \makebox[3cm]{\hrulefill}
b) The Mares on the Moon are seen to be less cratered than the rest, why:
\makebox[3cm]{\hrulefill}\\
c) Some craters show rays, what are these: \makebox[3cm]{\hrulefill} \\

\vspace{0.5cm}
\item
A wide binary is separated by 15\arcmin, and contains an A
star of apparent magnitude 6 and a K star of apparent magnitude
3.5.\\
a) What magnification is needed to just fit the binary in the field of
an eyepiece of apparent field 40\deg: \makebox[6cm]{\hrulefill} \\
b) Which star has the brighter absolute magnitude:
\makebox[6cm]{\hrulefill} \\ 
c) If the A star is a main sequence star, what is the K star likely to
be:  \makebox[3cm]{\hrulefill} 



\vspace{0.5cm}
\item
a) If a telescope is advertised as an 10 inch telescope, what does the 10
inches measure:  \makebox[3cm]{\hrulefill} \\
b) By what factor does this telescope collect more light than a 1 inch
telescope:  \\ \makebox[3cm]{\hrulefill} \\
c) How many magnitudes brighter is a star seen in the 10 inch,
compared to the 1 inch: \makebox[3cm]{\hrulefill}


\vspace{0.5cm}
\item
a) In time exposure photographs of the night sky, star trails are arcs.
Why are meteor trails  straight lines:
\makebox[6cm]{\hrulefill} \\
b) Meteors in showers are named after constellations: why:
\makebox[3cm]{\hrulefill} \\
c) Why do showers occur: \makebox[3cm]{\hrulefill} 


\vspace{0.5cm}
\item
Make your best estimates of: \\
a) The magnitude of Sirius: \makebox[3cm]{\hrulefill} \\
b) The naked eye limit in NY on a good night: \makebox[3cm]{\hrulefill} \\
c) The magnitude of Titan: \makebox[3cm]{\hrulefill}


\vspace{0.5cm}
\item
What is the approximate time period between:\\
a) Eclipse seasons: \makebox[5cm]{\hrulefill}  \\
b) Successive risings of the Moon, as seen from the north Pole:
\makebox[5cm]{\hrulefill}  \\
c) Successive risings of Saturn, as seen from NY: \makebox[5cm]{\hrulefill} 

\vspace{0.5cm}
\item
Sort the following names into 3 pairs, where each pair matches objects
of the same physical type: \\
Phobos, Mira, Cepheid, Beehive, Pleiades, Ceres \\
a) \makebox[5cm]{\hrulefill} \\
b) \makebox[5cm]{\hrulefill} \\
c) \makebox[5cm]{\hrulefill} 

\vspace{0.5cm}
\item
In which directions do the following occur , e-w or w-e: \\
a) Sun motion on the ecliptic (relative to stars):  \makebox[3cm]{\hrulefill} \\
b) Sunspots cross the Sun:  \makebox[3cm]{\hrulefill} \\
c) Mars' motion against the stars during retrograde motion:
\makebox[3cm]{\hrulefill} 



\end{enumerate}

\bigskip
{\bf END}
\end{document}




















