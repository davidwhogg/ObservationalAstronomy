\documentclass[11pt]{article}
\topmargin -0.6in 
\textheight 8.7in 
\oddsidemargin -.0in 
\textwidth 6.5in 
\pagestyle{empty}
\begin{document}
\def\arcsec{\ifmmode {^{\scriptscriptstyle\prime\prime}}
          \else $^{\scriptscriptstyle\prime\prime}$\fi}
\def\arcmin{\ifmmode {^{\scriptscriptstyle\prime}}
          \else $^{\scriptscriptstyle\prime}$\fi}
\def\deg{\ifmmode^\circ\else$^\circ$\fi}


%%%%%%%%%%%%%%%%%%%%%%%%% page 1

{    \hfill Last Name:\,\makebox[6cm]{\hrulefill}} 

{ \hfill First Name:\,\makebox[6cm]{\hrulefill}}

\vskip 0.8cm
\begin{center}
{\large \bf Observational Astronomy: Midterm Wed} \\
%\vskip 0.8cm
\end{center}

\noindent You need the Mag 5 Atlas.
Use the back of this paper for rough notes etc.

\bigskip

\begin{enumerate}

\item 
Figure 1 is an unlabeled map of the sky with Polaris at the center, and
going down to Dec = --50\deg\ around the edges.  Use the atlas to
navigate around the map. {\bf Circle (for the RA,Dec mark with +) and label}
carefully and unambiguously: 

\medskip
$\beta$ Cassiopeiae \ \ \ Sirius \ \ \ Auriga \ \ \ Castor
\ \ \  Orion's belt \ \ \ RA=0 hr, Dec=0\deg

\vskip 1.0cm
\item
Use the atlas to find in which constellations are the following: \\
a) RA = 23 hr, Dec = +25\deg \\
b) the autumnal equinox

\vskip 1.0cm
\item 
How far in degrees is it from alpha ($\alpha$) to delta ($\delta$) of the
Big Dipper: \\
How many full Moons, edge to edge, are needed to span this distance:

\vskip 1.0cm
\item 
Consider the circle of stars that forms the head of the whale in Cetus
(Map 3).\\
Which star is dimmest:\\
Which star crosses the meridian first:

\vskip 1.0cm
\item
Approximately where on Earth does the center of the head of the whale (previous
question) pass
directly overhead: \\ 
At what sidereal time will this occur:

\vskip 1.0cm
\item
Can you see all three of the stars in Orion's belt
at the same time in binoculars with a 6~deg diameter field of view:

\vskip 1.0cm
\item
In NY, how often are the following on the meridian between the zenith
and the south horizon in a typical day (continuously, once, twice, 4 times, hardly ever, never) \\
a) Dec = 15\deg \\
b) RA = 13 hr 


\vskip 1.0cm
\item 
Sirius is up now in the evening in NY.  \\
According to EST, Sirius crosses the meridian earlier/later/at the
same time compared to the previous night:

\vskip 0.5cm
According to sidereal time, Sirius crosses the meridian earlier/later/at the
same time compared to the previous night: 

\vskip 1.0cm
\item
You are at a latitude of +70\deg\ on Earth. \\
Stars of what Dec pass through the zenith:\\
Stars of what Dec appear on the northern horizon, i.e., are just circumpolar:

\vskip 1.0cm
\item
In NY, when you look at the stars near Polaris, in which direction do
they appear to move around the 
NCP, clockwise or counter-clockwise: \\
How long does it take them to go around once (be precise):

 
\vskip 1.0cm
\item
In which direction does the Sun move on the ecliptic against the background stars, E or
W: \\
How far does it move in RA in one week:


\vskip 1.0cm
\item
In which direction does the Moon move in its orbit against the
background stars, E or W: \\
How far does it move in RA in one week:


\vskip 1.0cm
\item
Describe the detailed motion of the Sun in the sky during 24 hr at the summer
solstice, as seen from the North pole.

\vskip 1.0cm
\item
Last Monday was the equinox. Calculate the sidereal time for today  at 10 pm.


 
\vskip 1.0cm
\item
Which feature of the Sun's motion on the celestial sphere is most
directly responsible for the lengthening of daylight hours in the
spring in NY: increasing Dec or increasing RA: \\
Explain why:


\vskip 1.0cm
\item
As seen from NY, if Kochab in Ursa Minor is
on the meridian directly above Polaris, what is the sidereal time:


\vskip 1.0cm
\item
At what time does a full moon rise: \\
At what time does a 1st quarter moon transit:  \\

\vskip 1.0cm
\item
Draw a waxing crescent moon as it appears to the naked eye i) rising over
the eastern horizon and ii) setting over the west horizon in NY (no mares or craters needed). Shade in the unlit part.

\vskip 1.0cm
 Horizon: \ \ i) \makebox[3cm]{\hrulefill} \ \ ii) \makebox[3cm]{\hrulefill} 

\vskip 1.0cm
\item
When you observe the Moon with the naked eye, there is an isolated,
dark oval area to the upper right. \\
What is the full name of this area: \\
What is the approximate angular size (along the long axis):

\vskip 1.0cm
\item
In what phase is a 17 day old Moon:

\vskip 1.0cm
\item
For how long does sunlight fall continuously in the center of the
Moon's disk: \\
Apart from the phases, the angular diameter of the Moon's disk changes; what is
the time between maximum and minimum angular size:

\vskip 1.0cm
\item 
In NY, all the planets rise and set in a typical 24 hour period: true/false



\vskip 1.0cm
\item
Of the planets easily visible in the evening sky at present:
\\ Which is currently the nearest:
\\ Which is currently closest to opposition:
 
\vskip 1.0cm
\item
Mars has an orbital radius of approximately 1.5 AU. \\
What is its minimum distance from Earth: \\
What is its maximum distance from Earth:


\vskip 1.0cm
\item 
When Mars is closest to Earth: \\
What phase is it in:   \\
At what time does it transit:



\end{enumerate}

\end{document}




















