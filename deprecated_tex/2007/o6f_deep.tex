
\documentclass[12pt]{article}
\usepackage{psfig}
\topmargin -.6in
\textheight 8.7in
\oddsidemargin -.0in
\textwidth 6.5in
\title{The Analysis of Starlight: Lab Projects}
%\renewcommand{\baselinestretch}{1.2}

\begin{document}
\setcounter{page}{1}
\setcounter{equation}{0}
\pagestyle{plain}
\thispagestyle{empty}  % suppress number on first page
%\pagestyle{myheadings}
\newcommand{\kms}{\hbox{km\,s$^{\rm -1}$}}
%\def\kms{\ifmmode {\,{\rm km\,s^{-1}}}                          % km s-1
%determine the RA of the Sun        \else {\hbox{$\,$ {\rm km$\,$s$^{\rm -1}$}}}\fi}
%\def\solar {\ifmmode_{\mathord\odot} \else $_{\mathord\odot}$\fi} % _solar
%\def\mo {\ifmmode {\,{\it M}\solar} \else $\,M$\solar\fi}       % M solar
\def\lo {\ifmmode {\,{\it L}\solar} \else $\,L$\solar\fi}       % L solar
\def\my {\ifmmode {\,{\it M}\solar\,{\rm yr^{-1}}}              % Msol/year
        \else {$\,M$\solar$\,$yr$^{\rm -1}$}\fi}
\def\BD {BD$\,$+30{\degr}3639}
\def\HUNO{\rm H$\,$I}                   % molecular hydrogen
\def\HDOS{\rm H$_2$}                    % molecular hydrogen
\def\arcsec{\ifmmode {^{\scriptscriptstyle\prime\prime}}
          \else $^{\scriptscriptstyle\prime\prime}$\fi}
\def\arcmin{\ifmmode {^{\scriptscriptstyle\prime}}
          \else $^{\scriptscriptstyle\prime}$\fi}
\def\deg{\ifmmode^\circ\else$^\circ$\fi}

\sloppy




\noindent
%\vspace{0.15in}
{\bf Observational Astronomy    \hfill}\\
{\bf Lab: O-3/6F-note} 


\bigskip

\medskip

\noindent
{\hfill \Large {\bf Observing Faint Objects} \hfill}


\bigskip

\noindent
{\bf 1. Finding objects}

\medskip
\noindent
We review some techniques for finding objects of known
coordinates with a telescope. The best technique depends on how bright
the object is.

\medskip\noindent $\bullet$ \ If the object is bright enough to be seen with
the naked eye, the procedure is very straightforward. Locate the object
by eye, point the telescope in that direction, center the object in the
finder scope, and it should appear in the main eyepiece (see also important
note below).

\medskip\noindent  $\bullet$ \ If the object is not visible to the
naked eye, but is bright enough to be seen in the finder, 
Align the RA and Dec setting circles using
Polaris and a standard star, as in the Dialing in the Stars lab. Then
simply dial in the object. This should
usually be accurate enough to bring the object within the field of
view of the finder. If it can be seen, center it, and it should become
visible in the main eyepiece (see also important
note below).

\medskip\noindent  $\bullet$ \ If the object is too faint to be seen
in the finder, we have to be more careful. One technique is to use the
setting circles but to offset from a star very near to the object, so
that the errors are minimized. To do this, find a nearby bright star of
known coordinates and center it in the main eyepiece. Read the setting
circles as accurately as possible and compare the results with the
listed values: the differences become corrections to the setting circle
values. Now dial in the object, using the listed coordinates plus or
minus (as appropriate) the corrections you have just found. If you do
this carefully, it should be possible to point the telescope to an
accuracy of about 0.5\deg.  With a low power 
eyepiece the object should be somewhere in the eyepiece field.

\medskip\noindent  $\bullet$ \ An alternative for faint objects is to
use a star chart, e.g,, from the Atlas or the Field Guide. The
technique here is to locate the pattern of stars near the object in
the finder, and then align the finder so that the unseen object lies on the
cross hairs. The faint object should then appear in the main eyepiece
field (see important
note below).


\medskip\noindent
{\bf Important Notes:}  1. The image in the finder is
\emph{inverted} -- upside down, but \emph{upright}
in the main eyepiece (although sometimes switched left right).
2. Usually the finder is not exactly aligned with the main
eyepiece. To overcome this, center a bright star in the eyepiece and
note where it appears in the finder. Then use this position in the
finder as if it was the cross hair position.


It is also important to have a rough idea of the
size of the patch of sky you are looking at. In the finder, this is
about 7\deg. In the main eyepiece, it is much smaller: about 48\arcmin\ for
an eyepiece of focal length 40 mm, and 30\arcmin\ for one of 25 mm.


\newpage



\noindent
{\bf 2. Making a Finding Chart}

\medskip
\noindent
A finding chart is a sketch of the patch of the sky around the object
you are looking for, and allows you to home in on it.

The field of view of our finding charts is $20\deg \times 20 \deg$
which is usually big enough to include some reasonably
bright stars.  The inner circle on the chart is 5\deg\ in diameter,
and is just smaller than the field of view of the finder scope. The
cross at the center will correspond to the location of the object. To
make the finding chart:

\medskip\noindent
Locate the object in the Mag 5 Atlas and the Field Guide (pp 228 +).

\medskip\noindent
Mark on the chart the brightest stars in the Mag 5 Atlas that are within the
20\deg\ field,  centered on the
object.  Use neat pencil dots about 2 mm in diameter for the stars.
In case you later get lost, circle one of the stars as your reference,
estimate its RA and Dec as accurately as possible, and write the
coordinates next to the star.

\medskip\noindent Now locate fainter stars in the FG and draw them in
on the chart in the central circle, and (if necessary) out as far as
your reference star. Use appropriate sized dots for plotting. You need
only include enough of the brighter stars so that you will be able to
recognize that pattern and hop, using the finder, from the reference
star to match the view in the finder with the stars in the circle on
your chart.  Remember, the object may not be visible in the finder.

\bigskip
\bigskip\noindent
{\bf 3. Using the Finding Chart Outside} 

\medskip\noindent
Approximately align the telescope.

\medskip\noindent
Locate your reference star in the finder. If you have difficulty
identifying it, choose a very bright star, set up the setting circles
using the very bright star's coordinates from the atlas (page 33),
and offset to your reference star.

\medskip\noindent
Once you can see your reference star in the finder, star hop from star
to star until the deep sky object according to your map is located on
the finder cross hair (or rather, the equivalent position that is
aligned with the view in the main eyepiece) . If you have been
careful, the object should be visible in the main eyepiece.

\bigskip\noindent
{\bf NOTE 1:} The finder inverts your view so you have to use your
chart upside down.


\medskip\noindent
{\bf NOTE 2:} Many deep-sky objects are very faint, so you need to be well
dark adapted, i.e., in the dark with only a red flashlight for 20
minutes or more.



\newpage
\pagestyle{empty}
\noindent
{\bf DEEP SKY OBJECT FINDING CHART    \hfill} {\bf  Name:\makebox[4cm]{\hrulefill}}

\vspace{1.0cm}

\noindent
{\bf OBJECT:} {\makebox[3cm]{\hrulefill}}   
{\bf CONST:} {\makebox[3cm]{\hrulefill}} 
{\bf RA and DEC:} {\makebox[3cm]{\hrulefill}}      

\vspace{16cm}

\noindent
{\bf OBSERVATIONS:}
Record what you see below as a rough sketch with notes (if relevant):




\end{document}










