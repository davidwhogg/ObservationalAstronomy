 
\documentstyle[11pt]{article} 
\topmargin -.6in 
\textheight 8.7in 
\oddsidemargin -.0in 
\textwidth 6.5in 
\title{The Analysis of Starlight: Lab Projects} 
\date{Fall 1997} 
\begin{document} 
\setcounter{page}{1} 
\setcounter{equation}{0} 
\pagestyle{empty} 
\parindent 0pt 
\parskip 8pt 
%\pagestyle{myheadings} 
\markright{{\bf LAB B: Measuring the Wavelength of Light} \
\hrulefill \ } 
\def\arcsec{\ifmmode {^{\scriptscriptstyle\prime\prime}}
          \else $^{\scriptscriptstyle\prime\prime}$\fi}
\def\arcmin{\ifmmode {^{\scriptscriptstyle\prime}}
          \else $^{\scriptscriptstyle\prime}$\fi}
\def\deg{\ifmmode^\circ\else$^\circ$\fi}   
 
\noindent 
%\vspace{0.15in} 
{\bf Observational Astronomy      \hfill  V85.0013}\\ 
 
\bigskip 
 
\noindent 
{\hfill \Large {\bf Review Sheet 3} \hfill} 
 
\begin{enumerate} 

\item 
In NYC which of the following can pass through the zenith:
the celestial equator; the ecliptic; the local meridian; RA = 1 hr
line; Dec = +30\deg.

\item On the equator, which of the following can pass
through the zenith: the celestial equator; the ecliptic;
the local meridian; RA = 1hr line; Dec = +30\deg.

\item
On the equator, what fraction of the celestial sphere can be seen:\\
a) at the same time  \ \  b) over time.

\item
What constellation is the sun in: at the vernal equinox, the summer
solstice, the fall equinox, the winter solstice (offset by one from
the signs of the zodiac).

\item
In NY, at the fall equinox, the Sun crosses the meridian at what
az and alt.

\item
In NY, at the winter solstice, the Sun crosses the meridian at what
az and alt.

\item
At the winter solstice, where on earth does the sun pass through the
zenith at noon.

\item
Describe what the Sun does in 24 hrs, as seen from the north pole on
March 21.

\item
Describe what the Sun does in 24 hrs, as seen from the north pole on
June 21.

\item
Describe what the Sun does in 24 hrs, as seen from the arctic circle,
lat = +66.5\deg, on June 21.

\item
How long does it take the sun to move in RA 
(i.e., against the background stars) a distance equal to its
diameter (0.5 deg).

\item A star crosses the meridian at 8 pm on Feb 12. At what time will
  it be there 1 month later. 

\item A star on the CE crosses the meridian at 10 pm on April 1. Where
  will it be at 10 pm,  3 months later. 

\item
In NY, what is the sidereal time when Betelgeuse is on the meridian.

\item
The sidereal time at midnight on March 21 is:

\item
The sidereal time at noon on Sept 23 is:

\item
The sidereal time at 7 pm  on March 21 is:

\item
The sidereal time at 8 pm on October 3 is:
   
\item
The Field Guide Fig. 2-5 shows a 1603 star atlas picture of Cassiopeia
(Atlas map 1). Find the
magnitude of the star at the end of her nose.

\item
Ursa Major (Atlas map 1) is an exception to the rules about labeling the stars
alpha, beta etc.  Which star in the Big Dipper is the brightest.



\end{enumerate} 


\end{document}










