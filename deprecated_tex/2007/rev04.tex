 
\documentclass[11pt]{article} 
\topmargin -.6in 
\textheight 8.7in 
\oddsidemargin -.0in 
\textwidth 6.5in 
\title{The Analysis of Starlight: Lab Projects} 
\date{Fall 1997} 
%\renewcommand{\baselinestretch}{1.2} 
\begin{document} 
\setcounter{page}{1} 
\setcounter{equation}{0} 
\pagestyle{empty} 
\parindent 0pt 
\parskip 8pt 
%\pagestyle{myheadings} 
\markright{{\bf LAB B: Measuring the Wavelength of Light} \
\hrulefill \ } 
\def\arcsec{\ifmmode {^{\scriptscriptstyle\prime\prime}}
          \else $^{\scriptscriptstyle\prime\prime}$\fi}
\def\arcmin{\ifmmode {^{\scriptscriptstyle\prime}}
          \else $^{\scriptscriptstyle\prime}$\fi}
\def\deg{\ifmmode^\circ\else$^\circ$\fi}


   
 
\noindent 
%\vspace{0.15in} 
{\bf Observational Astronomy      \hfill  V85.0013}\\ 
 
\bigskip 
 
\bigskip 
 
\noindent 
{\hfill \Large {\bf Review Sheet 4} \hfill} 
 
 
\bigskip 
%{\hfill {\bf Short Answers} \hfill} 
  
\begin{enumerate} 

\item 
In NY, you observe a 1st quarter moon on the meridian. Which
side is lit up, to your left or right.

\item 
In NY, you observe a waxing crescent moon setting. Do the horns point
(mainly) up or down. What about the case of a setting waning crescent.

\item 
At what time of day do the following Moons cross the meridian:\\
1st quarter \\
3rd quarter \\
full.

\item 
At what time of day (approx) do the following Moons rise: \\
3rd quarter \\
new \\
waxing gibbous (take mid-gibbous phase).

\item
At what time of day do the following Moons set: \\
full \\
1st quarter \\
waxing crescent (take mid-crescent phase).

\item In what phase is a 16 day old Moon.


\item
As the moon moves on the celestial sphere does its RA increase or decrease.


\item
As the Moon moves along the zodiac:\\
How many degrees per day does it move. \\
How long on average does it spend in each constellation. \\
When it is in the constellation Taurus, what constellation is it in next.

\item
How long does it take the moon to move a distance equal to its own
diameter (0.5 deg) against the background stars.

\item
How long is the Sun continuously above the horizon for someone on the
Moon's equator.

\item An astronaut at the center of the disk of a full Moon would
observe the Earth to be in what phase.

\item
In terms of motion on the celestial sphere, explain why the synodic
period of the Moon is longer than the sidereal period.


\item
What is libration of the Moon.

\item
At the equinoxes, what is the Dec of a full Moon.



\item
The angular diameter of the Moon as seen from Earth changes by 13 per cent
over time. Why.

\item
When one see a large mare on the Moon, its diameter is about the size
of: \\ The US/Texas/Manhattan/Washington Square.


\item 
What is the lunar terminator, and why is it useful.

\item
Name three examples of each of these lunar features: \\
Maria \\
Mountains \\
Craters.

\item
The mare areas of the Moon are much less cratered than the
highlands. Explain why.


\item
Describe the Moon's motion and phases as seen in December from
the North Pole.
\end{enumerate} 


\end{document}










