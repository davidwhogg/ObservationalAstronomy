 
\documentclass[11pt]{article} 
\topmargin -.6in 
\textheight 8.7in 
\oddsidemargin -.0in 
\textwidth 6.5in 
\title{The Analysis of Starlight: Lab Projects} 
\date{Fall 1997} 
%\renewcommand{\baselinestretch}{1.2} 
\begin{document} 
\setcounter{page}{1} 
\setcounter{equation}{0} 
\pagestyle{empty} 
\parindent 0pt 
\parskip 8pt 
%\pagestyle{myheadings} 
\markright{{\bf LAB B: Measuring the Wavelength of Light} \
\hrulefill \ } 
\def\arcsec{\ifmmode {^{\scriptscriptstyle\prime\prime}}
          \else $^{\scriptscriptstyle\prime\prime}$\fi}
\def\arcmin{\ifmmode {^{\scriptscriptstyle\prime}}
          \else $^{\scriptscriptstyle\prime}$\fi}
\def\deg{\ifmmode^\circ\else$^\circ$\fi}


   
 
\noindent 
%\vspace{0.15in} 
{\bf Observational Astronomy      \hfill  V85.0013}\\ 
 
\bigskip 
 
\bigskip 
 
\noindent 
{\hfill \Large {\bf Review Sheet 7} \hfill} 
  
\bigskip 



\begin{enumerate} 

\item
Place these magnitudes in order, brightest first: \\
$-$0.6 \ \ +28 \ \ +0.5 \ \ $-$2 \ \ +4.4 


\item Polaris has a magnitude of 2.1 and gamma Ursa Minor has a
magnitude of 3.1 \\
Which star is brighter: \\
By what factor is it brighter (if you measure it with a light meter):

\item
One star of the Big Dipper is much fainter than the rest: which is it:

\medskip
What common rule do the Big Dipper stars break:

\item 
Roughly how many stars of magnitude 3 do you need to cluster together
so they appear as magnitude 1: 

\item 
A star of  13th mag blows up, and reaches mag 3.  By what factor did
it brighten (as measured by a light meter): 

\item 
The Sun has an apparent magnitude of $-$26.5. How many stars of mag 3.5
do you need to cluster together to appear as bright as the Sun: 
10,000 \ \  100,000 \ \ 1 million \ \ 1 billion \ \  1 trillion 

\item
The stars have slightly different colors to the naked eye: why

\item 
What is the spectral type of the Sun:

\item 
A regular main sequence star of spectral type B0 is (hotter/cooler) and
(brighter/dimmer) than the Sun.

\item 
You observe a binary with a K giant and a white dwarf. \\
Which is redder:\\
Which is hotter: \\
Which is brighter: 

\item
Place these in order, shortest first: 1 lt yr, 1 pc, 1 AU, the
distance from Earth to Sirius

\item 
The standard distance for absolute magnitude is 10 pc. 
What is this in light years: \\
How many times farther away is this than the nearest star to us:

\item 
The Sun has an absolute magnitude of +4.8. What would be its apparent
magnitude if it were at a distance of 100 pc.

\item 
Betelgeuse has an absolute magnitude of $-$5.6. What would be its
apparent magnitude if it were only 1 pc away.

\item 
The long period variable omicron Ceti is called Mira -- the wonder
star. Describe its change in appearance over time to the naked eye.

\item 
What causes the variability of the star Algol (beta Per). \\
What evidence is there for your answer in the star's light curve:


\item 
Why are Cepheid variables called Cepheids: \\
Sketch a light curve for them: \\
Why does their brightness vary;

\item 
In a binary system about what point in space do the components orbit:

\item 
The two brightest stars in the Trapezium multiple star system are
separated by about 10''. The Trapezium is about 500 pc
away. Calculate the separation of the two stars in parsecs.

\item 
What is the proper motion of a star: \\
The typical proper motion of nearby stars are something like
0.1'' per year. If a constellation 10 deg in size has stars with proper
motions like this in random directions, estimate how many years it
takes until the constellation looks noticeably different.

\end{enumerate} 

Formulae: \ \ \ \ m $-$ M = 5 log(D/10) \hspace{2cm} theta(arcsec) = 206,000 s/d

\end{document}










