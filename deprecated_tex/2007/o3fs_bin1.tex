
\documentclass[12pt]{article}
\usepackage{psfig}
\topmargin -.6in
\textheight 8.7in
\oddsidemargin -.0in
\textwidth 6.5in
\title{The Analysis of Starlight: Lab Projects}
%\renewcommand{\baselinestretch}{1.2}

\begin{document}
\setcounter{page}{1}
\setcounter{equation}{0}
\pagestyle{plain}
\thispagestyle{empty}  % suppress number on first page
%\pagestyle{myheadings}
\newcommand{\kms}{\hbox{km\,s$^{\rm -1}$}}
%\def\kms{\ifmmode {\,{\rm km\,s^{-1}}}                          % km s-1
%determine the RA of the Sun        \else {\hbox{$\,$ {\rm km$\,$s$^{\rm -1}$}}}\fi}
%\def\solar {\ifmmode_{\mathord\odot} \else $_{\mathord\odot}$\fi} % _solar
%\def\mo {\ifmmode {\,{\it M}\solar} \else $\,M$\solar\fi}       % M solar
\def\lo {\ifmmode {\,{\it L}\solar} \else $\,L$\solar\fi}       % L solar
\def\my {\ifmmode {\,{\it M}\solar\,{\rm yr^{-1}}}              % Msol/year
        \else {$\,M$\solar$\,$yr$^{\rm -1}$}\fi}
\def\BD {BD$\,$+30{\degr}3639}
\def\HUNO{\rm H$\,$I}                   % molecular hydrogen
\def\HDOS{\rm H$_2$}                    % molecular hydrogen
\def\arcsec{\ifmmode {^{\scriptscriptstyle\prime\prime}}
          \else $^{\scriptscriptstyle\prime\prime}$\fi}
\def\arcmin{\ifmmode {^{\scriptscriptstyle\prime}}
          \else $^{\scriptscriptstyle\prime}$\fi}
\def\deg{\ifmmode^\circ\else$^\circ$\fi}

\sloppy



\noindent
%\vspace{0.15in}
{\bf Observational Astronomy    \hfill} {\bf First Name:\makebox[4cm]{\hrulefill}}\\
{\bf Lab: O-3FS} \hfill {\bf Last Name:\makebox[4cm]{\hrulefill}}


\bigskip

\medskip

\noindent
{\hfill \Large {\bf Binaries  } \hfill}


\bigskip

\noindent
{Objectives:} Binary hunting: for techniques, see sheet on finding
objects.


\bigskip
\noindent
{\bf 1. What are Binaries ?  }

\medskip\noindent 
Over half the stars  in our galaxy are members of
binary systems: i.e., they have one or more close companions and move
in mutual orbits under the force of gravity. The
word binary means double systems, but we use it for
multiple systems as well. Most binaries are too distant or the
stars too faint to be split (i.e., resolved into two),
but there are many that can be seen. Typical orbital periods are
100s of years, so we do not see the stars whizzing around each other,
but they appear frozen, locked together by gravity.

Searching for binaries is a test of
skill, and is a good opportunity to see the different
brightnesses and colors of stars. The two members of the binary are at
the same distance from us, so a difference in brightness is a
consequence of the \emph{intrinsic} difference between the stars.  Recall
also that the colors are determined by the temperature of the stellar
surface. They range from blue-white (more than 30,000 C), through
white, yellow, orange, to red (about 3,000 C).

%\newpage
\bigskip
\noindent
{\bf OUTSIDE}

\bigskip
\noindent
{\bf 2. Magnification and Field.}

\medskip\noindent Set up the telescope and note the eyepiece focal length
below.
Determine the
magnification and the field of view, so you know how much sky you can
see: 
M = 2000/f$_e$, where the f$_e$
is the focal length printed on the eyepiece, and the (true) field =
the apparent field (about 40\deg)/M. 

\medskip\noindent 
Eyepiece (mm):
\makebox[2cm]{\hrulefill} \ \ M: \makebox[2cm]{\hrulefill}
\ \ Field of View (\arcmin):\makebox[2cm]{\hrulefill}

\bigskip\noindent
{\bf 3. Observing}

\medskip\noindent
Find each of the following, sketch the binary with its actual
orientation in the field of view in the circles provided as proof that
you found it.  Label the stars with their colors, and indicate which
is the brighter. In most cases the binaries are close, and you will
need to expand the center of the field for the picture.  An example is
shown first.  The closest pairs where the angle is small are the
hardest to split.

\bigskip 
       
%\begin{minipage}[t]{9cm}{ Zeta Ursa}
%  {\psfig{figure={o3s_f1.eps},width=4.cm}}

\parbox[b]{8cm}{ Example\\
 \\ \\ }  \begin{minipage}[b]{8cm}{\psfig{figure={o3s_f1.eps},width=4.0cm}}\end{minipage}

\bigskip\noindent 

\parbox[b]{8cm}{ Star: \makebox[3cm]{\hrulefill}\\
RA: \makebox[1.5cm]{\hrulefill} \ \ Dec: \makebox[1.5 cm]{\hrulefill} \\
Brightest mag: \makebox[1cm]{\hrulefill} \\
Separation:  \makebox[1cm]{\hrulefill} \\ }   \begin{minipage}[b]{8cm}{\psfig{figure={o3s_f1.eps},width=4.0cm}}\end{minipage}

\bigskip\noindent 


\parbox[b]{8cm}{ Star: \makebox[3cm]{\hrulefill}\\
RA: \makebox[1.5cm]{\hrulefill} \ \ Dec: \makebox[1.5 cm]{\hrulefill} \\
Brightest mag: \makebox[1cm]{\hrulefill} \\
Separation:  \makebox[1cm]{\hrulefill} \\ }   \begin{minipage}[b]{8cm}{\psfig{figure={o3s_f1.eps},width=4.0cm}}\end{minipage}

\bigskip\noindent 


\parbox[b]{8cm}{ Star: \makebox[3cm]{\hrulefill}\\
RA: \makebox[1.5cm]{\hrulefill} \ \ Dec: \makebox[1.5 cm]{\hrulefill} \\
Brightest mag: \makebox[1cm]{\hrulefill} \\
Separation:  \makebox[1cm]{\hrulefill} \\ }   \begin{minipage}[b]{8cm}{\psfig{figure={o3s_f1.eps},width=4.0cm}}\end{minipage}

\bigskip\noindent 



\parbox[b]{8cm}{ Star: \makebox[3cm]{\hrulefill}\\
RA: \makebox[1.5cm]{\hrulefill} \ \ Dec: \makebox[1.5 cm]{\hrulefill} \\
Brightest mag: \makebox[1cm]{\hrulefill} \\
Separation:  \makebox[1cm]{\hrulefill} \\ }   \begin{minipage}[b]{8cm}{\psfig{figure={o3s_f1.eps},width=4.0cm}}\end{minipage}

\bigskip\noindent 


\parbox[b]{8cm}{ Star: \makebox[3cm]{\hrulefill}\\
RA: \makebox[1.5cm]{\hrulefill} \ \ Dec: \makebox[1.5 cm]{\hrulefill} \\
Brightest mag: \makebox[1cm]{\hrulefill} \\
Separation:  \makebox[1cm]{\hrulefill} \\ }   \begin{minipage}[b]{8cm}{\psfig{figure={o3s_f1.eps},width=4.0cm}}\end{minipage}

\bigskip\noindent 



\end{document}












