 
\documentstyle[11pt]{article} 
\topmargin -.6in 
\textheight 8.7in 
\oddsidemargin -.0in 
\textwidth 6.5in 
\title{The Analysis of Starlight: Lab Projects} 
\date{Fall 1997} 
%\renewcommand{\baselinestretch}{1.2} 
\begin{document} 
\setcounter{page}{1} 
\setcounter{equation}{0} 
\pagestyle{empty} 
\parindent 0pt 
\parskip 8pt 
%\pagestyle{myheadings} 
\markright{{\bf LAB B: Measuring the Wavelength of Light} \
\hrulefill \ } 
\def\arcsec{\ifmmode {^{\scriptscriptstyle\prime\prime}}
          \else $^{\scriptscriptstyle\prime\prime}$\fi}
\def\arcmin{\ifmmode {^{\scriptscriptstyle\prime}}
          \else $^{\scriptscriptstyle\prime}$\fi}
\def\deg{\ifmmode^\circ\else$^\circ$\fi}


   
 
\noindent 
%\vspace{0.15in} 
{\bf Observational Astronomy      \hfill  V85.0013}\\ 
 
\bigskip 
 
\noindent 
{\hfill \Large {\bf Review Sheet 2} \hfill} 
  

\begin{enumerate} 

\item
In NY, as you watch stars near the east horizon, they move up/down and
to your left/right.

\item
In NY, as you watch the stars at alt = 45\deg, az = 180\deg, 
is their main motion up, down, to your left, or to your right.

\item
In NY, as you watch the stars near Polaris, is the motion
clockwise or counterclockwise.

\item 
In NY, what are the  alt and az of Polaris. 

\item
In NY, what is the Dec of a star that just touches the north
horizon.

\item
In NY, what is the Dec in the zenith.

\item
In NY what is the alt of the celestial equator as it crosses the meridian.

\item
In NY, what is the Dec on the south horizon.

\item
In NY, what is the Dec on the east horizon.

\item
On the celestial sphere, which RA and Dec are directly opposite RA = 5
hr, Dec = 25\deg.

\item
In NY, when the RA line through the south horizon is 16 hr, what is the
RA on the north horizon.

\item
In NY, is a star at alt = 12\deg, az = 210\deg\  circumpolar or not.

\item
Read this precision coordinate in RA and Dec (the units are often
omitted in writing):  23:10:13.12 $-$22:23:16.4

\item How far apart in degrees are stars on the celestial equator
separated by 1 hr in RA.

\item 
How far part in \arcsec\ are two stars separated by 1 s on the celestial equator.

\item
What is the alt of Polaris as seen from latitudes of 90\deg, 20\deg, 0\deg.
\item
What is the alt of the SCP as seen from latitudes of 0\deg, $-20$\deg,
$-$90\deg.


\item You are marooned on a tropical island with a radio and your star
  atlas. You find that the bright star, Sirius (RA 17 hr, Dec
--17\deg) goes through the zenith. What latitude do you radio to
the search party.

\item
In the atlas, estimate the angular separation of Betelgeuse and
Bellatrix on the sky in degrees.


\item In the atlas, Consider the following stars:  
Rigel, Sirius, Aldebaran, Capella.
\\  Which is nearest the celestial Equator;
Which is nearest the Ecliptic \\ 
Which has the lowest Dec;
Which comes nearest the zenith in NY.


\end{enumerate} 


\end{document}










