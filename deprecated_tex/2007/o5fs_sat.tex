
\documentclass[12pt]{article}
\usepackage{psfig}
%\topmargin -.6in
\textheight 8.7in
\oddsidemargin -.0in
\textwidth 6.5in
\title{The Analysis of Starlight: Lab Projects}
%\renewcommand{\baselinestretch}{1.2}

\begin{document}
\setcounter{page}{1}
\setcounter{equation}{0}
\pagestyle{plain}
\thispagestyle{empty}  % suppress number on first page
%\pagestyle{myheadings}
\newcommand{\kms}{\hbox{km\,s$^{\rm -1}$}}
%\def\kms{\ifmmode {\,{\rm km\,s^{-1}}}                          % km s-1
%determine the RA of the Sun        \else {\hbox{$\,$ {\rm km$\,$s$^{\rm -1}$}}}\fi}
%\def\solar {\ifmmode_{\mathord\odot} \else $_{\mathord\odot}$\fi} % _solar
%\def\mo {\ifmmode {\,{\it M}\solar} \else $\,M$\solar\fi}       % M solar
\def\lo {\ifmmode {\,{\it L}\solar} \else $\,L$\solar\fi}       % L solar
\def\my {\ifmmode {\,{\it M}\solar\,{\rm yr^{-1}}}              % Msol/year
        \else {$\,M$\solar$\,$yr$^{\rm -1}$}\fi}
\def\BD {BD$\,$+30{\degr}3639}
\def\HUNO{\rm H$\,$I}                   % molecular hydrogen
\def\HDOS{\rm H$_2$}                    % molecular hydrogen
\def\arcsec{\ifmmode {^{\scriptscriptstyle\prime\prime}}
          \else $^{\scriptscriptstyle\prime\prime}$\fi}
\def\arcmin{\ifmmode {^{\scriptscriptstyle\prime}}
          \else $^{\scriptscriptstyle\prime}$\fi}
\def\deg{\ifmmode^\circ\else$^\circ$\fi}

\sloppy



%\markright{{\bf LAB E: Hubble's Law} \ \hrulefill \ }


\noindent
%\vspace{0.15in}
{\bf Observational Astronomy    \hfill} {\bf First Name: \makebox[4cm]{\hrulefill}}\\
{\bf Lab: O-5FS} \hfill {\bf Last Name: \makebox[4cm]{\hrulefill}}


\bigskip

\medskip

\noindent
{\hfill \Large {\bf Saturn} \hfill}


\bigskip

\noindent
{Objectives:} To observe the characteristics of Saturn, its rings and
moons.

\bigskip\noindent

\bigskip
\noindent
{\bf 1. Saturn }

\medskip
\noindent
In a small telescope Saturn readily shows a resolved disk and a bright
ring system. Although near opposition the angular diameter of the disk
reaches 19\arcsec, the surface features (due to cloud bands) are
fairly subtle and harder to discern than those of Jupiter. The rings
system extends out to 22\arcsec\  from the center (about 2.3 Saturn
radii). The four brightest moons in order from the center are Tethys
(10.5 mag), Dione (10.6 mag), Rhea (9.9 mag) and Titan (8.3
mag). Titan has a period of 16 days, and at opposition can extend
197\arcsec\ (10 Saturn diameters) from the center. The moons are much
more difficult to observe and identify than those of Jupiter.
 
If time permits, we shall try to observe Saturn at a later observing
session as well to record the changing moon configuration.

\bigskip
\noindent
{\bf 2. Observations}

\medskip
\noindent

\noindent $\bullet$ \ 
Record the date.

\medskip
\noindent $\bullet$ \ 
 Identify the naked eye stars around Saturn, and thereby locate its 
position in the Atlas. Estimate the RA and Dec from the map,
and identify the constellation.

\medskip
\noindent $\bullet$ \  
 Find Saturn  with one of the 8 inch telescopes.
As a preliminary, determine the directions of N, S, E, and W
in the eyepiece and label the figures on the observing sheet
accordingly. Note these may not be what you expect, due to the
optical setup: they can be determined by moving the fine controls and
seeing which way Saturn moves.

\medskip
\noindent $\bullet$ \ 
Examine Saturn's disc and rings, and sketch the results using the
larger circle on the observing sheet as a guide. If the phase is not exactly
full, shade in the limb of the disc that is missing (this should of
course correspond to the side away from the Sun).  Examine the
shape of the disc: it is usually possible to see that it is not
circular, with one dimension (the equator) being larger than the other
(apart from phases effects). Indicate in the figure where the disc is
largest.  Study any visible cloud patterns and record them on the
disk.  In sketching the rings indicate (where appropriate): any gaps
in the system; any shadow cast by the rings on the disc; any
shadow cast by the disk on the rings.

\medskip
\noindent $\bullet$ \ 
Search around Saturn (within 10 diameters from the center) for any
faint objects which are likely to be the brighter moons. Record their
position around the smaller figure in the observing sheet (the numbers
marked are Saturn diameters from the center). Note that the ring and
moon system in Saturn can be significantly tilted to the line of sight,
{\bf so the moons may appear to the side or above/below the planet.}



\newpage
\noindent
{\bf 3. Observation I}
\bigskip\bigskip
\noindent

Date: \makebox[2cm]{\hrulefill} \ \ 
Constellation: \makebox[2cm]{\hrulefill} \ \ 
RA: \makebox[2cm]{\hrulefill} \ \ 
Dec: \makebox[2cm]{\hrulefill} \ \ 

\vspace{3.0cm}


\begin{figure}[h]
\centerline{\psfig{figure={o4s_f1.eps},width=11.0cm}}

\vspace{3.0cm}

\centerline{\psfig{figure={o4s_f2.eps},width=2.5cm}}

\end{figure}
 \vspace{1.0cm}
\bigskip\bigskip\bigskip
\noindent
{\bf 4. Observation II (moons only)}
\bigskip\bigskip
\noindent

Date: \makebox[2cm]{\hrulefill} \ \ 
Constellation: \makebox[2cm]{\hrulefill} \ \ 
RA: \makebox[2cm]{\hrulefill} \ \ 
Dec: \makebox[2cm]{\hrulefill} \ \ 

\vspace{3.0cm}


\begin{figure}[h]
\centerline{\psfig{figure={o4s_f1.eps},width=11.0cm}}

%\vspace{1.0cm}


\end{figure}

\end{document}











