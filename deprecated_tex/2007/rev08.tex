 
\documentclass[11pt]{article} 
\topmargin -.6in 
\textheight 8.7in 
\oddsidemargin -.0in 
\textwidth 6.5in 
\title{The Analysis of Starlight: Lab Projects} 
\date{Fall 1997} 
%\renewcommand{\baselinestretch}{1.2} 
\begin{document} 
\setcounter{page}{1} 
\setcounter{equation}{0} 
\pagestyle{empty} 
\parindent 0pt 
\parskip 8pt 
%\pagestyle{myheadings} 
\markright{{\bf LAB B: Measuring the Wavelength of Light} \
\hrulefill \ } 
\def\arcsec{\ifmmode {^{\scriptscriptstyle\prime\prime}}
          \else $^{\scriptscriptstyle\prime\prime}$\fi}
\def\arcmin{\ifmmode {^{\scriptscriptstyle\prime}}
          \else $^{\scriptscriptstyle\prime}$\fi}
\def\deg{\ifmmode^\circ\else$^\circ$\fi}


   
 
\noindent 
%\vspace{0.15in} 
{\bf Observational Astronomy      \hfill  V85.0013}\\ 
 
\bigskip 
 
\bigskip 
 
\noindent 
{\hfill \Large {\bf Review Sheet 8} \hfill} 
 
 
\bigskip 

\begin{enumerate} 

\item
The Milky way forms a complete circle around the sky.  Explain why:


\item 
How many time a day does the Milky Way plane pass through the zenith
in NYC:


\item 
Which of the following deep sky objects are diffuse gases:
globular clusters, planetary nebulae, open clusters, H II regions

\item 
Name the most easily visible of the following (in the northern hemisphere): \\
planetary nebula, star cluster, H II region, external galaxy \\
Which can be seen with the naked eye:  

\item
What does the M in M1 and M31 stand for:


\item A typical nebula of total mag 8 is harder to see than a star of 
mag 8. Why:


\item 
The Coalsack is a  dark patch in the Milky Way. Explain why the patch
is seen:


\item 
What surrounds the Trapezium in Orion: \\
What makes it shine:

\item
In a cluster, the brighter stars on the main sequence are
redder/bluer: \\
If bright red stars are present they are:

\item 
What signature of a cluster is indicative of its age:

\item
Why are planetary nebulae called planetary nebulae: \\
They are typically more symmetric than H~II regions: why:

\item 
What was a planetary nebula just before it became a  planetary
nebula:\\
What does the central star evolve into:


\item 
Supernova 1987A was approximately 13 mag before it blew up, and
reached mag 3. By what factor did it brighten (as measured by a light meter):


\item 
The Crab nebula (M1) is about 5\arcmin\ in size. It blew up in 1054
AD. Assuming it expands s at a constant rate, how much bigger does it
get each year in arc seconds:


\item 
What is the most distant object visible to the naked eye (in a dark
sky): \\
Roughly how far away is it in light years:


\item 
How many Magellanic clouds are there: \\
What are they: \\
Where are they to be seen:


\item 
What are the different types of galaxies: \\
What type is the Milky Way:


\item 
The apparent mag of M31 is 3.7. \\
Its distance is 730,000 pc, calculate its absolute magnitude:\\
If the Sun's absolute mag is +4.8, how many times brighter is M31
than the Sun:


\item 
M31 appears roughly elliptical in general outline. \\
Explain the origin of its apparent shape: \\
What evidence is there for this in long exposure images:


\item 
What is the Hubble Deep field: \\
Nearly every object in it is a:


\end{enumerate} 

\end{document}










