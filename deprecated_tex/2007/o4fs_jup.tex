
\documentclass[12pt]{article}
\usepackage{psfig}
%\topmargin -.6in
\textheight 8.7in
\oddsidemargin -.0in
\textwidth 6.5in
\title{The Analysis of Starlight: Lab Projects}
%\renewcommand{\baselinestretch}{1.2}

\begin{document}
\setcounter{page}{1}
\setcounter{equation}{0}
\pagestyle{plain}
\thispagestyle{empty}  % suppress number on first page
%\pagestyle{myheadings}
\newcommand{\kms}{\hbox{km\,s$^{\rm -1}$}}
%\def\kms{\ifmmode {\,{\rm km\,s^{-1}}}                          % km s-1
%determine the RA of the Sun        \else {\hbox{$\,$ {\rm km$\,$s$^{\rm -1}$}}}\fi}
%\def\solar {\ifmmode_{\mathord\odot} \else $_{\mathord\odot}$\fi} % _solar
%\def\mo {\ifmmode {\,{\it M}\solar} \else $\,M$\solar\fi}       % M solar
\def\lo {\ifmmode {\,{\it L}\solar} \else $\,L$\solar\fi}       % L solar
\def\my {\ifmmode {\,{\it M}\solar\,{\rm yr^{-1}}}              % Msol/year
        \else {$\,M$\solar$\,$yr$^{\rm -1}$}\fi}
\def\BD {BD$\,$+30{\degr}3639}
\def\HUNO{\rm H$\,$I}                   % molecular hydrogen
\def\HDOS{\rm H$_2$}                    % molecular hydrogen
\def\arcsec{\ifmmode {^{\scriptscriptstyle\prime\prime}}
          \else $^{\scriptscriptstyle\prime\prime}$\fi}
\def\arcmin{\ifmmode {^{\scriptscriptstyle\prime}}
          \else $^{\scriptscriptstyle\prime}$\fi}
\def\deg{\ifmmode^\circ\else$^\circ$\fi}

\sloppy



%\markright{{\bf LAB E: Hubble's Law} \ \hrulefill \ }


\noindent
%\vspace{0.15in}
{\bf Observational Astronomy    \hfill} {\bf First Name: \makebox[4cm]{\hrulefill}}\\
{\bf Lab: O-4FS} \hfill {\bf Last Name: \makebox[4cm]{\hrulefill}}


\bigskip

\medskip

\noindent
{\hfill \Large {\bf Jupiter} \hfill}


\bigskip

\noindent
{Objectives:} To observe the characteristics of Jupiter and its four
brightest moons.

\bigskip\noindent

\bigskip
\noindent
{\bf 1. Jupiter and its moons}

\medskip
\noindent
In a small telescope Jupiter readily shows a resolved disk with some
surface features and four bright moons Io, Europa, Ganymede, and
Callisto (in order from the center).  Near opposition, the angular
diameter of the disk reaches 49\arcsec\ and allows some of the cloud
bands in its atmosphere to be seen, including the Great Red Spot.  The
brightest moons have a magnitude of about 5, and Callisto (the
farthest out) ranges up to 618\arcsec\ (13 Jupiter diameters) from the
center. The planet rotates very rapidly, in about 9 hr and 50 min, and
the configuration of the moons changes from day to day. We shall try
to observe Jupiter at more than one session during the semester to
record the changing configuration.

\bigskip
\noindent
{\bf 2. Observations} (for each observing occasion)

\begin{itemize}
\item Record the date.

\item Identify the naked eye stars around Jupiter, and thereby locate its
position in the Atlas. Estimate the RA and Dec from the map,
and identify the constellation.
 
\item Find Jupiter with one of the 8 inch telescopes. 
As a preliminary, determine the directions of N, S, E, and W
in the eyepiece and label the observing sheet
accordingly. Note these may not be what you expect, due to the
optical setup: they can be determined by moving the fine controls and
seeing which way Jupiter moves.

\item Examine the disc of Jupiter and sketch the
results in the larger circle on the observing sheet. If the phase is
not exactly full, shade in the limb that is missing (this should of
course be the side away from the Sun). Examine the shape
of the disc: it is usually easy to see that it is not circular, with
one dimension (the equator) being larger than the other (apart from
phase effects). Indicate in the figure where the disc is largest.
Study any visible cloud patterns and record them on the disk.

\item Study the location of the moons around Jupiter. Estimate as
accurately as possible the distance of each from the center in terms
of Jupiter diameters. Draw the moons around the smaller figure of
Jupiter provided on the observing sheet; the labelled tick marks are
in Jupiter diameters as a guide (write your actual estimate next to
each moon). Study the colors of the moons -- label those whose color
is distinctive. Note that some of the moons may be absent, behind
Jupiter or in its shadow. They may also just be visible passing
directly in front of the planet.
 

\end{itemize}


\newpage
\noindent
{\bf 3. Observation I}
\bigskip\bigskip
\noindent

Date: \makebox[2cm]{\hrulefill} \ \ 
Constellation: \makebox[2cm]{\hrulefill} \ \ 
RA: \makebox[2cm]{\hrulefill} \ \ 
Dec: \makebox[2cm]{\hrulefill} \ \ 

\bigskip
\bigskip
\bigskip
\begin{figure}[h]
\centerline{\psfig{figure={o4s_f1.eps},width=13.0cm}}

\vspace{1.0cm}

\centerline{\psfig{figure={o4s_f2.eps},width=3.5cm}}

\end{figure}

\bigskip\bigskip
\noindent
{\bf 4. Observation II}
\bigskip\bigskip
\noindent

Date: \makebox[2cm]{\hrulefill} \ \ 
Constellation: \makebox[2cm]{\hrulefill} \ \ 
RA: \makebox[2cm]{\hrulefill} \ \ 
Dec: \makebox[2cm]{\hrulefill} \ \ 

\bigskip
\bigskip
\bigskip
\begin{figure}[h]
\centerline{\psfig{figure={o4s_f1.eps},width=13.0cm}}

\vspace{1.0cm}

\centerline{\psfig{figure={o4s_f2.eps},width=3.5cm}}

\end{figure}

\end{document}











