
\noindent Today we will use the Palomar Observatory Sky Survey
plates to investigate the deep sky.  You have used these before in the
Milky Way lab.  Here we will take a somewhat more careful look. 

Recall that the images are negatives so the light of the stars etc. is
dark. The images come in pairs, labeled ``E'' and ``O'' in the small
rectangle in the upper left. The images labeled E are taken with a red
filter -- so the image records the red light. The ones labeled O are
taken with a blue filter, and so records the blue light. With
each pair of images we will supply a transparent overlay.

Pick one of the available pairs supplied by your instructors.  {\bf
Please treat them with great care: keep them flat; no pencils or pens
or food near them, and do not write on paper on top of the
prints. Thanks!}

\noindent
{\bf 1. Orienting yourself in the image:} 

\noindent Each image is labeled in the upper left by an approximate
R.A.~and Dec of the center of the image.  Note that because of
precession the R.A. and Dec we would use is somewhat different than it
was in the 1950s when the images were taken. Nevertheless (most of)
the stars aren't moving fast in angle relative to each other, so we
can determine what the current R.A. and Dec is in this image.

\noindent First, write down the approximate center listed on your
image:

\vspace{40pt}

\noindent To do so, pick the four (roughly) brightest stars in the
image. Make sure the four are fairly widely separated on the page.
Using the given R.A. and Dec as a starting point, use Starry Night to
identify those four stars. List below their names, R.A.s, Decs and
magnitudes. Also, draw their rough configuration in the image on the
blank page at the end of the lab.

\noindent Remember: in the images, with the rectangle in the upper
left, North (higher Dec) is up and East (higher R.A.) is to the left.

\clearpage

\noindent {\bf 2. Using the coordinate grid}

\noindent With either the red or blue images, place the transparent
grid you received on top of the image. Align it such that the
North-South line goes through approximately the center, and that the
declinations of your identified stars are correct. Use the marking
tape to identify where each star is, so you can put the transparency
back into place quickly.

\noindent Using the R.A.s of the stars, determine the approximate
spacing of the thick and thin lines of right ascension. If there are
two thin-line spacings list both.

\vspace{100pt}

\noindent What is the center R.A. line of your grid?

\vspace{100pt}

\noindent In Starry Night, find a tenth magnitude star in the
field. Using your RA and Dec grid to help you, locate it in on the
POSS image. Mark your map at the end of the lab with its name and
location.  

\noindent In Starry Night, zoom in around that star to a 15 arcmin
field-of-view.  Make sure you have All Sky Image on and no magnitude
limits for the stars. Compare how many the stars you see in the POSS
image to those in Starry Night. Which shows more (and fainter) stars?

\clearpage

\noindent {\bf 3. Finding galaxies and nebulae}

\noindent These images have been chosen because they contain one (or
more) interesting objects that are either distant galaxies or clusters
in our own galaxy. Find one of these, determine its RA and Dec, and
draw it roughly below:

\vspace{100pt}

\noindent Is it red or blue? How does its form differ between the two
images? 

\vspace{80pt}

\noindent Using the RA/Dec grid, estimate its rough angular size:

\vspace{40pt}

\noindent Look it up in Starry Night by looking at that RA and
Dec. What is the name and classification of the object? What is its
physical size?

