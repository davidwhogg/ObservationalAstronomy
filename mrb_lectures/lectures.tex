\documentclass[12pt, preprint]{aastex}
\usepackage{hyperref}
\hypersetup{
    colorlinks,
    citecolor=black,
    filecolor=black,
    linkcolor=black,
    urlcolor=black
}

\begin{document}

\newcommand{\degree}{\ensuremath{{}^\circ}}

\tableofcontents

\clearpage
\section{Lecture 1: The Celestial Sphere}

\subsection{Prologue}

In 1609, shortly after the invention of the telescope, Galileo Galilei
became the first person to point one at the night sky and record what
he saw. What he found led to a profound change in perspective
regarding the place of Earth and humanity in the universe. His
observations of the moons of Jupiter and the phases of Venus became
cornerstone of a ``new world system'' that placed the Sun at the
center of the Solar System and the Earth orbiting around it. He
unveiled the Milky Way---the band of light overhead visible in the
summer---as a collection of stars, revealing that there were not just
the few thousand stars visible to the naked eye, but countless numbers
of them. Since 1609, astronomers have built telescopes with greater
and greater care and with larger and larger sizes, and followed down
the path that Galileo's work pointed. We now know that the Solar
System is itself in orbit around the center of the Milky Way, which is
itself one of probably a few trillion galaxies in the observable
universe, containing probably $10^{23}$ stars, or about Avogadro's
number. The development of astronomy led directly to the development
of our understanding of fundamental physical laws, upon which all of
modern science rests.

In this course we are going to follow this path as well. You will
learn to observe the night sky. You will learn how to figure out what
objects will be visible at a given time on a given night at a given
location, and how to find it using its coordinates on the sky and with
respect to the constellation of stars within which it lies. You will
learn how to observe it with your eyes, with binoculars, and with a
telescope. The stars will be your familiar friends by the end of this
course and their patterns as recognizable to you as a map of the
continents. And you will have the skills to use binoculars and
telescopes to observe the fainter stars and other objects up there. 

In order to understand how to observe the night sky, you will need to
know about the motion of the Earth, the Moon, and the planets through
space. The rotation of the Earth creates night and day, and its orbit
around the Sun means that the stars visible at nighttime change
through the year. The phases of the Moon and the visibility of the
planets are determined by their motions around the Earth and Sun,
respectively. In this course, you will start to get a visceral sense
of these motions through space.

The paths of the Earth and other bodies are determined by the laws of
inertia and gravitation and are utterly predictable and inevitable and
unstoppable by any human action. It is an essential lesson in the
beautiful and frightening nature of physics, that we are at its
mercy. As much as the practical lessons we will learn, I want you to
remember this---that Nature and Physics will always be obeyed and for
humanity to survive we need to respect that.

\subsection{Practicalities}

Please carefully read the syllabus to understand how the course works.

We will have one lecture per week. This material cover what is
necessary for labs but also expand on that to understand more of the
underlying principles and the context. Your lab book has a list of
questions for you to complete following each lecture.

Starting next week, at the beginning of each lecture we will have a
short quiz. This quiz will be based on the lecture questions. In the
quiz, I will provide two questions, and your job will be to answer one
of them. Then we'll discuss the answer together.

The material regarding observing the sky is covered in the required
reading. Other background information I will cover in lecture. A very
good (but not required) book with a lot of this background information
is {\it Welcome to the Universe}.  Another good book (if you can find
it) which covers this material well and entertainingly is H. A. Rey's
{\it The Stars}; yes, the author who wrote Curious George.

The heart of the course is the lab; it will be unlike any other
physics lab you will have taken. We begin this week. You will be in
either the Monday or the Wednesday lab. You must stay in that lab,
because as you will find, the two labs will get out of synch.
Depending on the weather, we will conduct the labs either indoors or
outdoors. You won't know for sure until that evening, since we
sometimes cannot be sure about the weather. We will always meet in
Meyer 224.

Key things are:
\begin{itemize}
\item Be on time---if we are outdoors, you will not be able to get on
  the roof with us if you arrive late.
\item Be properly dressed for the weather, and warmer than you
  ordinarily would (see the lab notebook).
\item Bring your ID.
\end{itemize}

Finally, there are two sheets at the end of the lab, one about Polaris
\& Casseopia, and one about the Moon. You should fill them out {\it
  every week}.

There will be one midterm and one final. These exams will cover the
material in the lecture questions and the labs. The questions will be
similar to the lecture questions in style and complexity.

Any questions before we launch into astronomy?

\subsection{The Changing Sky}

The sky is filled with stars and planets and the Moon. Throughout the
night these move across the sky in unison.  But some objects move
independently, like the Moon and planets. And the positions of the
stars are not the same every night, but change throughout the year.

We need to learn how to interpret this view of the sky in terms of sky
charts. We'll need to learn the coordinates of this map, the
constellations, and the meaning of major features like the Ecliptic
and the Galactic plane.

In short we need to understand what is known as the Celestial Sphere
to make sense of what we see.

\subsection{Angles}

To begin with, we must learn the concept of an angle, because it is in
terms of angles that we determine the location of objects on the sky
with respect to one another.

Let's begin with simple geometry. There are 360 degrees in a circle,
so 90 degrees in a right angle. The degree is broken up into 60
arcminutes, and each arcminute is broken up into 60 arcseconds. 

In astronomy, sometimes we break up a circle into {\it hours} instead
of degrees; you will discover as we observe the sky why this is
convenient. There are 24 hours in a circle, and thus each hour is 15
degrees. Each hour is broken up into 60 minutes and each minute into
60 seconds. This means that each degree ($1/15$ of an hour) is 4
minutes, which is a useful thing to know.

It is important that you do not mix up {\it arcminutes} and {\it
  minutes}, or {\it arcseconds} and {\it seconds}. They are quite
different quantities.

When we refer to angles between objects on the sky, it tells us
something about the direction to those objects. Very often the {\it
  distance} to objects is extremely difficult to know. We will talk
about how we determine distances to very distant objects later. But
suffice it to say that much of the history of astronomy and cosmology
is one of finding better and better ways to measure this
difficult-to-pin down quantity of distance.

But if you do know the distance to two stars that are physically close
to each other both at the distance $D$, it is useful to understand
something called the {\it small-angle approximation}. In this
approximation, if the distance $D$ is very large relative to the
distance between the objects, $s$, then the triangle formed by the two
stars and the observer is almost a right triangle and we can write:
\begin{equation}
\label{eq:tantheta}
\tan\theta = \frac{s}{D}
\end{equation}
Furthermore, we can further approximate and find:
\begin{equation}
\label{eq:smallangle}
\frac{\theta}{60\degree} \approx \frac{s}{D}
\end{equation}
If you measure $\theta$ and know $D$, you can determine the size of
something by writing $s\approx D(\theta/60\degree)$; this is integral
for how we determine the sizes of the Moon, Sun, the planets, and many
other objects. If you know $s$ for some reason, you can figure out its
distance with $D\approx s(60\degree/\theta)$. 

You actually use this concept all of the time semi-consciously. For
example, even with one eye closed and therefore no stereoscopic
vision, you can tell how far away things are. When you see somebody
with your eyes, what your eyes actually {\it see} is the angular size
$\theta$ of who you see. You know how big humans are (around $s=5$
feet). That immediately translates into a distance $D$ that you can
guestimate. You of course aren't explicitly doing the math---but
something in your brain is doing it!

\subsection{The Celestial Sphere: RA and Dec}

We use angles all the time when we talk about positions on the
Earth. We can take a sphere like the Earth and cover it with a
grid. In one direction we divide the sphere with {\it great circles}
(any circle that would cut the sphere exactly in two equal
parts). Each great circle is a line of constant {\it longitude}. In
the perpendicular direction we divide the sphere with {\it small
  circles} (any other circle drawn on the sphere). Each of these small
circles is a line of constant {\it latitude}. Actually, though, the
Equator (line of latitude 0\degree is a great circle.  Once we have
done this, we can fully specify the position of any thing on the Earth
by its two coordinates: longitude and latitude.

We have some choices to make. First, which directions are the {\it
  poles}---the latitudes $90\degree$ and $-90\degree$ (or $90\degree$
North and South)? We choose these two be the along the axis of rotation
of the Earth, which is a sensible choice, but it principle we could
have picked different poles. Second, and more arbitrarily, where is
longitude $0\degree$? This location is known as the {\it prime
  meridian} and it really doesn't matter where it is. Humans have
decided to put the prime meridian at the longitude that corresponds to
the Greenwich Observatory outside of London, one of the greatest
observatories historically, and now one of the greatest science
museums in the world.

Once we have this coordinate system defined, we can make a flat map
with dimensions of latitude and longitude. The Mercator projection is
one example of such a map, but it is not the only one. All such
projections look distorted relative to what we would see in reality on
a sphere. The point is that we can draw on a flat piece of paper where
locations are on Earth with respect to each other.

In this system, New York City is about latitude 41\degree North and
longitude 74\degree West. We can also think of this as $360\degree -
74\degree = 286\degree$ East. You should think of East as the
``positive'' direction for longitude.

One of the first people to develop a system for mapping the world like
this was Ptolemy, who was a Greek astronomer who worked in the second
century of the Common Era (CE). Ptolemy's {\it Geography} contained
maps and coordinates (longitude and latitude) of features and cities
from what is currently Britain across the Mediterranean to
China. Although the original maps are lost, fifteenth century
reproductions based on the coordinates in the book give a sense of
what they may have looked like.

Not surprisingly, this system was also used by Ptolemy and his
predecessors to map the sky, and that system is known now as the {\it
  Celestial Sphere}. The coordinates we use are called Right Ascension
(like longitude) and Declination (like latitude). And just like
longitude and latitude, each star can be assigned a position in this
system.  This system is aligned with longitude and latitude, and like
longitude you should think of the ``positive'' right ascension
direction as East.  Unless they are moving in space relative to the
Sun, the coordinates of the stars do not change; now, the stars do in
fact move, but slowly enough that their coordinates change noticeably
only over many years (in most cases many thousands of years).

The units we use for these coordinates are different:
\begin{itemize}
\item Declination is usually expressed in degrees, arcminutes, and
  arcseconds.
\item Right Ascension is usually expressed in hours, minutes, and
  seconds.
\end{itemize}
Translating these units into decimal degrees can be done as
follows. For a declination of $D$ degrees, $M$ arcminutes, and $S$
arcseconds:
\begin{equation}
\label{eq:decdegunits}
\mathrm{Dec~in~deg} = D + \frac{M}{60} + \frac{S}{60\times 60}
\end{equation}
For a right ascension of $H$ hours, $M$ minutes, and $S$ seconds:
\begin{equation}
\label{eq:radegunits}
\mathrm{RA~in~deg} = \left(H+ \frac{M}{60} + \frac{S}{60\times
  60}\right)\times 15
\end{equation}

It is important to understand the relationship between your location
on Earth and the current time, and the right ascension and
declination. Imagine we are at some point on the Earth---which is easy
because we are! If we look directly up, that direction is called {\it
  zenith}.

Because the Celestial Sphere is aligned with longitude and latitude,
the declination of the zenith  direction always is equal to your
latitude. As the Earth turns, the stars rotate by, but the declination
of zenith doesn't change. It is this convenient fact that leads us to
align these two coordinate systems.

However, clearly the right ascension at zenith must be changing. In
fact, because the Earth turns once per day, the right ascension at
zenith will go through a full circle of 360\degree (or 24 hours of
angle) over the course of a day (as we will learn later, actually over
a tiny bit shorter than a day, 23h 56m, something called the sidereal
day). This reason is why it is useful to have right ascension in
hours, because the right ascension at zenith changes by about one hour
in angle per hour of time (a little faster than that).

But we have to choose some ``zero point'' in right ascension, just
like we did for longitude. There is no particular line of right
ascension that would more naturally be called $RA=0\degree$ than any
other. The arbitrary choice we make is as follows.

We choose a particular moment during the year, March 21 (the vernal
equinox) at 12:00 (noon) Universal Time (UT). What is UT? For our
purposes it is the standard time in Greenwich, England. We define the
right ascension $0\degree$ based on the direction of zenith at
longitude $0\degree$ at that time (i.e. the prime meridian).

From this fact, you can figure out the right ascension at zenith at
the moment at any other place. So in NYC, for example, we are at
around 284\degree East. So at 12:00 (noon) UT on March 21 (7am our
time) the zenith direction in NYC is $RA=284\degree\approx 19$
hours. As the Earth rotates, the RA at zenith increases (again at
about one hour per hour).

\subsection{Hour Angles and Dec}

RA and Dec are coordinates for stars that are the same for every
observer. But it is convenient to think about other coordinate systems
that depend on the location of the observer.

A convenient coordinate system that is similar to RA and Dec is {\it
  hour angle} and Declination. This coordinate system is aligned with
RA and Dec except that it is fixed with respect to the observer, and
the stars coordinate values change over time. The declinations remain
the same, but the hour angles change as the sky rotates by. HA$=0$
hours is the line connecting due North to due South and going through
zenith. This line is called the meridian.

\subsection{Time Above Horizon}

Stars will rise in the East and set in the West, and over this time
their hour angles will increase, again by about one hour per hour. The
point at which they rise above the horizon for a given observer, and
how long they will stay in the sky, depend on the declination of the
star and the latitude of the observer.

The diagram in the slides, which is appropriate for a northern
observer, shows how this works. As the Earth turns, from the viewpoint
of the observer, the star moves in a circle; which circle depends on
the declination of the star, but in every case it moves around the
circle for almost 24 hours. But the star will only be above the
horizon according to the observer when it is above the horizontal
circle (labeled NESW).

For example, a star rises with a negative HA hours, and sets at a
positive HA. Stars reach their highest point on the sky when the cross
the meridian. This highest moment is referred to as the {\it transit}
of the star.

If the star is on the Equator, that's the thick line.  The star rises
due East and sets due West, and half of the circle that it traverses
is above the horizon. That means it is above the horizon for 12
hours. Stars further north than the Equator have a larger fraction of
their circle above the horizon, so are above the horizon longer; the
opposite is true for stars further south.

When the stars are northern enough, they are above the horizon {\it
  all} of the time! These are called circumpolar stars. They are found
when Dec$>90\degree -$Lat (in the northern hemisphere) or when
Dec$<-90\degree -$Lat (in the southern hemisphere).

When the stars are southern enough, they {\it never} are above the
horizon.  They are found when Dec$<$Lat$-90\degree$ (in the northern
hemisphere) or when Dec$>$Lat$+90\degree$. (in the southern
hemisphere).

\subsection{Altitude and Azimuth}

A second convenient set of coordinates, that is fixed with respect to
the observer, are {\it altitude} and {\it azimuth}. These are also a
spherical coordinate system, but now with the pole defined towards
zenith.  The altitude is how far above the horizon a star is, and the
azimuth is the compass direction it is in.  The Alt$=90\degree$
direction is zenith, and Alt$=0\degree$ is towards the horizon;
negative altitudes are below the horizon. For a northern hemisphere
observer, the azimuth is defined to be $0\degree$ to the North,
$90\degree$ to the East, $180\degree$ to the South, and $270\degree$ to
the West.

Clearly as the sky rotates by, the altitude and azimuth of stars
change. For stars that rise and set, as they rise their altitude
increases, it reaches a maximum at the meridian, and then it
decreases. The azimuth changes in a more complicated way, depending on
the declination of the star and the latitude of the observer.

The altitude at transit is an important thing to know about a star,
because it tells you the highest the star will ever get in the sky
as seen from your location. It is determined by the declination of the
star and your latitude according to:
\begin{equation}
\label{eq:maxalt}
\mathrm{altitude~at~transit} = 90\degree - \left|\mathrm{Dec} -
\mathrm{Lat}\right|.
\end{equation}
So for example, a star on the Celestial Equator (Dec$=0\degree$) will
reach a maximum altitude of 49\degree as seen from New York City.

This equation is of course related to the equation above for which
stars never rise at a given latitude. If the altitude at transit is
negative, the object can never be observed. For example, from New York
City if the latitude is less than $49\degree-90\degree = - 41\degree$
then the star will never rise. In practice, the altitude actually will
have to be higher unless you are in a very flat area!

As we learn more about the rotation and orbit of the Earth, we will be
learning more on this subject and making use of these coordinate
systems. The first lab is entirely about the Celestial Sphere and many
of these issues will be clarified in that lab. The Edmunds Atlas is
also an excellent resource.

Next lecture we will learn about the basics of telescopes, the tools
we will be using in lab.

\clearpage

\section{Lecture 2: Telescopes}

Last time, I began the lecture by claiming that Galileo led a leap
forward in astronomy by pointing a telescope into the sky. Since that
time, humans have spent an enormous amount of effort building larger
and more sophisticated telescopes. Today the largest telescopes are a
pair of 10-meter diameter telescopes in Hawaii, the Keck
Telescopes. Several organizations across the world have plans to build
20--30 meter telescopes, and every few years we launch a new telescope
into space. The most famous of these is the Hubble Space Telescope,
launched over 30 years ago. In this lecture, I will describe what
telescopes are and how we use them, focusing on {\it optical}
telescopes, that is, telescopes which observe light at roughly the
wavelengths with which our eyes can see.

\subsection{How Professional Astronomers Use Telescopes}

In this course we will be using relatively small telescopes, of a type
suitable for amateur astronomy, but I want to first give you a sense
of what a professional astronomer does with telescopes. 

The first thing to realize is that we do not use telescopes to observe
the sky from New York City. Or even Buffalo, or any smaller town, or
actually anywhere East of the Mississippi in the U.S. You need to put
your telescope in a place that is dark (i.e. very few people), dry (so
it isn't often cloudy), and high (to reduce the effects of the
atmosphere). New Mexico and Arizona are great, and certain parts of
California. Chile is excellent, and Hawaii, and a few other isolated
spots around the world.

When we use these telescopes, typically we use them in one of two
modes. The first is what I do, which is {\it survey astronomy}. That
means that we make very large maps of objects in the sky: like a
comprehensive surveys of the stars in the Milky Way, or a large
three-dimensional map of the universe. In my project, the Sloan
Digital Sky Survey, we use two telescopes (one in New Mexico and the
other in Chile). I'm one of around a thousand astronomers involved in
the project. We decide what the telescope is going to do for about 5
year intervals, and then we have a dedicated team of observers who
live near the telescopes and execute the programs. The data streams
back to us and is stored in large servers, and we spend years
carefully analyzing the data in order to make our maps and then use
them to test our ideas about how the Milky Way formed and the history
of the universe. There are about 20--30 people who work at the
observatory---but hundreds who analyze and utilize the data.  And we
are still squeezing more information out of data we took 20 years ago.

The second is a more ``classical'' approach to astronomy. In this
case, an astronomer might fly to the observatory and work with the
team at the observatory for 5--6 nights to obtain the data you
want---basically pointing the telescope where you want it and
operating the cameras or other instruments that might be attached to
it. You might do this 2--3 times per year, and spend the rest of the
time analyzing the data. These days many observatories have adjusted
so you can do you your job as an astronomer remotely, which is more
convenient and is a smaller carbon footprint. Obviously space
telescopes must work this way!

But either way you do it---there's an immense amount of work {\it
  outside} of taking the data, because there is so much information
{\it inside} of the data. That is because telescopes are immensely
powerful at collecting this information, obviously much more powerful
than the human eye.

\subsection{What Makes Telescopes So Powerful?}

There are two properties of telescopes that are most important in
explaining why they are so powerful:
\begin{itemize}
\item They have a much larger {\it collecting area} than the eye, so
  can see fainter objects.
\item They can {\it magnify} an image so that finer details can be
  discerned. 
\end{itemize}

\subsection{Magnitudes}

To understand the brightness and colors of objects, it is useful to
learn how astronomers talk about brightness and colors. We use a
system of {\it magnitudes}. These arose from brightness
clsasifications originally used by the ancient Greek astronomers such
as Hipparcos. Each star was classified as magnitude 1, 2, 3, 4, or 5,
with 1 being the brightest and 5 being barely visible with the naked
eye. Once telescope came into use the system was extended to fainter
stars.

But this system was qualitative and relative---just a classification
based on a judgment of how the star looked. In the 1850s, an
astronomer studying asteroids, named Pogson, quantified this
system. By that time, it was understood that the ``brightness'' had to
do with the amount of energy per unit area per unit time, in the form
of light, was arriving at us from the star (or other object). Pogson
showed that the old system of magnitudes roughly obeyed:
\begin{equation}
\label{eq:mag}
\mathrm{mag} = - 2.5 \log_{10} \mathrm{flux} + \mathrm{constant}
\end{equation}
The minus sign means the brighter the star, the lower the
magnitude. Each magnitude difference of 2.5 is a factor of 10 in flux.

As an example, the star Sirius is the brightest in the sky and has
mag $\sim -1.5$. The very faintest visible stars with good eyes and a
dark site are mag $\sim 6$. So that's a difference of 7.5 mag, which
is three factors of ten or $10\times10\times 10= 1000$.

At a dark site, a pair of binoculars will get you to about 10th
magnitude or a bit fainter. A moderate sized telescope of about
4-meters can get to 25th magnitude. The Hubble Space Telescope can get
to 30th magnitude! That is 10 factors of 10 (or 10 billion times)
fainter than your eye can see.

\subsection{Collecting Area of Telescopes}

How do telescopes do this? Consider your own eye. It collects light
through its pupil. How much light it collects depends on the area of
the pupil. That's why in bright light, your eye automatically
contracts its pupil to reduce the incoming light, and in faint light,
your eye automatically dilates its pupil to increase it.

The area of a circle is related to its radius by $A=\pi r^2$, or to
its diameter by $A=\pi d^2/4$. If your pupil dilates by a factor of
two in diameter, you receive four times more light. If your pupil is 2
mm diameter, then a 2 meter telescope has a thousand times larger
diameter and therefore a thousand-squared, or a million, times more
collecting area.

\subsection{Magnification \& How Your Eye and Telescopes Focus Light}

To understand more about how telescopes work, it is useful to consider
how your eye works. As I noted before, there is an aperture (the
pupil) which lets light in. And on the back of your eye is the retina,
which is covered in special cells (rods and cones) that detect light
and report what they detect to your brain.

But if that is all there were, then the world would look very blurry
to you. Consider light coming from a star in some particular
direction. If the light just comes through the aperture, it makes a
circular image on the retina, and you would see that star as a big
circle. If there is another star in another direction, its light will
blend with the other star. You can see from the geometry of the image
in the lecture slides that even stars 20--30 degrees apart on the sky
would overlap considerably. This is how a pinhole camera works, if you
have ever made one of those; if you have, you know that to get good
resolution the distance from the pinhole to the image surface has to
be much much longer than the size of the pinhole. Obviously this is
not what your eye is doing!

Instead, your eye has a refracting lens that focuses the light. The
shape of the lines is carefully designed so that it focuses light from
different directions to different locations on the back of the
retina. That focusing power allows the eye to distinguish different
directions well.

In practice, an eye with good vision can ``resolve'' stars in two
different directions as close as a few arcminutes. To ``resolve''
the stars means they will look like two distinguishable stars instead
of just one. 

Telescopes work on the same principle. The first telescopes were
refracting telescopes that in fact worked very similarly, but
reflecting telescopes do the same thing. At the same time, the
telescope can be used to get a much higher resolution image. The
diagram of a simple refracting telescope shows how this works.

Light from very distant stars comes in as parallel rays. An objective
lens focuses the light, but then you let it diverge. Slightly further
down the optical path, you put in a converging eyepiece lens, and this
lens converts the light back into parallel rays. These rays then enter
your eye. However, if the focal length of the eyepiece is much shorter
than the focal length of the objective, this setup will converge fine
angular differences in the incoming light into larger angular
differences in the light delivered to your eye. This is what is known
as {\it magnification}.

The amount of magnification can be calculated from the focal length of
the objective and the focal length of the eyepiece:
\begin{equation}
\label{eq:magnification}
M = \frac{f_o}{f_e}
\end{equation}
So if we have an objective focal length of $f_o = 2000$ mm and an
eyepiece of 40 mm, then we have a magnification $M=50$. That means the
few arcminutes that your unaided eye can resolve is translated into a
few arcseconds on the sky. It is useful to increase that magnification further
sometimes, and in lab we have eyepieces with as small focal lengths as
8 mm, which would lead in this example to $M=250$.

You pay a price for high magnification, which is a reduced field of
view. The exit of the eyepiece along with your eye only admits a range
of angles spanning about 50\degree. This will be obvious to you in
looking through the telescopes, because you will see a distinct circle
with blackness around it. With magnification, the 50\degree your eye
percieves corresponds to a smaller angle on the sky. For example, at
$M=50$, through the telescope you only see a 1\degree patch of the
sky, and for higher magnification it is even smaller.

The other feature you will notice is that at high magnification the
blurring of the atmosphere and the instability of the telescope mount
will be obvious. The images of objects will look blurred and will
dance around the field of view somewhat, except under the best of
conditions. This takes some getting used to.

\subsection{Evolution from Refracting to Reflecting Telescopes}

The first telescopes were of the sort that Galileo used. Small tubes
with a pair of refracting lenses. You will make something like this in
one of the indoor labs; not an actually useful telescope but just the
optical setup. Over the course of the next three hundred years, people
became better at creating lenses of larger and larger sizes. These
larger sizes necessitated longer focal lengths, and thus longer tubes,
and a greater degree of engineering in order to manipulate precisely
the pointing of the telescope and keep it stable during observations.
Most astronomical work was done with refractors over the next 200--300
years.

By the late 1800s a problem was emerging regarding refracting
telescopes. The largest refractor ever built is the one in Yerkes
Observatory in Wisconsin, and its objective lens is about 1 meter in
diameter. Because of the great length of the tube, it needs to sit
inside an enormous dome; the mechanical system and the dome alone
become expensive because of their size. But the real problem is
constructing the lens. To build a lens, glass is melted and formed
into a rough blank, which is cooled until it is solid, then ground
down to a precise shape, and finally polished. A high quality lens
should have few imperfections, such as bubbles or cracks. As it cools,
if imperfections form they can lead to catastrophic cracks. Since
glass expands and contracts as it heats and cools, this meant that the
glass had to be carefully cooled slowly and evenly, or else the fact
that part of the glass was contracting differently would lead to
cracks forming. The cooling process is therefore is weeks or longer
for a large piece of glass. This makes the production of a large lens
extremely tricky and for the largest telescopes often multiple tries
were taken to build their largest pieces, and only a handful of firms
in the world could or would attempt to make them. For the largest
ones, only a single firm, Charles Feil's, would take them on---and it
was generally a great financial risk, since producing a single piece
of glass could take as much as eighteen tries.  By the late 1800s, the
people who made these lenses were reaching their limit---nobody could
made a quality lens greater than 1 meter in diameter.

In the late 1800s, ``glass'' meant literally glass. In the 1920s,
Pyrex and similar materials were developed for the purpose of
household goods. These new materials expanded and contracted much less
than regular glass. For baking, that is great because if you (for
example) pour cold water onto a hot pie dish made of Pyrex, it will
not shatter, whereas a glass one might. Yet even with these advances,
and improvements over the last hundred years in material science,
today the effective limit on the size of a lens that can be produced
with a finite amount of time and effort is still not much more than a
meter.

Thus it became necessary around 1900, if astronomers were to build
telescopes with a greater collecting area, to use a different
design. From the beginning, another design was available: the
reflecting telescope.  Newton designed a reflecting telescope, and
much of William and Caroline Herschel's work on ``nebulae'' (which
turned out to be distant galaxies) used reflecting telescopes. In the
slide I show an example of the simplest design, which is a parabolic
mirror, with a flat mirror that redirects the focus to the side. We
have some telescopes of this design that I will show you in lab.

Why is it easier to make a large reflecting telescope? For a lens you
need a large, thick piece of glass which is free of imperfections
throughout its body. To make a mirror, you can make the optical
surface itself much thinner; you need its surface to be smooth but you
do not care whether its interior is completely transparent. So
typically a reflecting telescope's mirror is thin, with a supporting
structure behind it or within it. It is the same principle as behind
an I-beam, which gives much stronger support than a rectangular beam
of the same weight would.

The reflecting telescope has some other advantages as well:
\begin{itemize}
\item No {\it chromatic aberration}. A lens bends light, but typically
  bends light of different wavelengths in different ways. Blue light
  is bent more than red light. This means that blue light focuses in a
  different place than red light. This leads to distorted-looking
  images. Mirrors reflect all light the same way and do not suffer
  this problem.
\item Smaller telescopes in length. Because the light is reflected
  back, and through often the use of multiple mirrors, a long focal
  length can be fit into a smaller length. This reduces the size of
  the dome considerably and reduces the engineering challenges
  associated with building the telescope. 
\end{itemize}

Note that both reflecting and refracting telescopes are characterized
primarily by an aperture size and a focal length, and that the
function of the eyepiece remains the same in both cases.

\subsection{Equatorial vs. Alt-Az Telescopes}

There is another watershed between types of telescopes, also generally
dividing older telescopes from more modern ones. This time the
division is between telescopes designed in the computer age (1980s or
later) versus earlier.

The difference has to do with how the telescope is designed to
point. The telescope is suspended from some support. You need to be
able to maneuver the telescope to point to any direction. That means
that you must allow it to rotate around that support in at least two
axes.

Now, if you point at a star with a telescope and just leave the
telescope alone, you will see the star ``drift'' out of the
field. This is because the Earth rotates, and thus the sky drifts by
(i.e. the stars rise and set).

Older telescopes were essentially always designed as {\it equatorial}
telescopes. These are designed to make it easy for them to track stars
as they traverse the night sky because of the turning of the
Earth. The concept is straightforward. If the Earth is turning in one
direction, the telescope has to turn in the opposing direction in
order to keep pointing at the same object. An easy way to do this is
to allow the telescope to rotate about the same axis.  So you make one
axis parallel to the axis of rotation of the Earth. As you rotate this
axis, you are rotating the telescope East and West in hour angle or
right ascension.  Then you make the other axis correspond to
declination.

Imagine now that you point at a star. If you want the telescope to
keep pointing at that star, you just need to rotate one axis of the
telescope (the right ascension axis) at the same rate as the Earth
(but the opposite direction). With our telescopes, that just means
pressing a button on a motor which just does one thing---drive that
one axis at that single speed. Before electric motors this was
trickier, but is still pretty simple mechanically. 

With professional telescopes, they are on fixed mounts where the right
ascension axis is always pointing in the same direction as the Earth's
axis of rotation. In our case, we will need to align our telescopes to
this axis at the beginning of each lab. Luckily there is a relatively
simpler procedure to do so. The goal of the procedure is to align the
telescope so that when the telescope is pointing along that axis of
rotation we see Polaris in the center of the field. Polaris is near
declination of 90\degree (actually $+$89\degree $15'$); i.e. at the
North pole it is directly overhead. So this means the axis of rotation
of Earth points at Polaris. And so if our telescopic axis is pointing
to Polaris, it is pointing within a degree of the axis of rotation of
the Earth.

As you will see in lab, an equatorial mount means tilting the
telescope in a somewhat unnatural fashion, at an odd angle relative to
the ground. At the Equator, you would need the telescope to be able to
rotate around an axis perpendicular to the ground, like a pig roasting
on a spit! As telescopes get bigger, this arrangement becomes more and
more mechanically difficult.

A much easier mechanical arrangement is the {\it alt-az} telescope,
where the telescope rotates around two axes, one of which alters
azimuth (just rotates the compass direction, so rotates around
zenith), and the other of which controls the altitude from the
horizon.  This is much simpler to control.

However, if you imagine trying to track a single star as it goes
across the sky, it is changing both in altitude and azimuth, and
neither at a constant rate, and so you need to constantly be adjusting
the speed of rotation of both axes to track a star. Figuring out how
much is a straightforward calculation, but until the 1970s or 1980s
wasn't something easy to do in real-time, because you needed a
computer to do the calculation and to control the motors. Affordable
high speed computers therefore made it possible to design alt-az
telescopes, and of course today you probably are carrying multiple
computers on your person that would be capable of this calculation.

In short, equatorial telescopes have mounts that control the hour
angle and declination; they make tracking simple but are mechanically
less convenient. And alt-az telescopes have mounts that control
altitude and azimuth; they make tracking complicated but are
mechanically convenient, especially for large telescopes. Either type
of telescope needs to be carefully calibrated before use in order to
control its pointing properly.

A couple of a brief asides. First, it sometimes can be convenient with
an equatorial telescope to roughly align it. If you are just observing
by eye, you may not need the telescope to track perfectly. Roughly
pointing the telescope towards Polaris within 5--10 degrees isn't so
hard, and can often suffice to reduce the drift of objects out of the
field of view to a tolerable level. For example, in 2017, in observing
the solar eclipse (with a specially designed telescope!! do not do
this with a regular telescope) I had to do the alignment with a
compass---I couldn't find Polaris during the day, and the solar
telescope wouldn't have been able to see it---and it worked well
enough.

Second, if you were to buy a telescope like the ones we use in lab,
you would have the option to buy an equatorial wedge or not. If you
did, you could use the telescope as an equatorial. Without the wedge,
you can only use it as an alt-az. But the telescopes these days always
come with a tracking system. You don't even have to level the
telescope well; you just point the telescope at three different stars,
and the tracking system will determine exactly how the telescope is
oriented, and it will be able to not just track but also to point to
any RA and Dec you would like on command. This is pretty awesome and I
would recommend using that system if you buy a telescope; but we won't
be using it here! We'll be doing things a bit more old school.

\subsection{Detectors}

Professional telescopes (and fancy amateur telescopes) have one other
feature that enhances the utility of a telescope relative to the human
eye, and those are detectors.

Your eye collects light, but it really only collects it for a fraction
of second before reporting back to your brain.  You would be able to
see much fainter things if your eye would collect light for a long
time, say an hour, and then it reported what it saw. You would be able
to see things thousands of times fainter. It would also be nice if you
could store what you saw in a precise way that could be shared with
others and quantitatively examined. Of course, astronomers have always
drawn pictures of what they've seen, but these sketches are hard to
measure precisely.

An enormous advance was made in the mid-1800s with the development of
photography. This advance allowed light to be collected over a long
period of time. Photographic emulsions are much less efficient at
collecting light than your eye---your eye catches nearly 100\% of the
light, but emulsion only about 5\%---but you can observe the sky for
many minutes, up to an hour or so at a time. This allows much fainter
objects to be seen. Furthermore, it allows the images to be
investigated at your leisure later with greater quantitative
precision.  Henry Draper, a doctor at the NYU School of Medecine in
the 1870s, was one of the early practicioners of astronomical
photography. His father took famous early photographs of the Moon, and
Draper operated a telescope on Worth Street and one a bit north of the
city in Hastings-on-Hudson (the building is still there, but not the
observatory). After Draper died relatively young, his wife used his
fortune to fund astronomical research using photography (at Harvard!
nobody else was doing astronomy at NYU at the time). We will pick up
this story later when we talk about stars and galaxies, but for now it
is enough to say that they created a great library of images that
after decades of study allowed astronomers to understand the
fundamentals of what stars our, how old and large the Galaxy and the
Universe it, and many other more abstruse things.

Today we use silicon detectors akin to the detectors you would find
in the camera on your phone, that are nearly 100\% efficient and
record infrormation in a convenient quantitative way. Again, you can
purchase these for an amateur telescope but it is a more difficult and
involved adventure to use them!

\subsection{Colors}

You may wonder how astronomical detectors can record the color of the
light that the telescope collects.

The ``color'' of something reflects its ``spectrum.'' Light is a type
of wave, and light to your eye has a very small wavelength---0.4 to
0.7 microns (one micron is one millionth of a meter). Smaller
wavelengths are ``blue'' and longer ones are ``red.''

Recall that your eye's retinae have 4 types of cells that detect
light---three types of cones, plus the rods. The cones are most color
sensitive, and they roughly correspond to red (long wavelengyth),
green (medium wavelength), and blue (short wavelength). When you see
something red that means that your red cone cells are responding more
than the blue ones; i.e. there is more light at 0.7 microns than at
0.4 microns.

(If you think about it very hard you will realize why a mixture of pure
red, pure green, and pure blue light can mimic (almost) any color).

In astronomical imaging, you can do something similar to what your eye
does by inserting filters that only let in light of certain
colors. Alternatively, you can divide each pixel into red, green, and
blue sensitive parts (this is basically what a digital camera
does). Either way, the final ``color'' images that you would see in a
magazine are made by combining these images in different wavelength
ranges.

The advantage of doing this with detectors is that you can use
detectors that are sensitive to light {\it outside} of 0.4--0.7
microns---light that is invisible to your eye!

\subsection{Professional Telescopes}

Professional optical telescopes today come in a number of
varieties. Some refracting telescopes are still in use today, usually
small ones, prized for their optical quality and lack of diffraction
spikes. Anything 1-meter or larger is a reflecting telescope. We
consider a 3--4 meter telescope to be ``moderate'' in size today, and
the largest ones in existence are 10-meter (the Keck Telescopes, for
example). In about 10 years we can expect telescopes of 20--30 meters
in size.

We also have telescopes in space, most famously the Hubble Space
Telescope. This is only 2.5 meters in diameter. Its great advantage is
that it is above the blurring of Earth's atmosphere and so can create
extremely sharp images. Within the next couple of years, the James
Webb Space Telescope will be launched, which will be 6.5 meters in
size.

Finally, there are telescopes that work at other wavelengths than the
optical---in fact, all the way from X-rays (extremely small wavelength
light) to radio waves. The principles upon which these telescopes are
designed are the same as those of optical telescopes, but in practice
they work quite differently and look quite different.  Unfortunately
we cannot explore all of these interesting difference!

% She funded Edward Pickering's lab, which had telescopes in Cambridge
% and later in Peru. Pickering is famous for having a large number of
% female assistants who analyzed the photographs, among them over the
% years Wilhelmina Fleming, Henrietta Leavitt, Antonia Maury, and many
% other now-famous names. At Cambridge, they created a great library
% of photographic images that they spent decades understanding. What
% emerged from those studies was a fundamental re

\clearpage

\section{Lecture 3: Rotation and Orbit of the Earth}

We will now return to some of the topics covered in the first lecture,
related to the Celestial Sphere. Specifically, we will discuss the
rotation of the Earth about its axis and the roughly circular orbit of
the Earth around the Sun. These motions largely---but as we will learn
later in the semester not completely---describe how the Earth moves in
space.

\subsection{Systems of Time}

Before beginning, we should talk a little bit about systems of
time. The most natural system of local time is {\it local mean time}.
This time system is the one in which the Sun transits at 12:00pm noon,
and tomorrow also at noon, etc. But this time is particular to your
longitude. The Sun transits in New York City a few minutes after it
transits in New Haven. And it would be annoying to have the time in
New York City differ by a few minutes from other local cities.

For this reason we define {\it local standard time}, which is defined
differently in different time zones by when the Sun transits at some
particular longitude within the time zone. In most cases this means
that local mean time and local standard time differ by at most 30
minutes (though time zones aren't defined strictly by longitude, so
this isn't always true). This difference is small enough that we can
effectively ignore it for this course---and New York City's local mean
time and local standard time only differ by a few minutes in any case.

Local time is inconvenient if we want to specify when some
astronomical event occurred or will occur, because it means something
different to each observer. Astronomers therefore usually specify time
in Universal Time or ``UT'', which is defined to be local mean tie at
Greenwich Observatory. So New York's local standard time is 5 hours
behind Universal Time.

Of course, most of the year (March through November) your clocks are
not set to any standard time. Instead they are set to Daylight Savings
Time, which is one hour ahead of standard time. Do not ask me why we
change the clocks twice a year---I honestly do not know. But the
consequence is that in daylight savings, New York's local dayligh time
is 4 hours behind UT.

\subsection{The Rotation and Orbit}

The Earth rotates about its axis (the line connecting the North and
South Poles). This creates the day-night cycle.

At the same time it orbits the Sun once per year. The orbit describes
(within about 1\%) a near-perfect circle, with the same sense as the
rotation (i.e.  counterclockwise in my diagrams, which view the Earth
from above the North Pole). The plane containing the circle of the
Earth's motion is known as the {\it Ecliptic Plane}.

But the rotation axis of the Earth is ``tilted'' with respect to the
orbital plane, by 23.5\degree, called the {\it Ecliptic Angle}.

The last thing to remember is that most objects on the sky---stars and
galaxies---are {\it much} further away than the size of the Earth's
orbit. The size of the Earth's orbit is an Astronomical Unit, or
150 million kilometers. Even the closest stars are almost 100,000 AU
away, and most of the visible ones are tens or hundreds of times
further than that even. 

These facts seem very simple but they have extremely important
consequences to us---both as astronomers and as humans.

\subsection{The Sidereal Day}

The period of rotation is not 24 hours, as you might think. It is 23
hours and 56 minutes, and this period is known as the {\it sidereal
  day}. If I sit on the Earth at a particular moment and point at
zenith, I am pointing at some direction in space. But the Earth is
rotating so that zenith will not point in that direction for long.
The amount of time for the Earth to rotate around all the way so that
zenith points in that direction again is the sidereal day.

We know that the {\it solar day} is 24 hours---the time between
today's noon and tomorrow's noon; that's just how the hour is defined.
Why do the solar and sidereal days differ?  Let's say I start at noon
and point towards zenith. I'll be pointing at the Sun. The Earth
rotates around one sidereal day, so that I am pointing in the same
direction again. But at the same time the Earth has moved in its
orbit. The zenith in 23 hours and 56 minutes will not be pointing at
the Sun. Because the Earth is orbiting in the same sense as its
rotation (i.e. in my sketches they are both counterclockwise), it has
to rotate a little more in order for Zenith to point at the Sun---just
about 4 minutes more. Another way to say it is that 1 sidereal day
later, it will be 11:56am; the Sun will not quite have reached its
apex.

Because we define days in terms of the Sun (i.e. the day-night cycle)
and not the stars, in everyday life we use the solar day. But the
sidereal day is the {\it actual} rotation period of the Sun (meaning
(for the physics majors in the class) the rotation period in an
inertial reference frame).

\subsection{Local Sidereal Time}

A consequence of the combination of the rotation of the Earth and its
orbit is that the stars that are visible at night change throughout
the year. As the Earth rotates, the stars rise and set. If you pick
local midnight, or any given local standard time, and look at what
stars are at zenith, over the course of a solar day the stars will
return to where they were and then move a bit farther for the last
four minutes. So if we return tomorrow night at the same time, the
stars will have rotated a bit further; over the course of the year
this means the visible sky at any given time will change. When a year
has passed, then the Earth will return to the original place in its
orbit that you started, and you will see the same sky again. Then the
cycle repeats.

There is a very convenient definition of time for astronomers that
allows us to keep track of what is visible, called {\it local sidereal
  time} or {\it LST}. Local sidereal time has 24 ``hours'' in one
sidereal day; thus the sidereal hours are just a bit shorter than
regular hours. The LST is defined as the right ascension that is
transiting at any given moment. Therefore it tells us exactly what
should be up in the sky at any given moment.

If we know the LST, we can find the hour angle of any given star whose
coordinates we know. Clearly for stars whose RA is equal to the
current LST, their HA is zero (they are transiting). And the HA should
increase as LST increases, since the stars are rising and
setting. Finally, it should repeat on the same cycle as LST. So we
have:
\begin{equation}
\label{eq:ha}
\mathrm{HA} = \mathrm{LST} - \mathrm{RA}
\end{equation}

It should be revealed now why we often use hours to express RA, since
the RA currently transiting corresponds to a definition of ``time,''
the LST. And, in fact, the RA transiting does change at just about one
hour per hour---just a little bit faster than than.

In fact, we can think of (and you may have already seen in lab) the
area of sky around the pole as an LST clock. If you look North from
our lab area, the diagrams in my slides show what you will see, but
instead of the actual sky I am showing the Edmunds Atlas. The right
ascension line going vertically (along the sky, towards zenith) is the
current LST.

The way these diagrams are written is that for 8pm local standard time
they list the day of the year for which LST$=$RA. That is, if you go
outside at some night at 8pm standard time, the stars transiting will
be the ones at the RA associated with that night.

Then, as the night progresses, the LST increases at about one hour per
hour, and so the RA that is transiting will increase. On the chart you
can see that that means that looking North the sky rotates
counterclockwise.

\subsection{Visibility Across the Year}

We now have the language to figure out what RA and Decs are visible
from any point on Earth, at any time of night, and any night of the
year.

Let us first consider the Earth at some particular time, let's say
late June. In this case, the Earth's North Pole is tipped as far as it
can be towards the Sun. At any point on Earth, night-time in late June
is always going to be a time when your part of the Earth is on the far
side away from the Sun. So the visible stars will be those that are on
that same side of the sky---the stars in the same direction of the Sun
are those up during the day, and you won't be able to see them.

As the Earth orbits the Sun, its North Pole still points in the same
direction. Three months later in late September, the night time stars
will be in a different direction. The pattern continues until the
following June. In each season the night time stars will be in
different directions than the other seasons.

We need to relate these directions to the right ascensions of the
visible stars. I have marked my diagram with right ascension, and you
can relate the way I've marked it to the definition of right ascension
in the first lecture. Recall that the RA=0\degree corresponds to the
direction of zenith at noon UT on the vernal equinox (March 21). That
is the upwards vertical direction in my diagram, because that is the
direction of the Sun as seen from Earth on March 21; and so also the
direction with the same RA as zenith at noon on that day. Then RA
increases eastward; since we are looking at the Earth from above the
North Pole, that means counterclockwise on this diagram.

In terms of the directions in RA visible at night throughout the year,
if RA$=$0h transits at noon on March 21, then RA$=$12h transits at
midnight on March 21. So the visible RAs will be those surrounding
RA$=$12h. The RA transiting at midnight increases throughout the year,
and must increase by 24 hours across the whole year (so as to repeat
back to 12h the following March).  So it increases at about 2h per
month, or 4 minutes per day---i.e. exactly by the difference between
the sidereal and solar day.  In June, RA$=$18h transits at midnight;
in September, RA$=$0h transits at midnight; and in December, RA$=$6h
transits at midnight.

\subsection{The Seasons}

The relative orientation of the Earth's rotation and its orbit are
responsible for the seasons. Thinking about it very simply, in June
the Earth's northern hemisphere is tilted towards the Sun, and in
December it is tilted away from the Sun. That makes the days longer in
summer and shorter in winter, and the Sun higher in the sky in summer
than in winter, and thus leads to a variation in temperature across
the year. Then the opposite is true in the South.

More technically, consider the position of the Sun above the Earth
relative to its Equator as a function of time of year. In June,
clearly the Sun sits above a Northern latitude---in fact it will be at
Zenith at a latitude of $+23.5\degree$, the ecliptic angle. That means
the declination of the Sun in the sky is $+23.5\degree$ at that time.
But in December, clearly the Sun sits above a Southern latitude; and
in fact the declination of the Sun in the sky is $-23.5\degree$ at
that time. Therefore throughout the year the Sun's declination varies.

The Sun's right ascension varies too. This is because the changing
position of the Earth changes which stars are behind the Sun from our
point of view. So on March 21 the Sun, which we know must transit at
LST$=$0h that day, is at RA$=$0h as it must be, and its right
ascension grows, at 2h per month so that it goes through a full circle
in 12 months (i.e. $12\times 2$h$=24$h$=360\degree$).

We can plot the Sun's path through the sky on an RA and Dec chart. It
moves from East to West on the sky (right to left if North is up). Its
points of highest and lowest declination are known as the soltices
(summer and winter), and the times its declination is zero are known
as the equinoxes (vernal and autumnal).

This path is known as the Ecliptic, because it is where any object in
the Ecliptic plane will appear. For this reason, the Moon and the
other planets will always appear near the Ecliptic, since they all
orbit the Sun in the same plane that the Earth does.

Just like a star of a given RA and Dec, we can deduce the time the Sun
spends above the horizon and the maximum altitude which it
reaches. These quantities are what set the seasons and to a large
degree the climate at different latitudes on Earth. 

Consider an observer in the Northern hemisphere. Just as for a star,
if the Sun has a northern declination (i.e., greater than 0\degree),
then it will stay above the horizon for more than 12 hours. If the Sun
has a declination of 0\degree, it will stay above the horizon for
exactly 12 hours. If the Sun has a southern declination (i.e. less
than 0\degree), then it will stay above the horizon for less than 12
hours.

At the equinoxes, the Sun has a declination of 0\degree, and so the
day is exactly 12 hours. At the (northern) summer solstice, it has a
declination of $+23.5\degree$ and therefore is above the horizon more
than 12 hours, and daytime is the longest on that day. At the
(northern) winter solstice, it has a declination of $-23.5\degree$ and
therefore is above the horizon less than 12 hours, and daytime is the
shortest on that day. So that explains how the length of the day
varies throughout the year.

At the same time, the Sun reaches different maximum altitudes in the
sky depending on one's latitude and the Sun's declination on that
given day. Recall Equation \ref{eq:maxalt} from the first lecture, and
you will find that if the Sun's declination equals your latitude, the
maximum altitude is 90\degree---i.e. the Sun reaches zenith. For
latitudes greater than $+23.5\degree$ or less than $-23.5\degree$, the
closer to summer solstice it is in that hemisphere, the higher in the
sky the Sun will reach.

If the Sun is high in the sky that means it is shining directly down
on the Earth and its light is spread over the minimum amount of
area. This means that the Earth absorbs more energy per unit
area.

The combination of these two effects---the changing length of the day
and the changing maximum altitude of the Sun---combine to create the
season as we know them. It might seem strange that the coldest months
{\it follow} the winter solstice instead of being right around them,
but that is just because the overall temperature lags the driving
force of these effects a bit---the climate takes a month or two to
respond.

\subsection{Climatic Zones on the Earth}

The climate differs as a function of latitude on the Earth, for
reasons that can be deduced from following the arguments in the last
section.

First, consider the region above latitude $+66.5\degree$ or below
latitude $-66.5\degree$---known as the Arctic and Antarctic
regions. In these regions, the extent of the circumpolar declinations
is very large. In the extreme case, at the poles, {\it all} the
visible stars are circumpolar. As noted above, the circumpolar region
is defined by Dec$>90\degree-$Lat (for northern latitudes) or
Dec$>-90\degree-$Lat (for southern latitudes). That means that in the
Arctic the circumpolar region includes declinations below
Dec$=23.5\degree$; since the Sun reaches these declinations near the
summer solstice, the Sun will be circumpolar! That is, it will never
set. And near the winter solstice, the maximum altitude of the Sun
will be negative, so it will never rise! A similar thing holds in the
Antarctic, but at the opposite times of year. The Arctic and Antarctic
regions therefore are defined to be those regions which experience
times of endless night or endless day.

Because of its dry and dark conditions, there is actually an
astronomical observatory at the South Pole. For example, a set of
radio telescopes there maps the cosmic microwave background, which is
light that comes from so far away that it was emitted when the
universe was only 400,000 years old (it is 13.5 billion years old
now).  To study this light, astronomers and technicians stay at the
observatory for more than six months at a time. ``Night'' begins near
the vernal equinox (March 21) and the Sun does not rise again until
September. In between it is extremely difficult to get in and out of
the South Pole base, so the team there stays confined that entire
period. It is an extraordinary commitment.

Second, consider the region between latitudes $-23.5\degree$ and
$23.5\degree$, called the Tropics. These are regions for which the
Sun's declination can be equal to one's latitude, so it can pass
through zenith. Furthermore, it includes the Equator, for which the
day is always exactly 12 hours, and in this region the day's length
varies relatively little. These facts mean that the Tropics are
generally characterized by hot weather and little seasonal variation.

In between these regions are the Temperate Zones, which we are in, and
which we usually characterize as having four seasons with fairly large
variations.

\clearpage
\section{Lecture 4: Finding Objects on the Sky}

In the lectures so far, we have learned about coordinate systems and
about the motion of the Earth. This information is enough for us to be
able to figure out whether a star is observable from a given location
and when. We have also learned a little bit about telescopes, and with
this knowledge we can figure out a strategy to point the telescope at
that star. Or cluster, or galaxy, or whatever. Now, you can also use
software like Starry Night, or more professionally-oriented software.
But a few principles actually suffice and are in fact useful things
that professional and amateur astronomers carry around in their heads.

There are a few basic things you want to consider about an object
before trying to observe it:
\begin{itemize}
  \item How bright is it? Is it bright enough to see with your
    instrument (eyes, binoculars, telescope) at whatever site you are
    at?
  \item How high in the sky does it get at your latitude
    (what is its {\it altitude at transit})? That determines how easy
    it will be to observe in the most optimistic case.
  \item When is it observable---what part of the night (if any)? That
    is, when is it within a few hours of transit?
  \item What strategy should I use to find it with the telescope? 
\end{itemize}

\subsection{Magnitude}

For stars, the magnitude is really the key guide as to
observability. Magnitudes less than 5 (or 6 in very dark sites) are
necessary to be visible by eye---but note that {\it finding} a
specific 5th magnitude star will still be hard because there are a lot
of them. In New York City, actually the limit is more like 3 or at
best 4. You will need to keep this in mind when choosing stars that
are good starting points for finding a target with your telescope.

If you assume your pupil is about 5mm diameter, then 50mm binoculars
will be able to look deeper by a factor of $(50/5)^2 = 100$. This
factor of 100 is 5 magnitudes. In New York City this is about
magnitude 8. Though we should be able to do better because the
magnification reduces the amount of sky background competing with the
object. Anyway, this gives you a sense of what should be visible with
modest binoculars; somewhere between about 8--10.

A 10-inch telescope is 254mm. This is 50 times 5mm, so a factor of
2500, which works out to be about 8.5 magnitudes. So this should bring
us to about 11.5--13.5.

For objects like galaxies and nebulae, we do not do as well, because
they are more easily hidden in the bright sky background compared to a
star with the same total magnitude.

These calculations give a starting point. But I'm an optimist so as
long as something isn't ridiculously faint, I always think it is worth
trying to find objects and seeing whether you can see them!

\subsection{Altitude at Transit}

Objects are most easily observed at transit (HA$=0$h). The further
they are from transit, the closer to the horizon they are and the
brighter the sky background will be. So we should check the altitude
at transit because that will be the ``best case scenario'' for
observing the object.

Recall the calculation:
\begin{equation}
\label{eq:maxalt2}
\mathrm{altitude~at~transit} = 90\degree - \left|\mathrm{Dec} -
\mathrm{Lat}\right|.
\end{equation}

If the transit occurs at a negative altitude, then this star will
never be visible!

\subsection{Time of Transit}

Now the most complicated part of this calculation is calculating what
time on a given night the object will transit. We can calculate this
just on the basis of the RA of the object.

To calculate this, it is helpful to first figure out what day of the
year the object transits at midnight. This can be deduced by
considering the Earth's orbit and the definition of RA. We know that:
\begin{itemize}
\item on March 21 the RA$=12$h transits at midnight local mean
  time
  \item On June 20, RA$=18$h transits at midnight local mean time.
  \item On Sept 23, RA$=0$h transits at midnight local mean time.
  \item On Dec 21, RA$=6$h transits at midnight local mean time.
\end{itemize}
Recall that local mean time is usually $\pm 30$ minutes from local
standard time, and in this course that is sufficiently close. Also
note that these statements are independent of where you are on the
Earth!

If we want to determine, for a given night, at what time a given star
transits, it is best to start from the nearest of these four dates and
then deduce what RA transits at midnight on our given night.  As our
example, we will consider the problem of determining at what time the
Great Nebula in Orion (M42), with an RA of 5h 30m, transits on January
31.  I can start by knowing that on Dec 21, RA$=6$h transits at local
midnight. The RA transiting at a given time increases by 2h each month
and 4m per day. Therefore, on January 31 (one month and 10 days
later), the RA transiting at midnight is 8h 40m.

Once we know what RA transits at local midnight, we can figure out
what time our given RA transits. We can use the fact that the
transiting RA increases at (close to) 1 hour of angle per hour of
time. So our desired object with an RA of 5h 30m transits 3h~10m
earlier than midnight, or around 8:50pm local standard time. That will
be the best time to observe M42---though it will remain relatively
high in the sky for the few hours before and after that.

We can compare this to the Edmund's Atlas. The dial on that atlas
tells us that on January 31, the RA$=4$h~40m transits at 8pm local
mean time. Therefore, 50 minutes later at 8:50pm the RA transiting
will be 5h~30m.It all agrees!

Another way of thinking about this is that we are seeking the time
that the hour angle HA$=0$h. Since HA$=$LST$-$RA, that means we are
seeking the date and time at which the local sidereal time
LST$=$RA. So we have determined that at 8:50pm on January 31, the LST
is 5h~30m.

What would have happened if we had chosen an RA (say) $\approx$
17h~40m and asked when that would have transited? Then we would have
found that it transited 9 hours after midnight, or 9am. Our conclusion
would then be that this star transited during the day and therefore
would be difficult to observe! It would be rising in the hours prior
to dawn, so might be observable at that time depending on the
declination and exactly when sunrise was at our latitude.

Another sort of question we might ask is on what day of the year a
star transits at a specific time chosen for our convenience. The 8pm
standard time used by the Edmunds Atlas is such a convenient time---it
will be usually after sunset but before bedtime (at least my
bedtime)---so let's use that. Consider the globular cluster M13 at
about RA$=16$h~40m; on what night does it transit at 8pm local
standard time? In this case, we start by realizing that the RA
transiting at 8pm on each solstice and equinox is 4 hours less than
that transiting at midnight---so RA$=$8h on March 21, RA$=$14h on June
20, RA$=$20h on September 23, and RA$=$2h on December 21. Now we can
decide which of those is closest to our desired RA and offset from
that date. Since RA$=$14h transits at 8pm on June 20, our RA$=$16h~40m
transits 1 month ($\sim$ 2 hours) and 10 days ($\sim$ 40 minutes)
later, on about July 30. So July 30 is a good time to observe M13 in
the early evening. Again, this just about agrees with the Edmunds
Atlas! And again, this is equivalent to having determined that on July
30 at 8pm standard time, the LST is 16h~40m.

Finally, we might want to already know what time and date we are
observing, and want to have an idea of where the object will be in the
sky. What we will be determining is the hour angle (HA). Let's
consider M92, a globular cluster very much like M13, at RA$=17$h 20m
or so. Let's assume it is October 1 at 8pm. Where will we be able to
find M92? We know that its hour angle will obey the equation:
\begin{equation}
{\rm HA} = {\rm LST} - {\rm RA}
\end{equation}
so to determine the HA we need to know the current LST. We again
offset from the nearest solstice or equinox. We know that at midnight
on September 23, the LST is 0h. Therefore at 8pm standard time on
September 23, the LST is 4 hours earlier, or 20h. Finally, October 1
is 8 days later, or equivalently $4\times 8=32$ minutes. So the LST on
October 1 at 8pm is 20h~30m. The HA of M92 is then 20h~30m {\it minus}
17h~20m, or $+$3h~10m. This means that M92 is well on its way to
setting, so we should look for it in the west.  From NYC, because M92
has a declination of 43\degree, it will still be relatively high in
the sky, however.

If the HA had been near zero, we would be finding the object near the
meridian. If the HA had been negative, it would be still rising so we
would look further east. If the absolute value of the HA had been a
lot more (like 5 hours or more), then quite likely the star would not
be visible.

Professional astronomers use these back-of-the-envelope methods {\it
  all of the time} when they are planning their observations, to the
extent that they often just have these numbers in their heads. Of
course, when it comes to actually pointing the telescope during a
specific observation, all of these calculations are done extremely
precisely with computers, accounting for lots of other effects, some
of which we will discuss later but many of which we will not!


\subsection{Actually Finding an Object}

Let's say you have determined that you have determined that at a
particular time, a star or other object will be at some reasonable
hour angle and likely observable from wherever you are.

If you are searching for a naked eye star, you use the constellations
to guide you. First, if necessary you look for the Big Dipper or
Casseopia in order to locate Polaris and figure out where North
is. Since you already have calculated the LST, you know what
constellations to expect overhead. You then can identify the region of
sky your object is in by its constellation, and a careful study of the
patterns of stars will allow you to identify it.

If you are searching for something through binoculars that is not
visible by eye, it is best to start from a naked eye star and
offset. For example, let's say you want to find Andromeda with
binoculars, which is perfectly doable in a sufficiently dark site. A
good way to do this is to look for Mirak (or Mirach) in the Andromeda
constellation. The Great Square of Pegasus can be an excellent
landmark, and two bright stars from Alpheratz you will find Mirak. If
you have found Mirak you can take a 90\degree\ right hand turn, follow
$\nu$Andromadae to $\nu$Andromedae, keeping pairs of stars in the
field of view to keep oriented. If you look closely at $\nu$Andr, you
will find it has two neighbors, one very close; these are charted in
the Pocket Sky Atlas; you can use those to verify you are looking at
the right star. When you have convinced yourself you are on
$\nu$Andromedae, you should see the Andromeda galaxy off center,
somewhat in the opposite direction from $\mu$Andromedae.

Now let's say you are looking through a telescope. Very often you are
looking for fainter objects and you need to proceed more
carefully. But I'll use the same example here of Andromeda. In this
case you can use the same strategy as with binoculars, but with the
finding scope, which has a simmilar field of view. Since the telescope
is more difficult to move, it is important to proceed carefully and
with a definite plan. Basically, to make a plan it is helpful to have
a deeper atlas than Edmunds, for example the Pocket Sky Atlas. Then
it is helpful to make a series of sketches of what you should see on
the path you should take.

There are two major strategies: first, star-hopping from bright star
to bright star. You try to plot a path where as best you can you keep
a bright star in the field of view at all times. You use the pattern
of faint stars to make sure you are proceeding in the right way. You
can also use the size of the hops as a sanity check with the dials of
the telescope.

Second, offseting by RA and Dec. In this strategy, you find a bright
star to center on. Then you calculate the difference in RA and Dec
between the bright star and your target object. Then you offset in RA,
and check that you see what you expect through the finder scope. Then
you offset in Dec, and do the same. If the pattern you see makes
sense, you should be in the right place. You of course can do the
offset in Dec first too, if that takes you through a more obvious
landmark. And you can do multiple steps too if the offsets are large
and you want to be careful.

Relative to the images I'm showing in class, this is nowhere near as
obvious when you are looking through a telescope! Sometimes you expect
a star to appear but it isn't quite bright enough, or there are stars
you don't expect. Doing this efficiently requires care and practice!

\clearpage

\section{Lecture 5: Stars}

So far we have taken it for granted that there are stars in the sky:
very distant points of light scattered across the sky. But it behooves
to know something about what they are. Early astronomers of course
could not know what they were. When people first came to understand
quantitatively how a light's apparent brightness decays with distance,
it became clear that if the stars were like our Sun, they had to be
what seemed to be {\it impossibly} far away. But the stars {\it are}
like our Sun, and they are {\it incredibly} far away. This example is
not the first or last time that humanity's progress has been stymied
by a lack of imagination as to what is and is not {\it impossible}.

\subsection{Temperatures and Luminosities}

Stars, we now know, are balls of incredibly hot gas, mostly hydrogen,
about 25\% helium, and 1-2\% of heavier elements. In order to
understand stars, it is useful to be able to measure the temperature
of the surface of the stars and to measure the total amount of energy
produced, the ``luminosity.''

The temperatures can be measured by measuring the color of a star. As
some of you have seen in lab, discerning colors by eye can be
difficult. But astronomers can measure the color using observations of
light through different filters, and can determine the magnitude as
observed in each filter. For example there is a commonly used bluish
filter called ``B'' and a commonly used greenish filter called ``V.''
The difference $B-V$ between these two magnitudes is the color.  Why
does this tell you about temperature? Because the thermal emission
hotter things tend to be {\it bluer} and colder thing {\it redder}.

The luminosities are more tricky to measure. We can easily measure the
magnitude, or the ``flux.'' But to figure out how intrinsically
bright, or how ``luminous,'' something is you have to find out how
distant it is. For example, a 100 Watt light bulb has some luminosity
(a total amount of energy per unit time, presumably 100 Watts?) but it
will appear brighter or fainter depending on how far away you are from
it. In particular, the brightness falls off as the ``inverse square''
of the distance:
\begin{equation}
{\rm flux} = \frac{{\rm luminosity}}{4\pi D^2}
\end{equation}
It turns out the distance is very hard to measure!

If you plot the colors of stars versus their magnitudes (roughly like
the temperatures vs their fluxes), the diagram doesn't look that
interesting. There is some sort of gap between very red and very blue
(known as the {\it Hertzsprung gap} after one of the early astronomers
to quantify it) but other than that, not much.

But if you plot the colors versus their luminosities (here the
logarithm of the luminosity) a very interesting pattern emerges, known
as the {\it Hertzsprung-Russell Diagram}. It was in part by measuring
this pattern that led to our modern understanding of stars and their
evolution. The HR diagram has several interesting features: the main
sequence and the red giant branch are the ones we will focus on.

The stars are classified in a manner that corresponds to their color,
in the order OBAFGKM, with ``O'' being the hottest (up to about 40,000
K) and ``M'' being the coolest (around 2500--3000 K). There is a
numerical ordering within each class (with ``1'' being the coolest and
``9'' being the hottest in each class). Our own star is a G2 star.

Stars are also classified according to their luminosity. In many old
texts, you will find a Roman numeral classification of luminosity from
I to V, but this is hardly used anymore. But for cool stars we still
distinguish between dwarfs (those on the main sequence) and giants
(those on the giant branches or for some other reason very bright for
their color).

The detailed classification of stars was originally done on the basis
of their spectra. The most important work was done as part of the
Henry Draper Memorial---recall Draper was a doctor at the New York
University School of Medicine, who also pursued his father John
Draper's work in astronomical photography. The Memorial was work at
Harvard funded by Anna Draper, his widow. This work was led by Edward
Pickering, and consisted of measuring images and spectra of stars and
studying them carefully. Many of the members of his lab made
pioneering discoveries: Wilhelmina Fleming, Henrietta Leavitt, Antonia
Maury, and Annie Jump Cannon. There is a great history of this work in
the book {\it The Glass Universe}, by Dava Sobel.

\subsection{Parallax}

How were these distances measured? We can't measure the distance very
easily because the stars are very far away. The story of how
astronomers predicted, searched for, and finally discovered {\it
  parallax} is recounted nicely in the book {\it Parallax}, by Alan
Hirschfeld.

Consider, for example, how you judge distance with your stereoscopic
vision. Each of your eyes has a slightly different perspective on the
world. Your brain uses this perspective to just distance, using the
difference in angle that an object is seen through each eye. Your
brain uses that angle and the spacing between your eyes to infer the
long side of the (nearly isoceles) triangle formed by your two eyes
and the object. This is a form of {\it parallax}---a difference in
angle from a difference in perspective.

In the context of astronomy, stars are very far away so we need a
large distance between two observations to create a big enough change
in perspective to matter. What we do in practice is observe the
positions of stars (relative to even more distant objects) as a
function of time across the whole year. In principle this gives a
short side of the triangle that is 2 AU, or 300 million km, in size.
From measuring the shift in angle across that distance, you can infer
the long side of the triangle, which is basically the distance.

We define a distance unit based on this triangle, which is the
distance if the angle is 2 arcsec. This is about 30 trillion km, or
about 3 light-years, and we refer to it in astronomy as a {\it parsec}
(``parallax-arcsecond''). Parsecs, kiloparsecs, megaparsecs,
gigaparsecs, are the typical units of distance used in astronomy (even
for distances much larger than a parsec).

This change in perspective is possible because the Earth is moving
around the Sun and not vice-versa. Ancient astronomers realized that
if this was the case, that they should see the shift in stars'
position relative to one another. This was an important argument in
favor of the geocentric model, in which the Earth did not move---if it
did move, you should see the ``reflex motion,'' or parallax, of the
stars, which they did not see.

However, the stars it turns out were just {\it much} further away than
the ancients imagined, such that the parallax is extremely small. In
fact, the closest stars are far away---actually just under a parsec,
so close to 30 trillion km. This is so far that their parallax is only
around one arcsecond! That is barely discernable even if such motion
occurred in the course of a single night---whereas to measure it you
have to measure it consistently across at least a year!  The result is
that parallax was not actually measured until the mid-1800s.

Most stars are even further than a parsec. Just the naked eye stars
range from a few parsecs to around 1000 parsecs, or a {\it
  kiloparsec}. And the fainter magnitude stars within our own galaxy
range to tens of thousands of parsecs, and distant galaxies (as we
will learn later) are much further than that.

\subsection{Formation of Stars and the Main Sequence}

To understand the Hertzsprung-Russell diagram requires understanding
what stars are, the circumstances under which they form, and how they
evolve. 

Our Sun is very hot ball of gas, $2\times 10^{30}$ kg in
mass---hundreds of thousands times more massive than the Earth. We
call this unit of mass the {\it solar mass}, labeled $M_\odot$. It is
shining at a roughly constant rate and has been for about 4.5 billion
years.

Why does it shine? It shines because the gas in its center is
undergoing a process known as {\it nuclear fusion}. The process is a
little complicated, but basically the gas is hot enough (10s of
millions of degrees K) and dense enough (about 100 times denser than
water) that protons can get very close to one another and essentially
merge to form larger nuclei. The primary energy-generating form of
nuclear fusion in the universe is the conversion of four individual
protons (hydrogen) into helium-4. Helium-4 has a mass about 1\% less
than four protons do, and since mass is a form of energy, and energy
needs to be conserved, that energy needs to go somewhere---it goes
into heat, which diffuses slowly out of the star and causes the star
to shine.  The surface of the Sun, however, is much cooler than the
center, at about 5800 K.

We call this phase of a star's life its {\it main sequence} phase, and
it is the longest part of a star's lifetime. The main sequence phase
will continue as long as their is enough hydrogen in the core of the
star, another 5.5 billion years in the case of our Sun---after that,
something new has to happen because the core will be filled with
helium, which doesn't burn as easily.

Within our galaxy and other galaxies, there are places filled with gas
which is very diffuse. But parts of this gas are denser than other
parts, and those parts pull themselves by gravitation together into a
star. Messier 42, the Orion Nebula, is a place where this is
happening, and there as in many other places there are many stars
forming together. The massive stars forming are very hot and ionize
the gas around them and make the nebula shine. The diffuse light in
the nebula comes not from the stars but from the gas in between the
stars. This sort of region is known as an {\it HII region}
(``aitch-two'').

The stars forming range from 100--200 $M_\odot$ at the massive end to
about $0.08$ $M_\odot$ on the low mass end. Smaller objects form too
but they are not massive enough to have hydrogen fusion at their
centers so they do not shine as brightly; these objects are called
{\it brown dwarfs}. So our Sun is a middling-mass type of star.

More massive stars, it turns out, are much more luminous and much
hotter at their surface than our Sun. For example $\delta$ Orionis is
a 20 solar mass star, but is 1000s of times more luminous and around
30,000 K. On the Hertzsprung-Russell diagram that makes it more
luminous and bluer, because hotter things are bluer. Since these stars
shine because they are burning their gas from hydrogen to helium, such
a star will run out of hydrogen at its center much faster than the Sun
will. Correspondingly, its main sequence phase will be much
shorter---only a few million years rather than 10 billion years.  For
less massive stars, the opposite is true. The lowest mass stars are
fainter and redder, and will live on the main sequence for trillions
of years.

\subsection{Open and Globular Clusters}

In the story of how we learned how stars worked, it was helpful to
look at systems like the Orion Nebula, but very much older, after the
gas has dissipated. These nebulae leave behind clusters of stars, some
of which are very long lived and some of which are expected to
eventually dissipate.

Let us consider young clusters of stars that all formed
together. These objects are known as {\it open clusters}, loose
associations of stars that are expected to disperse eventually. Most
such clusters are 10s or 100s of millions of years old, though a
handful are billions of years old. Comparing a very young open cluster
to a slightly older one, the latter has seemed to have lost its
massive, hot, luminous stars. What happened to them?

Let us also consider very old clusters, most of which are the {\it
  globular clusters}, some of the oldest objects in the universe, most
of them ranging from about 10 to about 12 billion years old. Look at
the Hertzsprung-Russell diagrams for these and you see the main
sequence, but you also see a luminous and red branch, known as the
{\it red giant branch}.

To understand these patterns, we have to look at the {\it post-}main
sequence evolution of stars---what happens to them when there is no
useful hydrogen left in their cores, just helium ``ash.''

\subsection{Post-Main Sequence of Massive Stars}

First consider the massive stars ($>8M_\odot$). From what I said
before, and from looking at the young clusters, it is these stars
which leave the main sequence first.

What happens to them is that they are not able to support their
centers against collapsing once they run out of hydrogen in their
cores. Their core therefore get denser and denser and hotter and
hotter. At a certain point, it turns out you can get the helium to
burn into carbon, which it will do for a few hundred thousand years.
Then you run out of helium, the core collapses more, and you start
burning carbon to neon for a few thousand years. Then neon burns into
oxygen for a few years. Oxygen will burn into silicon for a few
years. Finally for {\it a few weeks} the star burns silicon into
iron. At each stage the mass of the product is less than the mass of
the inputs, and therefore energy is released. However, there is no
fusion process that will turn iron into an element with less mass in
the output than in the input. There is no way for the star to burn
iron as fuel.

In these last phases, the stars are known as {\it supergiants}, and
they can oscillate between being very red and blue as these final
burning stages proceed.

At the end point for a supergiant, the center of the star collapses. A
very complicated process ensues, which ends with the outer parts of
the star being ejected into empty space, and the core of the star
falling into itself and forming either a neutron star or a black
hole. Neither of these systems shines brightly so we generally don't
see them in the stellar cluster anymore!

\subsection{Post-Main Sequence of Lower Mass Stars}

Second consider the less massive stars ($<8 M_\odot$), like our Sun.
Consider the Hertzsprung-Russell diagram for a globular cluster, which
is about $\sim 10$ billion years old. That means the stars remaining
on the main sequence are those that are less massive than the Sun. The
stars on the red giant branch are the stars that have just left the
main sequence---they are all just about the same mass, in this case
just about the mass of the Sun. For an older globular cluster, these
stars would be lower mass (and for a younger globular cluster, they
would be higher mass). These stars clearly become {\it much} more
luminous and also redder.

What happens to them is that when their cores fill with helium ash,
they are not massive enough to burn that helium into higher mass
elements. But what happens instead is that hydrogen ignites in a
shell around the core. Astrophysicists can predict what happens next
from simulations: the fusion proceeds {\it extremely} efficiently and
at the same time the star adjusts to this new situation by growing
extremely large. For example, our Sun will grow to nearly an AU in
size, potentially engulfing the Earth.

Why, exactly? Well the equations of stellar structure tell us this
will happen, and we see that it does in the real universe, but there
is not any very good ``intuitive'' way to understand exactly why this
is the thing that happens. It is a good example of why mathematics is
such a powerful language for describing the universe---more powerful
than regular languages, like English---because it can be used to make
a correct prediction like this even when we can't quite say in words
why.

In any case, these red giant phase doesn't last too long, typically
about 10\% of the duration of the main sequence phase. There are a
couple of later phases---the so-called {\it horizontal branch} phase
and then an {\it asymptotic giant branch} phase. In these last phases,
helium is finally burning in the center, into carbon and
oxygen. During the giant branch phases, the star's outer atmosphere is
only tenuously bound to the star because it is so big, and a wind
emanates from the star that is much stronger than our own Sun's solar
wind. At the end of the last giant phase, this wind becomes so strong
that it blows off the {\it entire} atmosphere.

What is left at the center is just the remains of fusion---a {\it
  white dwarf} typically 0.5--1 $M_\odot$ made mostly of carbon and
oxygen, and about the size of the Earth.

For about 50,000--100,000 years, the remnants of the atmosphere will
remain close enough to the white dwarf that it is lit up and
shines. This is what is known as a {\it planetary nebula}---so called
because they are often circular and so resemble a planet when seen
through a telescope, but are clearly gas nebulae. It is like an HII
region, but is much smaller. 

\subsection{The Hertzsprung-Russell Diagram Near Us}

Let us return to the HR diagram as seen around us. The stars around us
are mostly not in clusters anymore. They all formed in some sort of
cluster, mostly open clusters, that have since dissipated. What is
left is just a general population of stars that look mostly
unassociated with each others (except they are mostly in the disk of
the Milky Way). We can learn something about the history of the Milky
Way just by looking at the HR diagram.

The HR diagram shows a very full main sequence at all masses, {\it
  and} it shows a red giant branch. What this means is that the Milky
Way around us is currently forming stars (otherwise the short-lived,
massive stars would not be there) and it has to have been forming
stars for a long time, like over 10 billion years (otherwise the red
giant branch would not be there). The HR diagram we see can be thought
of as a combination of an open cluster's HR diagram (showing signs of
recent star formation) and a globular cluster's HR diagram (showing
signs of star formation in the deep past).

Astronomers can actually play this game much more quantitatively,
inferring fine details of the Milky Way's history and growth, from
diagrams like this measured in different parts of the Milky
Way---there is a lot of information in this diagram!

\clearpage
\section{Lecture 6: Variable and Binary Stars}

At first glance stars look like they are constants in the sky, of
unvarying brightness and position. But they do vary. All stars vary
over the longest timescales of millions or billions of years because
of the effects we talked about last time: running out of hydrogen fuel
in their cores and the changes they undergo after that point. But some
stars vary over human time scales.

There are at the broadest level two types: stars that vary on their
own, and stars that vary because they are in a {\it binary} system.

\subsection{Variable Stars}

There are many types of variables, and here I will discuss only the
most dramatic and important ones.

\subsubsection{Cepheid Variables}

{\it Cepheid variable stars} are quite luminous supergiants that vary
in a periodic (i.e. repeating) way over the course of days or
weeks. They are historically important because they were used to
establish the fact that our Milky Way is not the only galaxy in the
universe.  We will talk about that more next lecture.

The archetypical one is $\delta$ Cephei (``Delta Cephei''). It varies
with a 5.36 day period, by almost a magnitude (about a factor of two
in brightness). It was discovered to be variable in 1784, and over
time it became clear that it was one of a {\it class} of stars with
this regular pattern of variability.

Why does it vary when most other types of stars shine in a constant
fashion? Its atmosphere is ``unstable to pulsation.'' What does that
mean? That means that if the atmosphere expands a little, it will
start to ``pulsate'' around its original position, and that pulsation
will not decay. The pulsation is caused by the reaction of the opacity
of the star to its changing density as it pulsates; but this mechanism
is quite complicated to explain and took many years to understand. 

A major breakthrough in Cepheid science came in 1912. Henrietta
Leavitt, who was working in Pickering's lab at Harvard, was looking at
image plates that had been taken of what is now understood to be a
small, nearby galaxy orbiting the Milky Way, the Large Magellanic
Cloud. There are two of these galaxies (Small and Large), both easily
visible with the naked eye in the Southern Hemisphere. Pickering had
sent several astronomers to establish an observatory in Arequipa in
Peru, and they took photographic images on glass plates that they
shipped back to Boston for study.

Among these plates were many repeated images of parts of the LMC, and
Leavitt was studying a number of Cepheid variables studied in the
image. She noticed that the length of the period correlated with the
brightness of the star. Since all of the stars are in the same galaxy,
roughly the same distance away from us, this means their periods are
correlated with the luminosity---the so-called {\it period-luminosity}
relation. 

This relation allows us to estimate distances using the fact that:
\begin{equation}
f = \frac{L}{4\pi D^2}
\end{equation}
and that therefore:
\begin{equation}
D = \sqrt{\frac{L}{4\pi f}}
\end{equation}
For example, we know that $\delta$ Cephei, a 5.4 day period Cepheid that
is roughly $m=4.5$, is 273 pc away. Looking at Leavitt's plot, for
this star $\log_{10} P \sim 0.7$ and on that relation $m\sim 14.8$.
That is a difference of $\Delta m \sim 10 = 4 \times 2.5$, or a
difference in brightness of four factors of 10, $10^4$. Given the
relationship above, that means that these Cepheids in the LMC are
$\sqrt{10^4} \sim 100$ times further away than $\delta$ Cephei, or
about 30,000 pc (i.e. 30 kpc).

In reality, the LMC is about 50 kpc away---it turns out reading
numbers by eye off of 100-year-old plots and assuming their
calibrations are consistent isn't totally precise! But people can do
this analysis extremely precisely today and know the distance to the
LMC pretty well, though still arguing at the 5--10\% level.

When we talk about galaxies next time we'll describe more the role
this method played (and still plays!) in establishing the size of the
universe (which turns out to be big). 

\subsubsection{RR Lyrae}

Not far on the sky from $\delta$ Cephei is another variable star,
which is much fainter but has had a similarly important impact. This
is RR Lyrae, and like $\delta$ Cephei, it is an archetype of a whole
class of stars known as \ldots RR Lyrae stars.

Basically, these stars are much fainter versions of Cepheids (about 30
times fainter). They vary in a very similar way to Cepheids, and for
exactly the same reasons. But instead of being supergiants, they are
stars about the mass of our Sun, in their horizontal branch phase.

Historically, the similarity between the light curves of Cepheids and
RR Lyrae contributed to {\it a lot} of confusion---astronomers didn't
understand for many years that they were two essentially different
classes of star, which led to lots of trouble using them as a standard
candle. They lie on a different period-luminosity relation than
Cepheids!

Nevertheless, they played a very important role in understanding the
scale of our Galaxy and our place in it. In particular, an important
piece of work was that by Harlow Shapley on globular clusters. He was
working at Mt. Wilson Observatory in California and observed RR Lyrae
stars in these clusters; his paper of 1918 refers to them as
``Cepheids'' throughout, which turns out to be not appropriate, but he
didn't know that then. However, he was able to show that these
clusters were (a) distributed mostly on one side of the sky, towards
the Galactic Center, and (b) to get a distance measurement that
allowed him to determine how big the Galaxy was assuming that the
globular clusters were distributed about its center. It was too big by
a factor of about three---but still was a critical measurement
establishing that the Milky Way really is quite big.

\subsubsection{Long Period Variables}

Then there are various types of stars call ``long period variables''
that tend to be irregular. Most of them are in the midst of shedding
their outer atmospheres: some are extremely massive stars (like 100
solar masses) and are in very late stages before they go supernova;
others are stars like our Sun in their late stages of the red giant
phase, in the process of losing their outer atmosphere and becoming
white dwarfs.

Mira is a solar-mass star of the latter variety, which is dramatic
because it varies in luminosity by a factor of a thousand---at its
peak being $m\sim 2$ and then declining to $m\sim 10$.

\subsection{Binary Stars}

Another reason that stars might vary, or appear to vary, is that they
are part of a {\it binary} system of two stars (or sometimes more!).

\subsubsection{Novae}

A dramatic form of binary star is a nova. This is a star with a
dramatic brightening---many magnitudes---for just a short period,
maybe months, followed by many years of inactivity. The name ``nova''
indicates it is a ``new star,'' meaning just that it is newly
observable, not that it is actually a new star. This is the most
common ``brightening'' observed in stars.

Novae occur when red giant stars and white dwarf stars are orbiting
each other. When stars form, occasionally two stars form that are
close enough together that they orbit each other. Sometimes the orbit
is very far apart (for example, many times the size of our solar
system) and sometimes they are very close (for example, closer than
the orbit of Mercury!). Such stars are called {\it binary} stars.

If the stars are both around the mass of the Sun, then they will both
eventually become red giants and then white dwarfs. If one of them is
a little more massive than the other, then the more massive one will
become a red giant and then a white dwarf first. Later on, the less
massive one will become a red giant. At that moment, the system
consists of a red giant and a white dwarf orbiting each other.

The red giant's atmosphere is emitting a strong wind, as its
atmosphere will eventually be totally shed. Some of that material
falls down towards the white dwarf. The gravity around white dwarf is
rather strong, because it is about half the mass of the Sun but
compressed to the size of the Earth. The gas tends to arrange itself
in a relatively stable disk; gas slowly works its way inward,
ultimately falling onto the white dwarf.

In some systems, called {\it cataclysmic  variables}, the disk roughly
periodically fluctuates and heats up, getting much hotter and
brighter. These events are called {\it dwarf novae}.

In other systems, the disk also fluctuates but ends up delivering a
whole lot of gas onto the surface of the white dwarf at the same
time. The intense gravity at the surface of the white dwarf gas
undergoes a thermonuclear explosion, with hydrogen being converted to
helium for a brief period. This causes a much more intense brightening
called a {\it nova}.

\subsubsection{Wide Binaries}

Not all binaries are so dramatic. Consider Mizar and Alcor in the Big
Dipper. These are close on the sky and moving in nearly the same
direction with each other; it is unknown whether they are
gravitationally bound to each other or not, but they may be.

However, Mizar is known to be a binary star itself, with a companion
$\zeta$ Ursa Majoris (``zeta''). This was the first binary to be split
by a telescope, by Castelli and then Galileo in 1617.

\subsubsection{Spectroscopic Binaries}

When binaries are close together on the sky (including, but not
limited to, the red giant-white dwarf binaries that cause novae), they
often cannot be split even with a telescope---they are just two close
together in angle to be distinguised. 

However, they can still be detected based on the fact that they are
orbiting each other, through the {\it Doppler shift}.  The Doppler
shift is a shift in wavelength of light caused by relative motion
towards or away from the observer. If there is some specific
wavelength of light an object is emitting, if that object is moving
towards you, you will see that light shifted to the blue (lower
wavelength). If that object is moving away from you, you will see that
light shifted to the red (higher wavelength). These blueshifts or
redshifts are therefore signatures of motion.

All stars show some blueshift or redshift, because they are all moving
at least a little towards or away from us; those blueshifts and
redshifts don't change over time because the stars aren't typically
accelerating or decelerating much. But if the star is in a binary, it
also is moving around its companion, and that acceleration can be
detected. Basically, relative to its overall motion, half the time it
is moving away from us, and half the time coming back.

The larger the masses of the stars orbiting each other, the larger the
velocities at a given orbital radius. This fact leads us to be able to
relate the velocities and periods of the orbit to a mass. The specific
equation is $GM_{\rm tot} = v^3P / 2\pi$ for an orbit that is
perfectly aligned with the line of sight. But the important thing is
that this method is the fundamental source of our empirical knowledge
of the mass of stars, making spectroscopic binaries an extremely
important set of objects.

It turns out that $\zeta$ Ursa Majoris itself was the first
spectroscopic binary to be observed!  This was achieved by Pickering
and Antonia Maury (who also happened to be a niece of Henry Draper) in
1889.

\subsubsection{Eclipsing Binaries}

A final interesting case is that of an {\it eclipsing binary}. This is
a special case when the orbit of the stars is such that they can pass
in front of one another. The light from the star behind is blocked
when this happens (it is ``eclipsed'') and this makes the star dip in
brightness.

This happens in a periodic way that allows you to determine the period
of the orbit. It also means that the orbit {\it must} be nearly along
the line of sight. So when an eclipsing binary is also observed as a
spectroscopic binary, that is the most useful system for determining
masses of stars.

\subsubsection{Note on Exoplanets}

All of these techniques can be used to find much smaller companions of
distant stars. For example, the Sun wobbles a tiny bit because it has
planets orbiting it---this can be detected around other stars and used
to infer that they also have planets. When the planets' orbits are
correctly aligned, they pass between us and the star, and eclipse the
star's light a little bit (like a small fraction of a percent). Using
these methods, thousands of stars have been found to host planetary
systems of their own!

\clearpage
\section{Lecture 7: Extragalactic Nebulae}

Most objects in the sky that your naked eye can see are stars, which
are very small in angular extent so look like points of light. There
are some objects whose light is more extended, and before their nature
was understood they were often classified under the same name of {\it
  nebulae}. They are in fact a mix of gaseous nebulae, like HII
regions and planetary nebulae, star clusters, and distant galaxies.

\subsection{Canonical Catalogs of Nebulae}

Two catalogs that are still referred back to are Messier's catalog and
the New General Catalogue.

Charles Messier was a French astronomer of the late 1700s. He had a
particular interest in finding comets. Of course, comets are faint,
diffuse patches of light that are moving across the sky from night to
night. While finding comets, Messier did not want to be confused by
nebulae, which are much more distant and not orbiting the Sun. So he
made a catalog of {\it diffuse things which are not comets} (about a
third of which were ones he discovered). In his work, he used a
relatively small refractor (about 4 inches) and so the objects he
studied were bright. This grab bag of 110 objects is still referred to
today in the naming of the brightest nebulae in the Northern sky---as
in Messier 31 (the Andromeda galaxy), Messier 42 (the Orion Nebula),
and Messier 2 (a globular cluster), or as often abbreviated M31, M42,
and M2.

The New General Catalogue was compiled about a century later and
correspondingly contains many more and fainter objects, almost
8,000. These are also nebulae. The catalog was based on William and
Caroline Herschel's original General Catalogue from 1786, later added
to by them and then William's son John Herschel in the first half of
the 1800s. Like Messier's catalog, the objects are a bit of a grab
bag, but the majority of the NGC objects are galaxies. Again, you will
see their names referred to still today for the brighter galaxies on
the sky. For example, Messier 100 is also referred to as NGC 4321.

If you take a look at Herschel's sketches of nebulae you'll get a
sense of why their nature wasn't known for a long time! In telescopes
available to Herschel, they look like fuzzy blobs---are they nearby
small clouds of gas or very distant massive galaxies? It is hard to
tell, and the answer is: some of each!

\subsection{The Milky Way Galaxy}

One thing that astronomers were pretty sure about was the Milky
Way. This is a diffuse band of light spanning the sky visible in dark
skies. After Galileo first looked at it with a telescope and saw that
it consisted of many, many individual stars, it became clear that this
was a vast system of stars, and our solar system existed inside it.

But again, was the universe outside our Galaxy just empty space
forever? Or were some of the nebulae we saw actually galaxies like our
own but just extremely distant? It took 300 years to know the answer.

\subsection{The Great Debate}

This subject---whether our Galaxy was unique, or whether it was
surrounded by many other galaxies with vast empty regions in
between---was the subject of a so-called ``Great Debate'' in 1920
between two astronomers, Harlow Shapley and Heber Curtis.

In my classes, I used to frame the Great Debate as a way to understand
the astronomical questions at play, but I've come to realize that this
debate was so confused and confusing that instead of being a good
platform for pedagogy on the specifics of astronomy, it is better as
an example of what the frontiers of science look like. These
astronomers and their contemporaries were on the precipice of own of
the most fundamental discoveries ever made by humanity: that the
universe is unimaginably vast, that it had an identifiable beginning,
and that we could determine its age. Yet to look at their arguments,
while you see some brilliant insight and clever measurements, they
come with a strong admixture of muddled thinking, contradiction, and
just plain errors.

The major questions of the debate were:
\begin{itemize}
\item Are we near the center of the Milky Way or near the outskirts
  (right answer: the outskirts)?
\item Is the Milky Way large (100s of kpc) or small (10s of kpc)
  (right answer: 10s)?
\item Are the ``spiral nebulae'' (what we would now call galaxies)
  inside the Milky Way or very distant (right answer: distant)?
\end{itemize}
Even the debate is confused. At the time, people thought that one side
was ``being at near the edge of a large Milky Way with spiral nebulae
inside it'' (Shapley) and the other side was ``being at the center of
a small Milky Way with spiral nebulae outside of it'' (Curtis).  But
neither side was right or wrong---each had a handle on some good
evidence and on some bad.

And both were strongly affected by an incorrect prejudice that if the
Milky Way was large (and thus the spiral nebulae were likely to be
large too) then the distant galaxies would have to be {\it even
  further} away, which was just {\it too far}.  This intuition was
basically wrong.

Recall that Shapley had established that we were near the edge of the
Milky Way by looking at distances of globular clusters. However, as I
noted, he mixed RR Lyrae stars and Cepheids as the same type of star,
leading him to incorrectly find too large a scale for the Milky
Way. Curtis correctly rejoinded that if the Milky Way disk extended so
far, its most distant stars had to be much more luminous than closer
stars, which it turns out they are not. 

Shapley pointed out that rotation of distant spiral nebulae had been
observed. If they were galaxies like our own, they would have to be
millions of light years away, and to be visibly rotating the rotation
speeds would have to be faster than the speed of light. So they can't
be that far away! That is an excellent argument \ldots but the
galaxies were not detectably rotating. In fact, rotation of external
galaxies by looking at motions over time in their images was not
detected until just a couple of years ago, for very nearby
systems. The reported detection of rotation was a mistake.

One more piece of evidence that they debated was that the spiral
nebulae often had large {\it velocities of recession} or {\it
  redshifts}. That is, from the Doppler shift of light they seem to be
moving away from Earth at 100s or 1000s of km/s (much higher redshifts
are now known!). This piece of evidence did not figure strongly in the
debate, but became important very soon after! 

Many more arguments were made in the debate, but they all have this
character. They tend to illustrate the overall ignorance about the
Milky Way and nebulae of the time, which is not a statement about the
quality of the science, but a statement about the fact that this is
{\it always} true at the cutting edge of science.

For this reason, a lot of the work in science is sifting through
conflicting evidence and trying to reconcile seemingly
impossible-to-reconcile facts. You try to use the scientific method to
do this rigorously, but it is hardly a straight path---there is a lot
of zigging and zagging trying different hypotheses and ideas. It is
why we usually want several different independent confirmations of a
new idea before we take it seriously.

\subsection{Cepheid Variables and the Distances to Galaxies}

The immense distances to spiral nebulae is widely considered to have
been established definitively upon Edwin Hubble's determination of the
distance to Andromeda and Messier 33 (another, smaller mass galaxy
near Andromeda).

To find the distance to Andromeda, Hubble relied on Cepheid
variables. Recall that in the LMC, which we now know is 50,000 pc
away, Henrietta Leavitt had found that Cepheids with periods of around
30 days had an apparent magnitude around 13. In Andromeda, Hubble
found that similar period Cepheids had apparent magnitudes around
18. Since the Cepheids have the same period, they must have the same
luminosity. Therefore Andromeda must be further away. The difference
in magnitude is 5 magnitudes, so the difference in flux is a factor of
100. That implies a difference in distance of approximately 10. In the
event, Andromeda is actually about 15 times further away than the LMC,
or about 800,000 pc.

Note that actually the literature of the time does not really directly
discuss the Magellanic Clouds, which should have (based on their
Cepheids, which were obviously well known) evidently been quite
distant relative to most globular clusters. And the LMC at least to
the modern eye looks similar to a spiral galaxy with a strong bar. It
is a puzzle to me why this was not taken as evidence at the time for
the nature of the spiral nebulae. But the astronomers of the time
classified Andromeda in their minds as akin to the fainter spiral
nebulae, but didn't do the same for LMC. Again, it is a testament to
how confusing evidence can be at the moment it is being revealed.

No matter, the large distance to Andromeda was clear, and clearly
greatly exceeding even the larger estimates of the size of the Milky
Way. The other spiral nebulae were taken to be similar in nature but
since they were fainter than Andromeda, they must be further away,
which is correct.

\subsection{A Basic Cosmography}

This led astronomers to the following basic picture. We live on Earth,
a dense ball of rock, one of eight planets in our Solar System
orbiting the Sun. Between the planets is emptiness, except for a very
diffuse wind of gas flowing from the Sun. The size of the Solar System
is a few billion km, or a few light hours.

The Sun is just one of about 10 billion stars in the Milky Way;
clearly we can see that these stars are very far apart from each
other, since the nearest one is over a light year away, about one
parsec.  The size of the Milky Way is about 20,000 pc in diameter.  We
are somewhere near its edge.  In between the stars: more emptiness.

The Milky Way is just one of many similar galaxies. Some small
galaxies are relatively near us. The Small and Large Magellanic Clouds
are about 50,000 pc away; they are actually bound together
gravitationally to each other, and are falling into the Milky Way,
with which they will eventually merge. But the nearest galaxy similar
in size is the Andromeda galaxy, which is about 800,000 pc away. And
beyond that there are galaxies spread out across the universe, with
typical separations of a million pc or more.  In between the galaxies:
even more empty emptiness!

\subsection{Discovery of the Expanding Universe}

At about the same time that it was recognized that the ``spiral
nebulae'' (and other objects) were millions of light years away and
should be considered other galaxies, it was recognized that their
motions implied that the universe was expanding.

This was a surprise. Before around 1920, it was generally assumed
(when people thought about it at all) that the universe was just
stable. It was known from geological deposits that the Earth was very
old, probably at least hundreds of millions of years old. At the time,
the energy production in stars was not known, and their lifetimes
weren't known. But there was not any particular concept that {\it
  overall} the universe should be anything but roughly stable on the
largest scales. Again: to a modern eye this position looks
nonsensical, but it was a widespread assumption of the time. But
people only were driven away from the assumption by the accumulation
of evidence.

The first piece of evidence was the fact that most galaxies seem to be
receding from us. This was inferred from the large Doppler redshifts
of almost all galaxies with observed spectra, many of them gathered by
Vesto Slipher of Lowell Observatory in Arizona. Recall this was part
of the evidence discussed in the Great Debate, but at that time its
significance was not fully clear.

In the wake of the Hubble's establishment of the distance to
Andromeda, he and (independently) Georges LeMaitre around the same
time both put together estimates of distances of galaxies with
estimates of their velocities.

Hubble took very crude estimates of distance using the ``brightest
star'' in each galaxy---which very often was actually a large HII
region---as a crude standard candle. This is very approximate, but it
is extremely hard to do better. Only with the launch of the Hubble
Space Telescope in the 1990s, for example, did it become possible to
find and measure Cepheids in these galaxies---measuring distances to
galaxies is {\it still} very hard.

He plotted these distances against the
velocities of galaxies and saw that they correlated. That is, they lay
along a line known today as the Hubble Law:
\begin{equation}
  v = H_0 d
\end{equation}
The velocity $v$ is proportional to the distance $d$, with a constant
of proportionality $H_0$, the Hubble Constant.  So what did Hubble
conclude? He concluded that \ldots the distances and velocities were
correlated. To the end of his life he was hesitant to say another more
definitive than that based on these observations.

LeMaitre came at the same thing very differently. He was a theoretical
physicist, and also a Catholic priest, working in Louvain in
Belgium. He was one of the first generation of physicists to steep
themselves in General Relativity, the theory of gravity postulated by
Einstein in 1915 to extend Newton's laws of gravitation. Study of this
theory more-or-less naturally leads to a dynamic universe---one that
expands forever, or expands and then is pulled by its own weight back
in on itself to collapse. Einstein recognized this and introduced a
term in General Relativity, the cosmological constant, to keep the
universe in balance; we now know that this term is much, much smaller
than it needs to be to do that.

Among his other contributions, LeMaitre took seriously the idea that
the universe was expanding and recognized that it would predict a
relationship between distance and velocity. He even took the same data
used by Hubble and estimated the rate of expansion! His interpretation
is basically right and is the one that we hold today.

\subsection{What the Expanding Universe Means}

What does it mean when we say the universe is expanding? It means that
if you look at all the galaxies in some sphere around us right now,
because they are all moving away from us, in a few billion years they
all will be in a bigger sphere. A few billion years ago, they were all
in a smaller sphere.

It is important to realize that this does {\it not} mean we are in a
special place. {\it Every} galaxy sees the same pattern of motion
around it. The Milky Way is not in the center of the universe!

It is also important to realize that the pattern of expansion implies
that there was a definite beginning to the universe. Think about a
galaxy \#1 some distance $d_1$ away and going velocity $v_1$ away from
the Milky Way. When would it have ``left''---i.e. when would it have
been in the same place as the Milky Way? That would have been:
\begin{equation}
t_1 = \frac{d_1}{v_1} = \frac{1}{H_0}
\end{equation}
What about some galaxy \#2? Well:
\begin{equation}
t_2 = \frac{d_2}{v_2} = \frac{1}{H_0}
\end{equation}
I.e. the same amount of time ago.

That is, all of the galaxies we see seem to have emanated from the
same place at the same time, about $1/H_0$ ago in time. Given what we
now know about the Hubble Constant, this is about 14 billion years. So
the universe is about 14 billion years old, and it started out as a
very dense, very small pocket of matter containing that material that
eventually became everything that is observable today.

There are some more details we worry about in modern cosmology, in
particular that the velocities of galaxies are not actually constant
over those 14 billion years---they at first decrease as gravity tugs
back on galaxies. Then they increase due to the (now measured)
cosmological constant that accelerates them. If that acceleration
continues, the universe will keep expanding forever at a faster and
faster pace!

\subsection{Mapping the Universe}

Once the Hubble Law is established, it can be used to map the universe
in a simple way. As I noted above, measuring thee distances even to
nearby galaxies is hard. But measuring their velocities is (relatively
speaking!) easier. But if we already {\it know} the Hubble Law exists,
we can use it to {\it determine} the distance from the velocity:
\begin{equation}
d = \frac{v}{H_0}
\end{equation}
This can allow us to make {\it really} big maps of the universe.

This is the subject of my own research with the Sloan Digital Sky
Survey. With a pretty large team (hundreds of astronomers) over the
last twenty years we have created the largest map of the universe to
date. We started by making a large image of the sky---at the time, the
largest deep image of the sky---mapping the RA and Dec of all of the
galaxies down to some faintness.

This map is beautiful in itself, and you can see how galaxies tend to
cluster together because of gravity. You can also see the variety of
galaxies, in particular the difference between the {\it elliptical}
galaxies and the {\it spiral} galaxies. The spirals were what the
astronomers of the 19th century referred to as spiral nebulae, and
they are similar to the Milky Way---mostly flat pancakes of stars,
most often with lots of newly forming stars. The ellipticals are
roundish and smooth, and generally older.

But then we can measure the velocities of galaxies by measuring their
redshifts, using features in their spectra that are well known. We
obtain the spectra many-at-a-time using fiber optic technology and a
specialized telescope facility in New Mexico (I will show more details
in class). And then we can determine their redshifts, and make maps of
the universe!

\clearpage
\section{Lecture 8: the Moon}

The Moon is probably the most familiar object in the night sky. In
this lecture, we will discuss the Moon's properties and its orbit. We
will discuss how phases of the Moon occur, which are one of the
easiest patterns in the night sky to observe. In a later lecture, we
will discuss the Moon's effects on ocean tides and its role in lunar
and solar eclipses.

\subsection{Formation of the Moon}

When stars form, a rotating disk of gas and dust forms around
them. The dust slowly agglomerates into large
rocks---asteroids---which themselves can come together to form the
moons in the solar system and the rocky planets.

For the solar system, all of this occured about 4.5 billion years
ago. In that early phase, the Early was still mostly molten because it
had not yet cooled down after the violent collisions causing its
formation. Our best understanding is that sometime around then, {\it
  another} forming planet, probably about the size of Mars (i.e. a
little smaller than the Earth), collided with the Earth. Neither was
destroyed, but the remnants of that second object remained in orbit
around the Earth and is now our Moon. At the beginning, the Moon was
very close to the Earth, but for reasons I'll explain later it has
slowly moved away over time (and is still moving away).

Why do we think this? The Moon is made of material more similar to the
outer parts of the Earth---i.e. its mantle and crust---and tends to
lack iron. Other more detailed chemical clues indicate that the
material on the Moon was brought to higher temperatures than that on
Earth. The main piece of evidence {\it against} this hypothesis is
that the Moon and Earth are too similar chemically overall (their
``isotopic abundances'' are uniquely similar to each other relative to
any other Solar System body)---nevertheless the current consensus is
that the giant impact explanation still holds up.

While this was happening, for the first 500 million years or so, the
solar system was filled with the smaller asteroids that formed in the
disk of gas. In the gap between Mars and Jupiter, asteroids survived;
they would (and still do) occasionally collide with one another, but
they aren't heavily perturbed or collide with any large planets.
Thus, in this region of the solar system we have an {\it asteroid
  belt}. 

But in the regions where there are planets, the asteroids were
deflected away {\it or} collided with the planets within that first
500 million years. This means that early in the solar system there was
a large number of collisions. As time has gone on they have become
less frequent (though they still occur!). We see the evidence of that
on the surface of the Moon.

You again might ask how we know this---the Apollo missions brought
back rocks and we have been able to use radiometric dating techniques
to age them, and place many of the events that created them at about 4
billion years ago. 

\subsection{Orbit of the Moon}

Just as the Earth orbits the Sun, the Moon orbits the Earth. However,
the Moon is much closer! The period of its orbit is about 27.3 days, a
time we will refer to as its {\it sidereal period}.  On average the
Moon is about 380,000 km away. The contemplation of this orbit was one
of the things that led Newton to his theory of gravity---which linked
the motions of the planets as described by Kepler to the motion of the
Moon to the gravitational forces that cause things (like his probably
apochryphal apple) to fall towards the center of the Earth.

But it is significant to us that the Moon's orbit is substantially
elliptical. At ``perigee'' (literally, ``closest to Earth'') it is
about 360,000 km, and at ``apogee'' it is about 400,000 km. So its
distance varies across its orbit by about 10\%. When the Moon is close
it will appear about 10\% bigger in radius (or about $1.1^2\sim1.2$
times the area).

We will also learn that it is significant that the Moon's orbit is not
{\it exactly} in the Ecliptic plane (the plane of the Earth-Sun
orbit). It is tilted by about 5 degrees. When we talk about eclipses
this will matter---it explains why we don't see eclipses every month.

Finally, it is worth noting that the Moon is also {\it rotating}. In
fact, it rotates at exactly the same rate it is orbiting! That means
that the same half of the Moon is always facing the Earth. From Earth
we can never see the {\it far side} of the Moon (not to be confused
with the {\it dark side} of the Moon).

\subsection{Change of the Moon's Orbit}

And as I noted before, the Moon is slowly moving outwards at about 3.8
cm per year! If the Moon and Earth were tiny point masses, this would
not happen; it is a consequence of the tidal forces between the Earth
and the Moon.

You might ask how we could possibly know that the Moon is moving away
at 3.8 cm per year---that's a pretty small rate. If so, you will be
even more surprised to know that this was suspected even in the 19th
century. When people looked at records of the Moon's location over
time and the time between perigee (the orbital period) they saw it was
very slowly getting longer. Because we have records of the Moon going
back 2,000 years or more, we can find very slowly occuring changes! A
very good explanation of the changes was that the Moon was slowly
receding.

However, it wasn't until we actually went to the Moon with rockets
that we could possibly measuring this recession directly. It is done
in a very clever way, and currently the best measurements are being
made at Apache Point Observatory (the same observatory I talked about
last time, which mapped the universe on the very {\it largest}
scales).

As part of the Apollo missions that NASA performed, and as part of the
robotic rover missions that the Soviets performed on the Moon, humans
left {\it mirrors} on the lunar surface. Not just regular mirrors, but
a set of {\it retroreflectors}, whose design is such that incoming
light is reflected off two mirrors at a right angle to each
other. This causes the light to be reflected precisely back to its
source. 

What this allows us to do is to send light to the Moon, and then time
how long it takes to get back. If this can be done very precisely, you
get a very exact distance to the Moon---it takes a nanosecond for
light to travel 2.5 cm, so you have to time the light very precisely!

In detail, what the APOLLO project does is to point a 3.5-m telescope
at where these reflectors are on the Moon. Then they send short laser
pulses {\it backwards} through the telescope. The light spreads out
quite a bit due to blurring by our atmosphere, so it ends up spread
out over a kilometer area on the Moon. But some of it hits the
retroreflectors and bounces back. The light is still spreading out so
only some of {\it that} actually comes back to hit the 3.5-m
telescope.  This takes about 2.5 seconds. The pulses are timed so that
{\it in between} the pulses, the telescope can record the light it
sees. Then you can measure when you receive the pulse back relative to
when you sent it.

Of course, the Moon's orbit is elliptical, so the distance is always
changing because of that.  There are a lot of little wobbles in the
Moon's rotation, and moonquakes, and the like.  Also the building that
the telescope is on, and the ground beneath it, also shift. So the
change in distance is a slow drift on top of a much bigger signal. But
it has been very definitively measured!

Among other things, this experiment is  performing very precise
measurements of General Relativity, Einstein's theory of
gravitation---the question is whether this theory has some flaw we can
measure if we do a precise enough measurement. They haven't found
anything so far, but these measurements have ruled out lots of
alternative theories that people were studying in order to understand
how the universe expands. So the Moon's orbit is {\it still} playing a
role in our understanding of gravity!

\subsection{Phases of the Moon}

But let us return to more easily made observations of the {\it phases}
of the Moon.

Depending on the relative position of the Earth, Sun, and Moon, we can
see different amounts of the Moon lit up by the Sun. If we are between
the Moon and the Sun, we see the Moon as {\it full}. If the Moon is
between us and the Sun, we can only see the dark side and we call it
{\it new}. In between one new moon and the next, we see it as {\it
  waxing crescent}, {\it 1st quarter}, {\it waxing gibbous}, {\it
  full}, {\it waning gibbous}, {\it 3rd quarter}, and {\it waning
  crescent}. 

Based on the necessary geometry, note that a crescent or new Moon will
always be transiting around when the Sun is transiting. It can't be
very far from the Sun on the sky. Conversely, a full Moon must be 180
\degree\ away from the Sun, so it will rise at sunset, transit at
midnight, and set at sunrise. A first quarter Moon will transit at
sunset, and a third quarter Moon will transit at sunrise. So you will
only see these phases high in the sky at specific times of
day---otherwise the position of the Sun and Moon would not be correct
to create that phase.

The time between one new Moon and the next, or one full Moon and the
next, is called the {\it synodic period}, and is 29.5 days. Why is it
not the same as the sidereal period? The answer is similar to why the
sidereal and solar days differ. As the Moon orbits the Earth over the
course of a month, the Earth is going around the Sun. So if the Moon
is full, and 27.3 days transpire and it is back at the same position
with respect to the Earth, because the Earth is moving, the Sun will
not be opposite the Moon. Since about $1/12$ of a year has gone by,
the Earth will be about $1/12$ around its 360\degree\ journey, and to
get back to full the Moon has to travel $1/12$ more along its orbit,
or about 2.2 days, which means it is a total of 29.5 days between two
full Moons.

The synodic period is longer than the sidereal, just like the solar
day is longer than the sidereal day. This is because the Moon is
orbiting the Earth in the same sense that the Earth is orbiting the
Sun (counterclockwise as viewed from above the North Pole). 

\subsection{Librations}

Even though the Moon is always facing us, we can view a little bit
``around the edges'' because of what are referred to as {\it
  librations}. These are slight changes in our viewpoint on the Moon.

The largest effect is the {\it libration in latitude}. The Moon rotates
about an axis that is not quite aligned with its orbit; it is tipped
by about 6.5\degree. So it is tipped towards us in one part of its
orbit, and away from us in the opposite parts. This libration allows
us to see a bit over the pole of the Moon

The next largest effect is the {\it libration in longitude}. This
effect occurs because of the ellipticity of the Moon's orbit. When the
Moon is closer to the Earth, it moves faster than when it is further
away. But its rate of rotation is always the same. So even though the
period of rotation is the same as the period of the orbit, it doesn't
result quite in maintaining  exactly same side pointing towards
Earth. The perspective shifts by a few degrees across the orbit.

The smallest effect is the {\it diurnal libration}. This is just the
change in perspective between seeing the Moon rising and setting---the
change in its parallax across the day. 

For these reasons when you look at a map of the face of the Moon, it
will show a little more than half of the Moon---it will show a few
degrees around the edge, because we can see those sometimes. Of course
on whatever night {\it you} are viewing the Moon you may not see those
features at the edge.

\subsection{Lunar Features}

The Moon has two basic features: bright areas and dark areas.

The bright areas are dominated by craters, caused by asteroids or
comets that plummeted into the Moon. The craters have flat centers,
sometimes with a dome at the center, and mountainous edges, which can
be kilometers in height.

You can conclude that these areas are {\it old}, in the sense that
these regions were not resurfaced by activity on the surface over the
past 3 billion years or so. If they had been, they would have had many
fewer craters! The craters you see are {\it mostly} very old. You can
tell which ones are older than others because the newer ones ``blot
out'' the old ones, i.e. a full crater overlaps a partial crater. Some
of the larger ones have distinctive ``rays'' emanating from them
consisting of material spewed out by the impact. For example, Tycho is
a large crater from about 100 Myrs ago with such rays. 

The dark areas, or {\it maria}, are the newer parts, though by ``new''
we mean about 3 billion years old.  For some reason---perhaps some
unusually large impacts about that time---there was significant
vulcanism that released huge amounts of lava, which spread out in
large flows. The material in the maria is basalt, similar to the
oceanic crust on the Earth.

Interestingly, if you look at the far side of the Moon, which we can
do now thanks to satellites, you find that there are basically no
maria! For some reason they only happened on the near side. A puzzle!

\clearpage
\section{Lecture 9: the Planets and their Motions}

Our solar system has eight planets (or nine if you count Pluto). It is
by no means unique---planets have been found around other stars. But
we really understand our own planets well: most have had close fly-bys
or have been landed upon. You can't do justice to our knowledge in a
lecture or in a semester. In college I took a course just on Mars, and
it only scratched the surface---and now we know a ton more than back
then in the early 90s!

\subsection{Basics}

The inner planets are ``rocky'' planets and relatively small, all
smaller than the Earth. They are primarily rocky bodies, though all
except Mercury have some atmosphere.

The outer planets are ``gas giants'' and considerably larger. Jupiter
is the largest, with about one thousandth the mass of the Sun, but
still 300 times more mass than the Earth. They consists of large
amounts of hydrogen and helium, methane (especially in the case of
Neptune and Uranus), around a small rocky core.

The asteroid belt is in between. Mercury, Venus, Mars, Jupiter, and
Saturn are all visible with the naked eye. Uranus and Neptune were
discovered only with telescopes.

Pluto is a bit of an oddball---a small icy body, among the largest of
a class of object we know now as ``dwarf planets,'' of which there are
many in the outer parts of the solar system. 

\subsection{A Potted History of Geocentrism \& Heliocentrism}

The word planet comes from the Greek word for wanderer. For the
ancient astronomers the defining properties of planets were
that---unlike the stars, which simply rotate through the night sky
over and over again with the same pattern year after year---the
planets moved in relation to the stars along the Ecliptic.

The ancient astronomers wanted a way to predict where the planets
would go and to explain their patterns of motion. They needed to build
models of their motions. There were two basic starting points: the
Earth was stationary with everything moving around it, or the Sun was
stationary with everything moving around it.

The ancient Greek astronomers generally concluded that the Sun,
planets, and stars all somehow moved around a stationary Earth. So the
day and night were caused by the whole Celestial Sphere rotating,
along with motions of the Sun and planets relative to the stars. Some
astronomers (Arybhata in India in the 500s, al-Silji in Iran in the
900s) postulated a rotating Earth, but these ideas were not really
accepted. Interestingly, some astronomers seemed to consider it a
basically a question unanswerable by science whether the Earth rotates
or everything rotates around it, which turns out to not be true (go to
any science museum with a working Foucault's Pendulum to see why!).

As for whether the Earth and other planets orbited the Sun, or whether
the Sun and planets orbited the Earth, again the ancient astronomers
considered both possibilities. Aristarchus of Samos, as early as
around 300 BCE, had promulgated the heliocentric model. However, for
centuries most astronomers accepted Aristotle's argument that if the
Earth moved ``there would have to be passings and turnings of the
fixed stars''---generally accepted as referring to the effects of
parallax. Of course, parallax exists but is so small that it wasn't
observed until 1838 with telescopes and extremely carefully designed
instruments unavailable to Aristotle. 

The solar system models that these astronomers built on the geocentric
model were quite complex, but seemed to explain the motions of planets
through the stars to within a few degrees. The complexity arose
mostly because the center of motion of planets really is closer to the
Sun than the Earth! 

In its basically final form as defined by the brilliant astronomer
Ptolemy, the planets orbit roughly in a circle around a point slightly
offset from the Earth called the {\it eccentric}. But they don't quite
go in a circle, they go in a epicycle whose {\it center} that moves
around the circle.  And also the motion of the center isn't always the
same---it moves with a constant angular speed with respect to a point
slightly offset from the eccentric called the {\it equant}. This is a
pretty complicated picture, and it sort of works.

The epicycles were particularly needed for Mars, which undergoes a
very obvious retrograde motion during some periods. It generally moves
East through the stars, but then periodically turns around and heads
west. They are also part of the Ptolemaic model for Venus and Mercury,
which in this model travel around the Earth between it and the Sun,
but with epicycles that explain their back and forth motion across it.

In 1543, shortly before he died, Nicolas Copernicus published his
heliocentric model. This model had the planets, including the Earth,
moving in perfect circles around the Sun. He had hesitated, not
unreasonably, in promulgating this theory. There was some sense at
that time that the motion of the Earth or not had theological
implications.

Interestingly, when it was published its editor added a preface that
commented that what really mattered was what it predicted for
observables, not whether it was {\it more true} than the geocentric
theory. This comment managed to do three things at once: (a) provide
the ultimate cop-out for the sake of appeasing those who would object
on dogmatic grounds; (b) very astutely presage a much more modern
attitude towards physical theory, that it is the {\it observable
  quantities} that matter, nothing else about the theory (in quantum
theory, known as ``shut up and calculate''); and (c) be extremely
wrong about what would be observable eventually to distinguish between
geocentrism and heliocentrism.

By 1600, the astronomer Tycho Brahe, under the patronage of King
Frederick II of Denmark for most of his career, had developed data
high enough quality to show that in detail the complicated, accepted
Ptolemaic model didn't work particularly well quantitatively. Also,
the Copernican model wasn't much better, and in some ways worse. Brahe
himself remained a geocentrist because of the lack of stellar
parallax---his favored model was not Ptolemaic though, he preferred
that the Sun traveled around the Earth, but all of the rest of the
planets circled the Sun.

At the end of his life, Brahe fell out with the successors to
Frederick, and moved to Prague to work for the Holy Roman Emporer
Rudolf II. When he died in 1601, his position was taken over by
Johannes Kepler.

In the first decades of the 1600s, Kepler provided the solution to the
problem of modeling planetary motion. Instead of having the planets
move in circles, he had them move in ellipses. This is a much simpler
alteration than the eccentrics, epicycles, and the equants of Ptolemy,
though he did have to make some particular assumptions about the rate
at which the planets moved through their ellipses. This model
explained the data extremely well, and was a natural and accurate
heliocentric model. He published it in 1609.

Now enters Galileo. The Keplerian model of the solar system is now
proposed and explains the data very well. Turning his telescope to the
skies, Galileo sees some interesting things. First, he sees the moons
of Jupiter orbiting it. Already this is suggestive! These objects are
obviously not orbiting the Earth. Then he sees the phases of Venus. In
the Ptolemaic system, Venus is never ``behind'' the Sun so its phases
are impossible to reproduce---in particular, in the Ptolemaic system
it can never be full.

These arguments figured largely in Galileo's famous {\it Dialog
  Concerning the Two Chief World Systems} that eventually got him in
trouble with Pope Urban VIII. The Roman Catholic Church for some
reason took exception to the heliocentric model because of perceived
conflicts with scripture. A complicated story ensued: Galileo was
enjoined in 1616 by the inquisition not to support the Copernican
picture; but was allowed and encouraged by the Inquisition to publish
his dialog in 1632; but his depiction of the geocentric side was
thought to mock the new Pope; both religious and scientific rivals
encouraged this and he ended up on trial for heresy. Eventually he was
sentenced to house arrest for the rest of his life in Arcetri, which
is probably the most beautiful place in the world you could be put
under house arrest.

\subsection{Kepler's Laws}

Eventually the predictive power of Kepler's Laws were simply too
good to ignore, and the astronomical community fully accepted them.  
Kepler's Laws are fairly simple to write down:
\begin{itemize}
  \item Planets move in ellipses (squashed circles) with one focus at
    the Sun (i.e. the Sun not at the center but closer to one side).
  \item Planets sweep out equal area in equal time (ie. they move
    fastest when closest to the Sun).
  \item The periods are larger and the speeds are slower if the sizes
    of the orbit are larger; specifically $P^2 = a^3$ if $P$ is in
    years and the ``semimajor axis'' $a$ is in Astronomical Units.
\end{itemize}

These laws were essential to Newton at the end of the 1600s in working
out the laws of gravity and motion. He explained how one object
orbiting a much heavier object must obey these laws. From this theory
of gravity you can also predict the motion the Sun undergoes (much
smaller!) because of a planet orbiting it, and you can predict the
effects of planets on each other's motions. The next centuries were
spent by mathematicians and astronomers refining their measurements
and their calculations to understand the motions and the laws behind
in finer and finer detail.

The third law's consequence means that the inner planets have short
orbits (88 days for Mercury!) and the outer ones have long orbits (12
years for Jupiter---most of you have lived through less then two
Jupiter years!). Pluto has an orbit of 250 years, but was discovered
80--90 years ago, so it has barely been through a third of its orbit
since we discovered it!

\subsection{Opposition and Conjunction}

But let's look at what these patterns mean for how we view the
planets. There are two very interesting positions a planet can be in:
\begin{itemize}
\item {\it conjunction}, when the planet is in the same direction as
  the Sun, which can be {\it inferior} if the planet is between the
  Sun and the Earth, and {\it superior} if the planet is on the other
  side of the Sun.
\item {\it opposition}, when the planet is in the opposite direction
  as the Sun.
\end{itemize}

Mercury and Venus can clearly only be in conjunction, not
opposition. They will be ``new'' in inferior conjunction and ``full''
in superior conjunction, though they will obviously be hard to see at
those times. The inferior conjunction is an interesting time because
it is when the planets may {\it transit} the Sun---which in this
context means something different than in the usage we used
before---in this case ``transit'' means ``pass in front of.''

When the inner planets are at their greatest {\it elongation} they
will be in half phase. This is also the time when they are easiest to
see, the best times being right before dawn (if they are at their
greatest western elongation) or right after sunset (if they are at
their greatest eastern elongation)

The outer planets will be easiest to see at opposition---they will be
up at midnight, they will be full, and they will be at their closest
point to us.

\subsection{Opposition and Conjunction}

The positions where planets are easiest to see are determined by the
relative position of the Sun, the Earth, and the planet. If the Earth
stayed still, these positions would repeat once a sidereal orbital
period for the planet. But the Earth is moving too. As with the Moon,
this means that the sidereal period for the planey is different than
its synodic period. Again the ``synodic period'' refers to the period
over which our view of the planet repeats---i.e. the time between
oppositions, or between superior conjunctions, or between any specific
phase.

Consider the Earth and an outer planet at opposition. Wait a year. Now
the Earth is back where it began, and the outer planet has progress on
its orbit, but less far (because of Kepler's third law). Earth will
catch up a little later. So the next opposition is a little longer
than a year later. The further out the planet the closer this time is
to a year. The closer in the planet is (while remaining further from
the Sun than Earth), the longer the synodic period is, because it is
only going a little slower than Earth. So Uranus has a synodic period
of 367 days, but Mars has a synodic period of 2.14 years.

The thinking is the same for an inner planet, but the roles are
reversed. So for a very short period inner planet, the synodic period
is just a bit longer than the sidereal period. But for an inner planet
closer to the Earth's orbit (but not outside it) the synodic period
can be long. So Mercury's synodic period is 116 days (compared to an
88 day orbit) but Venus's is 584 days (compared to a 225 day orbit).

The synodic period is useful because, for example, if Mars is nearly
at opposition, you know it will repeat that in about 2.14 years. And
the outer planets like Jupiter, you know will creep through the sky
and be at opposition again next year just a month later.

\subsection{Favorable Opposition}

The orbits of the planets are elliptical, not exactly circular.

This has two effects. First, the synodic periods (most notably for the
outer planets) are not {\it exactly} the times between oppositions,
because depending on where those planets are in their orbits they are
moving faster or slow.

Second, at certain points the orbits of the Earth and the planet will
get closer than others. This is a particularly big deal for Mars, whose
orbital eccentricity is the largest.

If you look at the orbits of Earth and Mars, because of the large
eccentricity of Mars, there is a certain point in space where the
orbits are closest. Because the synodic period of Mars is 2.15 years
(i.e. not exactly an integer number of Earth years), if Mars is
exactly at opposition on one night of the year (say Oct. 13 2020) then
it will again be at opposition two years and almost two months later
(which will be December 8 2022). Depending on what month that
opposition occurs, Mars will be either close to the Earth or far away.

The {\it favorable opposition} of Mars occurs around August 29. Not
{\it every} August 29! It requires Mars to be at opposition on that
night. So looking forward every two years and almost two months, we
can anticipate oppositions January 16 2025, February 19 2027, March 25
2029, May 4 2031, and June 27 2033, and September 15 2035. Notice how
the spacing is a bit uneven---when opposition occurs near apohelion
(the point furthest from the Sun) the synodic period is shorter
because Mars is moving more slowly.

But the main point is the oppositions in February and March (2027 and
2029) will be not great times to observe Mars because it is at a point
further away in its orbit. Given the diameter $s \sim 6,800$ km of
Mars and the distance $D \sim 100$ million km during its March 2029
opposition, we can calculate its angular diameter:
\begin{equation}
\theta \sim 60\degree\ \times \frac{s}{D} \sim 0.004\degree \sim 14''
\end{equation}
Whereas during the September 15 2035 opposition $D\sim 57$ million km
and so $\theta\sim 25''$.

\subsection{Retrograde Motion}

The outer planets, again most notably Mars, have something else
interesting happening near opposition, which is {\it retrograde
  motion}. Recall that the epicycles in the Ptolemaic model were
partly necessary in order to explain this unusual zig-zag in Mars's
motions. 

So what is actually happening? It is the effect of the Earth
``catching up'' to Mars. As the Earth enters the half of its orbit for
which Mars is visible at night, Mars appears to be traveling eastward
through the stars because its orbital motion is in that direction and
Earth is moving perpendicular to that motion. But as Earth gets to its
closest approach to Mars, its motion becomes parallel to, and faster
than, the motion of Mars, and Mars appears to slip backwards
west. Further along the orbit it moves eastward through the stars
again.

So in fact the Copernican or Keplerian model accounts for this effect
quite nicely.

\subsection{Transits of Mercury and Venus}

During an inferior conjunction, a planet may pass {\it directly} in
front of the Sun. This is called a transit and can be thought of as a
teensy-tiny eclipse of the Sun. This does not happen at every inferior
conjuction because the orbits of Mercury and Venus are slights tilted
from the Ecliptic.

Where these orbits cross the Ecliptic are the places such transits can
occur, so they will occur at very specific times of year: May and
November for Mercury, June and December for Venus. Because Mercury is
closer to the Sun, the transit takes up a larger fraction of its orbit
and is therefore more likely. So transits of Mercury happen about
every 7--14 years, but transits of Venus occur in pairs 8 years apart
from each other, with hundred or so year gap between pairs.

The transits of Venus played an important role in understanding the
structure of our solar system. Once people accepted Kepler's law and
the heliocentric model, they could figure out the relative distances
between the planets: their periods after all are related to their
sizes, so the ratios of the orbital radii of the planets became very
well known. But their {\it absolute} sizes were not. The lack of a
sizeable diurnal parallax meant that the distances were large, but
there wasn't much of a way to precisely measure the size.

But! If you were able to observe a transit of Venus simultaneously
from two well separated spots on Earth (and then those two observers
compared notes) you could observe the parallax of Venus against the
Sun from those two spots. Since you know the distance between the two
spots on Earth, you know the short side of the triangle $s$, and the
position of Venus as it crosses the Sun tells you the angle, so you
can determine the long side of the triangle:
\begin{equation}
D \sim \frac{\theta}{60\degree} \times s
\end{equation}
which is the distance between the Earth and Venus.

This method was very rapidly realized to be possible. Indeed, Kepler
predicted the 1631 and 1639 eclipses. However, he made an error in his
calculations.  Jeremiah Horrock in England managed to see it in 1639
anyway---dodging clouds until minutes before the transit occurred.
But not enough astronomers across the world saw it for a good
measurement of the scale of the solar system.

By the time of the next pair of transits in 1761 and 1769, the
astronomical world had arranged for many scientists across the world
to simultaneously measure the transit.  The most famous story is of
Guillaume Le Gentil, who traveled to India in 1760. But due to
complications arising from the Seven Year's War, he ended up on a boat
during the 1761 transit. He remained in India until the next transit,
establishing an observatory there and performing geographical
work. After his eight year wait, the day of the transit was
cloudy. His trip back to France was delayed, and he got dysentery, and
finally arrived home in 1771. Supposedly, he found his wife had
claimed him dead and had remarried, because his letters back home and
to the academy of sciences in France had apparently been waylaid.

\subsection{Uranus, Neptune, Pluto}

The brightest five planets are easily observable with the naked eye,
and their status as wanderers has been known always---i.e. since
before recorded history.

But Uranus, Neptune, and Pluto required telescopes to discover. Uranus
can be bright, technically a very faint naked eye star, but noticing
and measuring the motion of this very faint star was not in practice
possible. It is unclear, but it may be listed in Ptolemy's star
catalogs. When people started cataloging star positions with high
accuracy using telescopes, Uranus was sometimes observed---and then
later {\it not} observed---for example, Flamsteed observed it a number
of times, but did not recognize that it was moving, and didn't have a
good enough telescope to notice it wasn't a star.

William Herschel observed it in 1781 while scanning the sky with the
telescope he was using for observing nebulae, and recognized it was
not a star but a planet. Other astronomers found it and tracked its
motion, and concluded it was a planet---the first new one ever!

In tracking the path of Uranus, astronomers noticed it didn't seem to
follow Kepler's Laws perfectly, and even accounting for the influence
of other planets in the solar system it was not moving along its
orbital path the way it was predicted to. In the 1840s, several
astronomers independently hypothesized that the discrepancies were due
to another planet perturbing Uranus. After some false starts and
errors, one of them, Le Verrier, sent to the astronomer Johann Galle a
correct-enough prediction of the location of Neptune that Galle found
the object within a degree of the prediction and in under an
hour. This is an inspiring example of using the effect of gravity to
infer the existence of something not yet detected!

The search for solar system objects continued, and continues to this
day. One of these searches in 1930 led Clyde Tombaugh at Lowell
Observatory to find Pluto.

Pluto is an oddball among the planets, and according to the
International Astronomical Union is no longer classified as a planet
but a {\it dwarf planet}---the largest of what we now know as {\it
  Kuiper belt objects}.  I will discuss these objects a bit in a later
lecture, but it is worth pointing out why Pluto is different:
\begin{itemize}
  \item Unlike the rocky inner planets or the gas giants, Pluto is a
    mix of rock (mostly in the inner core) and water ice (in the
    mantle and crust). It is structurally quite different.
  \item Its orbit is not very well aligned with the Ecliptic.
  \item It has not ``cleared'' its orbit---there are many smaller
    objects that share its trajectory around the Sun that combined are
    more massive than Pluto by a long shot.
\end{itemize}
From a taxonomical point of view it does make sense to say Pluto is
like the smaller objects around it more than the other sorts of
planets---but our naming conventions should not blind us when we think
about what the object ``is.''

\clearpage
\section{Lecture 10: Moons in the Solar System}

The Earth is not the only planet with moons around it (although it is
the one  of the eight planets with a moon most similar in size to its
host planet). Mars, Jupiter, Saturn, Uranus, and Neptune all have
moons as well.

\subsection{Moons of Mars: Phobos and Deimos}

The two moons of Mars were discovered in 1877 by Asaph Hall at the United
States Naval Observatory, an institution which still does excellent
astronomy. They are relatively small---Phobos  is about 22 km in size,
and Deimos is about 13 km. They have short orbits around Mars---8
hours for Phobos, 30 hours for Deimos. Like our moon, they are tidally
locked, because the effects of the tides of Mars on them are strong
because they are so close.

In many ways, these moons look like asteroids, which we will look
carefully at later. It is possible that they were captured asteroids,
or that they were products of ejecta when an impactor hit Mars. 

\subsection{Moon of Jupiter: the Galilean Moons and more}

As we've discussed previously, Galileo discovered the moons of Jupiter
with close to his first observations with a telescope. His diagrams
are quite recognizable once you have seen the moons yourself. You can
see these moons quite easily with a telescope, and even through
binoculars.

The four moons he discovered are known as the Galilean moons, from the
inner to the outer called Io, Europa, Ganymede, and Callisto. They
range from size from 3100 km (Europa) to 5100 km (Ganymede). So
somewhat smaller than the Earth, more similar to our Moon. Their
orbits are short: Io is 1.8 days and Callisto is the longest at 17
days.

But these moons are just the largest of more than 60 moons of Jupiter!
The others are much smaller and hard to see. After Galileo's discovery
in 1610, the next one (Almathea) was discovered in 1892; that one is
only 170 km in diameter and is the largest of these other moons.

The Jupiter system has been studied in detail by spacecraft. First by
the Voyager spacecrafts in 1979, and then by the Galileo mission from
1995--2003, and most recently by the Juno mission starting in 2016 and
ongoing as of 2021. Let's look at the moons in detail.

{\it Io} is the closest moon of the Galilean moons, and in some ways
most interesting. It is highly volcanic. If you look at it (from a
nearby spacecraft!), you can see the volcanic features, and even catch
volcanos in action. You can tell that the surface of Io is relatively
``young'' because it doesn't have very many craters from asteroid
impact. However, at first it might seem odd that Io is volcanic. It is
smaller than the Earth, more like the size of the Moon, and you expect
it to have efficiently cooled down; like a Thanksgiving turkey
relative to small roasting chicken, you expect smaller bodies to have
cooled down more. 

Io is volcanic because it is being pulled on by gravitational
tides. It is in a ``resonant orbit'' with Europa and Ganymede. If the
moons all exactly line up, then four Io orbits later they will line up
again, and they'll also line up in a row at 1 and 2 orbits later. In
the ``lined up'' configuration, the gravitational forces induced the
other moons and Jupiter stretch Io's orbit just a little, making its
eccentricity about 0.004. This is still very circular, but it is
enough given how close Io is to Jupiter to induce strong tidal
forces. Just like the moon's tides induces the oceans on the Earth to
move in response (we will discuss this in detail later), Jupiter's
tides will induce Io to stretch. This actually is causing Io to
stretch out, and for this stretching to change throughout its orbit by
as much as 100 m---which is a lot given that Io is largely made of
rock. Just like if you bend a fork or paper clip back and forth it
will warm up, the friction associated with this tidal bending
generates heat, a lot of it! 

Where does this energy come from? It comes from the orbital energy of
Io, which is very large. In the ordinary course of events, the tidal
response of Io and its heating would reduce the orbital energy and
Io's orbit would rather quickly ``circularize''---the minimum energy
for a given angular momentum orbit is a circular orbit. It is the fact
that Europa and Ganymede are in a resonance forcing Io's orbit to stay
slightly eccentric that means the volcanism continues.

This volcanism is actually one of the more amazing theoretical
predictions I've heard of. The paper predicting the effect was written
while Voyager I was approaching but still pretty far away from
Jupiter. It appeared in {\it Science} on March 2, 1979, and keep in
mind that it would have been written, submitted, edited, and published
over the course of the year prior to that. Then on March 8, 1979,
Voyager I flew by Io and lo and behold---volcanos! That is one of the
quickest turnarounds from ``nontrivial prediction'' to
``confirmation'' that I have ever heard of!

{\it Europa} is the second closest moon and also has a ``young''
looking surface and evident geologic activity. Except this surface is
made of water ice. You can see the lineae (lines) clearly. They are
analogous to oceanic ridges in the Earth's crust; neighboring regions
of Europa's surface are pulling apart, and warmer material from below
bubbles up and freezes. Evidently the material below the ice surface
is slushy or liquid! We know the mass of Europa from the gravitational
effects on the other moons and on satellites; this indicates it is
much denser than water, so its interior (below the water ice surface)
needs to be rocky. If you had to pick another place in the solar
system which could currently host life, Europa would be near the top
of your list---it was for this reason that at the end of its eight
year mission Galileo was plunged into Jupiter's atmosphere, so as not
to contaminate any of the moons with stowaway bacteria that may have
survived on Galileo.

{\it Ganymede} is similar to Europa but larger, and is somewhat more
rocky. It also likely has a subsurface ocean. It seems to experience
less ``tectonic'' activity, with a substantial fraction of its surface
covered in craters.

{\it Callisto} is also a mix of rock and water ice. In this case a
subsurface ocean is possible but isn't known for sure. Its surface is
extremely old, and basically saturated with craters.

\subsection{Moons and Rings of Saturn}

The most obvious feature of Saturn is its rings, but it also has over
60 moons. And again, Galileo was the first to note either of these
features.

However, Galileo didn't really understand what the rings were. To him
they looked like two large accompanying moons. But then they seemed to
disappear! This happened because as the Earth moved the change in the
view towards the rings made them appear edge on and thus hard to
see. Then he saw later two half-circles. Only later in the 1600s did
Christian Huygens correctly identify these features as ``rings''
around Saturn. Astronomers then discovered features {\it within} the
rings, most prominently the Cassini gap between the outer ring (A) and
the inner ring (B), discovered by (who else?) Cassini in 1676.

Our view of these rings from spacecraft is excellent---the Voyagers got
great views in 1980 and 1981 and the Cassini mission explored the
whole system in detail between 2004--2017.

What are the rings? They are small bits of ice, ranging frmo
centimeters to meters in size. They don't amount to much---if it were
all gathered into one body, it would only be a few 100 meters
across! They are spread about 50,000 km radially, but are extremely
thin, just a few 100 m. They are likely kept thin by collisions
between ring particles.

The rings show tremendous structure, which is largely caused by
interactions with the many moons of Saturn. Many gaps correspond to
``resonances'' that enhance interaction with outer moons. Some
features, like the F ring on the outside, are kept in place by
``shepherd moons'' (Prometheous and Pandora inside and outside the F
ring, for example). 

They are likely to be young, because their gravitational dynamics
indicates that their orbits should decay slowly and the ring material
should fall into Saturn within a few hundred million years. 

But what caused them to form, then? Either an icey moon was ripped
apart by tides or collided with something, and the rings are part of
the remnant of that process. It is still a mystery!

Of the moons, there are about six easily visible in a telescope,
Iapetus, Titan, Rhea, Dione, Tethys, Enceladus, and Mimas.  By far the
largest, Titan is about 10 times further out than the extent of the
rings and has a period of about 16 days. It was discovered in 1655 by
Huygens, and the rest were discovered by Cassini in the 1670s and
1680s and by Herschel in the 1780s.

Titan is fascinating. It has an extremely thick atmosphere filled with
organic molecules, methane (CH$_4$), and ethane (C$_2$H$_6$), and
perhaps water and ammonia. But it is extremely cold, so the methane is
liquid! So there are giant lakes and clouds of methane and
ethane. There exist ethane rains, ethane rivers, and very likely
ethane springs from subsurface reservoirs. Again, this organic-rich,
dynamic environment lends itself to thoughts of potential life.

Among the other moons, Enceladus is worth a look. Like Io, it
undergoes tidal heating, which drives water geysers from underneath
its surface.

\subsection{The Other Moons}

What I've said so far still just scratches the surface:
\begin{itemize}
\item Uranus has rings. Herschel claims to have seen a ring in 1787,
  though this seems doubtful with his equipment. The rings were
  discovered in 1977 by noticing that they dimmed the light of stars
  behind them, and weren't imaged directly until Voyager flew by. We
  can observe the rings today from the ground with infrared
  imaging. Uranus also has at least 27 moons, the largest two of which
  (Titania and Oberon) were discovered by Herschel in those 1787
  observations.
  \item Neptune has over 14 moons, including the largest, Triton,
    which has an interesting geology. It is mostly ice and rock, with
    a thin nitrogen ice layer, and a thin nitrogen atmosphere. Triton
    was discovered just 17 days after Neptune itself, and is quite a
    bit bigger than all the rest of the moons.
    \item Pluto has a very big ``moon,'' Charon, that was only
      discovered in 1978.
\end{itemize}

There's just too much to cover for a single course, let alone a
lecture. But when you find these objects on the sky, and even through
our telescopes they just look like minute points, remember they are
each their own complex and unique world!

\clearpage

\section{Lecture 11: Precession of the Equinoxes}

Heretofore we have treated RA and Dec as a fixed system relative to
the stars. As you remember, RA and Dec are designed to be aligned with
the rotation axis of the Earth, which to this point we have always
assumed to point in the same direction. However, it turns out that the
Earth's axis of rotation is changing slowly over time---it is {\it
  precessing}. Here I discuss the nature of this precession, its
causes, and its effects.

\subsection{Brief Review}

Recall the basics of the definition of RA and Dec:
\begin{itemize}
\item The Celestial Poles are directly above the North and South Poles
\item The Declination at zenith is equal to the latitude of the
  observer
\item The lines of constant RA  are aligned with the lines of constant
  longitude on Earth.
\item RA$=$0h is defined to be the RA of the Sun on the vernal equinox
  (March 21).
\end{itemize}

\subsection{Precession}

However, the axis of rotation of the Earth changes direction. It
always rotates about the North Pole--South Pole axis, but the
direction that that axis is pointing changes.  It changes in such a
way that the Ecliptic Angle is kept constant. The precession means
that the direction of the rotation axis itself ``circles around'' the
direction perpendicular to the Earth-Sun orbital plane.
The precession period is about 25,770 years. 

In detail there is a little wobble around the precession too, called
the {\it nutation}, but we won't worry about that here.

Why does the Earth's axis of rotation change its direction? Because
the Earth's rotation causes a very slight bulge in its Equator. The
Earth's diameter measured at the Equator is about 1\% larger than the
diameter measured from the North to South Pole. It is squashed in an
``oblate'' fashion like you have pressed on a beach ball (as opposed
to the ``prolate'' fashion of a football or rugby ball).

This bulge is tugged on by the tidal forces of the Moon and the
Sun. The bulge is misaligned with the Ecliptic because the Earth is
tilted by 23.5\degree.  The forces of the Moon and Sun tend to try to
bring it into alignment. But the fact that the Earth is spinning
prevents that from happening and instead the Earth precesses, like a
top does when it is started spinning on a table at a tilt.

The physics here is akin to the physics of an orbit, in the sense that
although the tidal torques are in the direction to align the Earth's
bulge with the Ecliptic, instead of that happening, the alignment of
the Earth's bulge just rotates around in a circle.

\subsection{Change in Position of the Equinoxes}

Although it may seem odd that this effect is called the ``precession
of the equinoxes,'' the term is apt because it directly causes a
change in the location in Earth's orbit at which the Equinoxes occur.

Recall that the direction RA$=$0h is the direction from the Earth to
the Sun when the Earth is at Vernal Equinox, or on what we call on or
about March 21. On that same day, RA$=$12h transits at midnight. But
that position is defined by the direction of the tilt of Earth's axis;
it is tilted so that at the vernal equinox it is tilted in the
opposite direction of the Earth's motion.

If the direction of the Earth's tilt changes, this alignment will
change. The precession of the tilt is changing such that by the time
the Earth has traveled a full orbit, its tilt has moved slightly
clockwise as viewed from the North. So the next vernal equinox occurs
at a slightly different position in the orbit. At this point in its
orbit, in the coordinate system we have defined the Sun will be
slightly west of 0h (about 23h 59m 56.6s) and at midnight the RA
transiting will be 11h 59m 56.6s.

That means that actually Spring starts at {\it that} position. Over
the course of thousands of years this adds up, and the position of the
equinox slips further and further, going all the way around over the
course of 25,770 years.

\subsection{Calendar (Tropical) Year vs. Sidereal Year}

The nature of this effect is that the vernal equinox recurs in
slightly less time (by about 20 minutes, or 1/26,000 or so of a year)
than it takes for the Earth to actually orbit its full orbit.

The time between vernal equinoxes is the actual time we humans we use
to define the {\it calendar year} or {\it tropical year} of 365 days,
5 hours, 48 minutes, and 45 seconds. That's because we need the year
to reflect the periodicity of the seasons. The definition of the year
ought to tell us when to plant seeds, when to plan to let animals out
of the barn, when to plan ski holidays, when to plan summer holidays,
etc.

But the calendar year is 20 minutes shorter than the {\it sidereal
  year}, the time the Earth actually takes to actual finish exactly
one orbit.

We've seen this effect before, for the sidereal vs. synodic period of
the Moon, or the sidereal vs. solar day. But because the precession is
clockwise as seen from the North instead of counterclockwise (like the
orbit of the Earth and Moon), the sidereal year is slightly longer
than the calendar year instead of shorter (like the sidereal vs solar
day is). 

\subsection{Keeping RA \& Dec Aligned}

The precession causes an issue with the RA/Dec system, which leads to
this system having to be redefined periodically.

The issue is that RA and Dec are designed to be aligned with longitude
and latitude on the Earth. And we want stars' RA and Dec to stay
roughly fixed over time. But if the orientation of the spin axis of
the Earth is changing, then this alignment cannot hold over long
periods of time. 

For example, about 13,000 years from now, the rotation axis of the
Earth will have precessed all the way around and will be
47\degree\ ($=2\times 23.5\degree$) from its current position. So in
our current RA/Dec system the axis will not be pointing to
Dec$=+90$\degree\ as it is now, but instead to Dec$+=43$\degree and
RA$=18$h. This misalignment will mean that as the Earth rotates, it
will no longer be true that the declination at zenith will depend only
on your latitude---it will depend on time. And the meridian will no
longer be a single RA.  The same is true even 50 years from now, just
the misalignment over small time periods is much smaller.

How do we deal with this? Well clearly we want RA/Dec to stay aligned
with the Earth's axis of rotation. That means that this coordinate
system changes over time. In practice, astronomers redefine the system
every 50 years. So there was coordinate system based on the 1950
equinox, based on the 2000 equinox, and based on the 2050 equinox, and
so on.

The RA and Dec positions of stars changes by at most around
0.5\degree\ or so in each redefinition. For many purposes this is
small (e.g. your Mag 5 atlas is in 1950 coordinates and is still fine
to use); but it is huge relative to the accuracy that professional
telescopes need, so it is absolutely necessary to account for.

It is important to remember that this effect is not because the {\it
  stars} are moving. The change in their coordinates is because we are
changing the coordinate system to stay aligned with the Earth's
rotation.

\subsection{Stars and the Seasons}

What does change from a human point of view is that the stars which
are visible in different seasons will change over time. If you pay
attention to the stars in the early evening sky over the course of the year
you will notice the Orion constellation up in the winter, giving way
to Gemini and Leo, and then Hercules and Lyra in the summer, followed
by Cygnus and later Pegasus in the Fall.

But as the Earth's rotation axis changes, the location of the
equinoxes and solstices changes in space. This means that Winter
occurs at a different place in the Earth's orbit, and thus the stars
that are visible during specific seasons will change.

Incidentally, this also means that the constellation that the Sun is
in on a specific day of the calendar year will change too. If you look
at a published set of horoscopes, you will find that it associates
certain ranges of days with certain signs of the zodiac---i.e. which
constellation the Sun is in on those days.  But if you look carefully,
the Sun is actually not in those constellations on those days---the
published horoscopes refer to where the Sun would have been on those
days some year roughly 2000 years ago. Luckily this does not affect
the accuracy of the horoscopes (they are still completely bunk).

\subsection{Discovery of the Precession}

The precession of the equinoxes is a pretty subtle effect to the
casual observer. You definitely would never notice it just looking up
at the sky with your eyes over the course of your lifetime, even if
you were a careful observer.

With precise instrumentation, say what Tycho Brahe had or more modern
equipment, the precession can be measured over the course of
observations over a few decades. You might think that the discovery of
precession had to wait until that instrumentation exists.

However, because our predecessors on Earth included great geniuses,
the astronomer Hipparcos did manage to discover this precession around
2000 years ago. He could do so because astronomers were keeping careful
records over long periods of time---more than a single person's
lifetime---and by cleverly using lunar eclipses.

At the time of a lunar eclipse, you can tell exactly what position on
the sky is exactly opposite the Sun. Then you can estimate the
position of stars near the Moon at that time, relative to the position
opposite the Sun on that date of the year. If you get another lunar
eclipse later in the same area of the sky (which as we will learn
later is likely to take a long time to recur) then you can measure the
star again. You would ideally have two measurements on the same day of
the year (but separated by many years), but even if you don't you can
adjust for being a few weeks off pretty accurately.

Hipparcos measured the location of a star Spica in this way around 128
BCE, and compared its position relative to the direction opposite the
Sun on a specific night of the year. An astronomer named Timocharis
had in the prior century, sometime in the 200s BCE, made the same
measurement. Hipparcos found that the position of the star---and the
other stars---measured in this way had shifted about 2\degree. Ptolemy
wrote about this about 300 years later in 139 CE, and updated the
measurements with new observations. These were not perfectly accurate,
and derived a precessional period a bit too long, around 36,000 years
instead of 26,000 years. But it is an impressive observation
nevertheless, requiring hundreds of years and a number of astronomers,
and carefully recording, storing, and sharing astronomical information
over long periods of time!

There are various theories that previous civilizations, e.g. the
Babylonians and Egyptians, had discovered precession, but the evidence
seems to be thin. It does seem weird that the Egyptians, whose
monuments were designed to be carefully aligned with the stars, and
which stood for hundreds or thousands of years, didn't record the
effect.

Regardless, a final side note is worth making that none of the
astronomers who considered parallax, Hipparcos, Ptolemy, or later
ones, successfully attributed it to its cause---the changing rotation
axis of the Earth. Of course this is because they had concluded the
Earth was not rotating. It was Copernicus who first wrote down this
precession as one of the motions of the Earth necessary to explain
astronomical observations.

\clearpage

\section{Lecture 12: Tides and Eclipses}

We will look today in greater detail at the effects of tides and
eclipses. I've referred to these before but now we are going to learn
a bit more about how they work.

\subsection{Tides}

The basic reason for tides has to do with the fact that the force of
gravity depends on distance. For an object of large enough size,
different parts of it experience different gravitational forces. These
differences are what we call {\it tidal forces}.

The Moon and Earth pull on each other gravitationally and this keeps
them in orbit. If they were both very small masses that is all that
there would be to it.

But the force of gravity falls as the square of the distance $R$:
\begin{equation}
F = \frac{GM_{\rm Moon} M_{\rm Earth}}{R^2}
\end{equation}
The distance to the near side of the Earth is smaller than the
distance to the center, which is smaller than the distance to the far
side. So the force towards the Moon is a bit stronger on the near side
of the Earth than  on the center of the Earth, and the force towards
the Moon is a bit weaker on the far side of the Earth than on the
center of the Earth.

For the mathematically inclined, you will be able to show that the
difference in force across the Earth is:
\begin{equation}
\Delta F = F_{\rm near} - F_{\rm far} \approx D_{\rm Earth} \frac{{\rm d}
  F}{{\rm d}R} \propto - \frac{D_{\rm Earth} }{R^3}
\end{equation}
If the above equation is gibberish to you do not worry about it. All
it says is, is that the difference in force across the Earth is
proportional to the size of the Earth divided by the cube of the
distance between the Moon and the Earth.

What is the effect of this difference in forces? It pulls the near
tide of the Earth more towards the Moon, and the far side less. This
pulls the Earth into a slight egg shape. But the Earth is rotating
relatively quickly beneath the Moon, so even over time this effect
will not distort the solid parts of the Earth permanently (or
temporarily) into that shape. On the other hand, lots of the surface
of the Earth is covered in water, which is more easily and rapidly
moved by these forces.

The result is that the water responds and there is a (slight) bulge of
water that on average is roughly in the directions towards and away
from the Moon. Because the Earth is turning and it takes some time for
the water to respond to gravitational forces, the bulge slightly leads
the Moon, by about 10 degrees. 

The result for an observer fixed on Earth, say in the middle of the
ocean, is that as the Moon is transiting or the Moon is on the
opposite side of the sky (HA=0h or HA=12h), the tide is high (the
ocean is a bit deeper). When the Moon is setting or rising the tide is
low (the ociean is a bit more shallow).

The period of this effect is roughly half a day, since it repeats when
the Moon is at HA=0h and at HA=12h. In fact it is a bit longer than
half a day since the Moon is moving eastward on the sky, at a rate
equal to about $1/27.3$ of a full orbit per day. This means that the
periodicity of the tides is actually about 12h 25m. So if you go to
the beach at 1pm one day and it is high tide, that means that the next
high tide is at 1:25am, and that the next day high tide will be
1:50pm.

One complication to this story is the role of the Sun. The Sun is very
far away, but it is much more massive than the Moon. You can use the
equations above to show that the Sun contributes about half the tidal
force of the Moon---so it is quite significant! This will change when
the high tides are a little, and also means that those parts of the
month when the Sun and Moon are aligned (new and full Moon) have the
most extreme tides---these are known as ``spring'' tides. The first
and third quarters have the least extreme tides---these are known as
``neap'' tides. If you regularly spend time near the shore, and pay
attention to this effect, it is very noticeable!

Another important effect is how close the Moon is. At perigee, the
Moon is about 10\% closer than at apogee. Since the the strength of
the tidal forces scales as $1/R^3$, this means about 30\% stronger
tidal forces ($1.1^3 = 1.1\times1.1\times 1.1\approx 1.3$). So if the
Moon happens to be near perigee, the tides will be more extreme.

The strength of the tides also depends on your latitude. In the
direction of the tidal bulge in the Earth, the tides will clearly be
highest. In the tropical regions, where the Moon (like the Sun) can be
directly overhead, the high tides can be more extreme.

An even greater complication is that you will notice I was careful
above to refer to an observer in the middle of the ocean before. If
you are near the shore, or in a bay, or even a large sea like the
Mediterrean, there are a bunch of effects that will basically delay
the high tide. For this reason, in practice high tide is not always
near HA=0h or HA=12h for the Moon; again, if you pay attention you
will notice this. Simply put, it takes some time for the water to
respond and the delay depends on the details of the underwater
geography. This is an immensely complicated system, which is why every
locality has its own tidal charts and tables that you need to consult
to predict high and low tide. 

Incidentally, recall earlier in the semester I talked about the fact
that the Moon was drifting away from us. It is the tidal bulge on the
Earth that causes this. Since the tidal bulge ``leads'' the Moon
gravitational forces between the tidal bulge on the Earth and the Moon
tend to give the Moon more angular momentum. This pushes it out in its
orbit. This is a transfer of energy from the Earth's rotation to the
Moon's orbit. So it makes the Moon's orbit further away, and slightly
lengthens the length of the day (by about 0.002 seconds per century).

This process already happened to the Moon, in the sense that it was
once rotating faster than it is now. But for the same reasons, it was
slowed down until it exactly matched its orbital rate, at which point
the tidal bulge on the Moon was exactly aligned with its tidal forces.

\subsection{Eclipses}

A fascinating phenomenon happens due to the Moon's size and distance,
which it is just possible for the Moon to (barely) cover the Sun's
disk, and the Moon can also fit within Earth's shadow. For this
reason, we can experience eclipses of both the Sun and the Moon.

The Earth-Sun distance is about 150 million km, and the Sun's diameter
is 1,400,000 km, or about a factor of 100. The Earth-Moon distance is
about 380,000 km, and the Moon's diameter is about 3,500 km, also
about a fact of 100. In detail the angular size can be calculated to
about about 0.5\degree for the Sun:
\begin{equation}
\theta \sim 60\degree \times \frac{\rm Solar Diameter}{\rm 1 AU} \sim
0.5\degree
\end{equation}
It is similar for the Moon, though the angular size of the Moon varies
depending what point in its orbit it is in.

It is a bit of a coincidence that the Moon is just the right size to
cover the Sun! A billion years ago, the Moon would have been about 4
cm $\times10^9 = 100,000$ km closer, and would have been bigger on the
sky than the Sun by a bit, and a billion years from now, it will be
too small on the sky to produce a solar eclipse.

\subsection{The Geometry of Shadows}

If you put an object in front of a light source, it will cause a
shadow. Since the Sun outshines everything else in the solar system by
a lot, it makes a huge difference to be inside or outside the solar
shadow of an object.

There is a region called the {\it umbra} directly behind the object
which is ``fully'' in shadow---from within the umbra you can't see the
Sun at all. The umbra is shaped like a cone, with an opening angle of
0.5\degree for an object at 1 AU from the Sun (i.e. the angular size
of the Sun as seen from near the Earth's orbit). Adjacent to the umbra
is a region called the {\it penumbra} where the object ``partially''
covers the Sun---from within the penumbra, you can see parts of the
Sun but not others.

\subsection{Solar Eclipses}

We first consider the case that the object casting the shadow is the
Moon; these shadows cause solar eclipses. If the opening angle of the
umbra is 0.5$\degree$, we can calculate using the diameter of the Moon
$D_{\rm Moon}$ that the {\it length} $L$ of the Moon's umbra is:
\begin{equation}
L \sim D_{\rm Moon} \frac{60\degree}{0.5\degree} \sim 380,000 {\rm km}
\end{equation}
This is nearly the average distance of the Moon from the Earth. This
is really just a restatement of the fact that the Moon and Sun have
nearly the same angular size as viewed from Earth.

But looking at the umbra, we see that when the Moon is at perigee, the
umbra intersects the surface of the Earth! That means that the Sun can
be completely blocked by the Moon as viewed from the surface of the
Earth. This situation will cause a {\it total} solar eclipse for
anybody inside the umbra. When the Moon is at apogee, the umbra does
not intersect the Earth, so a total eclipse is not possible.

Even at perigee, the part of the umbra that intersects the Earth is
only a few hundred km in diameter. Furthermore, the Moon (and
therefore its shadow) is moving at about 4000 km/hour. So the shadow
will move across the diameter of the Earth in just a few hours. Since
the Earth is rotating in the same direction as the Moon is going, an
observer on Earth will see the velocity of the shadow as only about
half of its actual velocity, which lengthens the eclipse.

But the point is that to see the total eclipse, you need to be at a
very narrow region on the Earth at exactly the right time!

The same set of considerations governs when we can see an {\it
  annular} eclipse, where just the outer ring of the Sun is left
visible. These occur when the Moon is further out in its orbit, but
viewing the ring geometry (rathen than a crescent) is only possible in
a narrow part of the Earth.

The region of partial Eclipse is much larger. It is the size of the
penumbra. Given the large distance between the Earth-Moon system and
the Sun, the penumbra is about twice the size of the diameter of the
Moon, or 7,000 km. That is a sizeable chunk of the Earth, so partial
eclipses don't require you to be in as precise a location. Of course,
near the edge of the penumbra, only part of the Sun is covered, so the
effect is less dramatic.

\subsection{Lunar Eclipses}

The Earth also casts a shadow, and the Moon can fall into that
shadow. That is called a {\it lunar} eclipse and can also be {\it
  total} or {\it partial}. 

Since the Earth is bigger than the Moon, its umbra is larger and
longer. At the distance of the Moon, the, the diameter of the Earth's
umbra is about 9,500 km, so the Moon fits easily into it. When the
Moon passes through the umbra it is called a total lunar eclipse.
When the Moon passes through the penumbra it is called a partial lunar
eclipse. The Moon remains visible in either case, but will appear much
dimmer (and redder) in a total lunar eclipse and a little dimmer in a
partial lunar eclipse.

\subsection{Eclipse Seasons}

Why don't solar eclipses occur every new Moon and lunar eclipses every
full Moon? It is because the Moon's orbit is inclined relative to the
ecliptic. The eclipses will only occur when the Moon's orbit crosses
the ecliptic at the same time as it is either new or full. 

There are only two times a year when this can occur, which is when the
ellipse defining the Moon's orbit crosses the ecliptic right along the
Earth-Sun line. Within a few weeks of these points in the Earth's
orbit, if there is a new or full Moon in that period (which doesn't
always happen), an eclipse may occur. These are the {\it eclipse
  seasons}.

But the alignment of the Moon's orbit is changing, because of
gravitational interactions of the Moon with the Sun (basically, due to
similar physics to why the Earth's axis of rotation precesses). So the
intersection of the Moon's orbit with the ecliptic changes over
time. The consequence is that the eclipse season slips earlier each
year by about 2.5 weeks. The time between the center of one eclipse
season and the next is known as the {\it eclipse year} and is about
347 days.

NASA provides excellent predictions for when the lunar and solar
eclipses will occur. Keep an eye out in April 2024 for a total solar
eclipse in the US, and book your travel soon! 

\section{Small Bodies in the Solar System}

There are not just planets and moons in the solar system. As I've
alluded to before, there are many smaller objects, mainly asteroids
and comets, as well as much smaller bits of dust floating in the
tenuous gas between the planets. These are not just incidentally
important---they yield important clues to the origin of the solar
system and the primordial gas from which the Sun and planets
formed. Occasionally they provide spectacular shows in the night
sky---as comets periodically do---and occasionally they collide with
planets, including our own.

\subsection{Spherical vs. Non-spherical Objects}

We have excellent pictures of asteroids now from space missions within
the solar system. It is clear that they are quite lumpy and
non-spherical, whereas the larger moons and all planets are close to
spherical. This is an important watershed difference betweeen
``small'' and ``large'' objects.

Above a few 100 km in size, objects have enough gravity to pull
themselves into a spherical shape. If a large lump (i.e. a mountain)
develops, it tends to erode (i.e. the material will slide down
eventually and reduce the size of the mountain).  But below that size,
objects aren't massive enough for that to happen---so if they form
lumpy or peanut shaped or whatever, they will stay that way!

Essentially all asteroids and comets are on the small side of this
divide. 

\subsection{Asteroids}

Asteroids are extremely common in the solar system. Down to the
smallest sizes (say 50--100 meters) there are probably about 100 million of
them. We know of a few million, mostly on the larger side. They are
primarily in the asteroid belt between Mars and Jupiter, though as I
discussed before in the early phases of the solar system they would
have been spread throughout.

They have survived in the asteroid belt because there is a gap between
Mars and Jupiter where the gravitational interactions with planets
will not disturb them. However, occasionally collisions between
asteroids or gravitational disturbances from the outer planets will
drive some of them out of the asteroid belt and into Earth-crossing
orbits or out of the solar system.

There are some other stable places in the solar system. For example,
the Trojan and Greek populations around Jupiter are in special stable
locations called Lagrange Points (after an excellent French
mathematical physicist).

They are basically made of rock, but two different sorts of rock. The
first is called {\it chondrites}, which are primitive, unprocessed
rocks. The chondrites are thought to be pretty much untouched since
the very first moments that dust in the solar system started to stick
together to form pebbles and then larger objects.

The second are more heterogeneous, but have been noticeably processed
or {\it differentiated}. When a large enough object forms, the iron
settles to the center, and its center becomes iron-rich, and the outer
parts are more iron-poor. If that object undergoes a collision and
shatters, some of its remnants will be iron-rich, and some will be
iron-poor. This description is a simplification of a much broader set
of chemical differences. This process has apparently happened, so a
lot of the asteroids actually were once part of a smaller number of
more massive objects, subsequently destroyed!

How do we find these asteroids? We look at the same part of the sky
over and over, and look for things that move! If something is moving
faster over just a few days or weeks, it must be in the solar
system. Surveys of the sky have found millions of these objects.  The
colors of asteroids in the images tend to indicate what ``family''
they live in, which is a signature that these groups are likely to
have formed together, or perhaps are the remnants of the same
progenitor.

\subsection{Near Earth Objects}

As humans who don't like large explosions destroying the places where
we live, we have a special interest in near-Earth asteroids. There are
not as many of these as there asteroids in the asteroid belt, but they
do exist.

When they pass by Earth, they give ground-based astronomers a chance
to study them carefully, and see their rotation and surface features
directly. But occasionally they come close enough to be a bit more
alarming. 

The Minor Planet Center at Harvard keeps track of their orbits on the
basis of amateur and professional observations. A recent survey of the
sky from Hawaii called ``Pan-STARRS'' discovered many such objects. A
new large survey at the Vera Rubin Observatory will start to observe
the ecliptic for very small asteroids in the next few years, which
will greatly increase our knowledge of the potentially hazardous
asteroids.

Currently, the closest anticipated approach of a known asteroid is
from 99942 Apophis, which is about 300-m in size and will come within
about 40,000 km in April 2029. It will be extremely dramatic! It will
appear like a 3rd magnitude star, but will move at about a degree per
minute across the sky. There is literally no chance it will hit us
though! 

Of course, asteroids of that size do occasionally hit---like every 50
million years or so---and the results are pretty devastating. Recall
that we see the resulting craters easily on the ancient surface of the
Moon, for example the Tycho crater which resulted from a collision
about 100 million years ago. That was probably caused by a few km
sized asteroid.

On Earth, these craters do not survive as long because of tectonic
activity on the Earth, the effects of erosion, wind, water, and
biology.  But remnants of old craters can be seen in geographical
features around the world, some dating back hundreds of millions of
years. Manicouagan in Quebec was formed 210 million years ago. It is
about 70 km wide. When the impactor hit, the center of the crater
melted to extreme depths (kilometers); subsequently, the magma
underneath lifted the weakened center, forming a ring valley.

Another famous example, the Chicxulub Crater in the Yucatan, was
formed also by an impactor about 10 km in size, about 66 million years
ago. It is not as obvious; the crater features are mostly
subsurface. The first evidence for it was found in the 1940s by Pemex,
the Mexican oil company. Pemex scientists realized in 1978 that it was
a crater remnant, and announced it in 1981, although they couldn't
release all the evidence, which was proprietary because of its value
for finding oil and gas. Because of the date of the impact, which
coincides with the end of the Cretaceous period, and a number of other
lines of evidence, this event has been associated with the killing off
of the dinosuars (oh and about 75\% of all other species), at least
those that didn't later evolve into birds.

Smaller impactors of course exist. A small crater from just 50,000
years ago survives in Arizona (``Meteor Crater'') caused by an
asteroid a 50-m in size or so. If the objects are much smaller than
that, they will tend to mostly vaporize in the atmosphere. They can
still cause damage---the event in Tunguska (remote Siberia) in 1908
destroyed about 800 square miles of forest, even though the impactor
was mostly vaporized a few miles up on the atmosphere.

But most impactors are even smaller than that. If a few meter size
object enters the atmosphere it will totally burn up but will cause a
dramatic fireball in the sky. The typical ``shooting star'' you might
see a few times a night is even smaller---a few mm in size! These are
often associated with the dust trails associated with comets; there
are dramatic ``meteor showers'' when these shooting stars are very
common, corresponding to the nights when Earth is crossing the orbit
of a comet (even if the comet passed many years previously!).

\subsection{Meteorites}

The remnants of any asteroid that reach the ground can be later found
and studied, and they are called {\it meteorites}. The largest
meteorite ever discovered is Ahnighito, found in Greenland and on
display in the American Museum of Natural History. It is a very
iron-enriched meteorite; you can go see yourself! But there are many
smaller ones.

The majority of asteroids are the unprocessed chrondrites, and they
form the majority of meteorites that reach Earth. But the majority of
meteorites collected in human collections are in fact iron
meteorites. That is because when you dig a big iron-y rock out of your
field, it looks obviously weird. Chrondrites look a lot more like
ordinary rocks, so must often go unnoticed, and also weather much more
readily. 

The best place to find meteorites on Earth is probably Antarctica,
where a few effects conspire to make them easy to find (once you have
gotten there!). First, Antarctica is covered in ice, which is white,
and meteorites are dark, so easy to see. Second, the ice in Antarctica
isn't stationary, but tends to flow, and part of that flow converges
at a mountain range, at which point the ice is pushed up and eroded by
wind (``ablated''). But the meteorites are left at the surface. So a
large part of the continent ``catches'' the meteorites, and over a few
thousand years delivers them to a concentrated strip adjacent to the
mountains, making them easier to find. 

As I've noted before, some of the meteorites that have been found have
been traced to other large objects in the solar system, for example
the Moon and Mars. So some object hit the Moon or Mars, some material
escaped, and then floated around the solar system for a few million
years until happening to hit the Earth! We aren't quite as isolated as
we tend to think of ourselves.

\subsection{Comets}

Comets form the other major class of small bodies in the solar
system. In general, they are found further out than asteroids.  If you
think of asteroids as very small rocky planets, you can think of
comets as very small dwarf planets. They are low density mixtures of
water ice and dust and rock, with maybe a light crust of rock.

They come in two varieties: short period comets ($<200$ years) that
tend to orbit in the ecliptic plane, and long period comets that seem
to come from any direction. So Comet Halley (76 years) is a short
period comet, but Comet Hyakutake is a long period comet (17,000
years). 

They can be very spectacular on the sky. When they are far from the
Sun, they are pretty inert and look pretty much like asteroids. But
when they approach the Sun, their surfaces become very active because
of the increased solar flux and the interaction with the solar
wind. They emit gas and dust from their surfaces, which each forms a
distinct tail. The gas is hydrogen, carbon monoxide, and cyanide gas,
which tends to be blow directly away from the Sun by the solar
wind. The dust lags a bit because it doesn't as easily get entrained
in the solar wind, and so forms a tail angled a bit away from the gas
tail, in a direction opposite the comet's motion. 

The activity on the surface of comets tends to affect their
orbits. Comets don't {\it exactly} follow Newton's Laws, but act as if
the Sun's gravity is a little weaker than for other objects. This
effect was discovered in the 1950s. What appears to be happening is
that the emission of gas and dust is preferentially on the sunward
side of the comet, and as the gas and dust flies off in that
direction, it pushes the comet in the opposite direction away from
the Sun.

Comets are reasonably fragile and can be ripped apart easily by tidal
forces, or because of their own emission of gas and dust, or for
apparently no reason at all. In 2007, Comet Holmes pretty much just
exploded, which was pretty cool to see. In the early 1990s, the Comet
Shoemaker-Levy was found to have disrupted due to a tidal interaction
wiht Jupiter. It was then realized that it would actually come back
and its remnants would {\it hit} Jupiter. Unfortunately, it hit the
far side of Jupiter so these impacts could not directly be seen, but
in the hours afterward Jupiter rotated around and we could observe the
effects of the impact on the atmosphere. There are good reasons to
study this effect to understand Jupiter's atmosphere better, but also
it is a pretty neat thing to see!

Where do the comets come from? It seems as if the short and long
period comets should have different sources---the long period ones
come from all directions and the short are from the ecliptic. The long
period comets have long been assumed to come from a roughly spherical
region that extends very far around the solar system and contains a
huge number of these objects, called the Oort Cloud, after the
astronomer who initialized hypothesized its existence. The periods of
the long period comets are so long, that these objects have to be so
far away (1,000s of AU) that they can't possibly be seen from Earth,
so it has never been discovered.

For this reason, it was hypothesized around 1980 that the short period
comets came from a region in the outer solar system where there were
many dirty iceballs; in the early 1990s objects in this region were
discovered! This region is now called the Kuiper Belt, after an early
solar system theorist who postulated a similar region in the early
solar system (but did not postulate that it still existed).

\subsection{Interstellar Objects}

If the Oort cloud extends to many thousands of AU, it starts to look
more and more like an interstellar system of objects. The comets we
observe are almost uniformly gravitationally bound to the Sun, however
tenuously. However, there have now been discovered exceptions to
this---objects that have passed through the inner parts of the solar
system at such high speed that they appear to be {\it
  interstellar}---to not be bound to the Sun and just ``passing
through.''

The first was discovered in 2017 by Pan-STARRS, and is named
\`{}Oumuamua. It is only about 100 meters in size. It appears to be
``tumbling'' because it is not rotating about an axis of symmetry, and
its tumbling leads its brightness to change. From studying the change
in brightness you can try to infer its shape. It is unclear exactly
what its shape is: it could be {\it very} elongated like a cigar
(about 8 times long than thick), or it could be flat like a pancake
(about 6 times wider than thin). Or it could have some more unusual
shape, but either of those shapes is pretty unusual. It also had an
usual effect that it seems to have accelerated on its way out of the
solar system (just a little bit). The amount of acceleration is
similar to what comets experience, but there wasn't any observed
outgassing like you see in a comet. So all in all, a very strange
object. However, none of this is strange or definitive enough to
justify the irresponsible speculation of some physicists that this was
an alien solar sail or something like that!

The second was discovered in 2019 by an amateur astronomer named
Gennady Borisov, and is called 2I/Borisov. It is pretty much just a
regular comet. It will leave the solar system at quite high velocity,
around 30 km/s, off towards the stars in the constellation Telescopium
(yes, that exists) in the southern sky.

\clearpage
\section{Final Word}

There is really too much to talk about in thirteen lectures, even
limiting to the material relevant to looking at the sky with
binoculars and small telescopes. If you want to keep observing the
sky, I encourage you to go outside every week or so and take a look at
the stars and try to find one new thing. Either a faint star by
looking with your eyes, or perhaps an open or globular cluster with
binoculars. If you find yourself wanting more, I'd recommend a
reasonable quality telescope with a solid mount; the best bang for the
buck are Dobsonian mount telescopes like the long cylindrical ones in
lab. Of course if you become very serious you will have to spend
serious money on a telescope like the Meades that we have been mostly
using. But you don't have to be {\it very serious} to enjoy and
understand the night sky a little more every week!

\end{document}
