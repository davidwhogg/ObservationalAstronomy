\noindent{\bf Observational Astronomy / PHYS-UA 13 / Spring 2016 /
  Syllabus }

\noindent This course will teach you how to observe the sky carefully  
with your naked eye, binoculars, and a small telescope. You will learn
the basics of observable lunar and planetary properties, and the
basics of astronomical coordinates and observations. The goal is for
you to be able to understand and describe what you see in the sky at
night, and to be able to use charts and coordinates to predict it.

\noindent The instructors are: Prof. Michael Blanton (Meyer 523, {\tt
  mb144@nyu.edu}), whose office hours are Weds 11am--12:30pm (or by
appointment), and TA Nitya Mandyam Doddamane ({\tt nityamd@nyu.edu}).
 
\noindent The primary textbooks are: 
\begin{itemize}
\item {\it The Ever-Changing Sky}, by James Kaler
\item {\it Edmund Mag 5 Star Atlas} 
\item {\it Peterson Field Guide to the Stars and Planets}, by Jay
Pasachoff
\item This laboratory manual. 
\end{itemize}

\noindent Each week you will attend one lecture (at 3:30pm Monday in
Bobst LL139) and one lab. 

\noindent {\bf Arrive for the lab (on time!) at 7:00pm in Meyer 224}, where
we will discuss the contents of the lab and then (when appropriate) go
to the observatory. {\bf Starting the week of March 21, we will move
the lab time later to 7:45pm} to accommodate the later sunset and
daylight savings time.

\noindent You cannot switch between the lab sections, because in
general they will be on different schedules.  The timing of the indoor
and outdoor labs for each section will be driven mostly by the
weather. Welcome to observational astronomy!

\noindent For the labs: {\bf you MUST arrive on time}, or else you will not
be able to access the observatory.  In addition, please dress
appropriately for remaining outside for an extended period, including
hats and gloves when appropriate.  {\bf Dress warm!}

\noindent {\bf Attendance in lab is not optional!}  You can miss one lab
during the semester without penalty: you must however contact the lab
instructor explicitly {\bf beforehand} to claim this credit. If you
are absent for any other without good cause you will lose credit for
that lab.  If you miss more than three sessions without good cause,
you will not be given a passing grade no matter how you perform in the
class otherwise.

\noindent {\bf Do not use your phones during lab unless directed
to!}

\noindent Grades are based on labs (20\% on written material, 5\% on
extra good work), homeworks (10\%), the midterm (30\%) and the final
(35\%). 

\noindent For the homework, there is a sheet of ``Lecture Questions''
to answer. These are due at the beginning of each lecture.  {\bf Late
homeworks will not be accepted.}  Two of these questions will be
chosen to be graded each week (obviously, we won't tell you which two
beforehand).

\noindent For the midterm and the final you are responsible for
material in the labs, the reading, and the homework. In preparing for
the exams, use the homeworks as a guide to which material I believe is
essential.

\clearpage

\baselineskip 0pt
\begin{sidewaystable}
\small
\begin{tabular}{|c||c|c|}
\hline
{\it Jan.~25} 
& The Celestial Sphere: angles \& coordinates 
& Kaler Ch.~1; Edmund pp.~1--9; Peterson pp.~495--498 
\cr 
{\it Feb.~1} 
& Introduction to telescopes
& Kaler Ch. 13.8--13.16; Peterson pp. 503--508
\cr
{\it Feb.~8} 
& Rotation and Orbit of the Earth
& Kaler Ch. 2.1--2.6, 2.12, 3.1--3.13
\cr
{\it Feb.~15} 
& PRESIDENT'S DAY: Note there is still Wednesday lab!
& ---
\cr
{\it Feb.~22} 
& Stars
& Kaler Ch.~4.1--4.10; Peterson pp.~144--170
\cr
{\it Feb.~29} 
& Finding your way in the sky
& Edmund pp.~30--32; Peterson pp.~7--20, pp.~32--38, pp.~46--50
\cr
{\it Mar.~7} 
& {\bf Midterm exam in class!}
& ---
\cr
{\it Mar.~14} 
& SPRING BREAK
& ---
\cr
{\it Mar.~21} 
& Variables and Binaries
& Kaler Ch.~4.11--4.17; Peterson pp.~194--208, pp.~149--185
\cr
{\it Mar.~28} 
& Galaxies
& Kaler Ch.~4.13, 4.17; Peterson pp.~171--186
\cr
{\it Apr.~4} 
& The Moon
& Edmund p.~34; Kaler Ch.~9.1--9.5; Peterson pp.~348--359
\cr
{\it Apr.~11} 
& Planets and their motions
& Kaler Ch.~11.1--11.13; Peterson pp.~385--395
\cr
{\it Apr.~18} 
& Moons of Jupiter and Saturn
& Kaler Ch.~12.1; Peterson pp.~418--432
\cr
{\it Apr.~25} 
& Precession \& nutation
& Kaler Ch.~5.1--5.10
\cr
{\it May.~2} 
& Tides \& Eclipses
& Kaler Ch.~10; Peterson pp. 359--363, pp.~481--491
\cr
{\it May.~9} 
& Asteroids, Comets, Meteors
& Kaler Ch.~12.1--12.5; Peterson pp.~455--473
\cr
{\it May.~16} 
& {\bf FINAL EXAM 6:00pm Bobst LL139}
& ---
\cr
\hline
\end{tabular}
\end{sidewaystable}

\clearpage

\baselineskip 0pt
\begin{sidewaystable}
\small
\begin{tabular}{|c||c|c|c|}
\hline
{\it Jan.~25--27} 
& Orion, Andromeda, Uranus
& Full Moon (visible 25th)
& Sunset 5:05pm EST
\cr
{\it Feb.~1--3} 
& Orion, Andromeda, Uranus, Jupiter rising
& ---
& Sunset 5:13pm EST
\cr
{\it Feb.~8--10} 
& Orion, Andromeda, Uranus, Jupiter rising
& ---
& Sunset 5:22pm EST
\cr
{\it Feb.~17}
& Orion, Andromeda, Uranus, Jupiter rising
& ---
& Sunset 5:33pm EST
\cr
{\it Feb.~22--24}
& Orion, Andromeda, Uranus, Jupiter
& Full Moon (rising 8pm on 24th)
& Sunset 5:39pm EST
\cr
{\it Feb.~29--Mar.~2}
& Orion, Jupiter
& ---
& Sunset 5:47pm EST
\cr
{\it Mar.~7--9}
& Orion, Jupiter, M3 rising
& ---
& Sunset 5:55pm EST
\cr
\hline
{\it Mar.~14}
& Spring Break
& ---
& New lab time: 7:45pm EDT
\cr
\hline
{\it Mar.~21--23}
& Orion, Jupiter, M3 rising
& Full Moon
& Sunset 7:10pm EDT
\cr
{\it Mar.~28--30}
& Orion, Jupiter, M3 rising
& ---
& Sunset 7:17pm EDT
\cr
{\it Apr.~4--6}
& Jupiter, M3, Mercury 10$^\circ$ at 7:30pm
& ---
& Sunset 7:24pm EDT
\cr
{\it Apr.~11--13}
& Jupiter, M3, M5 rising, Mercury 10$^\circ$ at 8:00pm
& Waxing Crescent to 1st Quarter Moon
& Sunset 7:32pm EDT
\cr
{\it Apr.~18--20}
& Jupiter, M3, M5 rising, Mercury 10$^\circ$ at 8:25pm
& Approaching Full Moon 
& Sunset 7:39pm EDT
\cr
{\it Apr.~25--27}
& Jupiter, M3, M5 rising, Mercury 10$^\circ$ at 8:15pm
& ---
& Sunset 7:46pm EDT
\cr
{\it May 2--4}
& Jupiter, M3, M5, Mars at 10pm
& ---
& Sunset 7:53pm EDT
\cr
{\it May 9}
& Jupiter, M3, M5, Mars at 10pm, Saturn later
& Waxing Crescent Moon 
& Sunset 8:01pm EDT
\cr
\hline
\end{tabular}
\end{sidewaystable}

\clearpage

Interesting events of Spring 2016:
\begin{itemize}
\item Late January, Early February: Comet Catalina, maybe?
\item April 22--23: Lyrids Meteor Shower
\item May 6--7: Eta Aquarids Meteor Shower
\item May 9: Transit of Mercury, 7:12am through 2:42pm EDT
\end{itemize}

