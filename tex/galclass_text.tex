
\noindent {\bf Objectives}:
\begin{itemize}
\item To understand the concept of \emph{color} in astronomy.
\item  To be able to classify galaxies based on their morphology and colors
\item  To investigate how galaxies evolve over time
\item  To search for, analyse and interpret information from large galaxy surveys including the Sloan Digital Sky Survey (SDSS)
\end{itemize}

\noindent
%{\bf Part 1. Virgo Cluster}
{\bf  Introduction}:

\begin{figure*}[ht]
        \begin{center}{\psfig{figure={Hubblefork.eps},width=10.0cm}}
        \caption{Hubble’s famous Tuning Fork diagram created by Dr Karen Masters, based on an activity designed by Las Cumbres Observatory Global Telescope Network (LCOGT).}\label{hubblefork}
        \end{center}
                \end{figure*}

\noindent
Edwin Hubble created the first classification of galaxies. He produces a diagram, called the Tuning Fork diagram (figure \ref{hubblefork}) based on features galaxies have in common. He came up with three distinct groups - ellipticals, spirals and irregulars. Ellipticals have no spiral arms or a disk and are classified by how round they look, E0 is very round but an E7 is very flat. This number is actually the ellipticity of the galaxy (the ratio of the semi major axis to the semi minor axis). Spirals show a spiraling structure, spiral ``arms'', and can be further split into barred (SB) and un-barred (S). They are also classified by how tight their spiral arms are ``wound''. There is a transition type called S0 which have no spiral arms but they have a central bulge and a disk. Astronomers now use a slightly different naming system with two major groups called early-type (including ellipticals and S0s) and late-type galaxies (spirals). Hubble thought that galaxies in time moved from left to right in his Hubble Tuning Fork diagram but he was wrong. We will see why.

\noindent{\bf 2. The color of galaxies}

\noindent
The Sloan Digital Sky Survey SDSS has images a large portion of the sky, and found more than 80 million galaxies. Classifying them by eye would take huge amount of time. SDSS cleverly called upon the public to help, and asked volunteers to look at images of new galaxies, compare them with typical Early and Late type galaxies, and classify the new objects accordingly. This project is called the \emph{Galaxy Zoo}, {\tt http://www.galaxyzoo.org}. A quicker way, easier to implement robotically, to classify Galaxies, is by using their color. 

\noindent {\bf Question:} Look at the Hubble Tuning Fork diagram reproduced above. How does the color differ for different galaxy types?



\begin{figure*}[ht]
        \centerline{\psfig{figure={SDSSgals.eps},width=18.0cm}}
        \caption{A selection of SDSS galaxies. }\label{SDSSgals}
         \end{figure*}


\noindent 
Figure \ref{SDSSgals} shows a selection of SDSS galaxies. The SDSS survey takes sky images in multiple filters. By combining these images color images of astronomical objects are obtained. 

\noindent First, look at each galaxy in Figure \ref{SDSSgals} and classify it according to its \emph{shape} according to both classification schemes: as \emph{early} or \emph{late} type, and as \emph{elliptical} E0-6, S0, \emph{spiral} S or SB a,b or c.

\vspace{20pt}

\noindent a \makebox[2cm]{\hrulefill}  \makebox[2cm]{\hrulefill}\makebox[1cm] d \makebox[2cm]{\hrulefill}  \makebox[2cm]{\hrulefill}\makebox[1cm] g \makebox[2cm]{\hrulefill}  \makebox[2cm]{\hrulefill}

\noindent b \makebox[2cm]{\hrulefill}  \makebox[2cm]{\hrulefill}\makebox[1cm] e \makebox[2cm]{\hrulefill}  \makebox[2cm]{\hrulefill}\makebox[1cm] h \makebox[2cm]{\hrulefill}  \makebox[2cm]{\hrulefill}

\noindent c \makebox[2cm]{\hrulefill} \makebox[2cm]{\hrulefill}\makebox[1cm] f \makebox[2cm]{\hrulefill}  \makebox[2cm]{\hrulefill}\makebox[1cm] i \makebox[2cm]{\hrulefill}  \makebox[2cm]{\hrulefill}

\vspace{20pt}


\noindent Now go to the SDSS Object Explorer Tool:

\noindent {\tt http://cas.sdss.org/dr5/en/tools/explore/obj.asp}

\noindent and type in their co-ordinates (‘Search by’ - top left hand menu). You can click Save to Notes for each galaxy to save the information automatically. Then you can just click on ‘Show Notes’ to see your measurements. 
I mentioned that SDSS observed in \emph{multiple filters} to obtain color information about astronomical objects. The SDSS filters are: \u, \g, \r, \i \ and \z; \u \ is the bluest filter, \z \ is the reddest.

\noindent 
You want to make a \emph{color-color diagram} of these galaxies. A color-color diagram is a very useful astronomical tool: it is a plot of one particular color, against another, for the same object.  In astronomy colors are defined as difference in magnitude in different filters. For this particular color-color diagram you want to plot the SDSS \g-\r \ color, against the \u-\g \ color, i.e. \u-\g \ is on your $x$-axis, \g-\r \ is on the $y$-axis. Now pay attention: because you are using magnitudes, where the larger the number the brighter the object this may be counterintuitive:

\noindent {\bf Question:} Which is bluer: a galaxy with a larger value of \u-\g \ or with a lower value of \u-\g?

\noindent 
Use the box below for your color-color diagram. Mark the bottom-left and top-right corner with the word BLUE or RED according to where you expect the bluest and the reddest objects to fall. Draw an arrow that indicats in which direction in the plot objects get redder.

\noindent
Plot your \g-\r \ vs \u-\g \ colors in the box below. Briefly note whether you see any patterns.
\clearpage

\begin{center}
\bigskip

\noindent
\g-\r ~{\framebox[8.0cm]{\rule[-4cm]{0cm}{8cm}\,} \hspace{0.5cm}}

\u-\g
\end{center}

\noindent
Now draw a line $$y=-x+2.2.$$ This means draw a line the join points that have $y$-axis value equal to 2.2 - the $x$-axis value. It was found studying the SDSS sample that early type galaxies have ($u$ -− $r$) values higher than 2.2, and late type galaxies have ($u$ -− $r$) values lower than 2.2, indeed they are split by the equation ($g$ -− $r$) = −($u$ -− $g$) + 2.2 (Strateva et al. 2001). Do you see this pattern in the data you analyzed? 

\bigskip
\noindent
Mark the box regions according to the type you expect them to host in the plot: mark the regions above and below your $y=x+2.2$ line with either EARLY or LATE, whichever you think is appropriate. 

\clearpage

%\begin{center}{\bf Extra credits/Homework}\end{center}

\noindent
{\bf Clusters of galaxies}:

\noindent
George Abell classified \emph{thousands} of galaxy clusters, publishing them in 1958. We will use a few of them to see how galaxy colors change with redshift and draw conclusions about galaxy evolution!
The galaxy cluster Abell 2255 is at co-ordinates RA = 258.1292\deg\  and Dec = 64.092\deg .

\noindent
To obtain the colors of galaxies in Abell 2255, use the SDSS archive Navigation Tool: 

\noindent{\tt http://cas.sdss.org/dr5/en/tools/chart/navi.asp.}

\noindent
You will need to type in the coordinates of the Abell 2255, click on ``Get Image'' and zoom out once or twice. Now click on roughly 20 galaxies that you think are part of the cluster (make a note of what criteria you use to decide if these are cluster members). On the right hand side of this webpage, you will see the colors listed for the galaxy you clicked on, write these down. You can click Save to Notes for each galaxy to do this automatically or do it by hand by typing/writing them. If you save the notes on the webpage just click on ‘Show Notes’ to see your measurements. You can export to CSV format if you wish to work on it in Excel, or any computational language, but that is not necessary.
Don’t forget, however, to include all of your measurements in your lab diary (stapled print-outs are fine).

\noindent
Important: think carefully about how you know which galaxies are part of Abell 2255, and which are just other galaxies at different distances in the same part of the sky. Briefly describe your 20 galaxies. Are they similar? How are they different? Make a color-color plot (as above) of your sample of Abell 2255 galaxies. Again, draw the line that ‘separates’ ellipticals from spirals. How many of these galaxies are ellipticals and how many spirals?



\clearpage

\noindent
{\bf Do Galaxy Colors Change with Redshift?}

\noindent
Now you have done this for a relatively nearby cluster (at a redshift of 0.081), try comparing the colors and classification of galaxies in clusters at different redshifts. The table below provides coordinates and redshift information fot three Abel clusters:

\begin{table}[ht!]
\begin{center}
\begin{tabular}{l c c c}
Name of cluster &  Redshift & RA (deg) & Dec (deg)\\\hline
Abell 2255 & 0.081 & 258.129 & 64.093 \\ 
Abell 0023 & 0.105 & 5.44 & -0.89 \\
Abell 0267 & 0.230 & 28.77 & 1.01
\end{tabular}
\end{center}
\caption{ Properties of the galaxy clusters used in this research project.}
\end{table}

\noindent
Follow the same procedure as before to get the SDSS data on each cluster and create color-color plots. You will have to assume your sample contains no foreground or background galaxies. Count the \emph{fraction} of Early and Late type galaxies in each cluster. 
\begin{itemize}
\item{How does it change with the redshift? }
\vspace{30pt}
\item{What does that mean?}
\vspace{30pt}
\item{Can you comment in the light of this result on Hubble's initial prediction, that Galaxies would evolve rightward in his plot?}
\vspace{30pt}
\item{ Can you relate this result to what you know about \emph{stellar} evolution?}
\end{itemize}

