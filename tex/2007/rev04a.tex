 
\documentclass[11pt]{article} 
\topmargin -.6in 
\textheight 8.7in 
\oddsidemargin -.0in 
\textwidth 6.5in 
\title{The Analysis of Starlight: Lab Projects} 
\date{Fall 1997} 
%\renewcommand{\baselinestretch}{1.2} 
\begin{document} 
\setcounter{page}{1} 
\setcounter{equation}{0} 
\pagestyle{empty} 
\parindent 0pt 
\parskip 8pt 
%\pagestyle{myheadings} 
\markright{{\bf LAB B: Measuring the Wavelength of Light} \
\hrulefill \ } 
\def\arcsec{\ifmmode {^{\scriptscriptstyle\prime\prime}}
          \else $^{\scriptscriptstyle\prime\prime}$\fi}
\def\arcmin{\ifmmode {^{\scriptscriptstyle\prime}}
          \else $^{\scriptscriptstyle\prime}$\fi}
\def\deg{\ifmmode^\circ\else$^\circ$\fi}


   
 
\noindent 
%\vspace{0.15in} 
{\bf Observational Astronomy      \hfill  V85.0013}\\ 
 
\bigskip 
 
\bigskip 
 
\noindent 
{\hfill \Large {\bf Review Sheet 4} \hfill} 
 
 
\bigskip 
%{\hfill {\bf Short Answers} \hfill} 
  
\begin{enumerate} 

\item 
right


\item
up, down

\item 
6 pm, 6am, midnight

\item 
midnight, 6 am, 3 pm

\item
6 am, midnight, 9 pm

\item 
waning gibbous


\item
increases


\item
13.2, 2.3 days, Gemini

\item
0.9 hr

\item
14.8  days


\item 
new 

\item
Moon moves faster than Sun in same direction; needs more than a
sidereal period to catch it up.

\item
Moon sways side-to-side so you can see around the edges

\item
0\deg




\item
Moon's orbit is ellipse: gets nearer and farther away

\item
Texas


\item 
shadow line; topography is best seen there

\item

Imbrium, Tranqillitatis, Serenitatis \\
Caucasus, Apennines, Alps \\
Plato, Archimedes, Huggins

\item
they flooded after the main bombardment


\item
It is seen for 2 weeks: it rises as 1st qr., spirals the sky to full at
alt $\sim 23.5\pm5\deg$, then spirals down and sets as 3rd qr.


\end{enumerate} 


\end{document}










