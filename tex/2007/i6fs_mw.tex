
\documentclass[12pt]{article}
\usepackage{psfig}
\topmargin -.6in
\textheight 8.7in
\oddsidemargin -.0in
\textwidth 6.5in
\title{The Analysis of Starlight: Lab Projects}
%\renewcommand{\baselinestretch}{1.2}

\begin{document}
\setcounter{page}{1}
\setcounter{equation}{0}
\pagestyle{plain}
\thispagestyle{empty}  % suppress number on first page
%\pagestyle{myheadings}
\newcommand{\kms}{\hbox{km\,s$^{\rm -1}$}}
%\def\kms{\ifmmode {\,{\rm km\,s^{-1}}}                          % km s-1
%determine the RA of the Sun        \else {\hbox{$\,$ {\rm km$\,$s$^{\rm -1}$}}}\fi}
%\def\solar {\ifmmode_{\mathord\odot} \else $_{\mathord\odot}$\fi} % _solar
%\def\mo {\ifmmode {\,{\it M}\solar} \else $\,M$\solar\fi}       % M solar
\def\lo {\ifmmode {\,{\it L}\solar} \else $\,L$\solar\fi}       % L solar
\def\my {\ifmmode {\,{\it M}\solar\,{\rm yr^{-1}}}              % Msol/year
        \else {$\,M$\solar$\,$yr$^{\rm -1}$}\fi}
\def\BD {BD$\,$+30{\degr}3639}
\def\HUNO{\rm H$\,$I}                   % molecular hydrogen
\def\HDOS{\rm H$_2$}                    % molecular hydrogen
\def\arcsec{\ifmmode {^{\scriptscriptstyle\prime\prime}}
          \else $^{\scriptscriptstyle\prime\prime}$\fi}
\def\arcmin{\ifmmode {^{\scriptscriptstyle\prime}}
          \else $^{\scriptscriptstyle\prime}$\fi}
\def\deg{\ifmmode^\circ\else$^\circ$\fi}





%\markright{{\bf LAB E: Hubble's Law} \ \hrulefill \ }


\noindent
%\vspace{0.15in}
{\bf Observational Astronomy    \hfill} {\bf First Name:\makebox[4cm]{\hrulefill}}\\
{\bf Lab: I-6FS} \hfill {\bf Last Name:\makebox[4cm]{\hrulefill}}


\bigskip

\medskip

\noindent
{\hfill \Large {\bf The Milky Way} \hfill}


\bigskip

\noindent
In this project we shall explore characteristics of the Milky Way and
some of the deep sky objects found in the Messier catalog.


\medskip \bigskip
\noindent
{\bf 1. The Local neighborhood} 

\medskip\noindent
The stars in our neighborhood are
dotted about at random, typically separated by about 1~pc or a few
light years (recall 1~pc = 3.26~lt~yr).  
To see our stellar neighborhood, launch SN and choose go/solar neighborhood
and it will give you a little tour. Many of the very faintest, nearby
stars are not included -- but it gives the correct general impression
of where we live. You can stop the clock and zoom in and move  about
using the interlocked arrows panel (with cntrl pressed) and the left
mouse. 

When you have had a look, click guides/galactic/reference axes to get 
rid of the 3-D grid.


\bigskip
\noindent
{\bf 2. The Milky Way} 

\medskip\noindent
On much larger scales of thousands of parsecs or light years our star
system is flat, so it appears brightest in a ring on the celestial
sphere that we call the Milky Way.

Return to NY, click settings/Milky Way and set it to a color of your
choice -- say blue. Set the clock turning and see how the MW moves
around the sky.  Click on equatorial coordinates (Eq at top) and see
that the MW is fixed in RA and Dec. Estimate the coordinates of the of
the MW at about 10 or more positions along its whole length, and plot the
results on the graph below. To do this, its easier to stop the clock,
and to get rid of the horizon so you can follow it all the way around.

\vspace{0.1cm}

\begin{figure*}[h]
        \centerline{\psfig{figure={i6s_f1.eps},width=11.0cm}}
        %\caption{}
         \end{figure*}

\newpage
\noindent
Join the dots you have plotted with a smooth line: this
shows the plane of the MW -- the Galactic plane. The
center of the system lies towards  RA = 17 hr 40
min, Dec = --29\deg\ about 8,000~pc away. Mark it with a cross.

\bigskip
\noindent
Is the Galactic plane aligned with the celestial equator? Or with the
the ecliptic?: \\ \makebox[4cm]{\hrulefill}

\bigskip
\noindent
How many times per day (if at all) does the MW pass through the zenith in NYC:
\makebox[4cm]{\hrulefill}

\bigskip
\noindent
When (if at all) does the Galactic Center cross the meridian today:
\makebox[4cm]{\hrulefill}

\bigskip
\noindent
When done, reset the MW color to its original black.

\bigskip
\bigskip
\bigskip
\bigskip
\noindent
{\bf 3. The Messier Objects} 

\medskip
\noindent
In addition to single and binary stars, our star system has many other
components. Examples are to be found in the Messier Catalog (about 100
in number), and in the later NGC catalog which is similar but with
many more entries (1000s).

These deep sky objects are classified as several main types: regions of ionized
gas (called H II regions); planetary nebulae (PN); open star clusters
(OC); globular clusters (GC); and galaxies. Supernova remnants are
very rare: M1 is the only example in the Messier catalog.  Note that
H II regions which are defined by their nebulosity often have a newly
formed star cluster in the center, and at least one of the stars is
blue and bright so that it can excite the nebula; in these objects
there is also often evidence of dark cloud (DC) obscuration where the
unexcited material blocks the light from the background.

\bigskip
\noindent
Look up the representative M objects in the following table (over the
page) in SN,
using the search feature (magnifying glass at the top, with the
Messier list chosen). 

\bigskip
\noindent
Sketch what each looks like in the box provided. Measure a
representative angular size in arc minutes (use the compass tool with
the cntrl key pressed), and record the result in
the space provided.

\bigskip
\noindent
From the distances and angular sizes of these objects: \\
Which is the physically largest: \makebox[3cm]{\hrulefill}\\
Which is the physically smallest.\makebox[3cm]{\hrulefill}

\newpage

\noindent
{\framebox[3.0cm]{\rule[-1.5cm]{0cm}{3cm}\,} \hspace{0.5cm}
M3: GC \hfill $\theta_{\rm diam}$: \makebox[3cm]{\hrulefill} \ \ Distance: 
30,000 lt yr}

\bigskip
\noindent
{\framebox[3.0cm]{\rule[-1.5cm]{0cm}{3cm}\,} \hspace{0.5cm}
M6: OC  \hfill $\theta_{\rm diam}$: \makebox[3cm]{\hrulefill} \ \ 
Distance:\ 2,000 lt yr}

\bigskip
\noindent
{\framebox[3.0cm]{\rule[-1.5cm]{0cm}{3cm}\,} \hspace{0.5cm}
M8: H II \hfill $\theta_{\rm diam}$: \makebox[3cm]{\hrulefill} \ \ 
Distance: 6,500 lt yr}

\bigskip
\noindent
{\framebox[3.0cm]{\rule[-1.5cm]{0cm}{3cm}\,} \hspace{0.5cm}
M27: PN \hfill $\theta_{\rm diam}$: \makebox[3cm]{\hrulefill} \ \ 
Distance: 1,200 lt yr}

\bigskip
\bigskip
\bigskip
\noindent
Using your experience, find the following with SN and classify them:

\medskip
M44: \makebox[3cm]{\hrulefill} 

\medskip
M57: \makebox[3cm]{\hrulefill}

\medskip
M13: \makebox[3cm]{\hrulefill}

\medskip
M16: \makebox[3cm]{\hrulefill}

\vspace{.5cm}
\bigskip

\noindent
{\bf 4. Where Are The Deep Sky Objects?}

\medskip
\noindent
Using SN, click the stars and planets off, and click on the Messier objects
and the more numerous NGC objects (which in the setup has omitted the
galaxies -- we will examine them next time). Set the clock running to
see where most of the deep sky objects (excluding galaxies) are to be
found. The distribution is quite striking.

\bigskip
\noindent
The best place to find a star cluster or a nebula of some kind is : 
\makebox[3cm]{\hrulefill}


\bigskip
\medskip
\noindent

\noindent
{\bf 5. A Close-up of the Milky Way}

\medskip
\noindent
To get a realistic picture of the Galactic plane, examine the two
photographic images of the MW from the Palomar Sky Survey. The two
images are of the same patch of sky with a field of view about $6\deg
\times 6\deg$ in size.  The correct orientation is with the small
rectangle in the top left corner. The bright star at the bottom edge
is Deneb, $\alpha$ Cygni (near 21 hr +45\deg).  The images are
negatives so the light of the stars etc. is dark. The image labelled E
(in the small rectangle) is taken with a red filter -- so the image
records the red light, and the one labelled O is taken with a blue
filter, and so records the blue light.

\medskip
\noindent
{\bf Please treat them with great care: keep them flat; no pencils or
  pens near them, and do not write on paper on top of the
  prints. Thanks!}

\medskip
\noindent
You will see that the MW is a pretty busy place. Each tiny speck is a
star, seen here down to about 20th magnitude. Locate the field in
the the Mag 5 and the FG, atlas chart 19. Study the images and answer
the following questions. Note: the large horizontal nebula structures which
cross the images are parts of a large, old supernova remnant. The star
exploded to the south of the picture.


\bigskip
\noindent
What color is the nebulosity, red or blue: \makebox[4cm]{\hrulefill}

\bigskip
\noindent
Identify and examine the star pairs labelled 1, 2, and 3, in the key
below.
Which of each pair (left or right) is the hottest (bluest) -- note it need not be the
brightest: \\
pair 1: \makebox[4cm]{\hrulefill} \  pair 2: \makebox[4cm]{\hrulefill}
\ \
pair 3:  \makebox[4cm]{\hrulefill}

\bigskip
\noindent
Classify the gas cloud marked with an H in the Key:  \makebox[4cm]{\hrulefill}

\bigskip
\noindent
What is at the center of gas cloud H, making it excited (check both images): 
\makebox[4cm]{\hrulefill}

\bigskip
\noindent
Classify the tiny feature seen in red a few cm up and to the right of
the cloud H:   \makebox[4cm]{\hrulefill}

\bigskip
\noindent
Explain the white patches that cross the feature I in the key: \makebox[4cm]{\hrulefill}


\bigskip
\noindent
Use SN with a field of 10\deg\ to identify the detailed star field of the
images.  Identify by name the star marked A in the Key:
\makebox[4cm]{\hrulefill} \\
It is 932 lt away. If you click on it with the right mouse/go there,
you can take a trip out there in 3-D.


\end{document}




