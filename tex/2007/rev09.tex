 
\documentclass[11pt]{article} 
\topmargin -.6in 
\textheight 8.7in 
\oddsidemargin -.0in 
\textwidth 6.5in 
\title{The Analysis of Starlight: Lab Projects} 
\date{Fall 1997} 
%\renewcommand{\baselinestretch}{1.2} 
\begin{document} 
\setcounter{page}{1} 
\setcounter{equation}{0} 
\pagestyle{empty} 
\parindent 0pt 
\parskip 8pt 
%\pagestyle{myheadings} 
\markright{{\bf LAB B: Measuring the Wavelength of Light} \
\hrulefill \ } 
\def\arcsec{\ifmmode {^{\scriptscriptstyle\prime\prime}}
          \else $^{\scriptscriptstyle\prime\prime}$\fi}
\def\arcmin{\ifmmode {^{\scriptscriptstyle\prime}}
          \else $^{\scriptscriptstyle\prime}$\fi}
\def\deg{\ifmmode^\circ\else$^\circ$\fi}


   
 
\noindent 
%\vspace{0.15in} 
{\bf Observational Astronomy      \hfill  V85.0013}\\ 
 
\bigskip 
 
\bigskip 
 
\noindent 
{\hfill \Large {\bf Review Sheet 9} \hfill} 
 
 
\bigskip
\begin{enumerate} 

\item Define the nodes of the Moon's orbit:

\item Total solar and lunar eclipses take place in roughly equal
numbers: but the lunar variety is more commonly seen. Why:

\item When a total eclipse of the Sun takes place, which side
disappears first, the side to the east or west in the sky.

\item
If the Moon's orbit were aligned with the celestial equator, when
would eclipses be likely to take place:

\item 
What circumstances cause an annular solar eclipse:

\item
What is solar granulation:

\item 
Sun-spots look black in typical images of the Sun's disk. Explain why:

\item 
Sun-spots or sun-spot groups usually come in pairs, one ahead of
  the other. What is the connection between them:

\item 
In which direction do sun-spots move across the disk of the Sun: E to W
or W to E. \\
What is the maximum time a spot can be tracked from day to day across
the disk:


\item 
It is well known that the Sun has an 11 (or 22 yr) cycle. Does this mean
individual sun-spots can be tracked for this period or what:

\item 
Binoculars are simple telescopes, but the image one sees is not upside
down. How is this achieved:

\item 
Refracting astronomical telescopes have objectives made of at least 2
kinds of glass. What is the main purpose of this:

\item 
What magnification do you get in a simple astronomical telescope made
with two lenses of focal length 2000 mm and 40 mm:

\item 
To make the telescope in the above question, how far apart do you put
the lenses:

\item 
What does the dimension of 20 inches refer to in a 20 inch telescope:

\item 
If your naked eye can just resolve an angle of 3', what is the
minimum magnification you need to split a binary of separation 5'':


\item 
When observing with a Schmidt-Cassegrain telescope (like our 8 inch
telescopes), how many times does the light pass along the tube:

\item With a telescope with an objective of focal length 2000~mm, what
focal length eyepiece (of apparent field 40 deg) is needed so that the
full Moon just fits in the field of view.

\item 
How many magnitudes fainter can you see with a 20 inch telescope,
compared with a 2 inch telescope.

\item 
If the Earth had no atmosphere, what color would the sky be:

\item
Stars twinkle more at lower altitudes. Why:


\item 
What is Ceres:


\item 
Some comets are not seen periodically but appear to come out of 
nowhere. Where are they believed to come from:

\item 
Roughly how far up is a typical meteor that you see in the sky:

\item
Where are most of the asteroids:

\item 
In a time-exposure of the night sky, star trails are arcs from the
turning celestial sphere, but meteor
trails are roughly straight lines. Why:

\item
The meteors that come in showers originate where:

\item 
Meteor showers are known by the names of constellations, e.g.,
Leonids, Geminids.  Why:

\item 
What are fireballs or bolides:

\item 
The meteorites that one sees in museums are typically stony or
metallic. Where have they come from: 



\end{enumerate} 


\end{document}










