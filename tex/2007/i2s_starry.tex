
\documentclass[12pt]{article}
\usepackage{psfig}
% revised 02/05/02 to follow fall 01
%\topmargin -.6in
\textheight 8.7in
\oddsidemargin -.0in
\textwidth 6.5in
\title{The Analysis of Starlight: Lab Projects}
%\renewcommand{\baselinestretch}{1.2}

\begin{document}
\setcounter{page}{1}
\setcounter{equation}{0}
\pagestyle{plain}
\thispagestyle{empty}  % suppress number on first page
%\pagestyle{myheadings}
\newcommand{\kms}{\hbox{km\,s$^{\rm -1}$}}
%\def\kms{\ifmmode {\,{\rm km\,s^{-1}}}                          % km s-1
%        \else {\hbox{$\,$ {\rm km$\,$s$^{\rm -1}$}}}\fi}
%\def\solar {\ifmmode_{\mathord\odot} \else $_{\mathord\odot}$\fi} % _solar
%\def\mo {\ifmmode {\,{\it M}\solar} \else $\,M$\solar\fi}       % M solar
\def\lo {\ifmmode {\,{\it L}\solar} \else $\,L$\solar\fi}       % L solar
\def\my {\ifmmode {\,{\it M}\solar\,{\rm yr^{-1}}}              % Msol/year
        \else {$\,M$\solar$\,$yr$^{\rm -1}$}\fi}
\def\BD {BD$\,$+30{\degr}3639}
\def\HUNO{\rm H$\,$I}                   % molecular hydrogen
\def\HDOS{\rm H$_2$}                    % molecular hydrogen
\def\arcsec{\ifmmode {^{\scriptscriptstyle\prime\prime}}
          \else $^{\scriptscriptstyle\prime\prime}$\fi}
\def\arcminm{\ifmmode {^{\scriptscriptstyle\prime}}
          \else $^{\scriptscriptstyle\prime}$\fi}
\def\deg{\ifmmode^\circ\else$^\circ$\fi}





%\markright{{\bf LAB E: Hubble's Law} \ \hrulefill \ }


\noindent
%\vspace{0.15in}
{\bf Observational Astronomy    \hfill} {\bf First Name:\makebox[4cm]{\hrulefill}}\\
{\bf Lab: I-2S} \hfill {\bf Last Name:\makebox[4cm]{\hrulefill}}


\bigskip

\medskip

\noindent
{\hfill \Large {\bf The Starry Sky} \hfill}


\bigskip

\noindent
{Objectives:} To tour the sky and explore the way in which it moves,
using the sky simulation program Starry Night Pro. Check out the
information sheet on SN first, and try some of the controls. The
ability to visit other locations and to run the sky clock as fast as
we choose is very helpful in learning how the sky moves.

\bigskip
\noindent
{\bf 1. Finding your way around the sky}

\medskip
\noindent
Launch SN and check that it is set up for NY at the present time,
pointing South.  Explore the horizon by pressing the N, S, E and W
buttons, and look around more generally using the keyboard arrows, or
the sliders below and to the right of the screen. The stars can be
identified by pointing at them with the mouse pointer. Compare the
view with the Field Guide, sky map 12 (in the sections after page 53),
and the Mag 5 atlas (mainly maps 1 and 3).

The 88 official constellation boundaries and their names can be shown
by clicking (at the top of the screen)
guides/constellations/boundaries and labels. Within each constellation
the brightest stars are called $\alpha$, $\beta$, and
so on through the Greek alphabet (usually in order of decreasing
brightness), followed by the constellation name
in Latin (in the genitive case) or its abbreviation. Bright stars also
have proper names, e.g., Betelgeuse and Sirius, and these are usually
given by the SN pointer. The constellation figures can be shown by
clicking guides/constellations/classical illustrations. Most of these
are based on Johannes Bayer's 1603 star atlas \emph{Uranometria}. As
you look around the sky you will see it is pretty full of heros, heroines,
animals, and so on.


\bigskip
\noindent
{\bf 2. Measuring angles in Starry Night}

\medskip
\noindent

\noindent
We shall frequently be measuring angles with SN, and we start here
with the angles between stars which we have seen before with the
celestial globe. Locate the pairs of stars below and measure the
distance between them, to the nearest degree (see SN information sheet
on how to do this). The first pairs are to the South, the others to
the North. Merak and Dubhe are the ``pointers'' at the end of the
familiar Big Dipper, which is actually part of the constellation Ursa
Major, the Great Bear.  $\gamma$ Cas is the middle star of the W shape
in Cassiopeia. Knowing the rough distance of these stars from Polaris
is useful for locating Polaris when you are outside.
 
\begin{center}
\begin{tabular}{lc} \hline \\ [-6pt]
Pair   & \hspace{1cm} Separation \hspace{1cm} \\ [6pt]
\hline
The ends of Orion's belt    &        \\ \hline
Betelgeuse -- Sirius         &       \\ \hline
Merak -- Dubhe            &      \\ \hline
Dubhe -- Polaris          &      \\ \hline
$\gamma$ Cas -- Polaris &  \\ \hline
\end{tabular}
\end{center}

\newpage

\bigskip
\noindent
{\bf 3. Sky motion}

\medskip
\noindent
We shall now see how the sky moves. Return to looking S, set the
interval in the time panel to 10 mins, and click the forward arrow. As
time passes you will see the stars move. During the night in NY, new
stars come into view, and as morning approaches the sky brightens and
the sun rises.

To examine the motion in more detail, set the clock interval to 3
mins, and click the blue, daylight box at the top of the screen to
keep the sky dark during the day (you can still see the Sun). 
Focus on single stars and examine how they move close to the
horizon. In each box below draw a line with arrows to show how the
stars move (in about 10\deg\ boxes) at the compass horizon points. In the North also look
farther up in the sky to see the motion around Polaris.


        \begin{figure*}[h]
        \centerline{\psfig{figure={i2s_f1.eps},width=13.0cm}}
        %\caption{}
         \end{figure*}


\noindent
Now set the location of the observer (using the location panel) to a
latitude of +90\deg, the North Pole, and examine the sky motion
there. Fill out the boxes.



        \begin{figure*}[h]
        \centerline{\psfig{figure={i2s_f1.eps},width=13.0cm}}
        %\caption{}
         \end{figure*}

\noindent
Change the location of the observer to the equator (latitude 0\deg)
and repeat the operation.


        \begin{figure*}[h]
        \centerline{\psfig{figure={i2s_f1.eps},width=13.0cm}}
        %\caption{}
         \end{figure*}



The overall picture should (hopefully) be clear: you appear to live
under a large, rotating, celestial sphere that carries the stars
around. Note too that as you go South, new stars never seen in the NY
sky came into view.

 
\bigskip
\noindent

\newpage
\noindent{\bf 4. Polaris }

\medskip
\noindent
Return to NY, at the current time, and create an Az-Alt grid in the sky
by clicking the Lo (short for Local) button at the top of the screen. You will see that
Alt increases from the horizon up, and Az increases Eastwards from the
North horizon point. The vertical lines converge in the zenith (Alt =
90\deg). Check this by looking up with the Z (for Zenith) button at the
top of the screen.

If you start the clock, you will see that the Az-Alt
grid lines stay in exactly the same place all the time, but the stars
move. However, there is one special place, the North Celestial Pole
(NCP), that keeps the same Az and Alt all the time in NY. This is
close to Polaris (and we will often use the two names interchangeably,
although this is not exactly correct).

A key question in understanding star motion is where this special
point is when you go to different places on the Earth?  We have
already seen the rule in class: the altitude of Polaris (or more
exactly the NCP) is equal to the latitude (when you are in the northern
hemisphere).

\bigskip
Check this in NY by estimating the altitude and azimuth of Polaris
from the grid lines:
\bigskip

Polaris:\ \ Alt = \makebox[4cm]{\hrulefill} \ \ Az =  \makebox[4cm]{\hrulefill}

\bigskip
You will also have verified the rule for latitude = 0 in your drawing
of the N horizon picture in section 3.

\vspace{4cm}
\bigskip\noindent
{\bf 5. Motion in Az-Alt}

\medskip\noindent
It will be clear from the previous sections that not all stars as seen
from NY do the same thing. We will examine 2 examples here: Dubhe (at
the tip of the Big Dipper) and Mintaka (the highest star in Orion's
belt). The latter is of special interest since it lies almost exactly
on the celestial Equator, midway between the NCP and the SCP.  For
each star, estimate the Az and Alt at 10 or more roughly evenly spaced
positions in its motion (e.g., use single time steps of a few
hours). Plot the points on the grids below, and join them with smooth
curves with arrows to show the motion.

\newpage
        \begin{figure*}[h]
        \centerline{\psfig{figure={i2s_f2.eps},width=12.0cm}}
        %\caption{}
         \end{figure*}

\,
\bigskip
\noindent
Stars move (clockwise/counter-clockwise) around Polaris:\
\makebox[4cm]{\hrulefill}\\ 
For stars that rise in the east, they move up and to the
(left/right): \makebox[4cm]{\hrulefill}\\ 
For stars that set in the west, they move down and to the
(left/right): \makebox[4cm]{\hrulefill}\\ 
At what two Az values do  stars reach their highest in the sky ?\
\makebox[4cm]{\hrulefill} \\


\bigskip\noindent
{\bf 6. RA--Dec motion}

\medskip\noindent
Replace the Az-Alt grid with an RA-Dec grid, by clicking the Eq button
at the top of the screen. Also turn on the CE by clicking
guides/equatorial/equator. Set the sky turning, and you will see the
moving graph paper in which the stars are fixed.


\newpage
\noindent
There are some basic aspects of the RA-Dec system that follow from the
geometry of the celestial sphere we have explored above. You can
probably figure them out directly from our discussion in class (e.g.,
using the center, bottom diagram inside the cover of the Mag 5 Atlas
assuming NY is at lat = 40\deg): but
you should also check them on the screen: 

\bigskip
\noindent
What Dec is at the zenith?\  \makebox[4cm]{\hrulefill}

\noindent
What is the lowest Dec of stars visible in NY?  \makebox[4cm]{\hrulefill}

\noindent
What is the minimum Dec of stars that are circumpolar in NY (i.e.,
never set)? \makebox[3cm]{\hrulefill}

\noindent
What is the difference in RA between the N and S horizon
points? \makebox[4cm]{\hrulefill}

\noindent
At what compass points does the CE meet the horizon:\makebox[3cm]{\hrulefill}

\noindent
What is the difference in RA between the points where the CE meets the
horizon? \makebox[3cm]{\hrulefill}


\end{document}








