 
\documentclass[11pt]{article} 
\topmargin -.6in 
\textheight 8.7in 
\oddsidemargin -.0in 
\textwidth 6.5in 
\title{The Analysis of Starlight: Lab Projects} 
\date{Fall 1997} 
%\renewcommand{\baselinestretch}{1.2} 
\begin{document} 
\setcounter{page}{1} 
\setcounter{equation}{0} 
\pagestyle{empty} 
\parindent 0pt 
\parskip 8pt 
%\pagestyle{myheadings} 
\markright{{\bf LAB B: Measuring the Wavelength of Light} \
\hrulefill \ } 
\def\arcsec{\ifmmode {^{\scriptscriptstyle\prime\prime}}
          \else $^{\scriptscriptstyle\prime\prime}$\fi}
\def\arcmin{\ifmmode {^{\scriptscriptstyle\prime}}
          \else $^{\scriptscriptstyle\prime}$\fi}
\def\deg{\ifmmode^\circ\else$^\circ$\fi}


   
 
\noindent 
%\vspace{0.15in} 
{\bf Observational Astronomy      \hfill  V85.0013}\\ 
 
\bigskip 
 
\bigskip 
 
\noindent 
{\hfill \Large {\bf Review Sheet 7a} \hfill} 
  
\bigskip 



\begin{enumerate} 

\item
$-$2  \ \ $-$0.6 \ \  +0.5 \ \ +4.4 \ \ +28


\item 
Polaris, by about 2.5


\item
delta, not labelled in order of brightness

\item 
6 (2 mags = 2.5 * 2.5 = about 6)

\item 
10,000  (10 mags, 5 mags is x 100 )

\item 
1 trillion  (30 mags different, i.e. 100*100*100*100*100*100)


\item
different surface temperatures 


\item 
G2

\item 
hotter, brighter

\item 
K, wd, K


\item
1 AU,  1 lt yr, 1 pc,  Earth-Sirius

\item 
32.6 lt yr, 8 times.

\item 
9.8 (use formula)

\item 
-10.6

\item 
It reaches mag 2-3 and then fades from view below naked eye level; it
reappears on a regular schedule of about a year.

\item 
Eclipses: the light curve is flat except for two dips where the
eclipses occur.

\item 
The archetype is in Cepheus (delta); roughly a sine curve;  
star physically pulses


\item 
Center of mass; balance point.

\item 
0.02 pc; use formula

\item 
Angular motion on the celestial sphere.
If the stars move randomly by about 5 deg the pattern will be very
different. 5 deg is 5 x 60 x 60 = 18,000 arcsec. Speed is
0.1 arcsec per yr, so the time is 180,000 yr (roughly).

\end{enumerate} 

Formulae: \ \ \ \ m $-$ M = 5 log(D/10) \hspace{2cm} theta(arcsec) = 206,000 s/d

\end{document}










