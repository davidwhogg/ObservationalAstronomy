
\documentclass[12pt]{article}
\usepackage{psfig}
\topmargin -.6in
\textheight 8.7in
\oddsidemargin -.0in
\textwidth 6.5in
\title{The Analysis of Starlight: Lab Projects}
%\renewcommand{\baselinestretch}{1.2}

\begin{document}
\setcounter{page}{1}
\setcounter{equation}{0}
\pagestyle{plain}
\thispagestyle{empty}  % suppress number on first page
%\pagestyle{myheadings}
\newcommand{\kms}{\hbox{km\,s$^{\rm -1}$}}
%\def\kms{\ifmmode {\,{\rm km\,s^{-1}}}                          % km s-1
%determine the RA of the Sun        \else {\hbox{$\,$ {\rm km$\,$s$^{\rm -1}$}}}\fi}
%\def\solar {\ifmmode_{\mathord\odot} \else $_{\mathord\odot}$\fi} % _solar
%\def\mo {\ifmmode {\,{\it M}\solar} \else $\,M$\solar\fi}       % M solar
\def\lo {\ifmmode {\,{\it L}\solar} \else $\,L$\solar\fi}       % L solar
\def\my {\ifmmode {\,{\it M}\solar\,{\rm yr^{-1}}}              % Msol/year
        \else {$\,M$\solar$\,$yr$^{\rm -1}$}\fi}
\def\BD {BD$\,$+30{\degr}3639}
\def\HUNO{\rm H$\,$I}                   % molecular hydrogen
\def\HDOS{\rm H$_2$}                    % molecular hydrogen
\def\arcsec{\ifmmode {^{\scriptscriptstyle\prime\prime}}
          \else $^{\scriptscriptstyle\prime\prime}$\fi}
\def\arcminm{\ifmmode {^{\scriptscriptstyle\prime}}
          \else $^{\scriptscriptstyle\prime}$\fi}
\def\deg{\ifmmode^\circ\else$^\circ$\fi}





%\markright{{\bf LAB E: Hubble's Law} \ \hrulefill \ }


\noindent
%\vspace{0.15in}
{\bf Observational Astronomy    \hfill} {\bf First Name:\makebox[4cm]{\hrulefill}}\\
{\bf Lab: I-3FS} \hfill {\bf Last Name:\makebox[4cm]{\hrulefill}}


\bigskip

\medskip

\noindent
{\hfill \Large {\bf The Ecliptic} \hfill}


\bigskip

\noindent
{Objectives:} To explore the effects of Earth's orbital motion on what
we see in the sky: the Sun's passage through the zodiac, the equinoxes
and solstices, and the seasons.

\medskip
\bigskip
\noindent
{\bf 1. The Sun's path around the Zodiac}

\medskip
\noindent
Launch SN and check that it is set up for NY, pointing South.  Set the
time for March 21 at noon: the Sun will be due South. Switch off the
daylight (blue button at top) so we can see which constellation the
Sun is in. It is not so easy to recognize! Switch on
guides/constellations/classical illustrations and Zodiac. Of course,
it's Pisces, one of the Zodiac constellations.

During the year the sun passes through all 12 of the Zodiac
constellations. To see this, click planets in the tool box, click
on the Sun, then the lock symbol (this will keep our view of the Sun
centered as we follow it around the sky). Set the time interval for 2
days, and set the Sun in its journey round the sky. Write down the
constellation names in sequence as the Sun passes through.

\medskip
Pisces \hrulefill


 \hrulefill

\bigskip
\noindent
{\bf 2. Ecliptic motion}

\medskip
\noindent
\emph{Angular speed.} The exact path of the Sun around the Zodiac is called the ecliptic. Click
guide/ecliptic/ecliptic to switch it on. The angular speed of the Sun
along the ecliptic is an important number that we can measure directly.
Locate the stars $\epsilon$ (epsilon) Piscium and $\lambda$ (lambda)
Aquarii with the pointer in the tool box.  Both lie close to the
ecliptic.  $\epsilon$ Psc is one of the brighter stars associated with
the string holding the fish, and $\lambda$ Aqr lies at the mouth of
the watering pot.  Line up the Sun with $\lambda$, and with single
day time steps, count how many days it takes to reach
$\epsilon$. Then measure the angle between the stars using the compass
in the tool box, to the nearest tenth of a degree.  Use these numbers
to calculate the angular speed of the sun on the ecliptic in \deg/day,
to two decimal places.  

\medskip Separation $\epsilon$ Psc -- $\lambda$
Aqr: \makebox[3cm]{\hrulefill} Time interval: \makebox[3cm]{\hrulefill} 

\medskip 
\centerline{{\bf Angular speed of the Sun:}
\makebox[4cm]{\hrulefill}  }

\bigskip\noindent
\emph{Relation to RA-Dec.} The relation of the ecliptic to the RA-Dec
system is basic to many phenomena. The key relations are given in the
table below. The change in Dec by $\pm 23.5$\deg\ is the
result of the tilt of Earth's spin axis by 23.5\deg\ relative to the
plane of its orbit.  The place where the ecliptic cuts the CE in
spring defines the 0 hr marker for the RA system. Turn on the
CE and the equatorial grid (guides/equatorial/equator and grid), and
set the Sun in motion around the ecliptic as in section 1 (make sure
it is locked in the planet panel). Verify the entries in the table.


\begin{center}
\begin{tabular}{lccc} \hline \\ [-6pt]
Name & Date  & \hspace{1cm} RA \hspace{1cm} &Dec \\ [6pt]
\hline
Vernal equinox & March 21  & 0 & 0    \\ \hline
Summer solstice & June 21  & 6 & +23.5 \\ \hline
Autumnal equinox & Sept 23 & 12 &  0     \\ \hline
Winter solstice & Dec 22 & 18  & $-$23.5  \\ \hline
\end{tabular}
\end{center}



\medskip

In which general direction does the Sun move, to the East or West ?  \makebox[2cm]{\hrulefill}

From the table:

How many hrs (to nearest hr) of RA does the Sun move in a month?  \makebox[2cm]{\hrulefill}

How many minutes (to nearest min) of RA does the Sun move in a day ? 
\makebox[2cm]{\hrulefill}

\smallskip
\noindent Note: if you convert hrs and mins to degrees, the above should
agree approximately with the angular speed you measured in \deg/day.)

\bigskip
\noindent
{\bf 3. The stars and the seasons}


\medskip
\noindent
One consequence of the motion of the Sun relative to the stars is that
different parts of the celestial sphere are seen in the evening sky at
different dates during the course of the year. 
To see this, set up SN in NY for 8 pm, pointing S,
with the RA grid lines showing (Eq button at the top of the screen),
and with the Sun unlocked; also turn off daylight.  For the equinoxes
and solstices in turn at 8 pm, draw in and label the brightest star
visible in the 100\deg\ field, and label the RA line on the meridian
with its value.

        \begin{figure*}[h]
        \centerline{\psfig{figure={i3s_f2.eps},width=13.0cm}}
        %\caption{}
         \end{figure*}

It will be clear that the stars and constellations visible in the
evening change with the seasons, and that the sidereal time (the RA on
the Meridian) changes as well.
To see the transition from one of these pictures to the next, set up
back at the vernal equinox at 8 pm. Set the time step to 1
solar day and run the clock. You will see that at the same clock time
on successive days, the stars and constellations slip round the sky and
eventually are no longer seen in the evening. 

\medskip  
Each day at the same time the stars slip a little to the (east/west):
\makebox[2cm]{\hrulefill}

\bigskip
(For experts: If you change the time interval on the clock to 1
sidereal day, i.e. 23 hr and 56 min, you will find that the stars
return to \emph{exactly} the same positions every sidereal day.)


\medskip 
\bigskip

\newpage
\noindent
{\bf 4. The seasons on Earth}

\medskip
\noindent
An important effect of the changing Dec of the Sun during its journey
around the ecliptic is the occurrence of the seasons on Earth. This
affects several things: the most noticeable being the length of
daylight and the height achieved by the Sun during the day.  Set up SN
for NY, looking South, for the equinoxes and solstices in turn, and
measure the times of sunrise and sunset (to 2 minute accuracy, using
2 minute, single time steps near the rising and setting points)
and hence the day light time. Also estimate the maximum altitude, as
the sun transits. (For this you can use the Az, Alt grid given by the the
Lo button at the top of the screen; you should also be able to
calculate the Alt exactly from the Dec positions.)

\begin{center}
\begin{tabular}{lcccc} \hline \\ [-6pt]
Name & \hspace{0.5cm} Rise time \hspace{0.7cm} Set time \hspace{1cm}
 & Day length \ \ \ \ \ \ \ \ &
Max Alt (approx.)  \hspace{1cm}\\ [6pt]
\hline
Spring/fall equinox &   &  & &     \\ \hline
Summer solstice & & & &  \\ \hline
Winter solstice  & & & &  \\ \hline
\end{tabular}
\end{center}

In NY the changes in the daylight hours and the altitude of the Sun 
from the summer to winter solstices are quite striking. You will also
notice that the azimuth at which the Sun rises and sets also varies
remarkably during the year



\bigskip
\noindent
\emph{The Sun at other locations.} Stranger things happen at other
parts of the globe, that you should be able to work out from our
previous studies of where the NCP and the CE appear in the sky at
different latitudes.  Two instructive cases follow.  Describe what the
sun does in the sky for the locations and dates specified.  Then check
them out with SN by resetting the location and date. Turn the daylight
on, so you know whether it's day or night.

\medskip
 The summer solstice at the North Pole. Describe what the Sun
does during 24 hrs:

\medskip
 \hrulefill

\medskip
The winter solstice on the arctic circle, lat =
+66.5\deg.  Describe what the Sun does during 24 hrs:

\medskip
\hrulefill



\bigskip
\bigskip
\noindent
{\bf 5. A space view of the seasons}

\medskip
\noindent
For later comparison with other planets it is interesting to view the
changing of the seasons on Earth from space. The SN setup is more
complicated than any we have done so far. So be patient!

Set the date for March 21. Set the time at 6 pm.  Set the location of
the observer
on the equator, with Lat = 0\deg.  In the settings menu (at the the top
of screen), go to planet/sun and set: sharp transition and
guides. These will make the shadows on Earth easier to see, and will
draw some lines of longitude and latitude.

Now in the elevation window (of the tool box) choose radius $\times$
2, then choose the window again and click hover (at the top of the
list of choices). We are going to soar above the Earth and see what it
does. Finally, click planets, click Earth, then the lock symbol, to
turn the view to Earth. Voila, here it is!

Set the time step to 10 min and set the Earth spinning. Since it
is the vernal equinox, you will see that for everywhere (except the exact
Poles) the days and nights are 12 hrs. The sunlight just touches each
Pole.

Now speed things up by setting the time interval to 1 solar day. The
daily spin is not apparent, but the Earth is moving in its orbit. You
can snoop around, especially near the North Pole by clicking the
curved crossed arrows in the tool box: this allows you to move around
the planet up and down and side to side (by dragging on the screen
with the mouse). You will need to move about a bit to get the best
view of the Poles.  Note how as the days move into summer in the
North, the sunlight spreads over the arctic regions. (If you need to
check what happens in one day, set the time step back to 10 minutes.)
As time goes on the sunlight gradually recedes from the North, and as
winter approaches, the arctic regions enter their long dark
period. Check that the opposite is happening down in the southern
hemisphere.

\bigskip
\bigskip
\noindent
{\bf Appendix}

\bigskip 
If you finish early, or for later review.

\medskip
From the discussion in section 3, the key to finding what is in the
sky at any given instant is clearly the sidereal time. We have already
discussed in class one approximate way to determine this for any time
and day during the year as follows.

\begin{itemize} 

\item{\bf Determine the RA of the Sun} for the day in question.

\item {\bf Add/subtract the time after/before noon} on the day in
question.

\end{itemize}
The idea is that the RA of the sun gives you the sidereal time at noon
(ignore daylight saving) so for any time on that day you just add or
subtract the amount from noon. To get the RA of the Sun, we know the
RA on the equinox or solstice; we add to that 2 hr for each complete
month since then, and 4 min for each extra day.

\bigskip
\noindent
{\bf Example:} Oct 28 at 9 pm. The RA of the sun is 12 hr (at the
equinox on Sept 23) plus 2 hr (1 month to Oct 23) plus 20 min (5 days
at 4 min each) = 14 hr 20 min. 9 pm is 9 hr after noon so the ST =
23:20. Check it with SN.

\bigskip
\noindent {\bf Your turn:} Calculate the sidereal time for today at 8
pm. (Check it with SN): \makebox[2cm]{\hrulefill}


\end{document}




