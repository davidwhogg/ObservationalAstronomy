\documentclass[11pt]{article}
\topmargin -0.6in 
\textheight 8.7in 
\oddsidemargin -.0in 
\textwidth 6.5in 
\pagestyle{empty}
\begin{document}
\def\arcsec{\ifmmode {^{\scriptscriptstyle\prime\prime}}
          \else $^{\scriptscriptstyle\prime\prime}$\fi}
\def\arcmin{\ifmmode {^{\scriptscriptstyle\prime}}
          \else $^{\scriptscriptstyle\prime}$\fi}
\def\deg{\ifmmode^\circ\else$^\circ$\fi}


%%%%%%%%%%%%%%%%%%%%%%%%% page 1

{ \hfill Last Name:\,\makebox[6cm]{\hrulefill}}

{ \hfill First Name:\,\makebox[6cm]{\hrulefill}}

\vskip 0.8cm
\begin{center}
{\large \bf Observational Astronomy: Midterm Mon} \\
%\vskip 0.8cm
\end{center}

\noindent You need the Mag 5 Atlas.
Use the back of this paper for rough notes etc.  

\bigskip
\begin{enumerate}

\item 
Figure 1 is an unlabeled map of the sky with Polaris at the center, and
going down to Dec = --50\deg\ around the edges.  Use the atlas to
navigate around the map. {\bf Circle (for the RA,Dec mark with +) and label}
carefully and unambiguously: 

\medskip
$\beta$ Ursa Majoris, \ \ \  $\alpha$ Cassiopeiae \ \ \ Capella \ \ \
Gemini \ \ \ Rigel  \ \ \  RA=6 hr, Dec = 0\deg.

\vskip 1.0cm
\item 
Using the atlas, determine the RA and Dec of the star $\alpha$ in the
constellation of Camelopardus (The Giraffe) to within 10
min and 1 degree, respectively (consult Map 1):

\vskip 1.0cm

\item 
If you exactly center Procyon (in Canis Minor) in binoculars with a true
field of view of  6\deg, how many other stars mag 5 or brighter can you see. 

\vskip 1.0cm
\item
Consider the stars $\alpha$, $\beta$, $\gamma$, $\delta$ in Gemini.\\
Which star crosses the meridian first:\\
Which star does the Sun come closest to:

\vskip 1.0cm
\item 
In NY, at what altitude does Sirius cross the meridian: 

\medskip
In Auckland, New Zealand, (Lat = --37\deg) at what altitude does Sirius
cross the meridian:

\vskip 1.0cm
\item
Specify exactly by their coordinates all stars that are circumpolar as
seen from NY: 

\medskip
Specify exactly by their coordinates all stars that are circumpolar as
seen from the North Pole:


\vskip 1.0cm
\item
Circle all of the following that pass through the zenith in NYC at
some time during the year:\\
a) R.A. = 18 hr \ \   b) the ecliptic \ \  
c) Dec = 75\deg \ \ d) the celestial equator

\vskip 1.0cm
\item 
You are at a latitude of +50\deg\ on Earth. \\
Stars of what Dec pass through the zenith:\\
Stars of what Dec appear on the south horizon:

\vskip 1.0cm
\item
In NY, you see a bright star at az = 100\deg, alt = 10\deg. If you
re-observe the star 1 hour later: \\
The az has increased/decreased  (circle your choice)\\
The alt has increased/decreased  (circle your choice)

\vskip 1.0cm
\item  
As seen from NY, if Kochab in Ursa Minor is
on the meridian below Polaris, what is the sidereal time:

\vskip 1.0cm
\item 
Sirius is up now in the evening in NY.  \\
According to EST, Sirius crosses the meridian earlier/later/at the
same time compared to the previous night:

\vskip 0.5cm
According to sidereal time, Sirius crosses the meridian earlier/later/at the
same time compared to the previous night: 

\vskip 1.0cm
\item
If the Sun is in the constellation Libra (check the atlas),  \\
What is the season: \\
What constellation will it enter next:
 
\vskip 1.0cm
\item
Today is the vernal equinox.\\
What is the sidereal time at noon today: \\

What is the sidereal time at 10 pm tonight:


\item
Havana, Cuba, is close to a latitude of +23.5\deg. 
What are the maximum and minimum altitudes of the sun at noon in Havana
during the year. \\
Maximum: \\
Minimum:

\vskip 1.0cm
\item
One can sometimes see the dark side of a crescent moon lit up
faintly.

\smallskip
Where does this light originate: \\
Explain how it got there:



\vskip 1.0cm
\item
At what time does the full moon rise: \\
At what time does the 1st quarter moon set:

\vskip 1.0cm
\item 
Which lunar phase (new, full, 1st quarter, 3rd quarter) immediately
precedes a waxing gibbous phase: \\
Draw a waxing gibbous moon as it appears in NY crossing the meridian. Shade
in the unlit part.

\vskip 1.0cm

\item 
In what phase (specify precisely) is a 17 day old Moon:

\vskip 1.0cm
\item 
On one recent evening, the Moon was close to a bright star. On the following
evening the Moon was distant from the star by approximately: \\ 
1 fist (at arms length) / one pinkie (at arms length) / 45\deg / 90\deg  \\
to the E / W of the star (circle your choices)

\vskip 1.0cm
\item 
Name the two planets now visible in the evening sky: 
\\ 
Which one is currently closer to Earth:


\vskip 1.0cm
\item 
In NY, all the planets rise and set in a typical 24 hour period: true/false


\vskip 1.0cm
\item 
Venus has an orbital radius of approximately 0.72 AU. \\
What is its minimum distance from Earth: \\
What is its maximum distance from Earth:

\vskip 1.0cm
\item 
Is Mars is closer to the Earth at opposition or conjunction: \\
What phase is it in when closest:

\vskip 1.0cm
\item Jupiter (whose orbital period is 12 yr) will be in opposition in
May 2006, in the constellation of Libra. \\
At what time will it transit the meridian:\\
When will it next be in opposition after that:  


\vskip 1.0cm
\item
You have set up and aligned the dials of an equatorial telescope. \\
When you read the RA dial against the tick mark in the center of the
fixed telescope mount (not the one that moves as the telescope moves), 
what  quantity does it give: \\

If you point at Polaris, and then move the RA axis by plus 6 hrs,
where does the telescope point: \\


\end{enumerate}

\end{document}




















