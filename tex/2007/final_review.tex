\documentclass{article}
\begin{document}
\begin{enumerate} 
\item 
In which constellation is the star Polaris (find it in Atlas Map 1)?
\item
In which constellation is the asterism the Big Dipper (find it in
Atlas Map 1)?
\item
Most people know the signs of the Zodiac which correspond to the
constellations along the Zodiac in the sky: Aries, Taurus, ... What
three come next?
\item
The greek alphabet goes: $\alpha$ alpha, $\beta$ beta, $\gamma$ gamma,
$\delta$ delta ... what three come next?
\item
What is the approximate angular width (in degrees) of a pinky at arms
length?
\item
How many fists at arm's length end-to-end, reach from the horizon due
north to the horizon due east?
\item
Your pinky at arms length can cover up the Moon: true or false?
\item
What is the altitude and azimuth of a star due east, half way up the
sky from the horizon.
\item
What is the angle between the zenith and the position alt=0~deg,
az=45~deg?
\item
Write this precise angle as a decimal number of degrees: 20~deg\
35~arcmin\ 40~arcsec.
\item
What is the approximate angular diameter of the Moon in arcmin?
\item
What property would a star labeled beta (followed by the constellation
name) normally have?
\item
Which animals correspond to the following constellations: Aries, Ursa
Minor, Canis Major
\item 
The Field Guide Fig 4-2 shows a picture of Ursa Major from the 1801
Bode atlas.  Using the Atlas (map 1) , find the proper name of the
star at the end of the tail.
\item 
The Field Guide, Fig 2-13 shows the 1603 Beyer atlas picture of
Orion. From the Atlas (map 3) find the proper names of the two bright
stars at Orion's shoulders.
\item 
The Field Guide, Fig 2-5 shows the Beyer atlas picture of
Cassiopeia. From the Atlas (map 1) find the proper name of the bright
star at Cassiopeia's throat.
\item
In NY, as you watch stars near the east horizon, they move up/down and
to your left/right.  [Circle the correct words.]
\item
In NY, as you watch the stars at alt=45~deg, az=180~deg, is their main
motion up, down, to your left, or to your right?
\item
In NY, as you watch the stars near Polaris, is the motion clockwise or
counterclockwise?
\item 
In NY, what are the alt and az of Polaris?
\item
In NY, what is the Dec of a star that just touches the north horizon?
\item
In NY, what is the Dec in the zenith?
\item
In NY what is the alt of the celestial equator as it crosses the
meridian?
\item
In NY, what is the Dec on the south horizon?
\item
In NY, what is the Dec on the east horizon?
\item
On the celestial sphere, which RA and Dec are directly opposite
RA=5~hr, Dec=$+25$~deg?
\item
In NY, when the RA line through the south horizon is 16 hr, what is the
RA on the north horizon?
\item
In NY, is a star at alt=12~deg, az=210~deg circumpolar?
\item
How far apart in deg are stars on the celestial equator separated by 1
hr in RA?  Is the answer more or less if they are have that RA
separation but are both at Dec=45~deg?
\item 
How far part in arcsec are two stars separated by 1~s on the celestial
equator?
\item
What is the alt of Polaris as seen from latitudes of 90~deg, 20~deg, 0~deg?
\item
What is the alt of the SCP as seen from latitudes of 0~deg, $-20$~deg,
$-90$~deg?
\item
You are marooned on a tropical island with a radio and your star
atlas. You find that the bright star, Sirius (RA=17~hr, Dec=$-17$~deg)
goes through the zenith. What latitude do you radio to the search
party?
\item
In the atlas, estimate the angular separation of Betelgeuse and
Bellatrix on the sky in deg.
\item
In the atlas, Consider the following stars: Rigel, Sirius, Aldebaran,
Capella. Which is nearest the celestial Equator? Which is nearest the
Ecliptic? Which has the lowest Dec? Which comes nearest the zenith in
NY?
\item 
In NYC which of the following can pass through the zenith:
the celestial equator, the ecliptic, the local meridian, the RA=1~hr
line, the Dec=+30~deg line?
\item
If you are observing from the Earth's the equator, which of the
following can pass through the zenith: the celestial equator, the
ecliptic, the local meridian, the RA=1~hr line, the Dec=$+30$~deg
line?
\item
If you are observing from the equator, what fraction of the celestial
sphere can be seen \textsl{(a)}~at one instant of time, and
\textsl{(b)}~over the course of time?  How will these answers be
different observing from NYC?
\item
What constellation is the Sun in at the vernal equinox, the summer
solstice, the fall equinox, and the winter solstice?
\item
In NY, at the fall equinox, the Sun crosses the meridian at what
az and alt?
\item
In NY, at the winter solstice, the Sun crosses the meridian at what
az and alt?
\item
At the winter solstice, where on Earth does the Sun pass through
the zenith at noon?
\item
Describe what the Sun does in 24 hr, as seen from the north pole on
March 21.
\item
Describe what the Sun does in 24 hr, as seen from the north pole on
June 21.
\item
Describe what the Sun does in 24 hr, as seen from the arctic circle,
lat=$+66.5$~deg, on June 21.
\item
How long does it take the sun to move in RA (that is, relative to the
background stars) a distance equal to its diameter (0.5 deg)?
\item
A star crosses the meridian at 8 pm on Feb 12. At what time will it be
there 1 month later?
\item
A star on the celestial equator crosses the meridian at 10 pm on April
1. Where will it be at 10 pm, 3 months later?
\item
In NY, what is the sidereal time when Betelgeuse is on the meridian?
\item
What is the sidereal time at midnight on March 21, noon on Sept 23, 7
pm on March 21, and 8 pm on October 3?  Assume all times are given in
local solar time, not daylight time.
\item
The Field Guide Fig. 2-5 shows a 1603 star atlas picture of Cassiopeia
(Atlas map 1). Find the magnitude of the star at the end of her nose.
\item
Ursa Major (Atlas map 1) is an exception to the rules about labeling
the stars alpha, beta etc.  Which star in the Big Dipper is the
brightest?
\item 
In NY, you observe a 1st quarter moon on the meridian. Which side is
lit up, to your left or right?
\item 
In NY, you observe a waxing crescent moon setting. Do the horns point
(mainly) up or down?  What about the case of a setting waning crescent?
\item 
At what time of day do the following Moons cross the meridian: 1st
quarter, 3rd quarter, and full?
\item 
At what time of day (approx) do the following Moons rise: 3rd quarter,
new, and waxing gibbous (take mid-gibbous phase)?
\item
At what time of day do the following Moons set: full, 1st quarter, and
waxing crescent (take mid-crescent phase)?
\item
In what phase is a 16 day old Moon.
\item
As the moon moves on the celestial sphere does its RA increase or decrease?
\item
As the Moon moves along the zodiac, how many deg per day does it move
(relative to the stars)?  How long on average does it spend in each
constellation?  When it is in the constellation Taurus, what
constellation is it in next?
\item
How long does it take the moon to move a distance equal to its own
diameter (0.5 deg) against the background stars?
\item
How long is the Sun continuously above the horizon for someone on the
Moon's equator?
\item
An astronaut at the center of the disk of a full Moon would observe
the Earth to be in what phase?
\item
In terms of motion on the celestial sphere, explain why the synodic
period of the Moon is longer than the sidereal period.
\item
What is libration of the Moon?
\item
At the equinoxes, what is the Dec of a full Moon?
\item
The angular diameter of the Moon as seen from Earth changes by 13 per cent
over time. Why?
\item
The mare areas of the Moon are much less cratered than the
highlands. Why?
\item 
The planets are found on the celestial sphere close to what line.?
\item 
What is 1 AU in km?  How many Earths, placed edge-to-edge, are needed
to stretch from the Sun to the Earth.  (Earth's radius is 6,400~km.)
\item 
To an observer in NY during a typical 24 hr period, all the planets
rise and set: true or false?
\item 
To an observer on the North Pole during a typical 24 hr period, all the
planets rise and set: true or false?
\item 
For an observer in NY, roughly when does an outer planet like Saturn in
conjunction rise?
\item 
For an observer in NY, roughly when does an outer planet like Neptune in
opposition cross our meridian?
\item
Jupiter has a sidereal period of 11.9 years. How long on average does
it spend in each of the zodiacal constellations?
\item
An outer planet like Saturn moves (east/west) against the stars,
except near (conjunction/opposition/east quadrature/west quadrature),
when it moves (east/west) in retrograde motion.  [Circle the correct
words.]
\item
For an observer in NY, roughly when does Venus in inferior conjunction
set?
\item
When Venus is in western elongation, is it a morning or evening star?
\item
When Venus is an evening star, is it approaching or receding from
Earth?
\item
Place the following in order of brightness (at their maximum),
brightest first: Sirius, Jupiter, Venus?
\item
In what phase are the following: Jupiter at opposition, Jupiter in
conjunction, Venus at greatest eastern elongation, Venus at superior
conjunction, Mercury at inferior conjunction?
\item
When Mars is closest to Earth, what phase is it in, and at what local
time is it highest in the sky?
\item
Neptune has a sidereal period of 165 years.  Roughly what is its
synodic period?
\item
Which planet has: the longest orbital period, the shortest synodic
period, and he synodic period closest to one year?
\item
Venus has an orbital radius of 0.72 AU. How far away is it at its
closest approach to Earth, and how far is it at its farthest from
Earth?
\item
Saturn has an orbital radius of 9.5 AU. How far away is it at its
closest approach to Earth, and how far is it at its farthest from
Earth?
\item
What is a ``transit of Venus''?
\item
Using the diameter and orbital radius for Mars (consult the texts),
calculate its minimum and maximum angular diameters as seen from Earth
in arcsec.
\item
Using the diameter and orbital radius for Saturn (consult the texts),
calculate its minimum and maximum angular diameters as seen from Earth
in arcsec.
\item
From the above, and images of Saturn in the texts, estimate the maximum
angular diameter of Saturn's rings.
\item
If the atmosphere in NY smears out the image of Mars (at maximum size)
to give ``pixels'' of 2~arcsec, how many pixels is it across the
diameter of the Mars image?
\item
The maximum angular diameter of Venus is about 60~arcsec. Calculate
the minimum.
\item
The relative change of the angular size of Neptune from largest to
smallest is less than that of Venus. Why?
\item
What phase is Venus in when its angular size is a maximum?
\item
Is the angular size of Venus in a crescent phase larger or smaller
than when it is in the gibbous phase?
\item
What phase is Saturn in when its angular size is a maximum?
\item
Which of the planets are gas giants?
\item
Name one of the principal constituents of Jupiter's atmosphere?
\item
What color is Saturn to the eye?
\item
Order the letters A,B,F,G,K,M,O into temperature order?
\item
What is the temperature of the hottest stars you can see?
\item
What about the coolest?
\item
What is the principal difference between Betelgeuse and Rigel that
makes Rigel bluer?
\item
Why, in a very young open cluster, are the brightest stars bluest?
\item
Why, in a slightly older cluster, are the brightest stars reddest?
\item
Draw a diagram of the Pleiades, being as accurate as possible, with
the stars labeled, and the brighter stars shown as darker/bigger dots.
\item
Order these by luminosity, brightest first: Red giant, main-sequence G
star, white dwarf.
\item
What makes the Orion nebula glow red?
\item
What is the ultimate power source for the Orion nebula?
\item
What lives at the very center of our Galaxy?
\item
What makes the dark patches and bands in the Milky Way?
\item
What are the Magellanic clouds?
\item
How do we know or observe the rotation period of the Sun?
\item
How did we first measure the distance to the Sun?
\item
What causes a Lunar eclipse?
\item
Why are solar eclipses so much rarer than lunar eclipses?  Explain
your answer with a \emph{diagram.}
\item
If the focal length of a telescope is 2000\,mm, what focal-length
eyepiece will give it a magnification of 125?
\item
Why are the telescopes we buy reflectors and not refractors?
\item
Name a comet that was visible to naked-eye stargazers this year.

\end{enumerate} 
\end{document}
