
\documentclass[12pt]{article}
\topmargin -.6in
\textheight 8.7in
\oddsidemargin -.0in
\textwidth 6.5in
\title{The Analysis of Starlight: Lab Projects}
%\renewcommand{\baselinestretch}{1.2}

\begin{document}
\setcounter{page}{1}
\setcounter{equation}{0}
\pagestyle{plain}
\thispagestyle{empty}  % suppress number on first page
%\pagestyle{myheadings}
\newcommand{\kms}{\hbox{km\,s$^{\rm -1}$}}
%\def\kms{\ifmmode {\,{\rm km\,s^{-1}}}                          % km s-1
%        \else {\hbox{$\,$ {\rm km$\,$s$^{\rm -1}$}}}\fi}
%\def\solar {\ifmmode_{\mathord\odot} \else $_{\mathord\odot}$\fi} % _solar
%\def\mo {\ifmmode {\,{\it M}\solar} \else $\,M$\solar\fi}       % M solar
\def\lo {\ifmmode {\,{\it L}\solar} \else $\,L$\solar\fi}       % L solar
\def\my {\ifmmode {\,{\it M}\solar\,{\rm yr^{-1}}}              % Msol/year
        \else {$\,M$\solar$\,$yr$^{\rm -1}$}\fi}
\def\BD {BD$\,$+30{\degr}3639}
\def\HUNO{\rm H$\,$I}                   % molecular hydrogen
\def\HDOS{\rm H$_2$}                    % molecular hydrogen
\def\arcsec{\ifmmode {^{\scriptscriptstyle\prime\prime}}
          \else $^{\scriptscriptstyle\prime\prime}$\fi}
\def\arcminm{\ifmmode {^{\scriptscriptstyle\prime}}
          \else $^{\scriptscriptstyle\prime}$\fi}
\def\deg{\ifmmode^\circ\else$^\circ$\fi}





%\markright{{\bf LAB E: Hubble's Law} \ \hrulefill \ }


\noindent
%\vspace{0.15in}
{\bf Observational Astronomy    \hfill} 
% {\bf   } \hfill {\bf Last Name:\makebox[4cm]{\hrulefill}}


\bigskip

\medskip

\noindent
{\hfill \Large {\bf Starry Night: Information Sheet} \hfill}


\bigskip

\noindent
Starry Night is a sky simulation program that we shall use often in
the indoor labs. Our version is called Starry Night Professional. It
has lots of features that are fairly intuitive to use once you get the
ideas behind it. This sheet provides a brief introduction and some simple
commands to start you off. Try them out.

\medskip
\noindent
{\bf 1. Overview of operation}

\smallskip
\noindent
When you enter SN you see a view of the sky in a particular direction
at a particular time and date. The view may be from Earth, in NY or
elsewhere, or from some other planet or point in space. The default
field of view is 100\deg\ across which corresponds to a typical view
you get with your eyes. This field of view can be made to change by
zooming in or out. Other changes that can be made include the
direction in which you are looking, the location of the observer, and
the date and time. There are also options like switching on or off the
horizon (so you can see below your feet, which is obscured by the
Earth) and switching on or off daylight (so you can trace the stars
when the Sun is above the horizon). It will be set up so that when you
enter the program, the initial time is the real time, and the location
is NY.

\medskip
\noindent
{\bf 2. How to change things}

\smallskip
\noindent
There are several different ways to change things. The main console is
the Tool Box, the thin, vertical window with the PRO logo at the top. Other buttons and pull
down menus are along the top. The main features of the tool box are:

\begin{itemize}
\item {\bf Mouse controls} (upper panels).  Most useful: the arrow --
makes the mouse
pointer identify stars; the arrow with box -- makes the mouse identify
constellations; the compass -- makes the mouse into an angle measuring device.

\item {\bf Time panel.} Toggles on/off the time control; it should be
usually left on.  This runs like a CD player control: the filled box
is stop, the forward arrow is go. All the entries in the panel
including date, time, and (most important) time-step, can be edited by
over typing.

\item {\bf Planets panel}. Toggles on/off planet control.

\item {\bf Location panel}. Change the place you are observing from,
e.g. latitude. The Lat and Lon are shown in the readout below this panel.

\item {\bf Display panel}. Toggles on/off display options. Usually keep it
off, some of the changes can be made from the buttons along the top of
the screen instead.

\item {\bf Rocket panel}. More on this at a later date.

\item {\bf Magnifying glasses}. These zoom in and  out. The field of
view is given in the readout below them. Alternatively the readout can
be edited. 


\end{itemize}

\medskip
\noindent
{\bf 2. Some basic operations}
\begin{itemize}

\item {\bf To look at the horizon to the N, S, E, or W:} click the
  buttons labeled N etc. at the top of screen.

\item {\bf To look up/down or side to side:} use keyboard arrow keys, or sliders at
  bottom and right of screen.

\item{\bf To turn on/off horizon:} click tree at top, or display panel, then horizon.

\item{\bf To turn on/off daylight:} click little blue box at top or display
  panel, then daylight.

\item{\bf To identify a star:} Toggle arrow in tool box (usually already
  on) point at star with  mouse.  

\item{\bf To identify a constellation:} Toggle arrow plus box in tool
  box, then click on constellation. To de-activate, click right mouse and
  De-select with left mouse.

\item{\bf Alt/Az or Ra/Dec grid lines:} Toggle Lo or Eq buttons at the top.

\item{\bf To change location of observer:} Click Location button in
  tool box and edit lat/lon or move red circle.

\item{\bf To change the time or date:} In time box (activated from tool
  box but best left on) click time or date and overtype, or click
  little arrows to increase/decrease.

\item{\bf To change time step:} In time box, click units (minutes,
  hours, etc.) to change on pop-up menu. For values, click numbers and overtype or use
  increase/decrease arrows. 

\item{\bf To start/stop time:} The time box buttons from left to right
  are: single step (back) -- back -- stop -- forward -- forward (in real
  time) -- single step (forward).


\item{\bf To zoom in/out:} In tool box, click +/$-$ magnifying
  glass. To return to standard 100\deg\ field click square magnifying glass.

\item{\bf To measure an angle:} In tool box, click compass. Then hold down
  control on keyboard, and  drag
  with left mouse across the angle to be measured.  Note if done
  without the control down, it skips to nearest stars, which is
  sometimes not what you want to measure.  

\end{itemize}


\medskip
\noindent
{\bf 2. Emergency! Lost in space or time?}

\smallskip
\noindent
Click on the location panel, click on the
return home button, click on the S button along the top. This should
bring you back home, looking south. Check the time and reset ``now''
if that's important.




\end{document}




