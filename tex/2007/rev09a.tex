 
\documentclass[11pt]{article} 
\topmargin -.6in 
\textheight 8.7in 
\oddsidemargin -.0in 
\textwidth 6.5in 
\title{The Analysis of Starlight: Lab Projects} 
\date{Fall 1997} 
%\renewcommand{\baselinestretch}{1.2} 
\begin{document} 
\setcounter{page}{1} 
\setcounter{equation}{0} 
\pagestyle{empty} 
\parindent 0pt 
\parskip 8pt 
%\pagestyle{myheadings} 
\markright{{\bf LAB B: Measuring the Wavelength of Light} \
\hrulefill \ } 
\def\arcsec{\ifmmode {^{\scriptscriptstyle\prime\prime}}
          \else $^{\scriptscriptstyle\prime\prime}$\fi}
\def\arcmin{\ifmmode {^{\scriptscriptstyle\prime}}
          \else $^{\scriptscriptstyle\prime}$\fi}
\def\deg{\ifmmode^\circ\else$^\circ$\fi}


   
 
\noindent 
%\vspace{0.15in} 
{\bf Observational Astronomy      \hfill  V85.0013}\\ 
 
\bigskip 
 
\bigskip 
 
\noindent 
{\hfill \Large {\bf Review Sheet 9a} \hfill} 
 
 
\bigskip
\begin{enumerate} 

\item
Points at which the orbit crosses the ecliptic.

\item 
The totality path for a solar eclipse is a narrow strip (few 100 miles
wide at most)
whereas a total lunar eclipse can be seen from anywhere on the
side of Earth facing the Moon.

\item 
West. The Moon moves west to east relative to the Sun.


\item 
At the equinoxes, where the orbit crosses the ecliptic.

\item 
When the angular size of Sun is larger than Moon. This effect is a maximum
when the Earth is at perihelion (closest to Sun), and the Moon is at 
apogee (farthest from Earth) 

\item 
The rough mottled solar surface, caused by convection cells.


\item 
They are cooler, and emit less intense radiation than the average
surface, and therefore look relatively dark.


\item 
The pairs are the surface penetration points of loops of magnetic field.

\item 
East to west. Spot can be tracked about two weeks. Rotation period is
25--30+ days.


\item 
Spots are short lived. The 11 yr cycle is the period of increase and
decrease of the average number of spots (and related phenomena).

\item 
They have little prisms inside that flip the image upright.

\item 
To correct for chromatic aberration.

\item 
50.

\item 
2040 mm.

\item 
Diameter of the objective.

\item 
36.

\item 
3 times.

\item 
Moon is 0.5 deg. To get Moon to 40 deg, need M= 80.
Thus need focal length 25~mm.

\item 
Light collection depends on area, so it collects 100 times more light,
or 5 magnitudes.


\item 
Black - no light.

\item 
More atmosphere to see through.



\item 
Largest asteroid.

\item 
Infall from the very extended comet cloud (the Oort Cloud) way outside
the orbit of Pluto.


\item 
About 100 km up.


\item 
Between Mars and Jupiter.


\item 
Meteors are transient (few secs max) so are not smeared out by the
rotation of the celestial sphere.


\item 
Debris of comets.

\item
The radiants lie in the constellations referred to.

\item 
Larger of the meteor bodies that produce very bright, sometimes
explosive events. These usually also produce meteorites, the chunks
that land.

\item 
Asteroids.


\end{enumerate} 

\end{document}










