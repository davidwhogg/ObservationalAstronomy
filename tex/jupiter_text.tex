
\noindent
{Objectives:} To observe the characteristics of Jupiter and its moons.

\bigskip\noindent

\bigskip
\noindent
{\bf 1. Jupiter }

\medskip
\noindent
In a small telescope Jupiter is a nicely resolved disk, and its
brightest moons are easily visible. They are Io, Europa, Ganymede, and
Callisto (in order of distance from Jupiter). They were originally
discovered by Galileo --- one of the first astronomical discoveries
using a telescope.

If time permits, we shall try to observe Jupiter at a later observing
session as well to record the changing moon configuration: the orbital
times of the moons are days to weeks.

\bigskip
\noindent
{\bf 2. Observations}

\medskip
\noindent

\noindent $\bullet$ \ 
Record the date.

\medskip
\noindent $\bullet$ \ 
 Identify the naked eye stars around Jupiter, and thereby locate its 
position in the Atlas. Estimate the RA and Dec from the map,
and identify the constellation.

\medskip
\noindent $\bullet$ \ 
Find Jupiter with one of the 8 inch equatorial
telescopes.  As a preliminary, determine the directions of N, S, E,
and W in the eyepiece and label the figures on the observing sheet
accordingly. Note these may not be what you expect, due to the optical
setup: they can be determined by moving the fine controls and seeing
which way Saturn moves.

\medskip
\noindent $\bullet$ \ 
Examine Jupiter's moons, and sketch the results using the
larger circle on the observing sheet as a guide. If the phase is not exactly
full, shade in the limb of the disc that is missing (this should of
course correspond to the side away from the Sun). 
Study any visible cloud patterns and record them on the
disk.  

\medskip
\noindent $\bullet$ \ Search around Jupiter (within 10 diameters from
the center) for any faint objects which are likely to be the brighter
moons. Record their position around the smaller figure in the
observing sheet (the numbers marked are Jupiter diameters from the
center). Note that the moon system around Jupiter can be significantly
tilted to the line of sight, {\bf so the moons may appear to the side
  or above/below the planet.}

\newpage
\noindent
{\bf 3. Jupiter Observation I}
\bigskip\bigskip
\noindent

Date: \makebox[2cm]{\hrulefill} \ \ 
Constellation: \makebox[2cm]{\hrulefill} \ \ 
RA: \makebox[2cm]{\hrulefill} \ \ 
Dec: \makebox[2cm]{\hrulefill} \ \ 

\vspace{3.0cm}


\begin{figure}[h]
\centerline{\psfig{figure={o4s_f1.eps},width=11.0cm}}

\vspace{3.0cm}

\centerline{\psfig{figure={o4s_f2.eps},width=2.5cm}}

\end{figure}
 \vspace{1.0cm}
\bigskip\bigskip\bigskip

\newpage
\noindent
{\bf 4. Jupiter Observation II (moons only)}
\bigskip\bigskip
\noindent

Date: \makebox[2cm]{\hrulefill} \ \ 
Constellation: \makebox[2cm]{\hrulefill} \ \ 
RA: \makebox[2cm]{\hrulefill} \ \ 
Dec: \makebox[2cm]{\hrulefill} \ \ 

\vspace{3.0cm}


\begin{figure}[h]
\centerline{\psfig{figure={o4s_f1.eps},width=11.0cm}}

\vspace{1.0cm}


\end{figure}

