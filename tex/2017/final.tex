\documentclass[12pt]{article}
\usepackage{enumitem}
\usepackage{url, graphicx, epstopdf}

% page layout
\setlength{\topmargin}{-0.25in}
\setlength{\textheight}{9.5in}
\setlength{\headheight}{0in}
\setlength{\headsep}{0in}
\setlength{\parindent}{1.1\baselineskip}

% problem formatting
\newcommand{\problemname}{Problem}
\newcounter{problem}

% math
\newcommand{\dd}{\mathrm{d}}
\newcommand{\e}{\mathrm{e}}

% primary units
\newcommand{\rad}{\mathrm{rad}}
\newcommand{\kg}{\mathrm{kg}}
\newcommand{\m}{\mathrm{m}}
\newcommand{\s}{\mathrm{s}}

% secondary units
\renewcommand{\deg}{\mathrm{deg}}
\newcommand{\km}{\mathrm{km}}
\newcommand{\cm}{\mathrm{cm}}
\newcommand{\mi}{\mathrm{mi}}
\newcommand{\h}{\mathrm{h}}
\newcommand{\ns}{\mathrm{ns}}
\newcommand{\J}{\mathrm{J}}
\newcommand{\eV}{\mathrm{eV}}
\newcommand{\W}{\mathrm{W}}

% derived units
\newcommand{\mps}{\m\,\s^{-1}}
\newcommand{\mph}{\mi\,\h^{-1}}
\newcommand{\mpss}{\m\,\s^{-2}}

% random stuff
\sloppy\sloppypar\raggedbottom\frenchspacing\thispagestyle{empty}

\begin{document}

\section*{Observational Astronomy---Final Exam}

\textsl{This midterm is composed of short-answer questions about the
  reading, lecture, and lab components of this course. Do your work on
  your own (though feel free to do research on the internets or in
  your books). If you discuss these questions with another class
  member, say who that is on your answer sheet. Your answers are due
  at Prof Hogg's office (726 Broadway, room 939; put it under the door
  if Prof Hogg isn't there) before \textbf{2017 Monday May 15 17:00
    EDT}.}

\paragraph{Repeat questions from the midterm:}
\begin{enumerate}
\item  What is the declination of the Sun on March 21? June 21?
  September 23? December 21?
\item In Tallahassee, at what declinations are the stars that never set?
  that never rise?
\item Say you observe a star at Dec = 0 passing through a telescope's
  field of view in 4 minutes. What is the angular size of that field
  of view?
\item If a star is transiting at midnight tonight, when will it
  transit tomorrow night? A month from now?
\item Which planets are visible to the naked eye?
\item Why does Venus have ``phases'' like the Moon but Jupiter does
  not?
\end{enumerate}

\paragraph{New reading questions:}
\begin{enumerate}[resume]
\item What is the difference between an equatorial telescope and an
  alt-az telescope?
\item If an equatorial telescope is tracking, how long does it take to
  complete one revolution?
\item What is an umbra? What is a penumbra? Draw them.
\item What are the two tails of a comet?
\item How do we know the distances to the nearest stars?
\item What is the Doppler shift?
\end{enumerate}

\paragraph{Indoor Lab questions:}
\begin{enumerate}[resume]
\item What is the angular separation between Betelgeuse and Sirius?
\item What is the angular speed of the Sun \emph{relative} to the
  constellations on the ecliptic?
\item What is the Alt and Az of Polaris in New York City?
\item The new Moon sets in New York City two hours after the Sun
  sets. That makes it a crescent Moon. Which way to the horns of the
  crescent Moon point at Moonset?
\item What is the maximum elongation of Venus?
\item When in retrograde, is Mars biggest or smallest? Why?
\end{enumerate}

\paragraph{Outdoor Lab questions:}
\begin{enumerate}[resume]
\item In what constellation is Betelgeuse?
\item On what Mag-5 Atlas chart is Betelgeuse?
\item Why did we begin our telescope labs pointing the telescopes at
  Polaris?
\item Roughly what kind of field of view (in degrees) did your
  telescope have when you measured or estimated it?
\item What is the name of a Moon crater you saw in your outdoor lab?
\item How many moons of Jupiter are easily visible?
\end{enumerate}

\paragraph{Lecture questions:}
\begin{enumerate}[resume]
\item A star transits at 10:01 pm on one night. At what time does it
  transit the next night?
\item At what time of day does a first-quarter moon rise?
\item What is the difference between a short-period and a long-period comet?
\item If you have a 2-m focal-length telescope with a 20-mm eyepeice,
  what is the angular magnification you get?
\item What is the most common kind of planet? That is, what radius
  does it have, roughly?
\item What is the most stable atomic nucleus?
\end{enumerate}

\end{document}
