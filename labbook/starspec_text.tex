
\noindent Today we will examine the colors, the ``spectra,'' and the
velocities of stars. Most of the stars we will look at are {\it much}
fainter than you can see with your eye --- however, they are very
similar physically to those you are familiar with, just further away
in our Galaxy. We will also take a look at a handful of galaxies, and
try to understand how the stellar content of the galaxies affects
their images and spectra.

\noindent Below is a spectrum of the Sun, in black, and a blackbody
spectrum at T=5777 Kelvin in green. Notice how the peaks coincide:
this indicates that the temperature of the sun is close to a
temperature of T=5777 Kelvin, and notice the features in the sun
spectrum: absorption and emission from various elements, while the
blackbody spectrum is smooth.

\begin{figure*}[h!]
\begin{center}{\psfig{figure={Sunspec.eps},width=10.0cm}} \caption{A
spectrum of the Sun.}
\end{center}
\end{figure*}

\noindent To begin, open a Mozilla Firefox browser on your computer to
the URL {\tt http://data.sdss.org}.  This site gives you access to a
set of real astronomical data publicly released by a team of
astronomers (including us at NYU).  In the menu, go to ``Optical
Spectra'' and under that ``Bulk Search.''

\noindent
{\bf 1. Colors of stars}

\noindent Enter the right ascensions and declinations listed below
into the ``bulk search'' form. Use a ``Search Range'' of 0.05
arcminutes. When you submit the search, it should return a table of
rows.  Each row corresponds to a star, and by hitting ``Plot'' in the
appropriate row you can look at the image and the spectrum of the
star. You can also hit ``CAS'' which will open a new tab, that also
shows the image of the object.

\begin{center}
\begin{tabular}{ccccccc} \hline \\ [-6pt]
R.A. (deg)  & Dec (deg)  & \hspace{0.2cm} Type \hspace{0.2cm} 
& \hspace{0.2cm} $\lambda_{\mathrm{peak}}$ (Ang.) \hspace{0.2cm} 
& \hspace{0.2cm} T (K) \hspace{0.2cm} 
& \hspace{0.2cm} H\&K? \hspace{0.2cm} 
& \hspace{0.2cm} Balmer?  \hspace{0.2cm} 
\\ [6pt]
\hline
228.06551 &  6.865659 & & & & &    \\ \hline
16.437201 &  -10.7071  & & & & &   \\ \hline
17.848558 &  16.327376 & & & & &    \\ \hline
151.95765 &  34.369592 & & & & &    \\ \hline
47.249486 &  38.01783  & & & & &   \\ \hline
64.948151 &  5.261293  & & & & &   \\ \hline
146.77222 &  62.631055 & & & & &    \\ \hline
\end{tabular}
\end{center}

\noindent First consider just the colors of the stars in the image on
the bottom left.  

\noindent Note that these images, in addition to
being far deeper than what you can see with your eye, exhibit somewhat
more color contrast as well.

\noindent Under the ``Type'' column, rank the list above from bluest
to reddest according to the image that you see. Use the labels O, B,
A, F, G, K, M from bluest to reddest.

\noindent The bluer of the stars are the hotter ones, and usually the
more massive and younger stars as well.  

\noindent
{\bf 2. Relationship of colors to the spectrum}

\noindent Now consider the ``spectrum'' shown on each plot.  The
curve shown is the amount of light emitted from each wavelength, from
blue (short) wavelengths on the left to red (long) wavelengths on the
right.  The unit of length here is the Angstrom, which is $10^{-10}$
meters. Your eye is sensitive to light between about 4000 and 7000
Angstroms.

\noindent For a perfect ``blackbody'' spectrum, the temperature of a
star would be related to its peak wavelength of emission by the simple
formula: $\lambda_{\mathrm{peak}} T \approx 3$ mm Kelvin. Although
stars are not perfect blackbodies, there is still an approximate
relationship between the color and the temperature.

\noindent Look at each spectrum and estimate the peak wavelength.  If
the spectrum continues to increase past the left or right edge of the
spectrum, indicate the maximum or minimum possible wavelength. 

\noindent Use the peak wavelength to deduce the temperature (or the
minimum or maximum temperature where appropriate).  Notice that the
temperature is closely related to the color classification in the
first section. This temperature is the effective temperature at the
surface of the star --- for all these stars the temperatures in their
centers where nuclear fusion is occurring is much hotter!

\noindent
{\bf 3. Features in the spectra}

\noindent There are a number of clear absorption features (called
``lines) in the spectra, seen as deep troughs at particular
wavelengths. There are many sets of lines associated with a number of
different absorption features, but we'll be interested today in just
two sets of them: the calcium H and K lines, and the hydrogen Balmer
lines.

\noindent The calcium H and K lines are near the left edge of these
spectra, at around 3933 Angstroms (K) and 3969 (H) Angstroms. They are
produced by trace amounts of calcium (gaseous calcium!) in the outer
atmospheres of the stars. Look for these in the spectra and mark the
table above according to whether each star has visible H \& K.  Which
star has the strongest calcium features?

\noindent The Balmer lines are features due to hydrogen in the outer
atmosphere. There is a large set of them: H$\alpha$ at 6563 Angstroms,
H$\beta$ at 4861 Angstroms, H$\gamma$ at 4341 Angstroms, H$\delta$ at
4102 Angstroms, H$\epsilon$ at 3970 Angstroms, etc.  Note in the above
table which stars have these, and which star has the strongest
features.

\noindent 
There is always hydrogen in stellar atmospheres, but it has to be at
the right temperature to produce deep hydrogen lines. Using the table
above, guess what that temperature is.

\vspace{30pt}

\noindent Which type(s) of stars have neither H \& K lines nor Balmer
lines?

\vspace{30pt}

\noindent
{\bf 4. Doppler shifts in galaxies}

\noindent Now we will look at a handful of galaxies.  As you probably
know, galaxies are distant analogs to the Milky Way, and consist of
conglomerations of 10s of billions of stars. This fact will be evident
when we look at their spectra. Search for the following spectra in the
``bulk search'' window (but keep your results window for the stars
open for comparison). 

\begin{center}
\begin{tabular}{ccccccc} \hline \\ [-6pt]
R.A. (deg)  & Dec (deg)  & \hspace{0.02cm} Star type \hspace{0.02cm} 
& Young/old?
& \hspace{0.02cm} K line (Ang.) \hspace{0.02cm} 
& \hspace{0.02cm} H$\delta$ (Ang.) \hspace{0.02cm} 
& \hspace{0.02cm} Velocity (km/s) \hspace{0.02cm} 
\\ [6pt]
\hline
153.4605 &  38.764896 & & & \\ \hline
140.03594 &  8.1503628 & & & \\ \hline
35.170815 &  1.0521941 & & & \\ \hline
185.7200 &  6.6798709 & & & \\ \hline
173.41101 &  4.1961732 & & & \\ \hline
\end{tabular}
\end{center}

\noindent You will notice for some galaxy there are large ``spikes,''
which we call ``emission lines.'' The light at these wavelengths is
not coming from the stars, but instead from gas in HII regions within
the galaxy --- their presence is usually a sign of ongoing
star-formation in the galaxy (though certain lines indicate the
presence of a black hole at the center of the galaxy).  Note that the
most prominent set of lines are Balmer {\it emission} lines!

\noindent Look for the H and K lines of calcium and the Balmer lines
for each spectrum.  Because Balmer emission makes it hard to see the
lines, you will have to search for Balmer emission in the H$\delta$
line to see it in the cases where it is there.  List above the
location in the spectrum of the K line and the H$\delta$ line in cases
where you see them. 


\begin{figure*}[h]
\centerline{\psfig{figure={doppler.eps},width=13.0cm}}
 \end{figure*}

\noindent These galaxies are moving quickly away from us. Remember 
that light travels at a finite speed.  Just like the sound of a siren
is affected by the relative motion of the siren (the source) and us
(the listener), the motion of celestial objects away from us modifies
the frequency of the light emission. When an ambulance drives toward
us the sound will be higher pitched: higher frequencies and lower
wavelengths. When the ambulance moves away the frequency decreases and
the wavelength gets larger. This is a purely geometric effect: look at
the figure in the next page to visualize it. Similarly, these galaxies
are receding from us due to the expansion of the universe, and the
light they emit will appear at a larger wavelenth than it was emitted
at: it will be ``reddened''. We call this ``cosmological redshift''.
Therefore, you will notice that the wavelengths of these lines are
{\it redshifted} from their positions in the stellar spectra, which
are almost at rest with respect to us. This Doppler shift is related
to the velocity by the equation:

\begin{equation}
\frac{\lambda_{\mathrm{obs}}}
{\lambda_{\mathrm{rest}}} = 1 + \frac{v}{c}
\end{equation}

\noindent where $c=299792$ km/s is the speed of light. Use this
equation to deduce the velocities of recession for the galaxies above,
and list it in the table.

\noindent
{\bf 5. Colors of galaxies}

\noindent Compare the spectra of the galaxies to the spectra of the
stars.  Write down which type of star each galaxies spectrum is most
similar to. The bluest stars are the shortest-lived, and the reddest
stars are the longest lived --- based on this fact write down whether
each galaxy is young or old (consider O, B and A stars ``young'' and
any cooler star to be ``old''). To put these ``ages'' in perspective,
A stars live for a billion years!

\noindent Look at the colors in the images of the galaxies --- you
should be able to notice that the ``younger'' galaxies are in general
bluer than the ``older'' galaxies.

\noindent Look up the galaxy at RA $=$ 215.08136 and Dec $=$
3.9327125. Click on the image on the bottom left to get a larger
view.  In general, where does the star-formation in the three big
galaxies you see seem to be occuring? Nearer their centers or further
out?

\vspace{30pt}

\noindent
{\bf 6. How faint are these stars and galaxies?}

\noindent The images and spectra you looked at in this lab are MUCH
fainter than visible with the human eye --- they were obtained with a
2.5-meter telescope. To get a sense, enter the coordinates of the star
Castor (RA $=$ 113.65, Dec $=$ 31.888) into the Navigate page for CAS
({\tt http://skyserver.sdss.org/dr15/en/tools/chart/navi.aspx}); this
is the same imaging you were looking at for the earlier objects in
this lab. Castor is about as bright as Polaris.  You will need to zoom
back in the image to see what is going on. Castor is much brighter
than all of the other objects we have been looking at!

\clearpage

