
\noindent
{\bf 1. Finding Objects.}  

\noindent The best technique for finding an object depends on how
bright it is. In this lab we will begin with objects which are naked
eye, or small offsets from bright stars. We will use the equatorial
telescopes. In subsequent labs we will try more difficult cases.

\medskip\noindent In particular, for this lab, we will only rely on
the Edmund atlas.  As you will see later, this is insufficient for
finding anything not very close to a reasonably bright star.

\medskip\noindent As you proceed below, for each object begin with a
long eyepiece, so a large field of view, when you are first trying to
find it. Once you have centered the object of interest using that
eyepiece, you can insert a smaller one for a more magnified view if
desired.

\medskip\noindent Note that the image is \emph{inverted} -- upside
down -- in the finder, but \emph{upright} in the main eyepiece. It is
also important to have a good idea of the size of the patch of sky
you are looking at. In the finder, this is about 7\deg, and in the
main eyepiece it is much smaller.

\bigskip
\bigskip
\bigskip
\noindent
{\bf 2. Magnification and Field.}

\medskip\noindent {\bf Field estimate.} TF = AF/M, where M
is the magnification given by $f_o/f_e$. For your eyepiece, estimate
the magnification and the true field assuming AF =40\deg.

\medskip\noindent $f_o$ (mm): \makebox[1.5cm]{\hrulefill} \ $f_e$
(mm): \makebox[1.5cm]{\hrulefill} \ M: \makebox[1.5cm]{\hrulefill} \ \
Field of View (\arcmin):\makebox[2cm]{\hrulefill}


\medskip\noindent {\bf Field Measurement.} Time the crossing of a star
near Dec = 0\deg\ in the eyepiece field (with tracking turned off).

\medskip\noindent Time in minutes: \makebox[2cm]{\hrulefill} \ \ \ Field
of View (\arcmin): \makebox[2cm]{\hrulefill}

\bigskip
\bigskip
\bigskip\noindent
{\bf 3. Observing.} 

\medskip\noindent You should try to find four or more objects, in
increasing order of difficulty.  These are suitable for lab times in
late January or early February; most of them are Map 3 in Edmund.

\begin{enumerate}
\item Naked eye target: Any currently visible planets, and/or Albireo
(in Cygnus). Albireo is a double star system about 420 light years
away. There is an orange star in a red giant phase (which itself has a
fainter companion orbiting it that your telescopes will not
resolve). The bluer one is a main sequence B star. 
\item Within finder scope of naked-eye:
M29, near Sadr in Cygnus. This is a small open cluster about 5000
light years away and only about 10 million years old.
\item Using offset from naked-eye star:
M2: to get it, offset West from $\alpha$ Aquarii or North from $\beta$
Aquarii.
This globular cluster is about 40,000 light years away. The stars in
it formed about 13 billion years ago, making them among the oldest
things in the universe. Note that although I'm reasonably sure we can
observe this successfully, I haven't looked for it in NYC before, so
we will just have to find out!
\end{enumerate}

\medskip\noindent 
On the next page, record the characteristics of the
objects you find, according to their type.

\medskip\noindent 
Binary: sketch the binary with its actual orientation in the field of
view in the circle provided.  Label the stars with their colors, and
indicate which is the brighter. Estimate the angular separation of the binary.

\medskip\noindent 
Open cluster: sketch the brightest 5--10 stars with the actual
orientation in the field of view in the circle provided. Estimate the
angular size of the open cluster.

\medskip\noindent 
Planet:  sketch in the disk, indicate any features, shadows, phase,
moons etc. Estimate the angular size of the planet.

\medskip\noindent
Globular Cluster: sketch the shape and the main star field in its
vicinity. Estimate the angular size of the cluster.


\newpage

\parbox[b]{8cm}{ Object: \makebox[3cm]{\hrulefill}\\
RA/Dec: \makebox[3cm]{\hrulefill} \\
Type: \makebox[3cm]{\hrulefill} \\
Size/separation:  \makebox[1.5cm]{\hrulefill} \\ }   \begin{minipage}[b]{8cm}{\psfig{figure={o3s_f1.eps},width=4.0cm}}\end{minipage}

\bigskip\noindent 

\parbox[b]{8cm}{ Object: \makebox[3cm]{\hrulefill}\\
RA/Dec: \makebox[3cm]{\hrulefill} \\
Type: \makebox[3cm]{\hrulefill} \\
Size/separation:  \makebox[1.5cm]{\hrulefill} \\ }   \begin{minipage}[b]{8cm}{\psfig{figure={o3s_f1.eps},width=4.0cm}}\end{minipage}

\bigskip\noindent 

\parbox[b]{8cm}{ Object: \makebox[3cm]{\hrulefill}\\
RA/Dec: \makebox[3cm]{\hrulefill} \\
Type: \makebox[3cm]{\hrulefill} \\
Size/separation:  \makebox[1.5cm]{\hrulefill} \\ }   \begin{minipage}[b]{8cm}{\psfig{figure={o3s_f1.eps},width=4.0cm}}\end{minipage}

\bigskip\noindent 

\parbox[b]{8cm}{ Object: \makebox[3cm]{\hrulefill}\\
RA/Dec: \makebox[3cm]{\hrulefill} \\
Type: \makebox[3cm]{\hrulefill} \\
Size/separation:  \makebox[1.5cm]{\hrulefill} \\ }   \begin{minipage}[b]{8cm}{\psfig{figure={o3s_f1.eps},width=4.0cm}}\end{minipage}

\bigskip\noindent 

\parbox[b]{8cm}{ Object: \makebox[3cm]{\hrulefill}\\
RA/Dec: \makebox[3cm]{\hrulefill} \\
Type: \makebox[3cm]{\hrulefill} \\
Size/separation:  \makebox[1.5cm]{\hrulefill} \\ }   \begin{minipage}[b]{8cm}{\psfig{figure={o3s_f1.eps},width=4.0cm}}\end{minipage}

\bigskip\noindent 


