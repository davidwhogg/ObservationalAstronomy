
\noindent The point of this lab and other similar labs is to 
to locate an object on the sky using the telescope.

\noindent Unlike in the ``First observations'' lab, here we will not
limit ourselves to the regions around bright stars.  Instead, you will
use the Sky \& Telescope Atlas to help you scout out the route to your
objects.  

\begin{enumerate}  
\item Find the RA and Dec of Messier 5. This globular cluster is
around 10 billion years old and has a mass of about $10^6$ solar
  masses. It lies around 25,000 lightyears away. It was originally
  discovered in 1702. 
\item Determine the following things about the object:
\begin{enumerate}
\item At what altitude does it transit when observed from New York City?
\item What hour angle will it have at around 
8pm Eastern Time tonight?
\item Will it be rising or setting?
\end{enumerate}
\item Look at the object on the map in Edmund for which it is closest
to the center of the map (that is, not one where it is very near the
edge). What is the nearest naked-eye visible star ($m<3.5$ in NYC!)?
\vspace{40pt}
\item Can you put the bright star and your object both in the finder
field (say it has 7 deg diameter)? 
\begin{enumerate}
\item If so,  what it will look like
(assuming the finder does not invert the view, which our finders do
not)?
\vspace{180pt}
\clearpage
\item If not, plot on a separate sheet a plan for how to star-hop from
  the nearest bright star to this one, as described in class, by
  drawing below the positions of the bright stars and how you would
  try to move the field of view of the telescope to close to the
  target position. Use the Sky \& Telescope Atlas to include faint
  stars not included in Edmunds: those will be essential. This will
  work best if you use a ruler!
\item Draw up a similar plan for offsetting the dials of an equatorial
  telescope.  Again, draw the path and the intermediate steps, and use
  Peterson's Guide. 
\end{enumerate}
\vspace{220pt}
\item When you actually
  observe the object, draw the view in the telescope below or on a
  separate sheet. Can you estimate the angular size of the object? 
\end{enumerate}

\clearpage

~
