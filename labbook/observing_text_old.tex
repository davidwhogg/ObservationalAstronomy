
\noindent
{\bf 1. Finding Objects.}  

\noindent The best technique for finding an object depends on how
bright it is. In this lab we will begin with objects which are naked
eye, or small offsets from bright stars. We will use the equatorial
telescopes. In subsequent labs we will try more difficult cases.

\medskip\noindent In particular, for this lab, we will only rely on
the Edmund atlas.  As you will see later, this is insufficient for
finding anything not very close to a reasonably bright star.

\medskip\noindent As you proceed below, for each object begin with a
long eyepiece, so a large field of view, when you are first trying to
find it. Once you have centered the object of interest using that
eyepiece, you can insert a smaller one for a more magnified view if
desired.

\medskip\noindent Note that the image is \emph{inverted} -- upside
down -- in the finder, but \emph{upright} in the main eyepiece. It is
also important to have a good idea of the size of the patch of sky
you are looking at. In the finder, this is about 7\deg, and in the
main eyepiece it is much smaller.

\bigskip
\bigskip
\bigskip
\noindent
{\bf 2. Magnification and Field.}

\medskip\noindent {\bf Field estimate.} TF = AF/M, where M
is the magnification given by $f_o/f_e$. For your eyepiece, estimate
the magnification and the true field assuming AF =40\deg.

\medskip\noindent $f_o$ (mm): \makebox[1.5cm]{\hrulefill} \ $f_e$
(mm): \makebox[1.5cm]{\hrulefill} \ M: \makebox[1.5cm]{\hrulefill} \ \
Field of View (\arcmin):\makebox[2cm]{\hrulefill}


\medskip\noindent {\bf Field Measurement.} Time the crossing of a star
near Dec = 0\deg\ in the eyepiece field (with tracking turned off).

\medskip\noindent Time in minutes: \makebox[2cm]{\hrulefill} \ \ \ Field
of View (\arcmin): \makebox[2cm]{\hrulefill}

\bigskip
\bigskip
\bigskip\noindent
{\bf 3. Observing.} 

\medskip\noindent You should try to find four or more objects, in
increasing order of difficulty. Choose at least one from each category
below. These are suitable for lab times in late January or early
February; most of them are Map 3 in Edmund.

\begin{enumerate}
\item Naked eye targets:
\begin{enumerate}
\item Any currently visible planets
\item Mirfak (early in semester)
\item Aldebaran (early in semester)
\item Alcyone (early in semester)
\item Mizar/Alcor (middle, view blocked early)
\item Algeiba (later in semester)
\item Porrima (later in semester)
\end{enumerate}
\item Within finder scope of naked-eye:
\begin{enumerate}
\item Theta Tauri, 6 arcminute double (early)
\item Lambda Orionis, very close binary (early)
\item NGC 663, open cluster in Casseiopia (early)
\item Tau Leo, 90 arcsec binary (later)
\item 38 Lynx (later)
\item M13, bright globular (later)
\end{enumerate}
\item Using offset from naked-eye star:
\begin{enumerate}
\item 30 Aries: wide double; to get it, offset East from Hamal (early)
\item M34: open cluster; to get it, offset West from Algol (early)
\item NGC 2244: sparse open cluster; to get it, offset South from 
  Alhena (early)
\item UU Aurigae: very red variable, offset East from Theta Auriagae (middle)
\item M44: large open cluster (later)
\item M67: open cluster (later)
\item SS Virgo: very red variable, offset West from Porrima (later)
\end{enumerate}
\item Finally, reward yourself by finding M42, the Great Nebula in
  Orion (early in the semester).
\end{enumerate}

\medskip\noindent 
On the next pages, record the characteristics of the
objects you find, according to their type.

\medskip\noindent 
Binary: sketch the binary with its actual
orientation in the field of view in the circle provided.
Label the stars with their colors, and indicate which
is the brighter.  

\medskip\noindent 
Open cluster: sketch the brightest 5--10 stars with the actual
orientation in the field of view in the circle provided.
you found it.  

\medskip\noindent 
Planet:  sketch in the disk, indicate any features, shadows, phase, moons etc.

\medskip\noindent
Nebula: sketch the nebula shape and the main star field in its vicinity.

\medskip\noindent 
In all cases you will need to expand the center of the field for the picture.

\newpage

\parbox[b]{8cm}{ Object: \makebox[3cm]{\hrulefill}\\
RA/Dec: \makebox[3cm]{\hrulefill} \\
Type: \makebox[3cm]{\hrulefill} \\
Size/separation:  \makebox[1.5cm]{\hrulefill} \\ }   \begin{minipage}[b]{8cm}{\psfig{figure={o3s_f1.eps},width=4.0cm}}\end{minipage}

\bigskip\noindent 

\parbox[b]{8cm}{ Object: \makebox[3cm]{\hrulefill}\\
RA/Dec: \makebox[3cm]{\hrulefill} \\
Type: \makebox[3cm]{\hrulefill} \\
Size/separation:  \makebox[1.5cm]{\hrulefill} \\ }   \begin{minipage}[b]{8cm}{\psfig{figure={o3s_f1.eps},width=4.0cm}}\end{minipage}

\bigskip\noindent 

\parbox[b]{8cm}{ Object: \makebox[3cm]{\hrulefill}\\
RA/Dec: \makebox[3cm]{\hrulefill} \\
Type: \makebox[3cm]{\hrulefill} \\
Size/separation:  \makebox[1.5cm]{\hrulefill} \\ }   \begin{minipage}[b]{8cm}{\psfig{figure={o3s_f1.eps},width=4.0cm}}\end{minipage}

\bigskip\noindent 

\parbox[b]{8cm}{ Object: \makebox[3cm]{\hrulefill}\\
RA/Dec: \makebox[3cm]{\hrulefill} \\
Type: \makebox[3cm]{\hrulefill} \\
Size/separation:  \makebox[1.5cm]{\hrulefill} \\ }   \begin{minipage}[b]{8cm}{\psfig{figure={o3s_f1.eps},width=4.0cm}}\end{minipage}

\bigskip\noindent 

\parbox[b]{8cm}{ Object: \makebox[3cm]{\hrulefill}\\
RA/Dec: \makebox[3cm]{\hrulefill} \\
Type: \makebox[3cm]{\hrulefill} \\
Size/separation:  \makebox[1.5cm]{\hrulefill} \\ }   \begin{minipage}[b]{8cm}{\psfig{figure={o3s_f1.eps},width=4.0cm}}\end{minipage}

\bigskip\noindent 


