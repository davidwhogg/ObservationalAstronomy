
\noindent The point of this lab is to prepare to locate an object on the
sky that would be observable at a given time and place. Unlike the
other labs, in this one I expect you to make detailed preparations
before the lab. The next time we go outside (and don't have something
else exciting to observe), you will bring this lab as
preparation. There are two copies of this lab in your book because we
will do this exercise twice during the semester.

\noindent Unlike in the ``First observations'' lab, here we will not
limit ourselves to the regions around bright stars.  Instead, you will
use the Sky \& Telescope Atlas to help you scout out the route to your
objects.  

\noindent Depending on our luck, we may do one of these labs as an
indoor lab. In this case, you should try to find the object on the sky
WITHOUT using the ``heads-up display'' in Starry Night (the names of
the stars, etc.  To do this, go to ``Preferences'' under the ``File''
menu.  Choose the ``Cursor Tracking (HUD)'' set of options in the
upper left tab.  In the ``Cursor Tracking'' options, make sure both
the ``When'' boxes (``These keys are down'' and ``Mouse is idle for'')
are NOT CHECKED. This should make it so the information on each object
your cursor touches does NOT come up on the screen.  This will more
closely emulate the problem you will face in the actual sky, which
emphatically does not come with a heads-up display!

\begin{enumerate}
\item Pick an object from the list below, and look up its RA and Dec
  (list it below next to the one you pick):
\begin{enumerate}
\item 31 Cygni
\item Rho Cancri
\item Tau Leonis
\item 27 Coma Berenices
\item NGC 1647
\item Rho Ophiuchi
\item SS Virginis
\item Messier 5
\item Messier 13
\item Messier 35
\item Messier 41
\item Messier 36
\item Messier 37
\item Messier 44
\item Messier 67
\item Tau-one Hydra
\item Theta Virginis
\item IC 4665
\item Eta Perseus
\item 12 Lynx
\item NGC 2264
\item UU Aurigae
\end{enumerate}
\item However, make sure the object you pick satisfies the following
criteria:
\begin{enumerate}
\item Transits at an altitude greater than 25 deg when observed from New York City.
\item Transits within 3 hours of 8pm Eastern Time on the date
the instructor provides. Make sure you know whether we will be on
Daylight or Standard time that day!
\end{enumerate}
there are some objects on the list above that do not, so you have to
check!
\item Look up the object on wikipedia or another on-line resource
(virtually anything listed above is very well known). Give a brief
(3-4 sentence) description of the object below.
\vspace{80pt}
\item What altitude does the object transit at?
\vspace{40pt}
\item Will it be rising or setting at 8pm EDT on that night?  What is its
hour angle?
\vspace{40pt}
\item Look at the object on the map in Edmund for which it is closest
to the center of the map (that is, not one where it is very near the
edge). What is the nearest naked-eye visible star ($m<3.5$ in NYC!)?
\vspace{40pt}
\item Can you put the bright star and your object both in the finder
field (say it has 7 deg diameter)? 
\begin{enumerate}
\item If so,  what it will look like
(assuming the finder does not invert the view, which our finders do
not)? (When you work through this in the indoor lab with Starry Night,
draw what it actually looks like in Starry Night).
\vspace{180pt}
\clearpage
\item If not, plot on a separate sheet a plan for how to star-hop from
  the nearest bright star to this one, as described in class, by
  drawing below the positions of the bright stars and how you would
  try to move the field of view of the telescope to close to the
  target position. Use the Sky \& Telescope Atlas to include faint
  stars not included in Edmunds: those will be essential. This will
  work best if you use a ruler!
\item Draw up a similar plan for offsetting the dials of an equatorial
  telescope.  Again, draw the path and the intermediate steps, and use
  Peterson's Guide. 
\end{enumerate}
\vspace{220pt}
\item Only to be done outside: Is this object ACTUALLY visible at that
  time? If not, at what date is it transiting at that time? Each group
  should observe each object chosen by each member.  When you actually
  observe each object, draw the view in the telescope below or on a
  separate sheet.
\end{enumerate}
