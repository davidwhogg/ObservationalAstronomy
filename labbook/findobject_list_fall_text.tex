
\noindent In the subsequent outdoor labs, you will be finding one or
more objects selected by the instructor. Interesting objects visible
from New York City are listed here and we will be choosing from this
list.

\begin{itemize}  
\item {\bf Mizar/Alcor.} This pair of stars actually  consists of
perhaps six stars in the same system, or at least moving
together. Alcor is a binary, but you will not be able to see its
companion, which is very low mass. Mizar is a quadruple system; with
your telescope you will resolve this into two ``stars,'' each of which
is actually a binary star. They are around 14 arcseconds apart. Mizar
was the first binary discovered with a telescope, by Galileo in 1617.
\item {\bf Double Cluster (NGC 884 and 869).} This pair of open
  clusters is very far away, at around 7,500 light years. They are
  just a few hundred light years from each other. They are very young
  (12 million years) and bright.
\item {\bf Messier 52} This open cluster is young (around 35 million
  years) and far away, somewhere between 3000--7000 lightyears.
\item {\bf $\delta$ Cepheii} This is a double star. The brighter star
varies on a 5.4 day period (between 3.6 and 4.3 mag) and is the
  archetypical ``Cepheid variable'' star on the basis of which the
  distances of galaxies were established. The fainter one is about 41
  arcsec away and is more stable.
\item {\bf Messier 13.}
  This globular cluster is around 12 billion years old and has a mass
  of about $10^6$ solar masses. It lies around 22,000 lightyears
  away. It was originally discovered in 1714 by Edmund Halley. It is
  only observable near the beginning of the semester.
\item {\bf Messier 92.}
  This globular cluster is one of the oldest known; contraints on its
  age are around 13--14 billions years, which is as old as you can
  yet. Otherwise it is very similar to M13, but a bit fainter because
  it is 27,000 light years away. It was originally discovered in 1777
  by Bode. Like M13, we will only see it in the beginning of the
  semester.
\item {\bf $\epsilon$ Lyrae.} This is a pair of double stars. 
Each pair has a very close separation of about 2--3 arcsec, so you
  will need good conditions to resolve them. Each pair is a true bound
  pair. The $\epsilon$-2 pair has a period of about 500 years and the
  $\epsilon$-1 pair has a period of about 1000 years. The two pairs
  are nearly the same distance (about 500 light years) are so are
  likely related physically.
\item {\bf Messier 10.} A globular cluster about 14,000 light years
  away. A bit fainter than M13 or M92, more comparable to M2. About 11
  billion years old, discovered by Charles Messier himself in 1764.
\item {\bf Messier 11.}
  An open cluster about 6200 light years away and 200 million years
  old, discovered in 1681 by Gottfried Kirch. It is one of the largest
  open clusters.
\item {\bf Messier 71.} A globular cluster about 13,000 light years
  away and about 9 billion years old. It was discovered by Philippe
  Loys de Ch{\'e}seaux in 1745. It is fairly diffuse and low in mass.
\item {\bf Messier 27 (Dumbbell Nebula).}
  This is a planetary nebula, the remnant of a red giant star shortly
  after its death. The nebula is gas surrounding a faint white
  dwarf. The Sun will undergo this phase in about 5 billion years. I
  do not know if this nebula can be observed from NYC.
\item {\bf Omicron 1 \& 2.} This is actually a quadrupole star
  system. You should be able to see the close companion of Omicron 1
  (bluer than Omicron 1 itself). Both Omicron 1 and 2 are eclipsing
  binaries; they both have close stellar companions (too close to
  resolve) with several year orbital periods that eclipse the stars.
\item {\bf Messier 15.}
  This globular cluster is around 12 billion years old and is around
30,000 light years away. It was originally discovered by Jean-Dominique
Maraldi in 1746.
\item {\bf Messier 39.}
  This open cluster is about 300 million years old and is around 800
  light years away. It is possible that Aristotle cataloged this
  object based on naked eye observations in 325 BCE, having mistook it
  for a comet.
\item {\bf Messier 2.}
  This globular cluster is around 12 billion years old and is around
40,000 light years away, on the opposite side of the Milky Way
galaxy. It is around 13 billion years old, and therefore one of the
oldest systems in the Universe. It was also discovered by Maraldi in
1746.
\item {\bf $\zeta$ Aquarii.} This binary star has a 587 year period and
a maximum separation of about 7.6 arcsec. But right now they are very
close, less than 2 arcsec apart. So conditions must be quite good to
resolve them! Both stars are F stars, somewhat more massive than our
Sun.
\item {\bf NGC 7789.} This open cluster has many faint stars. It is a
quite old open cluster (about 1.7 billion years) and is far away (over
20,000 light years), and both facts contribute to its faintness. It
was discovered by Caroline Herschel in 1783. I am unsure how easy it
will be to see from NYC; generally I steer clear of NGC objects. But
it may be worth a try!
\item {\bf Andromeda Galaxy.} This relatively massive galaxy is observable
with the naked eye in the darkest sites, but only barely visible with
  our telescopes when in New York City. It is about $10^{11}$ solar
  masses and is around two million light years away. If you find it,
  the faint smudge you see is really just the center of the galaxy,
  which extends over several degrees. 
\item {\bf Almak ($\gamma$ Andromedae).} A nice double star with 10
  arcsec separation.
\item {\bf Mira.} In Cetus, this is a long-period variable of about
300 days period, which doesn't quite vary exactly periodically. It
  varies from magnitude 2 to 10---i.e. from one of the brightest stars
  on the sky to needing a telescope to find.
\item {\bf Messier 33 Galaxy.} This galaxy is not far from Andromeda
  physically and on the sky, but it is much less massive and
  correspondingly harder to see. It is unclear who first cataloged it
  but it probably was Giovanni Battista Hodierna sometime before 1654.
\item {\bf 30 Arietis.} The two components of this binary star (A and
  B) are separated by 38 arcsec, or about 1500 AU at their distance of
  130 light years.  They are each a bit more massive than the Sun
  (1.1--1.3 solar masses) and are about a billion years old.
\item {\bf Messier 34.} This open cluster is around 1,500 light years
  away. It is about the same size as Messier 44 but is younger
  (200-300 million years) and will appear smaller and less obvious on
  the sky due to its greater distance. 
\item {\bf $\lambda$ Orionis (Meissa).} This binary star with
a 4 arcsec separation is very young (around 5 million years old), and
  consists of two massive stars (one 30 solar masses and one around 10
  solar masses). They are part of a large open cluster, and are
  ionizing a huge region of the surrounding gas. This pocket of star
  formation is at about the same distance ($\sim 1300$ light years) as
  the Great Nebula, Messier 42, which you have already seen.
\end{itemize}
