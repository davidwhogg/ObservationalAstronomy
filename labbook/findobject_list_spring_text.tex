
\noindent In the subsequent outdoor labs, you will be finding one or
more objects selected by the instructor. Interesting objects visible
from New York City are listed here and we will be choosing from this
list.

\begin{itemize}  
\item {\bf Mizar/Alcor.} This pair of stars actually  consists of
perhaps six stars in the same system, or at least moving
together. Alcor is a binary, but you will not be able to see its
companion, which is very low mass. Mizar is a quadruple system; with
your telescope you will resolve this into two ``stars,'' each of which
is actually a binary star. They are around 14 arcseconds apart. Mizar
was the first binary discovered with a telescope, by Galileo in 1617.
\item {\bf Andromeda Galaxy.} This relatively massive galaxy is observable
with the naked eye in the darkest sites, but only barely visible with
  our telescopes when in New York City. It is about $10^{11}$ solar
  masses and is around two million lightyears away. If you find it,
  the faint smudge you see is really just the center of the galaxy,
  which extends over several degrees. In the Spring, we will only
  search for this if the first few labs are extremely good weather.
\item {\bf Double Cluster (NGC 884 and 869).} This pair of open
  clusters is very far away, at around 7,500 light years. They are
  just a few hundred light years from each other. They are very young
  (12 million years) and bright.
\item {\bf 30 Arietis.} The two components of this binary star (A and
  B) are separated by 38 arcsec, or about 1500 AU at their distance of
  130 light years.  They are each a bit more massive than the Sun
  (1.1--1.3 solar masses) and are about a billion years old.
\item {\bf Messier 34.} This open cluster is around 1,500 light years
  away. It is about the same size as Messier 44 but is younger
  (200-300 million years) and will appear smaller and less obvious on
  the sky due to its greater distance. 
\item {\bf Messier 35.} This open cluster is around 4,000 light years
  away. It is a few thousand solar masses, and is physically
  quite large. Within a degree, to the south and west, is the old open
  cluster NGC 2158, which is much further away (over 10,000 light
  years) and fainter (I'm not sure whether NGC 2158 is visible from
  New York!). 
\item {\bf $\lambda$ Orionis (Meissa).} This binary star with
a 4 arcsec separation is very young (around 5 million years old), and
  consists of two massive stars (one 30 solar masses and one around 10
  solar masses). They are part of a large open cluster, and are
  ionizing a huge region of the surrounding gas. This pocket of star
  formation is at about the same distance ($\sim 1300$ light years) as
  the Great Nebula, Messier 42, which you have already seen.
\item {\bf Messier 41.} This open cluster is around 2,000 light years
  away, and is a few hundred solar masses, and around 200 million
  years old.
\item {\bf Messier 46.} This open cluster is similar to Messier 41,
  but around twice the distance.
\item {\bf Messier 44 (Beehive, or Praesepe).} This open cluster looks like a
  number of bright stars (the ``bees'') within a quadrangle of bright
  stars (outlining the ``hive''). It is around 600 light years away,
  and has a mass of a few hundred solar masses. It has an age of
  600--700 million years. The cluster is gravitationally bound, but
  will probably be ultimately destroyed by the tidal forces within the
  Milky Way. 
\item {\bf 38 Lyncis.} This binary has a separation of 2--3 arcsec, so
  will be barely resolved for us in New York City. The brighter star
  is about twice the mass of the Sun, whereas the fainter is only a
  little more massive than the Sun. If they formed together, they are
  both reasonably young at around 500 million years.
\item {\bf Algeiba.} This pair has a period of over 510 years. Its
  binarity was discovered by William Herschel in 1782, which means
  that we have not observed even half of its full orbit. Currently the
  pair is about 4 arcsec apart. The stars are both much more luminous
  than the Sun, and one is Sun-like in color (a G star) with the other
  a little redder (a K star).  They are about 130 light years away and
  are each 2--3 times the mass of the Sun and are about 500 million
  years old, in a late phase of their evolution.
\item {\bf $\tau^1$ Hydrae.} This is quite a wide binary (about 67
  arcsec separation) with one star bluish (an F star about 1.2 times
  the mass of the Sun) and the fainter one redder (a K star about 0.9
  times the mass of the Sun).
\item {\bf Porrima.} These two stars orbit around each other every 169
  years. They are currently very close --- probably a bit less than 2
  arcsec. We may or may not be able to separate them! They are nearly
  identical F stars, each about 1.5 solar masses.
\item {\bf $\theta$ Virginis.} This is a quadruple system, one
  spectroscopic binary A (not splittable by your telescope), and two
  other stars B and C. A and B are about 9 arcsec apart (and are each
  about 4th magnitude), and C is more distant at 70 arcsec (and much
  fainter at 10th magnitude). A and B are fairly massive stars, about
  3 solar masses (classified as A stars).
\item {\bf Messier 3.}
  This globular cluster is over 11 billion years old and has a mass of
  about $5\times 10^5$ solar masses. It lies around 34,000 lightyears
  away. It was originally discovered by Messier himself in 1764. It is
  rather hard to find from New York because of the lack of bright
  stars around it.
\item {\bf Messier 5.}
  This globular cluster is around 10 billion years old and has a mass
  of about $10^6$ solar masses. It lies around 25,000 lightyears
  away. It was originally discovered in 1702.
\item {\bf Messier 13.}
  This globular cluster is around 12 billion years old and has a mass
  of about $10^6$ solar masses. It lies around 22,000 lightyears
  away. It was originally discovered in 1714 by Edmund Halley. We will
  only be able to see it if we have good weather in the latest labs in
  the course. 
\item {\bf Messier 92.}
  This globular cluster is one of the oldest known; contraints on its
  age are around 13--14 billions years, which is as old as you can
  yet. Otherwise it is very similar to M13, but a bit fainter because
  it is 27,000 light years away. It was originally discovered in 1777
  by Bode. Like M13, we will only see it with luck in the very last
  week of lab.
\end{itemize}
