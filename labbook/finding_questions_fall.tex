
\begin{enumerate}
\item You go outside one night at midnight standard time in New York
  State, and see Merak and Dubhe pointing down vertically at
  Polaris. What month is it? 
\vspace{80pt}
\item Say you stick around a few hours.  How long will it be until
  Merak and Dubhe are pointing horizontally at Polaris?  Will they be
  to the left or to the right of Polaris?
\vspace{80pt}
\item On Mar 15 at about 2am standard time you notice an interesting
  star rising.  Because of bad weather you don’t get a chance to see
  the star again until Jun 15.  Within $\pm$ 20 minutes, at what
  standard time does the star rise that night?
\vspace{80pt}
\item What is the typical field-of-view of a telescope finder?
\vspace{80pt}
\item What is field of view of a 2m focal length telescope with a 40mm
  eyepiece? 10mm?
\clearpage
\item Complete these questions for Messier 2, a globular cluster. Keep your answers for
  use in Lab later in the semester.
  \begin{enumerate}
    \item At what altitude does it transit when observed from New York City?
\item What hour angle will it have at around 
8pm Eastern Time tonight?
\item Will it be rising or setting?
\end{enumerate}
\vspace{20pt}

Now look at the object on the map in the Edmund Atlas for which it is closest
to the center of the map (that is, not one where it is very near the
edge). What is the nearest naked-eye visible star ($m<3.5$ in NYC!)?

\vspace{40pt}

Can you put the bright star and your object both in the finder
field (say it has 7 deg diameter)? 

\vspace{20pt}

\begin{enumerate}
\item If so, draw (below) what it will look like
(assuming the finder does not invert the view, which our finders do
not)?
  \end{enumerate}
\end{enumerate}
