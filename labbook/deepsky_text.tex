
\noindent Today we will use Sloan Digital Sky Survey images
to investigate the deep sky.  

The images are negatives so the light of the stars etc. is dark. The
images come in pairs, labeled on the back with ``r'' and ``g.''  The
images labeled ``r'' are taken with a red filter -- so the image
records the red light. The ones labeled ``g'' are taken with a bluer
filter, and so records the blue light.

Pick one of the available pairs supplied by your instructors.  {\bf
Please treat them with great care: keep them flat; no pencils or pens
or food near them, and do not write on paper on top of the
prints. Thanks!}

\noindent
{\bf 1. Orienting yourself in the image:} 

\noindent Each image is also labeled on the back by the 
R.A.~and Dec of the center of the image. The images are 3 degrees on a
side.

\noindent First, write down the center listed on your
image:

\vspace{40pt}

\noindent Verify the location of the field on the sky. To do so,
identify the four (roughly) brightest stars in the image. Make sure
the four are fairly widely separated on the page.  Using the center
R.A. and Dec as a starting point, use Starry Night to identify those
four stars. List below their names, R.A.s, Decs and magnitudes. Also,
draw their rough configuration in the image on the blank page at the
end of the lab.

\noindent Remember: in the images, with the rectangle in the upper
left, North (higher Dec) is up and East (higher R.A.) is to the right
in this case (the print-out is a mirror image of how the sky looks).

\clearpage

\noindent {\bf 2. Coordinates in the image}

\noindent Using a ruler, estimate the ``plate scale'' of the image. Do
this by measuring the distance between any two stars, and then
calculating the distance $d$ between them in degrees based on their RA and
Dec values. Because you are on the sphere this distance is
(approximately):
\begin{equation}
\theta = \sqrt{(\delta_1 - \delta_2)^2 + \cos^2(\delta_c)(\alpha_1
- \alpha_2)^2} 
\end{equation}
where $\alpha$ is R.A., $\delta$ is Dec, and $\delta_c$ is either
$\delta_1$ or $\delta_2$ (it doesn't matter much which).  The plate
scale in mm per deg is then then just $d/\theta$.

\vspace{100pt}

\noindent Check that this makes sense given my above statement that
the image is 3 deg on a side.

\vspace{100pt}

\noindent In Starry Night, find a tenth magnitude star in the
field. Using the closest of your four reference stars, use its RA and
Dec to find it in your printed image. 

\noindent In Starry Night, zoom in around that star to a 15 arcmin
field-of-view.  Make sure you have All Sky Image on and no magnitude
limits for the stars. Compare how many the stars you see in the SDSS
image to those in Starry Night. Which shows more (and fainter) stars?

\clearpage

\noindent {\bf 3. Finding galaxies and nebulae}

\noindent These images have been chosen because they contain one (or
more) interesting objects that are either distant galaxies or clusters
in our own galaxy. Find one of these, estimate its RA and Dec, and
draw it roughly below:

\vspace{100pt}

\noindent Is it red or blue? How does the way it looks differ between the two
images?

\vspace{80pt}

\noindent Using your value for the plate scale, estimate its rough angular size:

\vspace{40pt}

\noindent Look it up in Starry Night by looking at that RA and
Dec. What is the name and classification of the object? What is its
physical size?

% plate 2241 - Coma 
