\documentclass[11pt, preprint]{aastex}
\usepackage{psboxit}

\setlength{\footnotesep}{9.6pt}

\newcounter{thefigs}
\newcommand{\fignum}{\arabic{thefigs}}

\newcounter{thetabs}
\newcommand{\tabnum}{\arabic{thetabs}}

\newcounter{address}

\begin{document}

\title{\bf Finding an object on the sky: due April 19, 2010}

~
\vspace{-30pt}

\noindent The point of this homework is to try to locate an object on
the sky that would be observable at a given time and place, and make a
plan for how to find it when you go outside with a telescope or
binoculars.

\noindent This homework is due back in LAB the week of April 19. 
Each lab group will try all of the
objects picked by members of the group.

\noindent For the indoors lab, you should try to find the object on
the sky WITHOUT using the ``heads-up display'' in Starry Night (the
names of the stars, etc.  To do this, go to ``Preferences'' under the
``File'' menu.  Choose the ``Cursor Tracking (HUD)'' set of options in
the upper left tab.  In the ``Cursor Tracking'' options, make sure
both the ``When'' boxes (``These keys are down'' and ``Mouse is idle
for'') are NOT CHECKED. This should make it so the information on each
object your cursor touches does NOT come up on the screen.  This will
more closely emulate the problem you will face in the actual sky,
which emphatically does not come with a heads-up display!

\noindent The set of tasks for this homework is to prepare for an
observation of a particular object. 

\begin{enumerate}
\item Pick an object from below, and look up its RA and Dec.  
\begin{enumerate}
\item Rho Cancri
\item Tau Leonis
\item NGC 1647
\item Messier 35
\item Messier 41
\item Messier 36
\item Messier 37
\item Messier 44
\item Tau-one Hydra
\item Eta Perseus
\item 12 Lynx
\item NGC 2264
\end{enumerate}
\item Make sure the object you pick satisfies the following criteria.
\begin{enumerate}
\item Be visible in the telescope but not naked-eye observable --- say
an object whose brightest component is in the range $5<m<10$. Pick
something interesting --- not just an individual faint star!
\item Transit at an altitude greater than 25 deg when observed from New York City.
\item Transit within 3 hours of 8pm Eastern Daylight Time on April 20. 
\end{enumerate}
\vspace{80pt}
\item Look up the object on wikipedia or another on-line resource
(virtually anything listed in Edmund is very well known). Give a brief
(3-4 sentence) description of the object below.
\vspace{80pt}
\item What altitude does the object transit at?
\vspace{40pt}
\item Will it be rising or setting at 8pm EDT on April 20?  What is its
hour angle?
\vspace{40pt}
\item Look at the object on the map in Edmund for which it is closest
to the center of the map (that is, not one where it is very near the
edge). What is the nearest naked-eye visible star ($m<3.5$ in NYC!)?
\vspace{40pt}
\item Can you put the bright star and your object both in the finder
field (say it has 7 deg diameter)? 
\begin{enumerate}
\item If so,  what it will look like
(assuming the finder does not invert the view, which our finders do
not)? (When you work through this in the indoor lab with Starry Night,
draw what it actually looks like in Starry Night).
\vspace{180pt}
\clearpage
\item If not, plot a plan for how to star-hop from the nearest bright
star to this one, as described in class, by drawing below the
positions of the bright stars and how you would try to move the field
of view of the telescope to close to the target position. (When you
work through this in the indoor lab with Starry Night, draw what the
last step actually looks like in Starry Night).
\end{enumerate}
\vspace{220pt}
\item Only to be done outside: 
Is this object ACTUALLY visible at that time? If
not, at what date is it transiting at that time?
When you actually observe this object,
draw the view in the telescope below or on a separate sheet.
\end{enumerate}

\end{document}
