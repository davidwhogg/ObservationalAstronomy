
\noindent
{Objectives:} To introduce some practical aspects of using an astronomical telescope.

\bigskip\noindent
{\bf INSIDE PREPARATIONS}


\medskip
\bigskip
\noindent
{\bf 1. Telescope basics}

\medskip
\noindent
\emph{Optics}. The telescopes we will use today are ``equatorial''
telescopes, and have an aperture of 8 inches diameter, which allows
them to collect much more light than the human eye (about 500 times
more) so that we can see much fainter objects.  The optics are quite
complicated. At the front is a transparent plate, and in the center a
small secondary mirror pointing inwards. The main optical component is
a converging mirror at the other end of the telescope. The light from
a distant star passes through the glass plate, bounces off the primary
mirror, comes back up the tube, is reflected back down the tube by the
secondary and passes through a hole in the primary mirror to the
eyepiece.  There is a knob on the back of the telescope that focuses
the optics. Once set up, this does not need to be adjusted unless you
change the eyepiece.

\medskip \noindent
\emph{Magnification}. When you look through the eyepiece the
view is magnified. We will not be concerned with this aspect today,
but for completeness note that the magnification is given by
the formula $m = f_o/f_e$ where $f_e$ is the focal length of the
eyepiece and $f_o$ is the focal length of the objective or primary
optical component. $f_e$ is usually written on the eyepieces in
mm. $f_o$ for our telescopes is about 2000 mm. So for a 25 mm
eyepiece the magnification is 2000/25 = 80.

\medskip \noindent 
\emph{Finder}. Attached to the side of the telescope is a
small finder telescope -- it acts a
bit like the sights on a rifle. It has a magnification of 6, and
sees about 5\deg\ of the sky -- much more than through the main
eyepiece. When you look though the finder there is a cross hair, that should
be aligned so that a star on the cross hair appears in the
eyepiece. Thus one way to observe a star is to line up the telescope
roughly in the direction of the star; then line up the star in the
finder. It should then appear in the main eyepiece.

\medskip \noindent 
\emph{Control}. The telescopes we will use have two moving axes that
correspond to the sky coordinates RA and Dec. Each axis has a scale
which enables you to dial in and point at a star of known
coordinates. The Dec circle is marked in degrees; the RA circle in
hours, with smaller 5 min ticks. To move over large angles: release
the axis brake; move the telescope; then reset the break gently -- not
tightly. For precision setting, there are control knobs for each axis
that move the telescope over small angles. 

\noindent
A correctly set up telescope can move in RA and Dec because it is
mounted at an angle so that one axis is aligned with the polar
axis. When you turn the telescope about this axis you are turning it
in RA; when you turn it about the other axis you are turning to
different Decs at the same RA. When you have set on a star, a motor
inside the telescope turns the polar axis to keep track of the star
(i.e., it keeps it pointing to the same RA and so compensates for
Earth's rotation). 

\noindent
We also have on the roof several 10 inch diameter telescopes, of the
``alt-az'' variety (rather than equatorial).  These telescopes do not
track without using their electronic devices (which we won't
do). However, they do have better optical quality. We will use them in
later labs.

\medskip\noindent
Before going out, review the recipe below as you will have to do it
outside. Also please remember:

\bigskip
\centerline{\bf Don't touch any part of the telescope optics} 


\medskip
\centerline{\bf Don't force any mechanical part of telescope} 

\medskip
\centerline{\bf Only put the axis brakes on gently}

\medskip
\medskip
\bigskip
\noindent
{\bf 2. Recipe for setting up telescope}

\medskip\noindent (a). First choose an eyepiece.  Recall that you will
want to be able to easily find your bearing, so you do not want a very
high magnification image --- thus you want a long focal length
eyepiece. The instructors will give you a tour of the lab room when
you arrive on the roof, and tell you where everything is.

\medskip\noindent
(b). Locate Polaris with the naked eye. (We will assume that it is
exactly at the NCP).

\medskip\noindent
(c). Rotate the telescope about the Dec axis so that it reads 90\deg.

\medskip\noindent (d). Turn the whole tripod of the telescope until
the fork points at Polaris. 

\medskip\noindent (e). Level the telescope by adjusting the legs and
using the spirit (or ``bubble'') level on the top of the tripod. 

\medskip\noindent (f).  Find Polaris in the finder, If it is off to
one side or not visible, readjust the tripod so that Polaris falls
close to the cross hair in the finder. Make sure the dial still reads
Dec = 90\deg, and readjust the legs if the telescope is no longer
level according to the bubble. You may have to repeat this process a
few times; the roof is not exactly horizontal!

\medskip
\noindent
(If Polaris remains always much too high or too low in the finder so
you cannot center it, consult the instructor -- the wedge needs
readjustment).

\medskip\noindent
(g). Once Polaris is centered make sure the telescope is switched
on. Look for Polaris through the main eyepiece. If it appears fuzzy,
focus it with the focus knob. If Polaris does not appear in the
eyepiece when centered in the finder, consult the instructor -- the
finder needs re-aligning. When this is done return to (b).

\medskip\noindent
(h). Once you have reached here, the telescope is approximately
aligned with the polar axis. Do not move the tripod from now on. Locate by
eye the reference star of known coordinates well away from the NCP
(see Section 3). Turn the telescope about both axes (releasing brakes
gently) to center the star in the finder, then in the eyepiece.
With the star centered, slide the RA circle around until the RA reads
the RA of the star. The Dec should already be OK to within a degree or
so.

\medskip\noindent
It is done: the setting circles roughly correspond to RA and
Dec on the celestial sphere. A star of known coordinates can be found
by moving the telescope until the dials read correctly (this should
be accurate enough to place the star in the finder). Conversely, the
coordinates of a star can be found by centering it, then reading the
dials. Remember -- be gentle with the brakes.


\noindent
{\bf OUTSIDE}

\bigskip
\noindent
{\bf 3. Setup}

\medskip\noindent
Set up the telescope according to the recipe given above, using as
reference star for RA that given below. The stars listed here are
appropriate for January or February lab times.

\medskip\begin{center}
NAME: \makebox[2cm]{Betelgeuse} RA: \makebox[2.5cm]{05h~55m}  Dec:
\makebox[2.5cm]{+07$^\circ$ 24$'$}
\end{center}

\bigskip\noindent
{\bf 4. Measuring Coordinates}

\medskip\noindent
Measure the coordinates of each of the stars in the table below. In
turn: locate the star by eye (e.g., using the sky maps in the field
guide after p 53); move the telescope so that the star is centered
first in the finder, then in the eyepiece; read off the star's
coordinates from the setting circles to within 5 min and 1\deg\ in RA
and Dec, respectively.

\begin{center}
\begin{tabular}{lcc} \hline \\ [-6pt]
\hspace{1cm}Name\hspace{1cm} &  \hspace{1cm} Your measured:  RA \hspace{1cm} & Dec \\ [6pt]
\hline
a) Betelguese  & &     \\ \hline
b) Castor  & &  \\ \hline
c) Algol  & &    \\ \hline
d) Sirius  & &   \\ \hline
  \end{tabular}
\end{center}

\bigskip\noindent
{\bf 5. Dialing objects}

\bigskip\noindent The table below lists the coordinates of four
objects. Dial them in on the setting circles and report what you see
at these locations. If you don't see anything obvious on the first
one, you probably have set up your telescope wrong!

\begin{center}
\begin{tabular}{lcc} \hline \\ [-6pt]
 \hspace{1.5cm}  RA \hspace{1cm} & Dec \hspace{1cm}& \hspace{3cm}Report\hspace{3cm} \\ [6pt]
\hline
b) 7h~39.29m  & $+5^\circ$ 13.3$'$ &     \\ \hline % Procyon
b) 3h~48.49m  & $+24^\circ$ 6.3$'$ &     \\ \hline % Pleiades
c) 5h~52.28m  & $+32^\circ$ 32.7$'$ &  \\ \hline % M37
d) 5h~35.14m   & $-5^\circ$ 25.9$'$ &    \\ \hline % Great Nebula in Orion
  \end{tabular}
\end{center}
