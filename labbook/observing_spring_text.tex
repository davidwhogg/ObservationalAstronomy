
\noindent
{\bf 1. Finding Objects.}  

\noindent The best technique for finding an object depends on how
bright it is. In this lab we will begin with objects which are naked
eye, or small offsets from bright stars. We will use the equatorial
telescopes. In subsequent labs we will try more difficult cases.

\medskip\noindent In particular, for this lab, we will only rely on
the Edmund atlas.  As you will see later, this is insufficient for
finding anything not very close to a reasonably bright star.

\medskip\noindent As you proceed below, for each object begin with a
long eyepiece, so a large field of view, when you are first trying to
find it. Once you have centered the object of interest using that
eyepiece, you can insert a smaller one for a more magnified view if
desired.

\medskip\noindent Note that the image is \emph{inverted} -- upside
down -- in the finder, but \emph{upright} in the main eyepiece. It is
also important to have a good idea of the size of the patch of sky
you are looking at. In the finder, this is about 7\deg, and in the
main eyepiece it is much smaller.

\bigskip
\bigskip
\bigskip
\noindent
{\bf 2. Magnification and Field.}

\medskip\noindent {\bf Field estimate.} TF = AF/M, where M
is the magnification given by $f_o/f_e$. For your eyepiece, estimate
the magnification and the true field assuming AF =40\deg.

\medskip\noindent $f_o$ (mm): \makebox[1.5cm]{\hrulefill} \ $f_e$
(mm): \makebox[1.5cm]{\hrulefill} \ M: \makebox[1.5cm]{\hrulefill} \ \
Field of View (\arcmin):\makebox[2cm]{\hrulefill}


\medskip\noindent {\bf Field Measurement.} Time the crossing of a star
near Dec = 0\deg\ in the eyepiece field (with tracking turned off).

\medskip\noindent Time in minutes: \makebox[2cm]{\hrulefill} \ \ \ Field
of View (\arcmin): \makebox[2cm]{\hrulefill}

\bigskip
\bigskip
\bigskip\noindent
{\bf 3. Observing.} 

\medskip\noindent You should try to find four or more objects, in
increasing order of difficulty.  These are suitable for lab times in
late January or early February; most of them are Map 3 in Edmund.

\begin{enumerate}
\item Naked eye target: Any currently visible planets, and/or Almach
($\gamma$ Andromedae). Almach is a system of four stars all
gravitationally bound about 350 light years away. Through your
telescope it will look like only a double. The fainter, bluer of the
two is in fact a triplet of stars; one lone star in a 64-year orbit
around a stellar binary with a period of 2--3 days.
\item Within finder scope of naked-eye:
$\theta$ Tauri, near Aldebaran. This is a visual binary. Both stars
are about 150 light years away in the Hyades open cluster, but they
are not gravitationally bound to each other. Each member of this
binary is itself a binary star, one with an orbit of 16 years (the
slightly fainter, redder one), and the other with an orbit of 140 days
(the slighter brighter, bluer one).
\item Using offset from naked-eye star:
M34: open cluster; to get it, offset West from Algol, or East from
Almach. This open cluster is about 1500 light years away. The stars in
it formed about 250 million years ago.
\item Finally, reward yourself by finding M42, the Great Nebula in
  Orion (early in the semester). This nebula is gas ionized by the
  light of very massive young stars that have formed in the past few
  million years. This system will look similar to M34 in a couple of
  hundred million years.
\end{enumerate}

\medskip\noindent 
On the next page, record the characteristics of the
objects you find, according to their type.

\medskip\noindent 
Binary: sketch the binary with its actual orientation in the field of
view in the circle provided.  Label the stars with their colors, and
indicate which is the brighter. Estimate the angular separation of the binary.

\medskip\noindent 
Open cluster: sketch the brightest 5--10 stars with the actual
orientation in the field of view in the circle provided. Estimate the
angular size of the open cluster.

\medskip\noindent 
Planet:  sketch in the disk, indicate any features, shadows, phase,
moons etc. Estimate the angular size of the planet.

\medskip\noindent
Nebula: sketch the nebula shape and the main star field in its
vicinity. Estimate the angular size of the nebula.


\newpage

\parbox[b]{8cm}{ Object: \makebox[3cm]{\hrulefill}\\
RA/Dec: \makebox[3cm]{\hrulefill} \\
Type: \makebox[3cm]{\hrulefill} \\
Size/separation:  \makebox[1.5cm]{\hrulefill} \\ }   \begin{minipage}[b]{8cm}{\psfig{figure={o3s_f1.eps},width=4.0cm}}\end{minipage}

\bigskip\noindent 

\parbox[b]{8cm}{ Object: \makebox[3cm]{\hrulefill}\\
RA/Dec: \makebox[3cm]{\hrulefill} \\
Type: \makebox[3cm]{\hrulefill} \\
Size/separation:  \makebox[1.5cm]{\hrulefill} \\ }   \begin{minipage}[b]{8cm}{\psfig{figure={o3s_f1.eps},width=4.0cm}}\end{minipage}

\bigskip\noindent 

\parbox[b]{8cm}{ Object: \makebox[3cm]{\hrulefill}\\
RA/Dec: \makebox[3cm]{\hrulefill} \\
Type: \makebox[3cm]{\hrulefill} \\
Size/separation:  \makebox[1.5cm]{\hrulefill} \\ }   \begin{minipage}[b]{8cm}{\psfig{figure={o3s_f1.eps},width=4.0cm}}\end{minipage}

\bigskip\noindent 

\parbox[b]{8cm}{ Object: \makebox[3cm]{\hrulefill}\\
RA/Dec: \makebox[3cm]{\hrulefill} \\
Type: \makebox[3cm]{\hrulefill} \\
Size/separation:  \makebox[1.5cm]{\hrulefill} \\ }   \begin{minipage}[b]{8cm}{\psfig{figure={o3s_f1.eps},width=4.0cm}}\end{minipage}

\bigskip\noindent 

\parbox[b]{8cm}{ Object: \makebox[3cm]{\hrulefill}\\
RA/Dec: \makebox[3cm]{\hrulefill} \\
Type: \makebox[3cm]{\hrulefill} \\
Size/separation:  \makebox[1.5cm]{\hrulefill} \\ }   \begin{minipage}[b]{8cm}{\psfig{figure={o3s_f1.eps},width=4.0cm}}\end{minipage}

\bigskip\noindent 


