\documentclass[11pt, preprint]{aastex}
\usepackage{rotating}

\begin{document}

\noindent{\bf Observational Astronomy / PHYS-UA 13 / Spring 2019 /
  Syllabus }

\noindent This course will teach you how to observe the sky carefully  
with your naked eye, binoculars, and a small telescope. You will learn
the basics of observable lunar and planetary properties, and the
basics of astronomical coordinates and observations. The goal is for
you to be able to understand and describe what you see in the sky at
night, and to be able to use charts and coordinates to predict it.

\noindent The instructors are: Prof. Michael Blanton (726 Broadway,
Rm 941, {\tt
  mb144@nyu.edu}), whose office hours are Weds 11am--12:30pm (or by
appointment), and TA Dou Liu ({\tt dl3182@nyu.edu}).
 
\noindent The primary textbooks are: 
\begin{itemize}
\item {\it The Ever-Changing Sky}, by James Kaler
\item {\it Sky \& Telescope's Pocket Sky Atlas}, by Roger W.~Sinnott
\item {\it Edmund Mag 5 Star Atlas} (will be supplied to you)
\item This laboratory manual. 
\end{itemize}

\noindent Each week you will attend one lecture (at 3:30pm Monday in
Waverly Building Room 370) and one lab. 

\noindent Grades are based on labs (20\% on written material, 5\% on
extra good work), homeworks (10\%), the midterm (30\%) and the final
(35\%). 

\noindent For the homework, there is a sheet of ``Lecture Questions''
to answer, based on the previous week's lecture. These are due at the
beginning of each lecture, starting the second lecture. {\bf Late
homeworks will not be accepted.}  Two of these questions will be
chosen to be graded each week (obviously, we won't tell you which two
beforehand).

\noindent For the midterm and the final you are responsible for
material in the labs, the reading, and the homework. In preparing for
the exams, use the homeworks as a guide to which material I believe is
essential.

\noindent {\bf Arrive for the lab (on time!) at 7:00pm in Meyer 224}, where
we will discuss the contents of the lab and then (when appropriate) go
to the observatory. {\bf Starting the week of March 11, we will move
the lab time later to 7:45pm} to accommodate the later sunset and
daylight savings time.

\noindent We will begin with Lab \#1 the first week, which is
indoors. This lab will be {\bf the very first week of class}. After
that the labs will proceed sequentially where possible, but whether we
are going on to the next indoor lab or next outdoor lab will depend on
the weather. Some outdoor labs will come out of order depending on
what is visible a given night. Finally, the last lab in this book
should have entries filled in {\bf every week}, and should be handed
in on the final lab date.

\noindent You cannot switch between the lab sections mid-semester,
because in general they will be on different schedules.  The timing of
the indoor and outdoor labs for each section will be driven mostly by
the weather. Welcome to observational astronomy!

\noindent For the labs: {\bf you MUST arrive on time}, or else you will not
be able to access the observatory.  In addition, please dress
appropriately for remaining outside for an extended period, including
hats and gloves when appropriate.  {\bf Dress warm!}

\noindent {\bf Attendance in lab is not optional!}  You can miss one lab
during the semester without penalty: you must however contact the lab
instructor explicitly {\bf beforehand} to claim this credit. If you
are absent for any other without good cause you will lose credit for
that lab.  If you miss more than three sessions without good cause,
you will not be given a passing grade no matter how you perform in the
class otherwise.

\noindent {\bf Do not use your phones during lab unless directed
to!}

\clearpage

\baselineskip 0pt
\begin{sidewaystable}
\small
\begin{tabular}{|c||c|c|}
\hline
{\it Jan.~28} 
& The Celestial Sphere: angles \& coordinates 
& Kaler Ch.~1, 3.15; Edmund pp.~1--9
\cr 
{\it Feb.~4} 
& Introduction to telescopes
& Kaler Ch. 13.8--13.13
\cr
{\it Feb.~11} 
& Rotation and Orbit of the Earth
& Kaler Ch. 2.1--2.6, 2.12, Ch. 3
\cr
{\it Feb.~18} 
& PRESIDENT'S DAY: Note there is still Wednesday lab!
& ---
\cr
{\it Feb.~25} 
& Finding your way in the sky
& Edmund pp.~30--32
\cr
{\it Mar.~4} 
& {\it NYU closed}
& 
\cr
{\it Mar.~11} 
& {\bf Midterm exam in class!}
& ---
\cr
{\it Mar.~18} 
& SPRING BREAK
& ---
\cr
{\it Mar.~25} 
& Stars
& Kaler Ch.~4.1--4.10, 4.15
\cr
{\it Apr.~1} 
& Variables and Binaries
& Kaler Ch.~4.11--4.13
\cr
{\it Apr.~8} 
& Galaxies
& Kaler Ch.~4.14, 4.16--4.17
\cr
{\it Apr.~15} 
& The Moon
& Edmund p.~34; Kaler Ch.~9.1--9.5, 9.9
\cr
{\it Apr.~22} 
& Planets and their motions
& Kaler Ch.~11.1--11.13
\cr
{\it Apr.~29} 
& Moons of Jupiter and Saturn
& Kaler Ch.~12.1
\cr
{\it May.~6} 
& Precession \& nutation
& Kaler Ch.~5.1--5.10
\cr
{\it May.~13 }
& Tides \& Eclipses
& Kaler Ch.~10
\cr
{\it May.~15 }
& {\bf FINAL EXAM, 6:00pm--7:50pm, Waverly Building Room 369}
& 
\cr
\hline
\end{tabular}
\end{sidewaystable}

\clearpage

\baselineskip 0pt
\begin{sidewaystable}
\small
\begin{tabular}{|c||c|c|c|}
\hline
{\it Jan.~28--30} 
& Orion, Andromeda, Mars, Uranus
& Full Moon
& Sunset 5:08pm EST
\cr
{\it Feb.~4--6} 
& Orion, Andromeda, Mars, Uranus
& ---
& Sunset 5:18pm EST
\cr
{\it Feb.~11--13}
& Orion, Andromeda, Mars, Uranus
& First Quarter Moon
& Sunset 5:28pm EST
\cr
{\it Feb.~20}
& Orion, Andromeda, Mars, Uranus
& Full Moon
& Sunset 5:36pm EST
\cr
{\it Feb.~27--Mar.~1}
& Orion, Mars, Uranus
& ---
& Sunset 5:45pm EST
\cr
{\it Mar.~4--6}
& Orion, Mars, Uranus
& ---
& Sunset 5:51pm EST
\cr
{\it Mar.~11--13} 
{\bf (New time: 7:45pm EDT)}
& Orion, Mars \& Uranus nearby
& Waxing Crescent Moon
& Sunset 6:58pm EDT
\cr
\hline
{\it Mar.~18}
& Spring Break
& ---
& ---
\cr
\hline
{\it Mar.~25--27}
& Orion, Mars
& ---
& Sunset 7:14pm EDT
\cr
{\it Apr.~1--3}
& M3, Mars
& ---
& Sunset 7:21pm EDT
\cr
{\it Apr.~8--10}
& M3, Mars
& Waxing Crescent Moon
& Sunset 7:28pm EDT
\cr
{\it Apr.~15--17}
& M3, M5 rising, Mars
& Full Moon
& Sunset 7:36pm EDT
\cr
{\it Apr.~22--24}
& M3, M5 rising, Mars
& ---
& Sunset 7:43pm EDT
\cr
{\it Apr.~29--May 1}
& M3, M5, Mars
& ---
& Sunset 7:50pm EDT
\cr
{\it May 6--8}
& M3, M5, M13 rising, M92 rising, Mars
& Waxing Crescent Moon
& Sunset 8:04pm EDT
\cr
{\it May 13}
& M3, M5, M13 rising, M92 rising, Mars
& Waxing Gibbous Moon
& Sunset 8:04pm EDT
\cr
\hline
\end{tabular}
\end{sidewaystable}

\clearpage

Interesting events of Winter \& Spring 2019:
\begin{itemize}
\item January 20: Lunar Eclipse (starts 10:30pm, maximum around
12:30am)
\item April 22--23: Lyrids Meteor Shower (associated with comet C/1861
G1 Thatcher) 
\item May 6--7: Eta Aquarids Meteor Shower (associated with Comet Halley)
\end{itemize}

\baselineskip 12pt

\end{document}
