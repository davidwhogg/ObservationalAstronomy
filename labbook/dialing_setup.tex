\medskip
\bigskip
\noindent
{\bf 1. Telescope basics}

\medskip
\noindent
\emph{Optics}. The telescopes we will use today are ``equatorial''
telescopes, and have an aperture of 10 inches diameter, which allows
them to collect much more light than the human eye (about 1000 times
more) so that we can see much fainter objects.  The optics are quite
complicated. At the front is a transparent plate, and in the center a
small secondary mirror pointing inwards. The main optical component is
a converging mirror at the other end of the telescope. The light from
a distant star passes through the glass plate, bounces off the primary
mirror, comes back up the tube, is reflected back down the tube by the
secondary and passes through a hole in the primary mirror to the
eyepiece.

\medskip \noindent
\emph{Focus}.
There is a knob on the back of the telescope that focuses the optics
(this is the longer knob; the shorter one locks and unlocks the
primary mirror and should not be touched). Once set up, the focus knob
does not need to be adjusted unless you change the eyepiece.

\medskip \noindent
\emph{Magnification}. When you look through the eyepiece the
view is magnified. We will not be concerned with this aspect today,
but for completeness note that the magnification is given by
the formula $m = f_o/f_e$ where $f_e$ is the focal length of the
eyepiece and $f_o$ is the focal length of the objective or primary
optical component. $f_e$ is usually written on the eyepieces in
mm. $f_o$ for our telescopes is about 2500 mm. So for a 25 mm
eyepiece the magnification is 2500/25 = 100.

\medskip \noindent 
\emph{Finder}. Attached to the side of the telescope is a
small finder telescope---it acts a bit like the sights on a rifle. It
has a magnification of 6, and sees about 5\deg\ of the sky---much more
than through the main eyepiece. When you look though the finder there
is a cross hair, that should be aligned so that a star on the cross
hair appears in the eyepiece. Thus one way to observe a star is to
line up the telescope roughly in the direction of the star; then line
up the star in the finder. It should then appear in the main
eyepiece. If it does not, the finder requires adjustment with the
small screws holding it in place, which you should ask the instructor
to perform.

\medskip \noindent 
\emph{Control}. The telescopes we will use have two moving axes that
correspond to the sky coordinates RA and Dec. Each axis has a scale
which enables you to dial in and point at a star of known
coordinates. The Dec circle appears just on one side of the telescope
and is marked in degrees; the opposite circle is unmarked. The RA
circle is marked in hours, with smaller 5 min ticks.

\medskip\noindent Each axis has a brake. The Dec brake is on the
unlabeled circle; the dial in the center of the circle screws and
unscrews. The RA brake is the silver lever next to the RA adjustment
knob.

\medskip\noindent To move over large angles: release the axis brake;
move the telescope; then reset the break gently---not tightly. For
precision setting, there are control knobs for each axis that move the
telescope over small angles.

\noindent
A correctly set up telescope can move in RA and Dec because it is
mounted at an angle so that one axis is aligned with the polar
axis. When you turn the telescope about this axis you are turning it
in RA; when you turn it about the other axis you are turning to
different Decs at the same RA. When you have set on a star, a motor
inside the telescope turns the polar axis to keep track of the star
(i.e., it keeps it pointing to the same RA and so compensates for
Earth's rotation).

\noindent
We also have on the roof several 10 inch diameter telescopes of the
``alt-az'' variety (rather than equatorial).  These telescopes do not
track without using their electronic devices (which we won't
do). However, they are fun to use and we may use them in later labs.

\medskip\noindent
Before going out, review the recipe below as you will have to do it
outside. Also please remember:

\bigskip
\centerline{\bf Don't touch any part of the telescope optics} 

\medskip
\centerline{\bf Don't force any mechanical part of telescope} 

\medskip
\centerline{\bf Only put the axis brakes on gently}

\medskip
\medskip
\bigskip
\noindent
{\bf 2. Recipe for setting up telescope}

\medskip\noindent (a). First choose an eyepiece.  Recall that you will
want to be able to easily find your bearing, so you do not want a very
high magnification image---thus you want a long focal length
eyepiece. The instructors will give you a tour of the lab room when
you arrive on the roof, and tell you where everything is.

\medskip\noindent (b). Turn on the telescope with the small switch on
the left, making sure the electronic paddle is attached. If it does
not turn on, it is out of battery and/or needs to be plugged in; ask
the instructor for help. Once on, wait for the telescope to finish its
initialization sequence. It will say something like ``0 to align, MODE
for settings.'' DO NOT PRESS ``0.''

\medskip\noindent (c). Using the paddle, press MODE. Use the up and
down arrow keys near the {\it bottom} of the paddle (not the ones near
the top) and the ENTER key to navigate to SETUP. If you press ENTER
and use the arrows, you should be able to find the following settings:
\begin{itemize}
\item DATE (set it to the date)
\item TIME (set it to the current STANDARD time)
\item DAYLIGHT SAVING (set it to NO)
\item TELESCOPE, under which are:
\begin{itemize}
  \item MOUNT (set it to POLAR)
  \item TRACKING RATE (set it to SIDEREAL)
  \item GPS ALIGNMENT (set it to OFF)
\end{itemize}
\item SITE (set it to NEW YORK)
\end{itemize}
Very likely the telescope will already have these settings, but you
should double check!

\medskip\noindent
(d). Locate Polaris with the naked eye. (If it is cloudy, we are just
going to pretend we know where it is).

\medskip\noindent
(e). Rotate the telescope about the Dec axis so that it reads
90\deg.

\medskip\noindent (f). Turn the whole tripod of the telescope until
the leg parallel to the RA axis points at Polaris. (If it is cloudy,
we are just going to pretend it is pointing correctly already).

\medskip\noindent
(g). Make sure the wedge angle of the mount below the telescope is at
41 deg. If it is not, ask the instructor for help fixing it.

\medskip\noindent (h). Level the telescope by adjusting the legs and
using the spirit (or ``bubble'') level on the tripod. 

\medskip\noindent (i).  Find Polaris in the finder, If it is off to
one side or not visible, readjust the tripod so that Polaris falls
close to the cross hair in the finder. At this stage you do not have
to be too precise; just make sure that Polaris roughly in middle of
the finder. (If it is cloudy, skip this step).

\medskip\noindent (j). Using the paddle, navigate to ALIGN, and follow
these steps:
\begin{itemize}
\item It should show EASY; using the arrows, change it to ONE STAR and
press ENTER. It should now display POLAR ALIGN; some instructions
appear (which you don't have to wait for).
\item Move the HA indicator on the RA circle to its zero position at
the bottom of the telescope. 
\item Press ENTER. The telescope will move, as it tries to center on
Polaris (which is slighly off the pole).
\item Now readjust the tripod with precision, by turning the tripod,
and releveling it, to center Polaris in the finder and the eyepiece to
the best of your ability. If Polaris remains always much too high or
too low in the finder, consult the instructor. DO NOT MOVE THE DIALS
OF THE TELESCOPE AXES! (If it is cloudy, skip this step).
\item Press ENTER. The telescope will move towards a bright star that it
chooses based on the time of night and the date to be in the sky
(e.g. Vega or Betelgeuse).
\item When it is done slewing, adjust the RA and Dec dials to center the
star. Press ENTER. (If it is cloudy, skip this step).
\end{itemize}

At this point, the telescope will be calibrated and will start to
track at the sidereal rate. DO NOT MOVE THE TRIPOD AT ALL AFTER THIS
POINT: you now will point at different stars just by adjusting the RA
and Dec axes of the telescope.

\medskip\noindent
(k) With the star centered, slide the RA circle with the hours labeled
around until the RA reads the RA of the star. There are two sets of
numbers increasing in opposite directions---one for the Northern
Hemisphere and one for the Southern Hemisphere---think about their
direction and use the right set! The Dec should already be OK to
within a degree or so.

\medskip\noindent
It is done: the setting circles roughly correspond to RA and
Dec on the celestial sphere. A star of known coordinates can be found
by moving the telescope until the dials read correctly (this should
be accurate enough to place the star in the finder). Conversely, the
coordinates of a star can be found by centering it, then reading the
dials. Remember---be gentle with the brakes.
