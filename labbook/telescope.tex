\noindent{\bf Observational Astronomy / Basic observatory instructions}

\noindent For every lab, we will meet in Meyer 224, and all
go together to the roof if we are having an outdoor lab (which won't
be decided until the last minute). {\bf Be on time, otherwise we may
have left already and you will miss lab!}

\medskip \noindent On the roof, it will be cold. {\bf DRESS MUCH MORE
WARMLY THAN YOU WOULD ORDINARILY BASED ON THE WEATHER}. You will be
able to take extra clothing off if it is unnecessary, but you won't be
able to put on extra clothing you didn't bring, so err on the side of
warmth!  This isn't like skiing or hiking or even walking in the cold
--- you will be standing stock still for a couple of hours. So dress
WARMLY.

\medskip \noindent Do not leave the roof without handing in your lab
and telling an instructor.

\medskip \noindent For the labs using telescopes (all outdoor ones
except the first), we will begin lab by moving the telescopes outside.
Make sure they are evenly spaced across the roof, and that each one
has a view of Polaris to the North. If you are using an equatorial
telescope, check that the batteries to its motor are working. Make
sure to get eyepieces from the lab room. Some useful tools are also
available in the lab room: binoculars, flashlights, and some sky
atlases, for example. 

\medskip
\noindent
Below we give some basic information about the telescopes. 

\medskip
\noindent
\emph{Optics}. The telescopes for this lab have apertures of 8 to 10
inches diameter, which allows them to collect much more light than the
human eye (about 500 times more) so that we can see much fainter
objects.  The optics are quite complicated. At the front is a
transparent plate, and in the center a small secondary mirror pointing
inwards. The main optical component is a converging mirror at the
other end of the telescope. The light from a distant star passes
through the glass plate, bounces off the primary mirror, comes back up
the tube, is reflected back down the tube by the secondary and passes
through a hole in the primary mirror to the eyepiece.  There is a knob
on the back of the telescope that focuses the optics. Once set up,
this does not need to be adjusted unless you change the eyepiece.

\medskip \noindent \emph{Magnification}. When you look through the
eyepiece the view is magnified.  The magnification is given by the
formula $m = f_o/f_e$ where $f_e$ is the focal length of the eyepiece
and $f_o$ is the focal length of the objective or primary optical
component. $f_e$ is usually written on the eyepieces in mm. $f_o$ for
our telescopes is about 2500 mm. So for a 25 mm eyepiece the
magnification is 2500/25 = 100. We have a variety of eyepieces
available in the lab room. Please take these from the lab room at the
beginning of each lab and return them at the end.

\medskip \noindent \emph{Finder}. Attached to the side of the
telescope is a small finder telescope -- it acts a bit like the sights
on a rifle. It has a magnification of 6, and sees about 5\deg\ of the
sky -- much more than through the main eyepiece. When you look though
the finder there is a cross hair, that should be aligned so that a
star on the cross hair appears in the eyepiece. Thus one way to
observe a star is to line up the telescope roughly in the direction of
the star; then line up the star in the finder. It should then appear
in the main eyepiece.

\medskip \noindent \emph{Control}. The telescopes come in two
varieties.  The ``equatorial'' telescopes have two moving axes that
correspond to the sky coordinates RA and Dec. Each axis has a scale
which enables you to dial in and point at a star of known
coordinates. The Dec circle is marked in degrees; the RA circle in
hours, with smaller 5 min ticks. To move over large angles: release
the axis brake; move the telescope; then reset the break gently -- not
tightly. For precision setting, there are control knobs for each axis
that move the telescope over small angles. The ``alt-az'' telescopes
have axes that correspond to altitude and azimuth; these telescopes
are easier to control, but do not naturally track the coordinates of
the sky the way equatorials do.

The first outdoor telescope lab (``Dialing in the Stars'') gives more
detailed information about how to set up and use the telescopes.
