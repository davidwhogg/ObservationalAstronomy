
\noindent
{Objectives:} To explore some aspects of telescopic observations of
the planets, including a tour, and the relations between telescopic
observations and where the planet is in the sky.

\medskip
\bigskip
\noindent
{\bf 1. A tour of the planets}

\medskip
\noindent
We are first going to use SN to take a look at the planets as they
would appear in a large telescope in perfect observing conditions: in
fact, much better views than any telescope on Earth. We shall set the
field of view to 3\arcmin\ (= 180\arcsec) which corresponds to a
typical eyepiece at ultra high magnification ($\sim 800$).  For
convenience, get rid of daylight and the horizon so we can visit all
the planets in one session. Make sure View/Solar System/Planets is
checked or you will see no planets.

\begin{center}
\begin{tabular}{ccc} 
\hspace{0.5cm} \framebox[4.0cm]{\rule[-2cm]{0cm}{4cm}\,}
\hspace{0.5cm} & 
\hspace{0.5cm}
 \framebox[4.0cm]{\rule[-2cm]{0cm}{4cm}\,}  \hspace{0.5cm} &
 \hspace{0.5cm} \framebox[4.0cm]{\rule[-2cm]{0cm}{4cm}\,}       \\  
 &  &        \\
Mercury  & Venus   & Mars       \\  
 &  &        \\
Rotation: \makebox[1cm]{\hrulefill}    & Rotation:
\makebox[1cm]{\hrulefill} & Rotation: \makebox[1cm]{\hrulefill}
\\
Moons: \makebox[1cm]{\hrulefill}   & Moons: \makebox[1cm]{\hrulefill}
& Moons: \makebox[1cm]{\hrulefill}
\end{tabular}
\end{center}

\begin{center}
\begin{tabular}{ccc} 
\hspace{0.5cm} \framebox[4.0cm]{\rule[-2cm]{0cm}{4cm}\,}
\hspace{0.5cm} & 
\hspace{0.5cm}
 \framebox[4.0cm]{\rule[-2cm]{0cm}{4cm}\,}  \hspace{0.5cm} &
 \hspace{0.5cm} \framebox[4.0cm]{\rule[-2cm]{0cm}{4cm}\,}       \\  
 &  &        \\
Jupiter  & Saturn   & Uranus       \\  
 &  &        \\
Rotation: \makebox[1cm]{\hrulefill}    & Rotation:
\makebox[1cm]{\hrulefill} & Rotation: \makebox[1cm]{\hrulefill}
\\
Moons: \makebox[1cm]{\hrulefill}   & Moons: \makebox[1cm]{\hrulefill}
& Moons: \makebox[1cm]{\hrulefill}
\end{tabular}
\end{center}

Click Options/Orientation/Ecliptic to orient the ecliptic from side to
side.  Then for each planet in turn, double click the box in the Find
panel to lock it, and make sure the field of view is correct (3\arcmin).

\medskip
Sketch the disc of each planet to scale (the box = the SN field of
view) with any features, and indicate any possible moons in the
field. Set the clock interval to 1 day. Observe the planet again with
a few single 1 day time steps (like observations from night to night)
and then continuously, and comment whether you can detect rotation of
the planet, and confirm the presence of moons (possible answers are
yes/no/maybe).

You will see that the planets exhibit very different angular sizes
(according to their true size and distance), and some of them show
distinct phases.

\medskip
\bigskip
\noindent
{\bf 2. Venus}

\medskip
\noindent
We now examine Venus in more detail as the most easily visible example
of an inner planet. We are going to observe it simultaneously in a
wide field, and close up, to see how the two views are related.

\medskip\noindent
{\bf Setup.} The setup requires some patience. Click
Options/orientation/ecliptic; turn off daylight. Click Sun and lock;
re-size the window to a long rectangle to cover the top half of the screen.
Make the field 100\deg\ across. Set the time interval
to about 12 hr. Run the clock and make sure you can see Venus shuttling back
and forth each side of the Sun (you can also see Mercury as
well). Stop the clock.

\medskip
\noindent Now open a second window using File/New. Do the same setup
steps as above, but this time set and lock on Venus, with a field of
2\arcmin. Run the clock with a 12 hr interval, and Venus should change
size due to approaching and receding from us, and change phase.

\medskip
\noindent With both windows running, click the arrow to the right of
the date to Synchronize Times in All Windows.  Check that
the dates change in the same way. 

\medskip
Watch the two windows carefully and see how the views correspond.

\medskip
In the table below, fill out the phase sequence that you observe that
corresponds to the sequence of configurations given in the first
column. Measure the maximum elongations and the angular size of the
disk as the planet goes through its phases. Note that there are two
types of conjunction (depending on where they occur in the sequence),
and elongation is the angle between Venus and the Sun. N.B. To measure
the elongations and the diameters you will need to stop the clock at
the right time; and then resynchronize the motion for the next
measurement.


	\begin{center}
\begin{tabular}{lcccc} \hline \\ [-6pt]
 &  \hspace{0.2cm}  Elongation \hspace{0.2cm} & \hspace{0.2cm} Phase
 \hspace{0.2cm} & \hspace{0.2cm} Angular diameter \hspace{0.2cm} &
 \hspace{0.2cm}   \\ [6pt]
\hline \\ [-6pt]
Conjunction  & 0\deg\  &  & &   \\ \hline \\ [-6pt]
Max elong. east/left &  &  &  &     \\ \hline \\ [-6pt]
Conjunction & 0\deg\ &  &        \\ \hline \\ [-6pt]
Max elong. west/right &  &  &  &      \\ \hline \\ [-6pt]
Conjunction & 0\deg\ &  &   &     \\ \hline \\ [-6pt]
\end{tabular}
\end{center}

\medskip\noindent
Note the dates between successive passages of Venus between two
superior conjunctions: \\
\\
Date 1: \makebox[2cm] \ \ Date 2:  \makebox[2cm]  
       

\medskip\noindent
{\bf Questions.}

\medskip\noindent Estimate the synodic period (in days) from your date
measurements.\makebox[1cm]{\hrulefill}

\medskip
\noindent
Compare this with the value from the relation to the orbital period P=225 days.
\[1/S = 1/P -1/E\]

\medskip\noindent
During roughly half of this synodic period, Venus is east of the Sun,
and is seen as the ``evening star'' after sunset, and for the other
half it is seen as the ``morning star'' before sunrise.   
From your observations, is Venus
approaching or receding from Earth during the evening star period ? \makebox[1cm]{\hrulefill}

\medskip\noindent
Can Venus ever be seen on the meridian at midnight in NY. Explain: \makebox[1cm]{\hrulefill} 

\medskip\noindent
Can Venus ever be seen in the sky at midnight in NY. Explain: \makebox[1cm]{\hrulefill} 



\medskip\bigskip
\noindent
{\bf 3. Mars }

\medskip
\noindent
We now take a similar look at Mars as an example of an
outer planet. Your will see that the behavior of the outer planets 
is somewhat different.

\medskip
Keep both windows open, and in one of them set on and lock Mars, with
a field of 30\arcsec, and in the other set on and lock Mars, with a
field of 100\deg. With a time interval of about 1 day, synchronize the
windows as before and watch.

Record the phases and angular sizes of Mars in the following table,
and record the dates of successive minima in size.



	\begin{center}
\begin{tabular}{lcccc} \hline \\ [-6pt]
   & \hspace{0.5cm} Phase
 \hspace{0.5cm} & \hspace{0.2cm} Angular diameter \hspace{0.2cm} &
 \hspace{0.5cm}   \\ [6pt]
\hline \\ [-6pt]
Min size  &  &  &    \\ \hline \\ [-6pt]
Max size &  &  &      \\ \hline \\ [-6pt]
Min size  &  &  &   \\ \hline \\ [-6pt]
\end{tabular}
\end{center}

\noindent
Also record the dates of successive minima in size.\\

\noindent
Date 1: \makebox[2cm] \ \  Date 2:  \makebox[2cm]  


\medskip\noindent
{\bf Questions.}

\medskip\noindent
Estimate the synodic period for Mars (in years) from your dates:
\makebox[4cm]{\hrulefill}

\medskip
\noindent
Compare this with the value from the relation to the orbital period
P=1.88 yr.
\[1/S = 1/E -1/P\]


\medskip\noindent 
In which direction does Mars usually travel against the background stars: \makebox[4cm]{\hrulefill}

\medskip\noindent
What is the least illuminated phase that Mars reaches: \makebox[4cm]{\hrulefill}

\medskip\noindent
What astronomical body passes Mars every time it is smallest in
angular diameter:
\makebox[4cm]{\hrulefill}

\medskip\noindent
When Mars retrogrades, is it biggest or smallest, and why: \makebox[4cm]{\hrulefill}



\medskip\noindent
From your measurements, what is the ratio of the maximum to minimum
angular size of Mars, i.e., max/min :\makebox[4cm]{\hrulefill}

\medskip\noindent
The radius of Mars' orbit is 1.5 AU. So its nearest approach to Earth is
\makebox[
1cm]{\hrulefill} AU and its farthest distance is
\makebox[1cm]{\hrulefill} AU. \ \
Based on this, what is the theoretical ratio of the max/min angular
sizes:\makebox[1cm]{\hrulefill}

\noindent
(It should be close to your measured ratio above.)


\bigskip
\noindent
{\bf 4. Jupiter and Saturn}

\medskip
While we are examining the planets it is worthwhile to revisit Jupiter
and Saturn.

\medskip
For Jupiter set the interval to 1 hr and study the motions of the 4
bright moons. For Saturn, a time step of 1 hr is also good for looking
looking at the moon motions. A longer time interval of 1 month allows
you to see the ring orientation change.







