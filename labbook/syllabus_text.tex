\noindent{\bf Observational Astronomy / PHYS-UA 13 / Fall 2022 /
  Syllabus }

\noindent {\bf Course Summary}: This course will teach you how to
observe the sky carefully with your naked eye, binoculars, and a small
telescope. You will learn the basics of observable lunar and planetary
properties, and the basics of astronomical coordinates and
observations. The goal is for you to be able to understand and
describe what you see in the sky at night, and to be able to use
charts and coordinates to predict it.

\noindent {\bf Instructors}: Prof. Michael Blanton (726 Broadway,
Rm 941, {\tt mb144@nyu.edu}) and TA Nick Faucher ({\tt
  ntf229@nyu.edu}).

\noindent {\bf Help \& Questions Time}: Prof. Blanton will be available in
  his office and/or on Zoom every Wednesday from 12:30pm to 2pm if you
  have questions or need help with the classwork. Feel free to email
  with an alternative appointment time if that time does not work.
 
\noindent {\bf Books}: The primary books are below.
\begin{itemize}
\item {\it Astronomy: A Self-Teaching Guide}, by Dinah Moche
\item {\it The Amateur Astronomer's Introduction to the Celestial
  Sphere}, by William Millar
\item This laboratory manual. 
\end{itemize}
There is a digital reserve copy of Moche available at the NYU
Library. For Millar, as of writing this syllabus, it is not
available through the library. This lab manual is available in PDF
format on Brightspace. We will also use the following books:
\begin{itemize}
\item {\it Sky \& Telescope's Pocket Sky Atlas}, by Roger W.~Sinnott
(we have copies in the lab rooms).
\item {\it Edmund Mag 5 Star Atlas} (will be supplied to you)
\end{itemize}

\noindent {\bf Lecture}: Each week you will attend one lecture
(at 3:30pm Monday in Room 369 of the Global Center of Academic \&
Spiritual Life). There is reading material listed for each week, and I
will distribute notes on Brightspace prior to each lecture too; it
will be best to read that material {\it before class}.  There will be
quizzes every lecture that will contribute to your grade, and other
activities that you will also learn from, so please attend! Let me
know {\it ahead of time} if you cannot.

\noindent {\bf Grading}: Grades are based on labs (20\% on written
material, 5\% on extra good work), weekly lecture questions (25\%),
weekly quizzes (10\%), one midterm (20\%), and the final (20\%).

\noindent {\bf Lecture Questions}: Each week (starting with the second week)
you will hand in the lecture questions associated with the previous
weeks' lecture. I will grade two of the questions each week (I won't
tell you which!).

\noindent {\bf Quizzes}: Each week (starting with the second week)
there will be a 5-minute quiz at the start of lecture.  Questions will
be based on the lecture questions due that day, and will involve the
conceptual questions (i.e. I won't ask the exact mass of the Sun or
other pure memorization questions).  Prepare for the quiz by answering
the lecture questions each week! For the quiz itself, I'll present two
question options and you can answer either one of the two questions.

\noindent {\bf Exams}: For the midterm and the final you are
responsible for material in the labs, the reading, and the
homework. In preparing for the exams, use the lecture question as a
guide to which material I believe is essential and the types of
questions I will ask.

\noindent {\bf Lab (meet in Meyer 224)}: Arrive on time at 7:00pm for
the lab. We will discuss the contents of the lab and conduct some
indoor activities. If weather permits, we will proceed to the rooftop
observatory. If weather doesn't permit, we will have indoor
activities.

\noindent {\bf What If I Have to Miss Lab?}: If you must miss lab due
to medical reasons or a truly unavoidance circumstance, contact the
professor {\it and} the TA with the justification. Other than such
circumstances, you can miss {\bf one lab} during the semester without
penalty: you must however contact the lab instructor explicitly {\bf
beforehand} to claim this credit. If you are absent for any other lab
without good cause you will lose credit for that lab.  If you miss
more than three labs without good cause, you will not be given a
passing grade in the course.

\noindent {\bf Weekly Lab Entries}: Finally, the last lab in this book
should have entries filled in {\bf every week}, and should be handed
in on the final lab date.

\noindent {\bf No Switching Between Labs}: You cannot switch
between the lab's Monday and Wednesday sections mid-semester, because
in general they will be on different schedules. The timing of the
indoor and outdoor labs for each section will be driven mostly by the
weather. 

\noindent {\bf Dress Warmly}: Please dress appropriately for remaining
outside for an extended period, including hats and gloves when
appropriate, which will be more often than you think.

\noindent {\bf Do not use your phones during lab unless directed
to!}

\noindent {\bf Accommodations}: I am committed to creating an
inclusive and accessible classroom environment for students of all
abilities. Students who may need academic accommodations are advised
to reach out to the Moses Center for Student Accessibility as early as
possible in the semester for assistance (212-998-4980 or {\tt
mosescsd@nyu.edu}). Information about the Moses Center can be found at
{\tt \url{http://www.nyu.edu/csd}}.  If you need any support in
connecting with the Moses Center or other resources, please also let
me know. Students who need accommodations for the laboratory component
of the class should talk to me about their needs so we can arrange a
solution.

\noindent {\bf Health}: I encourage all students to
attend to their general mental and physical health. To access the
University's extensive health and mental health resources, contact the
NYU Wellness Exchange. You can call its private hotline (212-443-9999)
or chat (in six different languages) via the Wellness Exchange app or
at {\tt wellness.exchange@nyu.edu}, available 24 hours a day, seven
days a week, to reach out to a professional who can help to address
day-to-day challenges as well as other health-related concerns.

\clearpage

\baselineskip 0pt
\begin{sidewaystable}
\small
\begin{tabular}{|c||c|c|}
\hline
{\it Sep.~12} 
& The Celestial Sphere: angles \& coordinates 
& Millar Ch.~2; Edmund pp.~1--9; Moche 1.1--1.15
\cr 
{\it Sept.~19} 
& Introduction to telescopes
& Moche 2.11--2.21
\cr
{\it Sept.~26} 
& Rotation and Orbit of the Earth
& Millar Ch.~4.1--4.5, Ch.~6; Moche 1.16--1.23
\cr
{\it Oct.~3} 
& Finding your way in the sky
& Edmund pp.~30--32
\cr
{\it Oct.~11} {\bf (Tuesday!)}
& Stars
& Millar Ch.~3; Moche Ch.~3
\cr
{\it Oct.~17} 
& Variables and Binaries
& Millar Ch.~3
\cr
{\it Oct.~24} 
& {\bf Midterm exam in class!}
& ---
\cr
{\it Oct.~31} 
& Galaxies
& Moche Ch.~6
\cr
{\it Nov.~7} 
& The Moon
& Millar Ch.~6; Edmund p.~34; Moche Ch.~10
\cr
{\it Nov.~14} 
& Planets and their motions
& Moche Ch. 8.5--8.9, Ch.~9
\cr
{\it Nov.~21} 
& Moons of Jupiter and Saturn
& Moche Ch.~9
\cr
{\it Nov.~28} 
& Precession \& nutation


& Millar Ch.~4.6--4.7
\cr
{\it Dec.~5} 
& Tides \& Eclipses
& Millar Ch.~7
\cr
{\it Dec.~12} 
& Asteroids, Comets, Meteors
& Moche Ch.~11
\cr
{\it Dec. 20 2:00pm--3:50pm}
& {\bf FINAL EXAM, GCASL 369}
& 
\cr
\hline
\end{tabular}
\end{sidewaystable}

\clearpage

\baselineskip 0pt
\begin{sidewaystable}
\small
\begin{tabular}{|c|c|c|c|c|}
\hline
{\it Sept.~12--14} 
& Jupiter, Saturn
& ---
& Waxing Gibbous Moon
& Sunset 7:07pm EDT
\cr
{\it Sept.~19--21} 
& Jupiter, Saturn
& Andromeda rising
& ---
& Sunset 6:56pm EDT
\cr
{\it Sept.~26--28}
& Jupiter, Saturn
& Andromeda
& ---
& Sunset 6:44pm EDT
\cr
{\it Oct.~3--5}
& Jupiter, Saturn
& Andromeda
& First Quarter Moon
& Sunset 6:32pm EDT
\cr
{\it Oct.~11--12}
& Jupiter, Saturn
& Andromeda
& Waxing Gibbous Moon
& Sunset 6:21pm EDT
\cr
{\it Oct.~17--19}
& Jupiter, Saturn
& Andromeda
& ---
& Sunset 6:10pm EDT
\cr
{\it Oct.~24--26} 
& Jupiter, Saturn
& Andromeda
& ---
& Sunset 6:00pm EDT
\cr
{\it Oct.~31--Nov.~2}
& Jupiter, Saturn
& Andromeda
& First Quarter Moon
& Sunset 5:52pm EDT
\cr
{\it Nov.~7--9}
& Jupiter, Saturn, Mars rising
& Andromeda, Pleiades
& Full Moon
& Sunset 4:44pm EST
\cr
{\it Nov.~14--16}
& Jupiter, Saturn, Mars rising
& Andromeda, Pleiades
& ---
& Sunset 4:37pm EST
\cr
{\it Nov.~21}\footnote{No lab on Wednesday, November 23 (night before
Thanksgiving)}
& Jupiter, Saturn, Mars
& Andromeda, Pleiades
& --- 
& Sunset 4:32pm EST
\cr
{\it Nov.~28--30}
& Jupiter, Saturn, Mars
& Andromeda, Pleiades
& First Quarter Moon
& Sunset 4:29pm EST
\cr
{\it Dec.~5--7}
& Jupiter, Saturn, Mars
& Andromeda, Pleiades
& Full Moon (close to Mars on 7th)
& Sunset 4:28pm EST
\cr
{\it Dec.~12--14}
& Jupiter, Saturn, Mars
& Andromeda, Pleiades
& ---
& Sunset 4:29pm EST
\cr
\hline
\end{tabular}
\end{sidewaystable}

\clearpage

Eclipses Fall 2022:
\begin{itemize}
\item Nov. 8: Total Lunar Eclipse (maximum around 6am ET)
\end{itemize}

\baselineskip 12pt
