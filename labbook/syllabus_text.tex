\paragraph{Observational Astronomy / NYU PHYS-UA 13 / 2023 Fall / Syllabus}

\paragraph{Course Summary:} This course will teach you how to
observe the sky carefully with your naked eye, binoculars, and a small
telescope. You will learn the basics of observable lunar and planetary
properties, and the basics of astronomical coordinates and
observations. The goal is for you to be able to understand and
describe what you see in the sky at night, and to be able to use
charts and coordinates to predict it.

\paragraph{Instructors:} Professor David W Hogg (726 Broadway,
Rm 1050, {\tt david.hogg@nyu.edu}) and TA Valentina~Tardugno ({\tt
  vt2189@nyu.edu}).

\paragraph{Help \& Questions Time:} Prof Hogg will be available in
  his office every Monday from 09:00 through 12:00 if you
  have questions or would like help on anything.

\paragraph{Books:} The primary books are
\begin{itemize}
\item {\it Astronomy: A Self-Teaching Guide}, by Dinah Moche
\item {\it The Amateur Astronomer's Introduction to the Celestial
  Sphere}, by William Millar
\item {\it Edmund Mag 5 Star Atlas} (this will be supplied to you)
\end{itemize}
There is a digital copy of Moche available at the NYU
Library. For Millar, as of writing this syllabus, it is not
available through the library. This lab manual is available in PDF
format on Brightspace. We may also use the following:
\begin{itemize}
\item {\it Sky \& Telescope's Pocket Sky Atlas}, by Roger W.~Sinnott
(we have copies in the lab rooms).
\end{itemize}

\paragraph{Lecture:} Each week there is one Lecture, and
there will be reading material assigned each week via Brightspace.
There will be Lecture Questions related to each lecture that
you can use to check that you are following the material.

\paragraph{Grading:} Grades are based on Labs (65\%) 
one Midterm (10\%), and the Final (25\%).

\paragraph{Exams:} For the Midterm and the Final you are
responsible for material in the Labs, in the Lectures, and in the assigned reading.
In preparing for the exams, use the Lecture Questions as a
guide to which material is essential (and the types of
questions to be asked).

\paragraph{Lab (in Meyer 224):} Arrive on time at 7:00pm for
the Lab. We will discuss the contents of the Lab and conduct some
indoor activities. If weather permits, we will proceed to the rooftop
observatory. If weather doesn't permit, we will have indoor
activities.

\paragraph{What If I Have to Miss Lab?} If you must miss lab due
to medical reasons, contact the
professor with the justification. If you are absent for any other lab
without good cause you will lose credit for that lab. If you miss
more than three labs without good cause, you will not be given a
passing grade in the course.

\paragraph{Weekly Lab Entries:} There is, in addition to the weekly
Labs, a full-semester Lab, which
should have entries filled in {\bf every week}, and should be handed
in on the final Lab date, as a final Lab.

\paragraph{No Switching Between Labs:} You cannot switch
between the lab's Monday and Wednesday sections, because
in general they will be on different schedules, and because there are
room and facility occupation rules. The timing of the
indoor and outdoor labs for each section will be driven mostly by the
weather.

\paragraph{Dress \& Safety:} Please dress appropriately for remaining
outside for an extended period, including hats and gloves when
appropriate, which will be more often than you think.
There will be some safety rules in the Labs and at the rooftop observatory.
Failure to obey the rules will get you removed from the Lab.

\paragraph{Accommodations:} We are committed to creating an
inclusive and accessible classroom environment for students of all
abilities. Students who may need academic accommodations are advised
to reach out to the Moses Center for Student Accessibility as early as
possible in the semester for assistance (212-998-4980 or {\tt
mosescsd@nyu.edu}). Information about the Moses Center can be found at
{\tt \url{http://www.nyu.edu/csd}}.  If you need any support in
connecting with the Moses Center or other resources, please also let
me know. Students who need accommodations for the laboratory component
of the class should talk to me about their needs so we can arrange a
solution.

\clearpage

\baselineskip 0pt
\begin{sidewaystable}
\small
\begin{tabular}{|c||c|c|}
\hline
{\it Sep.~12} 
& The Celestial Sphere: angles \& coordinates 
& Millar Ch.~2; Edmund pp.~1--9; Moche 1.1--1.15
\cr 
{\it Sept.~19} 
& Introduction to telescopes
& Moche 2.11--2.21
\cr
{\it Sept.~26} 
& Rotation and Orbit of the Earth
& Millar Ch.~4.1--4.5, Ch.~6; Moche 1.16--1.23
\cr
{\it Oct.~3} 
& Finding your way in the sky
& Edmund pp.~30--32
\cr
{\it Oct.~11} {\bf (Tuesday!)}
& Stars
& Millar Ch.~3; Moche Ch.~3
\cr
{\it Oct.~17} 
& Variables and Binaries
& Millar Ch.~3
\cr
{\it Oct.~24} 
& {\bf Midterm exam in class!}
& ---
\cr
{\it Oct.~31} 
& Galaxies
& Moche Ch.~6
\cr
{\it Nov.~7} 
& The Moon
& Millar Ch.~6; Edmund p.~34; Moche Ch.~10
\cr
{\it Nov.~14} 
& Planets and their motions
& Moche Ch. 8.5--8.9, Ch.~9
\cr
{\it Nov.~21} 
& Moons of Jupiter and Saturn
& Moche Ch.~9
\cr
{\it Nov.~28} 
& Precession \& nutation


& Millar Ch.~4.6--4.7
\cr
{\it Dec.~5} 
& Tides \& Eclipses
& Millar Ch.~7
\cr
{\it Dec.~12} 
& Asteroids, Comets, Meteors
& Moche Ch.~11
\cr
{\it Dec. 20 2:00pm--3:50pm}
& {\bf FINAL EXAM, GCASL 369}
& 
\cr
\hline
\end{tabular}
\end{sidewaystable}

\clearpage

\baselineskip 0pt
\begin{sidewaystable}
\small
\begin{tabular}{|c|c|c|c|c|}
\hline
{\it Sept.~12--14} 
& Jupiter, Saturn
& ---
& Waxing Gibbous Moon
& Sunset 7:07pm EDT
\cr
{\it Sept.~19--21} 
& Jupiter, Saturn
& Andromeda rising
& ---
& Sunset 6:56pm EDT
\cr
{\it Sept.~26--28}
& Jupiter, Saturn
& Andromeda
& ---
& Sunset 6:44pm EDT
\cr
{\it Oct.~3--5}
& Jupiter, Saturn
& Andromeda
& First Quarter Moon
& Sunset 6:32pm EDT
\cr
{\it Oct.~11--12}
& Jupiter, Saturn
& Andromeda
& Waxing Gibbous Moon
& Sunset 6:21pm EDT
\cr
{\it Oct.~17--19}
& Jupiter, Saturn
& Andromeda
& ---
& Sunset 6:10pm EDT
\cr
{\it Oct.~24--26} 
& Jupiter, Saturn
& Andromeda
& ---
& Sunset 6:00pm EDT
\cr
{\it Oct.~31--Nov.~2}
& Jupiter, Saturn
& Andromeda
& First Quarter Moon
& Sunset 5:52pm EDT
\cr
{\it Nov.~7--9}
& Jupiter, Saturn, Mars rising
& Andromeda, Pleiades
& Full Moon
& Sunset 4:44pm EST
\cr
{\it Nov.~14--16}
& Jupiter, Saturn, Mars rising
& Andromeda, Pleiades
& ---
& Sunset 4:37pm EST
\cr
{\it Nov.~21}\footnote{No lab on Wednesday, November 23 (night before
Thanksgiving)}
& Jupiter, Saturn, Mars
& Andromeda, Pleiades
& --- 
& Sunset 4:32pm EST
\cr
{\it Nov.~28--30}
& Jupiter, Saturn, Mars
& Andromeda, Pleiades
& First Quarter Moon
& Sunset 4:29pm EST
\cr
{\it Dec.~5--7}
& Jupiter, Saturn, Mars
& Andromeda, Pleiades
& Full Moon (close to Mars on 7th)
& Sunset 4:28pm EST
\cr
{\it Dec.~12--14}
& Jupiter, Saturn, Mars
& Andromeda, Pleiades
& ---
& Sunset 4:29pm EST
\cr
\hline
\end{tabular}
\end{sidewaystable}

\clearpage

Eclipses Fall 2022:
\begin{itemize}
\item Nov. 8: Total Lunar Eclipse (maximum around 6am ET)
\end{itemize}

\baselineskip 12pt
