
\noindent
In this project we shall explore some aspects of the Moon's orbit,
phases, and topography.


\medskip \bigskip
\noindent
{\bf 1. The Waxing Crescent Moon}

\medskip\noindent The Moon first becomes visible in its phase cycle a
few days after the new Moon -- in the waxing crescent phase. 
We shall first review how this looks in
the sky using SN.

Using the table in the FG (starts on p 350), find the date of the next
new Moon. Set up SN for NY, looking west at 5 pm, for the new Moon
day, plus two days, when the Moon is usually first seen. (Technical
aside: make sure the Moon is activated using View/Solar
System/Planets-Moons). Set the time rate to 100, and start the clock.

Watch the Sun set and the Moon appear. Perhaps you have seen this many
times in the real sky.

\medskip
The Moon sets down and to the (right/left): \makebox[2cm]{\hrulefill}

The horns of the crescent Moon point (up/down) and to the (left/right): \makebox[2cm]{\hrulefill}

\medskip\noindent
The latter should be obvious when you realize where the Sun is. Note
that if you set the date to that of the new Moon, the
Moon sets at nearly the same time as the Sun, and the edge is not lit:
so it is not seen.

\bigskip
\noindent
{\bf 2. The Changing Phases}

\medskip \noindent
In the days following the new phase, the Moon becomes more and more
prominent in the evening sky and the phase changes as well.
With the time at about 6:30 pm so it is just about dark, set the time
interval to 1 solar day and make single time steps to see what happens.

\medskip
Relative to the Sun, the Moon moves progressively to the (east/west): \makebox[2cm]{\hrulefill}

\noindent \medskip
How do we know where the Moon will be during the cycle? Well, the
following table covers the main points during the cycle. Turn on the
meridian (View/AltAz/Meridian), set the clock speed to 3000,
and for the date of each phase given in the FG for the next cycle,
stop the clock just as the Moon transits the meridian to the
south. (Note: the Dec can change a lot: why?) Enter the transit time to the
nearest hour in the Table.

\bigskip
\begin{center}
\begin{tabular}{lc} \hline
Phase  & \ \ \ \ Transit  \\
 \hline
New   &    noon    \\ \hline
First quarter   &        \\ \hline
Full   &    \\ \hline
Third quarter   &        \\ \hline
New Moon    &  noon     \\ \hline
\end{tabular}
\end{center}

\noindent For intermediate phases (e.g., waning crescent, waxing gibbous) you
can estimate the transit time between those in the table; and for
rising or setting times of the Moons you can subtract or add
approximately 6 hr.

\bigskip
\noindent
Based on this table, try the following. For each of the phases and
times listed, describe where the moon will be, and draw its phase as
seen in the sky (that is, assume down on the page is the direction
towards the horizon). If necessary, assume we are in New York City.

\medskip
i) A full Moon soon after sunset (or 6 pm). 

\centerline{\psfig{figure={o4s_f2.eps},width=2.5cm}}


ii) A first quarter Moon, just before midnight.

\centerline{\psfig{figure={o4s_f2.eps},width=2.5cm}}

iii) A third quarter Moon, soon after midnight.

\centerline{\psfig{figure={o4s_f2.eps},width=2.5cm}}

iv) A waning crescent Moon, just before sunrise (or 6 am).

\centerline{\psfig{figure={o4s_f2.eps},width=2.5cm}}

\vspace{9cm}

\bigskip
\noindent
{\bf 3. Moon's Orbit}

\medskip\noindent The speed with which the Moon changes its position
is quite remarkable. To see this set up as follows: Double click Moon
in Find to lock it; click off the daylight and horizon; click
Options/Orientation/Ecliptic and View/Ecliptic Guides/The Ecliptic (to
select the ecliptic); also turn on the constellation boundaries and
labels, and set the clock rate at 10,000.

\noindent You should see the moon traveling in its orbit against the background
stars and changing phase at the same time. It circles the sky in 27.3
days (the sidereal period).

\medskip
Using single time steps, count how many days on average the Moon spends in
each constellation:
\makebox[2cm]{\hrulefill}

\medskip
Calculate what the above should be from the sidereal period, assuming
the average constellation is 30\deg\ wide.  \makebox[2cm]{\hrulefill}

\medskip\noindent
{\bf The Nodes.}  You will see that the Moon does not follow the ecliptic exactly but crosses it
at some points; these are the nodes of the orbit which are important
for understanding eclipses, which we shall discuss later in the
semester.

\medskip
How many nodes are there: \makebox[2cm]{\hrulefill}

In which constellations are the nodes at present:
\makebox[2cm]{\hrulefill}
 
\medskip\noindent
{\bf Phase cycle.} The phases do not follow the sidereal period
exactly, because the Sun is also
moving (around the ecliptic); the phase period is 29.5 days as we saw
in class.

Carefully examine when the full Moons occur in the Moon's orbital
journey, and list the sequence of constellations in which successive
full Moons occur: \\
\makebox[12cm]{\hrulefill}

Explain the sequence: \makebox[12cm]{\hrulefill}

\bigskip
\bigskip
\noindent
{\bf 4. Moon Close Up} 

\medskip
\noindent
The Moon close up in SN is a bit like the Moon as it appears in
binoculars. To visit, make sure it is still locked, and set the field
of view to 1\deg. Run the clock until the Moon is full; then stop it.

\noindent Study the picture. Using the map in the back of the Atlas (turned
upside down) identify the main lunar features: the mares, the biggest
craters, and the craters with rays.

\medskip
\noindent
{\bf Sizes.}
How big is a mare or a crater? Well, the diameter of the Moon is
3,480~km (or 2,160 miles). Measure its angular diameter to the nearest
tenth of an arc minute using the angular measurement cursor, and enter
the result in the table below. Now measure the angular diameter of the
Mare Crisium (which is easy to identify with the naked eye), and the
crater Plato -- which is an example of a fairly large crater (you may
need to increase the magnification a bit to measure Pluto). Use your
measurement to calculate the size of Crisium and Plato in km.



\bigskip
\begin{center}
\begin{tabular}{lcc} \hline
     & \  size (in arc min) &\hspace{1.5cm} size in km  \\
 \hline
Moon           &  &       \\ \hline
Mare Crisium   &  &       \\ \hline
Plato          &  &       \\ \hline
\end{tabular}
\end{center}

\medskip
\noindent
{\bf Lunar Day.} How long does sunlight fall at a given place on the
Moon? Choose a crater, e.g., Copernicus. Single step the clock until
the terminator (shadow edge) just touches it as the Sun rises at this
point.  Note the date and time. Run the clock and stop it just as the
shadow reaches it. Note the date/time again. Give the difference, to
the nearest half day.

\medskip
Continuous sunlight on the Moon lasts: \makebox[4cm]{\hrulefill}

\medskip
What fraction of the 29.5 day phase cycle is this: \makebox[4cm]{\hrulefill}

\bigskip
\noindent
{\bf Orbit Effects}. Besides phases, the actual appearance of the Moon
has some subtleties. To see two of them, click Options/Solar
System/Planets-Moons/Surface Grid which sets up a grid on the Moon and
shows the poles. Set the clock speed at about 100,000.  You will see
two effects.

\medskip\noindent Libration. The Moon appears to roll slightly side
ways and up and down. Of course the changing phases make this less
obvious, but look at Mare Crisium and you will see the effect. The
effect is due mainly to the elliptical shape of the lunar orbit: the
Moon points towards the center of the ellipse, so we get to see around
the edges a bit, so see more than 50\% of the surface (actually
59\%).

\medskip\noindent Size. The second effect is that the Moon gets
bigger and smaller over time. By using single steps, estimate the time
(to within a day), between successive maxima.

\medskip
Time between maxima: \makebox[4cm]{\hrulefill}

Comment on whether this
seems reasonable: \makebox[4cm]{\hrulefill}


\bigskip
\noindent
{\bf 5. The Far Side} 

\medskip
\noindent
There is no permanent \emph{dark side} of the Moon, but there is a far
side that is not visible from the Earth. You can however visit it with
SN. It takes some space navigation.


First set the Moon to the new phase (so that the far side is
illuminated). Then in Find, click Moon, and in the pull down menu next
to it click Go There. When there, adjust the field of view, and using
the mouse drag round the Moon to see the lighted (far) side.


\medskip\noindent
There is one striking difference in appearance between the familiar
face and the far side.

\medskip
What is it: \makebox[12cm]{\hrulefill}
 

