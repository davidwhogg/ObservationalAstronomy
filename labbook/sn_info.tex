\noindent
{\hfill \Large {\bf Starry Night Information Sheet} \hfill}


\bigskip

\noindent
Starry Night is a sky simulation program that we shall often use in
the indoor labs. Our version is Starry Night Pro.\ 6.0. It has lots of
features that are fairly intuitive to use once you get the ideas
behind it. This sheet provides a brief introduction and some simple
commands to start you off. Try them out.

\bigskip
\noindent
{\bf 1. Overview of operation}

\smallskip
\noindent
When you enter SN you see a view of the sky as seen from a given
location, in a given direction, with a given field of view,
at a given time and date. The view may be from NY,
or some other place on Earth, or from another planet, or somewhere out
in space.

The most important information about the setup is
shown in the display windows along the top: date and time, rate of
time flow, your location (e.g. New York), direction of view (called
Gaze), and field of view (called Zoom). There are several ways to
change things:

\medskip
{\bf Pull Down Menus:} Along the top; note options are found
under both ``View'' and ``Options.''

\medskip
{\bf Tool bar:} Along top, from left:
\begin{itemize}
\item {\it mouse control}: controls what the mouse does (drag, ruler,
  etc)
\item {\it time and date}: set time and date to show (and whether
  standard or daylight savings time)
\item {\it time flow rate}: to speed up the flow of time in the
  display
\item {\it viewing location}: controls where the observer is located
\item {\it viewing direction}: direction of observation (most useful
  are NSEW buttons below)
\item {\it zoom}: angular field-of-view of window
\end{itemize}

\medskip
{\bf Tabs:} Along the left margin (most useful are ``Find'' and
``Options'').
\medskip

\medskip
\noindent
When you first enter the program, it is helpful to check that the
location is New York and that the time is now. If not, use the pull
down menu in the Viewing Location display to change, and click the Now button.

\clearpage
\noindent
{\bf 2. Basic Operations and How to change things}

\smallskip
\noindent


\begin{itemize}
\item {\bf Move around the sky:}  Hold down the left mouse button and drag
 the hand cursor.

\item {\bf Look at the horizon N, S, E, W, or at the  zenith:} Click the
  N, S, E, W, or Z buttons under the Gaze display.

\item {\bf Change field of view (zoom):} Use + $-$ buttons below Zoom
  display, or the pull down menu in the display. 

\item {\bf Change viewing location:} Pull down menu in display
  window. Can be specified by place name, or by latitude and longitude.

\item {\bf Change time/date:} Click in display window and type-over,
 or use keyboard arrows. 

\item{\bf Start/stop time:} Use CD like buttons below Time Flow display.

\item {\bf Change rate of time flow:} Click in Time Flow display and type-over or
 use keyboard arrows, or use drop down menu in display. 1 $\times$
 means as the sky moves in real time, which is quite slow.


\item{\bf Turn on/off horizon:} Pull down View menu, click Hide Horizon.

\item{\bf Turn on/off daylight:} Pull down View menu, click Hide Daylight.

\item{\bf Identify a star:} Point to it with the mouse cursor.  

\item{\bf Display Alt/Az grid lines:} Pull down View menu, click
  Alt/Az Guides, then grid. 

\item{\bf Display RA grid lines:} Pull down View menu, click
  Celestial Guides, then grid. 

\item{\bf Display constellation boundaries, labels, illustrations:}
  Pull down View menu, click constellations, then boundaries, labels,
  illustrations. 


\item{\bf Measure an angle on the sky:} Point to first star, and drag with left
  mouse button to the second star or position.

\item {\bf Find an object:} Click Find tab at left and select object. 

\end{itemize}

\bigskip
\medskip
\noindent
{\bf 3. Emergency! Lost in space or time?}

\smallskip
\noindent
Stop time flow on CD controls. Press Home button under Viewing
Location and check this is set for NY. Click S button under Gaze
window to look south.  

